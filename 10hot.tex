\chapter{Горячее и холодное}

\section{Может ли холодное нагреть горячее?}

Вопрос в заголовке конечно же имеет отрицательный ответ: при контакте двух тел тепло переходит от горячего к холодному.%
\footnote{Это является следствием второго закона термодинамики, экспериментально установленного принципа.}
Поэтому даже спрашивать следующее кажется глупым:

\paragraph{Задача.}
Можно ли, используя стакан воды нагретый до 100~°C, нагреть стакан молока с начальной температурой 0~°C до температуры выше чем 50~°C, то есть их общей температуры при смешивании?
Предполагаем, что стаканы одного размера.
Будем считать, что вода и молоко полностью идентичны всеми своими тепловыми свойствами.%
\footnote{В частности, у них одинаковые удельные теплоёмкости. То есть одинаковое количество тепла одинаково повышает температуры у равных масс воды и молока.}
Тепло не поступает извне, \emph{но разрешается использовать дополнительные сосуды.}

\paragraph{Решение.}
Это можно проделать, не нарушая второй закон термодинамики.
Для этого нам понадобится ещё один пустой стакан и крохотный ковшик.
Зачерпнём ковшик холодного молока, опустим его в горячую воду и подождём, пока температура не уравновесится.
Перельём содержимое ковшика в пустой стакан.
Повторим процесс, пока всё молоко не окажется в третьем стакане.
По пути в третий стакан каждая порция молока получает немного тепла от воды.
Я утверждаю, что после всех переливаний молоко окажется теплее воды.
Чтобы это понять, представьте, что вы вытаскиваете \emph{последний ковшик} молока из воды.
В этот момент молоко в ковшике той же температуры, что вода.
А остальное молоко ещё теплей, ведь ранние порции нагревались сильнее.
Значит, после добавления этого последнего ковшика к остальному молоку, оно окажется теплее воды.

\begin{figure}[ht!]
\centering
\begin{lpic}[t(2mm),b(2mm),r(0mm),l(0mm)]{pics/10.1}
\lbl{8,7;\parbox{20mm}{\footnotesize\centering молоко\\ при 0~°C}}
\lbl{41.2,10;\textcolor{white}{\parbox{20mm}{\footnotesize\centering вода\\ изначально \\была при\\ 100~°C}}}
\lbl{41.2,31;\parbox{28mm}{\footnotesize\centering температура\\уравновесилась}}
\lbl{74.4,3.7;\parbox{20mm}{\footnotesize\centering тёплое\\молоко}}
\end{lpic}
\caption{В стакане $N$ ковшиков, и
каждый ковшик молока при $0\degree$ охлаждает воду в $1+1/N$ раз.}
\label{pic:10.1}
\end{figure}

\paragraph{Приготовление лосося и постоянная Эйлера $\bm{e}$.}
Мы увидели, что этот метод нагревает молоко до температуры выше 50~°C, но насколько именно?
Если ковшик достаточно мал, то молоко нагреется приблизительно до 63~°C.
То есть можно обжечься, однако заметим, что это рекомендованная температура до которой надо прогревать внутренность филе лосося при приготовлении.

Математику должно показаться занятным, что предельно достижимая температура воды при этом методе равна
\[\frac{100\degree}{e},\]
где $e = 2{,}718\ldots$ — постоянная Эйлера, то есть предел последовательности $(1 \z+ 1/N)^N$ при $N \to \infty$.

Чтобы это обосновать,
давайте считать, что стакан вмещает $N$ ковшиков, где $N$ — целое число.
Один ковшик холодного молока при $0\degree$ и стакан с водой температуры $T$, придут в тепловое равновесие при температуре
\[
\frac{T}{1 + 1/N}.
\]
Действительно, при добавлении одного холодного ковшика к $N$ тёплым весь запас тепла $N$ ковшиков равномерно распределяется между $N+1$ ковшиками, и, следовательно, в расчёте на один ковшик, тепло  уменьшается в $\tfrac{N+1}{N} = {1+1/N}$ раз.
Значит и температура уменьшается в то же число раз.
Таким образом, при переходе от шага $k$ к шагу $k+1$, для температуры воды выполняется рекуррентное соотношение:
\[
T_{k+1} = \frac{T_k}{1 + 1/N},
\qquad T_0 = 100\degree.
\]
Во время всей процедуры исходная температура $T_0 = 100\degree$ делилась на одну и ту же величину $N$ раз, и в конце охладится до
\[
T_N = \frac{100\degree}{(1 + 1/N)^N} \;\;\approx\;\; \frac{100\degree}{e}.
\]

\paragraph{Температура тела.}
Взяв $N$ достаточно большим, получим, что $(1+1/N)^N \approx e = 2{,}718\ldots$,
и, значит,
\[
T_N \approx \frac{100\degree}{e} \approx 36{,}8\degree.
\]
Эта температура подозрительно близка к температуре человеческого тела.
Если вдруг нужно вычислить $e$ в экстренной ситуации и под рукой есть градусник, то можно измерить свою температуру в градусах Цельсия и подставить её в соотношение
\[
\frac{100\degree}{T_{\text{человек}}} \approx e.
\]
(Если у вас жар, то оценка получится заниженной, а если гипотермия, то завышенной.)
Получается, что натуральный логарифм (тот самый, что с основанием $e$) связан с человеческой натурой.

Раз уж пошла речь о совпадениях, напомню, что температура человеческого тела связана с оптимальной температурой для приготовленного лосося ($63\degree$):
\[
T_{\text{человек}} + T_{\text{лосось}} \;\approx\; 100\degree.
\]

\paragraph{Ещё теплей.}
Оказывается, что можно добиться почти идеального обмена температурами двух жидкостей — по крайней мере, в теории.
Для этого нужно мелко дробить обе жидкости, а не только молоко.
Практически это можно реализовать, пропуская воду и молоко в противоположных направлениях через две трубки, находящиеся в тесном тепловом контакте, как на рисунке \ref{pic:10.2}. Если прокачивать молоко слева, а воду справа, то мы получим почти идеальный обмен теплом.
Это простое устройство называется (противоточным) теплообменником.

\begin{figure}[ht!]
\centering
\begin{lpic}[t(2mm),b(2mm),r(0mm),l(0mm)]{pics/10.2}
\lbl[br]{6,17;\parbox{20mm}{\footnotesize\raggedleft холодное\\молоко}}
\lbl[tr]{6,1.5;\parbox{20mm}{\footnotesize\raggedleft холодная\\вода}}
\lbl[lb]{71,17.5;\parbox{20mm}{\footnotesize горячее\\молоко}}
\lbl[lt]{71,2;\parbox{20mm}{\footnotesize горячая\\вода}}
\lbl{39,9.5;{\footnotesize передача тепла}}
\end{lpic}
\caption{Противоточный теплообменник обеспечивает почти полный обмен температурами.}
\label{pic:10.2}
\end{figure}

Идея теплообменника используется в природе.
Например наши руки снабжены теплообменниками, ведь кровь глубоких вен идёт противотоком с кровью артерий.
В холодных условиях холодная кровь от кистей возвращается по этим венам, получая тепло от идущей наружу артериальной крови. Согретая поступающая кровь помогает поддерживать температуру тела.
При этом охлаждённая артериальная кровь, направляющаяся к конечностям, уже не отдаёт так много тепла наружу.
В жарких условиях этот механизм отключается: кровь идёт по поверхностным венам, помогая рассеивать избыточное тепло.

Теплообменниками снабжены собаки, овцы, верблюды и другие животные.
Они помогают поддерживать температуру мозга ниже, чем у остального тела: более холодная венозная кровь из рта и носа охлаждает артериальную кровь, питающую мозг (наиболее уязвимый к перегреву орган).
Кролики, у которых нет такого механизма, рискуют погибнуть от перегрева, если в жаркую погоду их долго преследует собака.
У серых китов по всей поверхности языка имеется множество противоточных теплообменников (язык не может быть утеплён слоем жира).
И этот механизм не ограничивается только млекопитающими: некоторые рыбы, например тунец, используют противоточные теплообменники, чтобы поддерживать температуру мышц на целых 14 °C выше температуры воды.

\section{Насос и молекулярный пинг-понг}\label{Насос и молекулярный пинг-понг}

\paragraph{Вопрос.}
Можно заметить, что насос нагревается когда вы накачиваете шину велосипеда.
Происходит ли это нагревание только из-за трения или ещё по какой-то другой причине?

\paragraph{Ответ.}
Основная причина не в трении.
Воздух нагревается при сжатии, и уже он нагревает его стенки.

\paragraph{Вопрос.} А почему сжатие нагревает воздух?

\paragraph{Ответ.}
Если вы решите сыграть в пинг-понг или теннис, то ответ окажется в ваших руках.
Ракетка, ударяющая по летящему шарику, подобна движущемуся поршню насоса, подтолкающего молекулу воздуха.
Благодаря движению ракетки шарик после удара приобретает б\'{о}льшую скорость (см. рисунок \ref{pic:10.3}).
(Прибавка скорости равна удвоенной скорости налетающей ракетки.
Предполагается, что столкновение совершенно упругое и масса шарика мала по сравнению с массой ракетки.)
Точно так же и молекулы, отталкиваясь от поршня, ускоряются и нагревают воздух.

\begin{figure}[ht!]
\centering
\begin{lpic}[t(2mm),b(2mm),r(0mm),l(0mm)]{pics/10.3}
\lbl{23,23,30;{\footnotesize холодный}}
\lbl{34,7,52;{\footnotesize горячий}}
\lbl[tl]{58,26;$v$}
\lbl[t]{66,23;$V$}
\lbl[t]{75,20;$V+2v$}
\lbl{65,7;\parbox{48mm}{\footnotesize\centering молекула ускоряется на\\ удвоенную скорость поршня}}
\end{lpic}
\caption{Молекула ускоряется на удвоенную скорость поршня.}
\label{pic:10.3}
\end{figure}

На рисунке \ref{pic:10.4} показано, как меняется температура воздуха внутри насоса.
Согласно графику, средняя температура в насосе выше, чем снаружи.
При этом стенка насоса нагревается до некоторого среднего уровня.

\begin{figure}[ht!]
\centering
\begin{lpic}[t(2mm),b(2mm),r(0mm),l(0mm)]{pics/10.4}
\lbl[t]{45,-1;{\footnotesize время}}
\lbl[b]{0,12.5,90;{\footnotesize температура}}
\lbl[t]{45,7;{\footnotesize температура снаружи}}
\lbl[l]{10,3.5;{\footnotesize сжатие}}
\lbl[b]{14,17;\parbox{22mm}{\footnotesize\centering выход\\ воздуха}}
\end{lpic}
\caption{Зависимость температуры в насосе от времени.}
\label{pic:10.4}
\end{figure}

\section{Теплонасос из велонасоса}

Теплонасос — это холодильник, которым пользуются для нагрева.
Холодильник перекачивает тепло изнутри наружу.
Если холодильник называют теплонасосом, то всего лишь хотят сказать что он используется для нагрева, а не охлаждения,
однако эти две функции неотделимы друг от друга.

\paragraph{Вопрос.}
Можно ли использовать велонасос как теплонасос, то есть перекачивать им тепло внутрь тёплого помещения из холодной зимней стужи снаружи?
Меня интересует лишь возможность этого в принципе, решение не обязано быть практичным.

\paragraph{Ответ.}
Насос с закрытым выходным отверстием — это просто поршень в цилиндре с воздухом (см. рисунок \ref{pic:10.3}).
Поместим насос снаружи, не сжатый и холодный.
Затем надавим на поршень достаточно сильно, чтобы сжатый воздух нагрелся выше комнатной температуры (сжатие вызывает нагрев).
Далее занесём насос внутрь чтоб он отдал часть тепла в помещению.
Как только его температура сравнится с температурой комнаты, вытащим его наружу и расслабим поршень.
Расширяющийся воздух станет холоднее уличного воздуха, ведь он отдал часть своего тепла помещению.
Став холоднее, насос будет втягивать тепло из холодного зимнего воздуха!
Это тепло компенсирует то, которое мы передали воздуху внутри.
После того как насос снова достигнет наружной температуры, цикл завершён и его можно повторять пока не надоест.

Настоящие теплонасосы устроены хитрей, но принцип работы тот же.
Вместо воздуха в них используют хладагент%
\footnote{например один из фреонов \pr}%
; сжатие и расширение заменены конденсацией и испарением хладагента.
Хладагент перекачивается по трубам, и нет нужды бегать с ним туда-сюда.

\paragraph{Эффективность.}
Удивительно, но тепловой насос расходует меньше энергии, чем сжигание топлива для получения того же количества тепла.
Это происходит потому, что часть работы, затраченная на сжатие поршня снаружи, возвращается, когда я снова выношу насос наружу и отпускаю поршень; то есть, поршень отдаёт обратно часть энергии, которую я потратил на его сжатие.

\section{Две комнаты}

\paragraph{Вопрос.}
Одна из двух одинаковых комнат в доме теплее другой.
Верно ли что у молекул воздуха в тёплой комнате
суммарная кинетическая энергия больше, чем в холодной?

\paragraph{Ответ.}
Энергия одинакова!%
\footnote{Я предполагаю, что воздух является идеальным газом и не учитываю вторичные эффекты, такие как расширение стен при нагревании и невозможность поддерживать температуру совершенно одинаковой во всей комнате.}
У воздуха в тёплой комнате средняя энергия молекулы выше чем в холодной, однако в тёплой комнате меньше молекул, так как нагретый воздух расширяется, и какая-то его часть просачивается в щель под дверью.
Эти два противоположных эффекта (молекулы быстрее, но их меньше) уравнивают друг друга.
Действительно, число молекул в комнате обратно пропорционально температуре $T$ (считая от абсолютного нуля), тогда как кинетическая энергия каждой молекулы прямо пропорциональна $T$.
Сейчас я разберу всё это подробней.

\paragraph{Подробное объяснение.}
Согласно уравнению состояния
идеального газа (достаточно хорошее приближение при рассматриваемых температурах),
\begin{equation}
    pV = NkT,
    \label{eq:10.1}
\end{equation}
где $p$ --- давление в комнате, $V$ --- объём комнаты, $T$ --- абсолютная температура воздуха,
$N$ --- число молекул в комнате, а $k$ --- константа, не зависящая от перечисленных величин
(она называется постоянной Больцмана).

С другой стороны, известно, что средняя кинетическая энергия $E$ молекулы прямо пропорциональна температуре газа: $E = (3k/2)T$.
Поэтому полная кинетическая энергия всех $N$ молекул в комнате равна
\[E_{\text{полная}}
= N E
= N \frac{3k}{2}T
= \tfrac{3}{2}NkT
\stackrel{\text{\eqref{eq:10.1}}}{=} \tfrac{3}{2}pV.
\]
Поскольку при нагревании комнаты и давление $p$, и объём $V$ остаются постоянными,
постоянной остаётся и $E_{\text{полная}}$, что и требовалось.

\section{Как морозить велосипедной шиной}


\paragraph{Вопрос.}
Как можно создать температуру ниже нуля в жаркий летний день, используя велосипедную шину?

\paragraph{Ответ.}
Надо просто открыть ниппель в накачанной шине.
Допустим, что шина надута до 3 атмосфер.
Это означает, что давление в внутри шины на 3 атмосферы больше чем снаружи, то есть давление в шине 4 атмосферы.
При выходе из шины, воздух значительно расширяется, ведь давление падает с 4 до 1 атмосферы.
В свою очередь, расширение вызывает охлаждение.
При таком падении давления абсолютная температура воздуха уменьшается процентов на сорок.
Начав с температуры $27 \,\text{°C} \approx 300\, \text{°K}$, мы получим $171 \,\text{°K} \approx -102\, \text{°C}$!
Здесь не учитывается нагрев за счёт вязкости при прохождении через узкое отверстие, но с уверенностью можно сказать, что температура будет ниже точки замерзания.

\paragraph{Вычисления.}
Рассмотрим небольшую область воздуха, которая за очень короткое время перемещается из шины наружу.
Поскольку всё происходит быстро, можно пренебречь теплообменом между этой областью и окружающим воздухом.
Для такого расширения без теплообмена (называемого \emph{адиабатическим расширением}) температура воздуха пропорциональна давлению в степени $2/5$:
\[\frac{T_2}{T_1} = \left(\frac{p_2}{p_1}\right)^{\tfrac{2}{5}},\]
где индексы температуры и давления обозначают начальное и конечное состояния.
Мы знаем, что $p_2/p_1=1/4$.
Возведя это в степень $2/5$, получаем примерно $0{,}57$.
Начав с $T_1 = 300 \,\degree\text{K}$ получаем
\[T_2 = 300 \cdot 0{,}57 \approx 171 \,\text{°K},\]
что соответствует примерно $-102\, \text{°C}$!
Таким образом, открывая ниппель, мы получаем самое холодное место на Земле (не считая криогенных лабораторий).
Хоть оно и очень маленькое и существует очень короткое время, но всё равно этим можно гордиться.
