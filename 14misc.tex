\chapter{Ещё задачи}

\section{Книга вместо штопора}

Следующий метод работает проверил я сам (дв\'{и}жимый научным любопытством и отсутствием штопора, не  скажу в каком порядке).

\begin{figure}[h!]
\centering
\begin{lpic}[t(0mm),b(0mm),r(0mm),l(0mm)]{pics/14.1(.9)}
\lbl[tl]{55,85;{\begin{tikzpicture}
  \node[cm={1,-.9,0,1,(0,0)}] at (0,0) {стена};
\end{tikzpicture}}}
\end{lpic}
\caption{Пробка будет понемногу выходить с каждым ударом,
но почему?}
\label{pic:14.1}
\end{figure}

Прижав книгу к стене, ударяйте дном бутылки о книгу.
(Во избежании травм, заверните бутылку в полотенце и наденьте защитные очки.
У того кто рискнёт проделать это с бутылкой шампанского, есть шанс получить премию Дарвина%
%\footnote{Обычно присуждается посметрно. \pr}%
.)
При каждом ударе пробка будет чуть выходить наружу, пока наконец её не удастся вытянуть.

Я проделал это давным-давно, ещё будучи советским студентом, кощунственно использовав том из собраний сочинений Ленина.
Эта книга помогла мне справиться с поставленной задачей, хотя штопор сработал бы лучше.%
\footnote{Это был редкий случай восстановления исторической справедливости, ведь в результате экономической политики, пропагандируемой в сочинениях Ленина,
с прилавков пропали многие вещи, включая, возможно, и  штопоры, однако я сумел использовать вместо штопора эти самые сочинения.}

\paragraph{Вопрос.} Что же выталкивает пробку?

\paragraph{Ответ.}
Краткий ответ — винный удар.
Похожий на гидравлический удар (или гидроудар) в водопроводе, возникающий при резкой остановке потока воды в трубе, который вполне может привести к повреждениям.

\begin{figure}[ht!]
\centering
\begin{lpic}[t(2mm),b(2mm),r(0mm),l(0mm)]{pics/14.2}
\lbl[bl]{0,60;(a)}
\lbl[bl]{0,38;(b)}
\lbl[bl]{0,12;(c)}
\lbl{27,53;\color{white}{\footnotesize вино}}
\lbl[l]{41,31;\color{white}{\footnotesize вино}}
\lbl[l]{41,7.5;\color{white}{\footnotesize вино}}
\lbl[t]{70,31,90;{\footnotesize книга}}
\lbl[b]{40,62;{\footnotesize ускорение}}
\lbl{1,30;\parbox{25mm}{\footnotesize\centering резкая\\остановка\\книгой}}
\lbl[r]{62,42.5;{\footnotesize удар}}
\lbl{48.4,53,90;{\footnotesize воздух}}
\lbl[r]{63,20.5;{\footnotesize сжатый воздух}}
\lbl[r]{22,24;{\footnotesize вакуум}}
\lbl[r]{20,0;{\footnotesize гидроудар}}
\end{lpic}
\caption{Как пробка выталкивается из бутылки.}
\label{pic:14.2}
\end{figure}

\pagebreak

Рисунок \ref{pic:14.2} даёт представление об этом процессе.%
\footnote{Надо признаться, что это всего-лишь моё мнение --- оно не подтверждено никакими измерениями или прямыми наблюдениями, скажем, с помощью скоростной камеры, и вряд ли организации, поддерживающие науку, захотели бы рекламировать финансирование подобных исследований.

(Переводчики проверили, что метод работает.
Кстати удобнее и безопаснее держать бутылку дном вниз в мужском сапоге и бить каблуком по стене.
Однако метод перестаёт работать если бутылка наполнена полностью вплоть до пробки, а также если в пробке просверлить дырочку миллиметрового диаметра.
Эти два эксперимента отчасти подтверждают, что дело именно в гидравлическом ударе.\pr)
}
На рисунке \ref{pic:14.2}a бутылка двигается с ускорением к стене.
В результате воздух подходит ко дну, а вино собирается слева, у пробки (точно так же, как пассажиров быстро разгоняющегося автобуса отбрасывает назад).
При встрече бутылки с книгой, вино продолжает двигаться по инерции, сжимая воздух справа и образуя пузырь вакуума около пробки (рисунок \ref{pic:14.2}b).
Сжатый воздух, действуя как пружина, замедляет вино, а затем толкает его обратно в сторону пробки.
Пузырь вакуума схлопывается — ведь вакуум, в отличие от воздуха, не смягчает удар!
В момент схлопывания вино ударяет в пробку, как молот по наковальне, без смягчающей подушки.
В разультате пробка немного выдвигается.
По сути, мы стучим вином по пробке изнутри!
После нескольких ударов пробка выходит достаточно, чтобы её можно было вытащить пальцами.

\paragraph{Кавитация.}
Схлопывающиеся пузыри вакуума создают огромные силы в менее желательных ситуациях, например возле гребных винтов лодок.
Пузырь вакуума образуется если гребной винт вращается слишком быстро.
Когда вакуум схлопывается, возникающий при этом удар может вывести винт из строя.

Во всех этих явлениях (кавитации, открывании бутылки и гидроударе) проявляется второй закон Ньютона $F\z=ma$, где огромное ускорения $a$ сопровождается огромной силой $F$.
Кстати, электрические токи также обладают инерцией%
\footnote{Она называется электромагнитной индукцией.},
подобный эффект возникает и в электрических цепях.
Благодаря этому эффекту можно получить сильный электрический удар от маленькой батарейки (см., например, Levi, The
Mathematical Mechanic, pages 178--79).

\section{«Оно живое!»}

\paragraph{Задача.}
Груз подвешен к потолку на системе из шнурков и пружин.
Если одну из пружин перерезать, опустится ли груз ниже?
В частности, что произойдёт с грузом на рисунке \ref{pic:14.3}, если перерезать среднюю пружину?%
\footnote{Этот парадокс был описан Дитрихом Брессом в контексте транспортных сетей.
Бресс обнаружил, что добавление лишней дороги может увеличить время в пути для всех (см. «Ueber ein Paradoxon der Verkehrsplannung», Unternehmensforschung 12 (1968), p. 258–268).
Механический эксперимент с пружинами, аналогичный изображённому на рисунке \ref{pic:14.3} аналогичен задаче о транспортных сетях, он обсуждается в статье C. M. Penchina и L. J. Penchina, «The Braess paradox in mechanical, traffic, and other networks», American Journal of Physics, May 2003, pp. 479---482.
Я признателен Полу Нахину за указание этих источников.}

\begin{figure}[ht!]
\centering
\begin{lpic}[t(2mm),b(2mm),r(0mm),l(0mm)]{pics/14.3}
\lbl[tr]{5,43;{\footnotesize шнурок № 1}}
\lbl[tl]{20,17;{\footnotesize шнурок № 2}}
\lbl[tl]{20,27;{\footnotesize эту пружину разрезаем}}
\lbl[tl]{20,45;{\footnotesize пружина}}
\lbl[br]{8,25;$A$}
\lbl[tl]{16,36;$B$}
\lbl[r]{8,15;$S_1$}
\lbl[l]{17,48;$S_2$}
\lbl[b]{11,35;$S$}
\end{lpic}
\caption{Опустится ли груз, если перерезать среднюю пружину?}
\label{pic:14.3}
\end{figure}

\paragraph{Решение.}
Груз поднимется.
Чтобы понять почему, давайте сначала зафиксируем груз и затем перережем пружину $S$.
В результате этого натяжения шнурков увеличатся, а натяжения пружин останутся теми же.
То есть, на груз действует б\'{о}льшая поднимающая сила, и если теперь отпустить груз, то он поднимется, как и утверждалось.

Иными словами, пружина $S$ тянет груз вниз, притягивая вниз точку $B$.
Поэтому, перерезав эту пружину, мы поднимаем груз вверх.

\section{Падение быстрее $g$: почему пол засасывает цепь?}

\paragraph{Удивительное всасывание.}
Следующее наблюдение наблюдение принадлежит Энди Руине;
видеоролик и статья про него доступны по ссылке
\href{http://ruina.tam.cornell.edu/research/topics/fallingchains/}{ruina.tam.cornell.edu/research/topics/fallingchains/}.%
\footnote{Для тех, кто поленится идти по ссылке, в видео ролике показано, что если отпустить лисенку как на рисунке \ref{pic:14.4} на стол с высоты 75 см, и одновременно дать идентичной ей лисенке свободно падать, то первая лесенка обгонит вторую примерно на 6---7 см.\pr}

Если держать верёвочную лестницу, схематично показанную на рисунке \ref{pic:14.4}, за верхний конец, и отпустить, то произойдёт удивительное.
Как только она коснётся пола, оставшаяся часть начнёт двигаться быстрее, свободно падающего тела.
Будет казаться, что пол всасывает падающую часть цепи.
Почему же это происходит?

\paragraph{Объяснение.}
Давайте разберёмся, что происходит, когда одно звено ударяется об пол.
Этот напоминает падающий карандаш.
Когда конец карандаша ударяется об пол, другой его конец резко дёргается вниз (если в момент удара  карандаш не был направлен почти вертикально).
Звенья цепи могут испытывать подобный резкий рывок вниз, если ударяются о поверхность под подходящим углом, и тогда они начинают тянуть за собой остальную часть цепи.

\begin{figure}[ht!]
\centering
\begin{lpic}[t(2mm),b(2mm),r(0mm),l(0mm)]{pics/14.4}
\lbl[r]{12,36;\parbox{25mm}{\footnotesize\centering ускоряется\\быстрей чем\\при свободном падении}}
\lbl[b]{33,37;{\footnotesize шнур}}
\lbl[t]{34,49;{\footnotesize стержень}}
\end{lpic}
\caption{Цепь втягиватся за счёт ударов в пол.}
\label{pic:14.4}
\end{figure}

\section{Ходьба по лодке с сопротивлением}

Давайте вспомним стандартную задачу о ходьбе вдоль лодки.
Человек, стоит на корме неподвижной лодки, и переходит к её носу.
На какое расстояние лодка сдвинется относительно берега?
Известны длина лодки и отношение масс человека и лодки.
Сопротивлением воды пренебрегаем.%
\footnote{Кратко напомню решение.
Пусть $\Delta p$ и $\Delta b$ --- сдвиги человека и лодки относительно земли;
нас интересует $\Delta b$.
Поскольку центр масс остаётся неподвижным, $m\Delta p = M\Delta b$.
Кроме того,
$\Delta p+\Delta b=L$.
Отсюда получаем $\Delta b=\tfrac{m L}{m+M}$.
Похоже на правду, ведь чем тяжелее лодка, тем меньше будет её смещение;
с другой стороны, очень тяжёлый человек заставил бы лодку сдвинуться почти на всю её длину.}
А вот интересная вариация этой задачи, предложенная Димой Бураго.

\paragraph{Задача.}
Предположим, что вода действует на лодку силой сопротивления $F$,
пропорциональной скорости лодки: $F = k v$, где $k$ — некоторая ненулевая константа.
На каком расстоянии от исходного положения окажется лодка после того,
как человек пройдёт от одного конца лодки к другому?
Изначально всё находится в покое.
Массы человека и лодки $m$, $M$, а также длина лодки $L$ известны.

\paragraph{Решение.}
В итоге лодка окажется там же, где была в начале!
Масса человека не имеет значения, так же как длина и масса лодки,
и даже величина коэффициента сопротивления $k$.
То есть все иходные данные были лишними!

\begin{figure}[ht!]
\centering
\begin{lpic}[t(2mm),b(2mm),r(0mm),l(0mm)]{pics/14.5}
\lbl[r]{25,25;{\footnotesize начало:}}
\lbl[t]{19.5,10;$b$}
\lbl[t]{38,9;$p$}
\lbl[r]{25,1;{\footnotesize конец ($t\approx\infty$):}}
\end{lpic}
\caption{Лодка в конце концов подойдёт к начальному положению.}
\label{pic:14.5}
\end{figure}

\paragraph{Описание движения.}
(Точное решение будет дано чуть ниже.)
Когда человек начинает идти вправо, лодка начинает двигаться влево (рисунок \ref{pic:14.5}).
Следовательно, сила сопротивления направлена вправо, и по второму закону Ньютона
центр масс системы человек+лодка ускоряется вправо.%
\footnote{Напоминает бегущую на месте мультяшную собаку: её лапы скользят по земле (как лодка скользит по воде), в то время как центр масс собаки ускоряется вперёд.
Однако мультяшные персонажи регулярно нарушают законы Ньютона.}
Значит центр масс этой системы двигается вправо и он продолжает своё движение по инерции
даже после того, как человек остановился.
Лодка постепенно замедляется из-за сопротивления воды.

Итак, лодка начала двигаться влево, что вызвало появление силы сопротивления, заставившей центр масс всей системы двигаться вправо — движение, которое сохранилось после того, как человек остановился.
Удивительным образом, лодка со временем приблизится к своему исходному положению.
Пока что наше рассуждение не объясняет почему возникает такое странное совпадение, но скоро всё станет ясно.

\paragraph{Обоснование} этого совпадения довольно простое, но требует матанализа.
Пусть $b = b(t)$ обозначает положение
центра масс лодки в момент времени $t$ (всё мерится относительно берега).
Аналогично, пусть $p = p(t)$ — положение человека
(рассматриваемого как материальная точка).
Центр масс системы лодка+человек вычисляется как взвешенное среднее двух положений:
\[C = C(t) = \frac{m p + M b}{m + M}.\]
Применим второй закон Ньютона (сформулированный на странице~???) в направлении движения лодки:
\[(m+M)\,\ddot{C} = -k \dot{b},\]
здесь каждая точка означает производную по времени.
Подставив выражение для $C$, получаем
\begin{equation}
m \ddot{p} + M \ddot{b} = -k \dot{b}.
\label{eq:14.1}
\end{equation}
Давайте проинтегрируем это равенство от $t=0$ до $t=\infty$.
По основной теореме анализа,
$\int_0^\infty \ddot{p}\,dt = \dot{p}(\infty) - \dot{p}(0)$.
Так как $\dot{p}(0)=0$ (вначале всё покоилось) и
$\dot{p}(\infty)=\lim_{t\to\infty}\dot{p}(t)=0$ (всё в конце всё остановилось),
заключаем, что
$\int_0^\infty \ddot{p}\,dt=0$.
Точно также получится, что
$\int_0^\infty \ddot{b}\,dt = 0$.
Проинтегрировав \eqref{eq:14.1}, получим
\[0= k(b(\infty) - b(0)).\]
В частности окончательное смещение лодки $b(\infty) - b(0) = 0$,
при $k > 0$.
Таким образом,
\[
b(\infty) = b(0),
\]
то есть лодка возвращается к своему исходному положению при $t \to \infty$.

То, что коэффициент $k$ оказался ненужным довольно удивительно.
Заметим однако, что значение $k$ влияет на скорость
приближения к исходному положению.
Чем меньше $k$, тем больше придётся ждать, а при $k = 0$, лодка вовсе не станет возвращаться.

\section{Корабль призрак без следов и усилий}

Начнём с вопроса для разминки.

\paragraph{Вопрос.}
Испытывала бы лодка сопротивление от абсолютно невязкой воды?

\paragraph{Ответ.}
Б\'{о}льшую часть энергии, вырабатываемой двигателем, расходуется на волны, вязкость не столь существенна.
Чем меньше волн оставляет лодка, тем лучше.
Корпус, форма которого почти не создаёт волн, экономил бы много энергии.

\paragraph{Вопрос.}
Можно ли, по крайней мере в принципе, спроектировать лодку, которая почти не будет оставлять следа на воде?
Пренебрегите вязкостью, пусть скорость будет постоянной и предположите, что изначально не было волн.

\paragraph{Решение 1.} (предложено Энди Руиной).
Можно сделать корпус из сот, образованных трубками;
вход и выход каждой трубки расположены на одной линии, как показано на рисунке \ref{pic:14.6}.
Вода входит в такую трубку в точке $A$ и выходит в точке $B$.
При отсутствии вязкости такой корпус «невидим» для воды: двигаясь равномерно, он не оставлял бы за собой никаких возмущений.

\begin{figure}[ht!]
\centering
\begin{lpic}[t(7mm),b(2mm),r(0mm),l(0mm)]{pics/14.6}
\lbl[b]{5,47;\parbox{25mm}{\footnotesize\centering однородный входящий поток}}
\lbl[t]{31,32;\parbox{35mm}{\footnotesize\centering такая трубка не влияет на внешний поток}}
\lbl[b]{3,8;\parbox{25mm}{\footnotesize\centering вход $A$}}
\lbl[b]{62,8;\parbox{25mm}{\footnotesize\centering $B$ выход}}
\end{lpic}
\caption{(a) Поток не замечает трубы.
(b) Корпус, сделанный из множества таких труб, не будет оставлять за собой возмущений (в идеальном мире).}
\label{pic:14.6}
\end{figure}

\paragraph{Решение 2.}
Чтобы предотвратить образование волн, можно одеть на лодку юбку; то есть диск, расположенный вровень с поверхностью воды, как на рисунке \ref{pic:14.7}.
Если юбка достаточно широкая, то волн почти не будет.
(Хотя в теории это может сработать, на практике такое решение вряд ли удобно по целому ряду причин, не говоря уже об эстетическом аспекте лодки в балетной пачке.)

\begin{figure}[ht!]
\centering
\begin{lpic}[t(5mm),b(2mm),r(0mm),l(0mm)]{pics/14.7}
\lbl[b]{25,17;\parbox{30mm}{\footnotesize\centering юбка находится на уровне воды}}
\lbl[b]{59,15;{\footnotesize зеркальная копия}}
\end{lpic}
\caption{Юбка вокруг лодки не позволяет волнам образовываться, тем самым уменьшая сопротивление.}
\label{pic:14.7}
\end{figure}

\paragraph{Парадокс Д’Аламбера и лодка в юбке.}
Приблизительно в 1752 году Д’Аламбер понял,
что сопротивление равно нулю при движении тела в идеальной жидкости\footnote{То есть несжимаемая, невязкая, безвихревая.
Кратко объясним эти термины.
Невязкая означает, что вязкость нулевая;
безвихревая --- нулевая завихренность (см. страницу ???).
Подробности можно найти в любом учебнике по гидродинамике, например, у Батчелора.}%
с постоянной скоростью (относительно жидкости).
При этом предполагается, что жидкость заполняет всё пространство.
Но давайте заменим мир над поверхностью на рисунке \ref{pic:14.7} зеркальным отражением мира под поверхностью; тогда вода заполнит всё пространство, а симметричная «лодка» превратится в подводную лодку.%
\footnote{Марк, стоит ли сюда добавить следующую сноску: «Гравитационное поле будет выгладеть экстравагантно, но это ничему не мешает.»}
Согласно парадоксу Д’Аламбера, лодка не будет испытывать сопротивления (при идеализированных условиях, указанных в сноске).
Можно думать, что юбка нужна лодке как раз, чтобы сделать парадокс Д’Аламбера (почти) применимым.

\section{Американские горки с постоянной перегузкой}

Можно ли придать трассе американских горок такую форму, что пассажир всё время испытывал бы постоянную перегрузку, скажем, $G=2g$, где $g$ — ускорение свободного падения?

\paragraph{Задача.}
Будем считать, что рельс идёт по кривой в вертикальной плоскости, а вагончик — это бусинка, скользящая по этой кривой без трения, под действием силы тяжести.

\paragraph{Решение.}
Ответ «да» ---
для любой перегрузки $G>g$ существует такая трасса; пример показан на рисунке \ref{pic:14.8}.
При подходящей начальной скорости пассажир будет ощущать свой вес равным $mG$, если на земле его вес равен $mg$.
Обратите внимание на б\'{о}льшую кривизну ближе к верхней части — она необходима для увеличения центробежной силы, чтобы скомпенсировать два эффекта: (1) уменьшение центробежной силы из-за меньшей скорости на высоте и (2) действие силы тяжести, стремящейся вырвать пассажира из сиденья.

\begin{figure}[ht!]
\centering
\begin{lpic}[t(2mm),b(2mm),r(0mm),l(0mm)]{pics/14.8}
\lbl[bl]{37,8;$\theta$}
\lbl[tl]{27,2;$s$}
\lbl[l]{15,13;$g$}
\end{lpic}
\caption{Eсли скользить по траектории, описанной уравнением Кеплера, то, стартовав с подходящей скоростью, мы бы испытывали бы постоянную перегрузку.}
\label{pic:14.8}
\end{figure}

\paragraph{Уравнение Кеплера.}
Интересно, что угол~$\theta$ с горизонталью
для такой трассы с постоянной перегрузкой удовлетворяет уравнению Кеплера:
\[
\theta - e \sin \theta = c s,
\qquad
e = \tfrac{g}{G} < 1,
\]
где $c$ — константа, связанная со скоростью в нижней точке,
а $s$ — длина, отсчитываемая вдоль трассы.
Выбирая разные значения константы $c$, мы будем получать американские горки разных размеров, но с одной и той же перегрузкой~$G$.
Объяснение того, как уравнение Кеплера появляется в задаче об американских горках, требует матанализа и мы его пропускаем.

Любопытно, что уравнение Кеплера возникло изначально в астрономии, при решении совсем другой задачи.

\section{Выстрел в тележку}

\paragraph{Задача.}
На тележке укреплён барабан как на рисунке \ref{pic:14.9};
барабан может вращаться, а тележка катиться без всякого трения.
Проведём два опыта.
В первом выстреливаем пулю так, чтоб она попадала в барабан в точке $A$,
заставив его вращаться, и после этого упала на платформу тележки.
Вся система — тележка, барабан и пуля, лежащая на платфрме, — начинает катиться.
Во втором опыте всё то же самое, только пуля попадает в точку $B$, и поэтому барабан не раскручивается.
Поскольку во втором случае энергия не тратится на вращение барабана, больше энергии остаётся на поступательное движение тележки.
Насколько быстрее будет двигаться тележка во втором случае?
Будем считать, что масса пули равна массе барабана, а массой тележки можно пренебречь.
Можно также предположить, что вся масса барабана сосредоточена на его ободе.

\begin{figure}[ht!]
\centering
\begin{lpic}[t(7mm),b(2mm),r(0mm),l(0mm)]{pics/14.9}
\lbl[b]{66.5,21;{\footnotesize барабан}}
\lbl[b]{27,29;\parbox{25mm}{\footnotesize\centering вращается\\ после удара}}
\lbl[b]{66.5,29;\parbox{25mm}{\footnotesize\centering не вращается после удара}}
\lbl[tl]{21,25.5;$A$}
\lbl[l]{58,18.5;$B$}
\end{lpic}
\caption{Насколько быстрее покатится вторая тележка?}
\label{pic:14.9}
\end{figure}


\paragraph{Ответ.}
Обе тележки будут двигаться с одинаковой скоростью!
Я спрятал ошибку, сказав (верно), что во втором случае остаётся больше энергии.
Однако я не уточнил, на что пойдёт эта лишняя энергия ,
а пойдёт она не на катании, а на тепло, возникающем при ударе пули о барабан.
Короче говоря, вращение в первом случае получит ту же энергию, что и лишнее тепло, выделяющееся во втором.

\paragraph{Пояснение.}
По закону сохранения импульса, скорости катания в обоих случаях одинаковы.
Главное, что вращение не меняет импульс барабана, так как импульс каждой частицы компенсируется равным и противоположным импульсом её антипода.
Поэтому независимо от того, вращается барабан после удара или нет, импульс всей системы (пуля + тележка + барабан) будет равен $Mv$,
где $M$ — полная масса, а $v$ — скорость.
Но весь этот импульс пришёл от импульса пули:
\[Mv = mV,\]
где $m$ — масса пули, а $V$ — её скорость до удара.%
\footnote{Я использую прописные буквы ($M, V$) для больших величин и строчные ($m, v$) для малых.}
Это и доказывает, что $v$, скорость тележки, не зависит от вращения.

\section{Как найти $\sqrt{2}$ используя кроссовку}

\paragraph{Задача.}
У вас в распоряжении секундомер и кроссовка, как найти приближённое значение $\sqrt{2}$?

\paragraph{Решение}

\begin{enumerate}
\item Подвесив кроссовку за её шнурок, мы получим маятник.
С помощью секундомера измерим число его колебаний в минуту;
обозначим результат через $n_1$.

\item Теперь сложим шнурок по палам и измерим новое число колебаний в минуту; обозначим результат через $n_{2}$.

\item Ответ: $\sqrt{2} \approx \frac{n_{2}}{n_{1}}$.
\end{enumerate}
Для большей точности берите интервал времени побольше
(и лучше использовать кроссовку поменьше).

\paragraph{Объяснение.}
Время $T$ одного полного колебания маятника задаётся формулой
\[
T = 2\pi \sqrt{\frac{L}{g}},
\]
где $L$ — длина нити, а $g$ — ускорение свободного падения.%
\footnote{Строго говоря, эта формула приблизителная, но она даёт хорошее приближение для колебаний малой амплитуды.}
Для периодов двух маятников длиной $L_{1}$ и $L_{2}$ получаем
\[
\frac{T_{1}}{T_{2}} = \sqrt{\frac{L_{1}}{L_{2}}}.
\]
Мы взяли $L_{1} = 2L_{2}$ и подсчитали число колебаний в минуту для каждой длины. Имеем
\[
T_{1} \approx \frac{1}{n_{1}}  \text{минут},
\]
так как за одну минуту совершается $n_{1}$ полных колебаний. Аналогично,
\[
T_{2} \approx \frac{1}{n_{2}}.
\]
Следовательно,
\[
\frac{n_{2}}{n_{1}}
\approx
\frac{T_{1}}{T_{2}}
=
\sqrt{\frac{L_{1}}{L_{2}}}
=\sqrt{2}.
\]
Теперь должно быть ясно как получить другие квадратные корни.
Например, чтобы вычислить $\sqrt{3}$, нужно сделать шнурок в 3 раза короче, добившись того, что $L_{1}/L_{2} = 3$.
