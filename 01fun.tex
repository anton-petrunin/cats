\chapter[Парадоксы, головоломки, задачи]{Физические парадоксы, головоломки и задачи}

\section{Введение}

Хороший физический парадокс --- это (1) неожиданность, (2) головоломка и (3) урок, и всё это в прикольной обёртке.
Парадокс часто основан на убедительном рассуждении, которое приводит либо к ошибочному, но правдоподобному, выводу, либо к верному, но неожиданному, выводу, который кажется ошибочным;
так, что трудно не поддаться искушению разобраться, что к чему.
В годы холодной войны ходила байка, что ЦРУ мешало работе советских военных НИИ, подбрасывая листовки с головоломками и логическими задачами.
Времена изменились, теперь те же задачи используются при приёме на работу.
Советский пропагандист сказал бы, что они остались орудиями капитализма.

Парадоксы не только увлекательны, но и полезны.
Они развивают интуицию, логическое мышление и критический подход, так, что
у человека развивается внутренний детектор лжи.
Хороший парадокс также учит скромности и осторожности, показывая, как легко ошибиться даже в элементарных вопросах физики.
Успокаивает то, что даже очень умные люди допускают ошибки в, казалось бы, очевидных вопросах.
А ведь объекты, с которыми имеют дело другие области --- астрономия, биология, медицина, экономика, климатология, политика и СМИ посложней%
\footnote{Я не пытаюсь сравнивать сложности наук, просто хочу сказать, что типичный объект в физике (например, кристалл) проще типичного объекта в других областях (например, клетки в биологи).}%
, чем в физике, а значит, там ещё легче ошибиться.
Кроме того, некоторые «ошибки» могут приносить пользу, по крайней мере, некоторое время.

В этой книге я хотел поделиться прикольными размышлениями о том, как устроен мир.
Надеюсь, они помогут вам понять суть некоторых физических явлений, и при этом не измучают%
\footnote{Я говорю о «мучениях» с иронией --- математика, разумеется, незаменима и прекрасна, по крайней мере для меня, уже потому, что это моя профессия.}
математикой.

Эта книга по физике — науке, которая ходит на двух ногах, одна нога это математика, а другая --- физическая интуиция.
К сожалению, школьная физика часто выходит хромоногой.

\paragraph{Сравнение с музыкой.}
Если бы музыке учили так же, как зачастую учат физике, то мы знали бы отдельные ноты, но не мелодии, которые из них получаются.
Увы, слишком много учеников видят в физике набор формул, которые надо лишь применить в подходящем случае, и,
как следствие, много способных учеников теряют к физике интерес.

\paragraph{Сначала интуиция.}
Эта книга призвана натренировать вашу физическую интуицию.
Слишком часто курсы физики пренебрегают интуицией, делая упор на подбор формулы, подходящей в конкретном случае.
В этой книге всё наоборот:
я хотел добиться максимума интуитивного понимания при минимуме формул.
Обсуждение волчка --- хороший тому пример;
без каких-либо формул я объясню, почему вращающийся волчок остаётся в вертикальном положении.
Нужны годы освоения математики и физики в достаточном объёме, чтобы научиться записывать дифференциальные уравнения движения волчка и понять, как из этих уравнений следует его устойчивость.
Пройдя весь этот путь, лишь немногие придут к интуитивному пониманию, почему волчок не падает.
Обидно, когда всё это время мощнейший инструмент --- физическая интуиция --- оказывается не у дел.

\section{Предварительные сведения}

Б\'{о}льшая часть книги (хоть и не вся) должна быть доступна читателям без специальной подготовки в физике;
все необходимые понятия объяснены в приложении.
Обычно математика остаётся в рамках алгебры, но изредка используется математический анализ.
Но даже в этих местах читатель, готовый принять на веру кое-какие математические выкладки, должен сносно себя чувствовать.%
\footnote{Речь идёт, например, о задаче с пращой на странице \pageref{Задача Давида и Голиафа}, где камень достигает бесконечной скорости за секунду.}

Тяга к новому — основной инстинкт большинства живых существ или, по крайней мере, млекопитающих.
Побуждая нас к исследованию, этот инстинкт помогает выживать — за исключением некоторых случаев, вроде лауреатов «Премии Дарвина» или героев шоу «Чудаки» (также известного как «Придурки», англ. Jackass).
Тот же самый инстинкт, который привёл Эйнштейна к его великим открытиям, толкает любопытного ребёнка разобрать механические часы и заглянуть внутрь.
Он же побуждает щенков и котят исследовать окружающий мир,
а у некоторых людей этот инстинкт настолько силён, что способен противостоять системе школьного образования.

\section{Источники}

Собирать подобные задачи посоветовал мне отец после того, как увидел одну мою головоломку;
она пришла мне в голову после школьного урока о капиллярном эффекте (см. страницу \pageref{Вечный двигатель на капелярной тяге}).
Из этой коллекции и выросла данная книга.
Mногие задачи книги моего собственного сочинения%
\footnote{Например,
\ref{Гелиевый шар}, \ref{Парадокс с кометой}, \ref{Хочешь медленнее},
\ref{sec:cork}, \ref{Пара рецептов}, \ref{sec:sails}, \ref{sec:iceberg},
\ref{Ванна на колесиках}, \ref{Углублённая задача}, \ref{Шарик под водой}---\ref{Проблема с весом},
\ref{Как двигаться в космическом корабле}---\ref{Вопрос о струйном принтере},
\ref{Разгон одним наклоном}, \ref{Как разогнаться на велосипеде, двигая только тело?}, \ref{Парадокс с ракетами}---\ref{Мячик из машины},
\ref{Парадокс с Кориолисом}, \ref{Почему праща не может работать}, \ref{Задача Давида и Голиафа},
\ref{Как удержаться на скользком куполе?},
\ref{Вечный двигатель на капелярной тяге},
\ref{Против ветра на велосипеде},
\ref{Могут ли пассаты замедлить вращение Земли?},
\ref{Американские горки с постоянной перегрузкой}, \ref{кроссовка}.}%
, но \emph{я уверен, что кто-то уже задумывался над ними или чем-то похожем задолго до моего рождения.}
Если мне известен автор или источник, то я его указываю.

\paragraph{Книжная полка.}
К счастью, %???
многое из основ физики можно понять, получив от этого удовольствие и (почти) без формул ---
несколько замечательных научно-популярных книг тому подтверждение;
среди них — \emph{The Flying Circus of Physics} Уолкера,
\emph{Thinking Physics} Эпштейна,
\emph{Mad about Physics} Яргодзки и Поттера,
а также классическая «Занимательная физика» Перельмана.
К сожалению, шикарная книга Маковецкого \emph{Смотри в корень}, которая разошлась тиражом более миллиона экземпляров в Советском Союзе, похоже, так и не была переведена на английский.
Книга Миннарта \emph{The Nature of Light and Color in the Open Air} (посвящёна оптическим явлениям в природе) никогда не устареет и доставит радость каждому любознательному человеку, которому посчастливится её открыть.
