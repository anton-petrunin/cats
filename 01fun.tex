\chapter[Парадоксы, головоломки, задачи]{Физические парадоксы, головоломки, задачи и приколы}

\section{Введение}

Хороший физический парадокс --- это (1) неожиданность, (2) головоломка и (3) урок и всё это должно быть подано в одном прикольном кульке.
Парадокс часто строится из убедительного рассуждения, которое приводит к неправильному выводу, кажущемуся верным, либо же к правильному выводу, кажущемуся неправильным или неожиданным,
так, что трудно не поддаться искушению разобраться что к чему.
В годы холодной войны ходила шутка, что Запад смог бы замедлить работу советских военных НИИ, разбрасывая над ними листовки с головоломками и логическими задачами.
Времена изменились, теперь те же задачи используются при приёме на работу, но в любом случае, всё это орудия капиталистов.

Парадоксы не только увлекательны, но и полезны.
Они развивают интуицию, логическое мышление и критический подход, так, что
человека становится слонее обмануть.
Хороший парадокс также учит осторожности и скромности, показывая, насколько легко ошибиться даже в элементарных вопросах физики.
Успокаивает то, что даже очень умные люди допускают ошибки в, казалось бы, очевидных вопросах.
А ведь объекты физики проще объектов других областей, например астрономии, биологии, медицины, экономики, климатологии, политики или СМИ,%
\footnote{Здесь я не сравниваю сложности разных наук.
Я лишь хочу сказать, что физику приходится иметь дело с более простыми объектами (например, кристаллами), чем биологу (например, с клеткой).} — а значит, в этих областях ещё легче ошибиться.
Кроме того, некоторые «ошибки» помогают, по крайней мере, некоторое время.

В этой книге я хотел поделиться прикольными размышлениями о том, как устроены вещи.
Надеюсь они помогут понять суть некоторых физических явлений без математических мучений.%
\footnote{Я говорю о «мучениях» с иронией — математика, разумеется, незаменима и, по крайней мере для меня, прекрасна, уже потому, что это моя профессия.}

Эта книга по физике — науке, стоящей на двух ногах, одна нога --- математика, а другая --- физическая интуиция.
К сожалению, школьная физика часто хромает.

\paragraph*{Сравнение с мусыкой.}
Если бы музыке учили так же, как зачастую учат физике, то мы бы знали об отдельных нотах, но ничего бы не знали о мелодиях, которые из них складываются.
Пугающе много учеников видят в физике набор формул, которые надо применять в подходящих задачах.
Неудивительно, что это часто отталкивает от физики умных учеников.

\paragraph*{Сначала интуиция.}
Самое \emph{полезное} из того, что даёт эта книга --- это тренировка физической интуиции.
Слишком часто курсы физики пренебрегают интуицией, делая упор на подбор формулы, подходящей к конкретной ситуации.
В задачах этой книги всё наоборот:
я хотел добиться максимума интуитивного понимания при минимумоме формул.
Задача о волчке --- хороший тому пример;
без каких-либо формул я объясню, почему вращающийся волчок остаётся в вертикальном положении.
Понадобятся годы, чтобы освоить математику и физику в достаточном объёме, научиться записывать дифференциальные уравнения движения волчка и понять, как из этих уравнений следует его устойчивость.
Пройдя весь этот путь, лишь немногие прийдут к интуитивному пониманию, почему волчок не падает.
И всё это время мощнейший инструмент --- физическая интуиция --- может остаться не у дел.

\section{Предварительные сведения}

Б\'{о}льшая часть книги (хоть и не вся) должна быть доступна читателям без специальной подготовки в физике.
Все используемые физические понятия объяснены в приложении.
Обычно математика остаётся в рамках алгебры, но изредка используется анализ.
Но даже в этих местах читатель, готовый поверить в какие-то математические факты, должен сносно себя чуствовать.%
\footnote{Речь идёт о задаче с пращой на странице \pageref{???}, где камень достигает бесконечной скорости за секунду.}

Тяга к новому — основной инстинкт большинства живых существ, или, по крайней мере, млекопитающих.
Побуждая нас к исследованию, этот инстинкт помогает выживать — за исключением некоторых случаев, вроде лауреатов Премии Дарвина или героев шоу Придурки (Jackass).
Тот же самый инстинкт, который привёл Эйнштейна к его великим открытиям, толкает любопытного ребёнка разобрать механические часы и заглянуть внутрь.
Он же побуждает щенков и котят исследовать окружающий мир,
а у некоторых людей этот инстинкт настолько силён, что остаётся с ними после окончания школы.

\section{Источники}

Эта книга выросла из моей коллекции задач.
Я составлял её по совету отца, после того как показал ему одну задачу, которая пришла мне в голову после школьного урока о капиллярном эффекте (см. страницу \pageref{???}).
Хотя некоторые задачи в этой книге придумал я сам,%
\footnote{Например, 2.1, 2.3, 2.4, 3.1, 3.2, 3.5, 3.6, 4.1, 4.2, 4.4---4.6, 5.3---5.8, 6.6, 6.7,
6.10---6.12, 8.2, 8.5, 8.6, 9.4, 11.1, 12.3, 13.2, 14.6, 14.8.???}
скорее всего, другие уже задумывались над ними или чем-то похожем задолго до моего рождения.
Если мне известен автор или источник, то я его указываю.

\paragraph{Хорошие книги.}
К счастью, многое из основ физики можно понять, получив от этого удовольствие и (почти) без формул ---
несколько замечательных научно-популярных книг тому поддвердение;
среди них — «The Flying Circus of Physics» Уолкера,
«Thinking Physics» Эпштейна,
«Mad about Physics» Яргодзки и Поттера,
а также классическая «Занимательная физика» Перельмана.
К сожалению, шикарная книга Маковецкого «Смотри в корень», которая разошлась тиражом более миллиона экземпляров в бывшем Советском Союзе, похоже, так и не была переведена на английский.
Книга Миннарта «The Nature of Light and Color in the Open Air», посвящённая оптическим явлениям, никогда не устареет и доставит радость любому любознательному человеку, которому посчастливится её открыть.
