\chapter{Парадоксы гироскопа}

\section{Волчок и земное притяжение}

Что удерживает вращающийся волчок от падения при том, что нет сил, противодействующих силе тяжести?
Дело обстоит довольно хитро, есть странная \emph{уклоняющая} сила --- сила, которая всегда направлена \emph{перпендикулярно} к движению оси волчка.
Эта уклоняющая сила противостоит неустойчивости: волчок начинает падать, но затем отклоняется в сторону, и в результате движется так, как показано на рисунке \ref{pic:9.3}.
В следующем абзаце я постараюсь показать, как эту странную «гироскопическую силу» можно вывести из второго закона Ньютона.

\paragraph{Антигравитационное колесо.}\label{Антигравитационное велоколесо}\rindex{велосипед}
\rindex{колесо}
Будем использовать велосипедное колесо как волчок.
Подвесим его на двух верёвках, как показано на рисунке \ref{pic:9.1}, и хорошо раскрутим.
Теперь перережем одну из верёвок.
Удивительно, но не поддерживаемый конец оси не упадёт вниз.%
\footnote{Предполагаем, что колесо хорошо раскручено.}
Вместо этого он начнёт медленно поворачивать (прецессировать).
Причём чем быстрее вращение колеса, тем медленнее прецессия.

\begin{figure}[ht!]
\centering
\begin{lpic}[t(2mm),b(2mm),r(0mm),l(0mm)]{pics/9.1}
\lbl[b]{15,31;$F$}
\lbl[b]{57,20;$F$}
\lbl[lb]{71,8;$1$}
\lbl[l]{67,20;$2$}
\lbl[lt]{71,30;$3$}
\lbl[b]{60,35;$1$}
\lbl[b]{65.3,36;$2$}
\lbl[b]{71,35;$3$}
\lbl[b]{10,8,95;{\footnotesize вращение}}
\lbl[r]{3,33;\parbox{22mm}{\footnotesize\raggedleft  ось\\ прицессии}}
\lbl[l]{38,21;{\footnotesize разрыв}}
\lbl[t]{66,3;{\footnotesize вид сверху}}
\end{lpic}
\caption{Инерция частиц гироскопа держивит ось горизонтально после разрыва верёвки.}
\label{pic:9.1}
\end{figure}

\paragraph{Вопрос.}
Что же мешает колесу упасть?
Как ему удаётся противостоять силе тжести?


\paragraph{Ответ.}
\label{Антигравитационное велоколесо:Ответ}
\rindex{колесо}
Можно сказать, что за это отвечает \emph{центробежная сила}, но не та, что первой приходит на ум, а другая --- перпендикулярная ей!

\begin{figure}[ht!]
\centering
\begin{lpic}[t(2mm),b(2mm),r(0mm),l(0mm)]{pics/9.2}
\lbl[t]{3,3;\parbox{22mm}{\footnotesize\centering  момент\\ силы}}
\lbl[]{17,38.6,-20;\parbox{22mm}{\footnotesize\centering  центробежная\\ сила}}
\lbl[b]{67,20;{\footnotesize вид сверху}}
\lbl[tl]{21,16;$L$}
\end{lpic}
\caption{Разбор гироскопического эффекта.}
\label{pic:9.2}
\end{figure}

Давайте посмотрим, что происходит с колесом, когда его ось поворачивается.
На рисунке \ref{pic:9.2} показана траектория одной частицы обода, вблизи верхней точки колеса.
Эта траектория искривлена из-за поворота оси.
Повенуясь инерции, частица пытается двигаться прямо, сопротивлясь отклонению с некоторой центробежной силой $F$ показаной на рисунке.
По тем же причинам, сила $-F$ действует на частицу вблизи нижней точки.
Получается, что на колесо действавует момент фиктивных сил вокруг линии $L$; именно этот момент и держит ось горизонтально.

\paragraph{Странная сила.}
\label{Антигравитационное велоколесо:Странная сила}
\rindex{колесо}
Вращающееся колесо демонстрирует пример очень странной силы, напоминающей силу магнитного поля на движущийся заряд.
В отличие от трения, эта странная сила направлена \emph{перпендикулярно} направлению движения оси гироскопа.
Чтобы выразиться точнее, превратим наше колесо в волчок, закрепив один конец оси на земле как показано на рисунке \ref{pic:9.3} (этот конец не может скользить, но может поворачиваться во все стороны).
Что случится, если попытаться сдвинуть свободный конец $A$ оси?
Чтобы не путаться в мелочах, пренебрежём силой тяжести.

\begin{figure}[ht!]
\centering
\begin{lpic}[t(2mm),b(2mm),r(0mm),l(0mm)]{pics/9.3}
\lbl[tl]{-4,37;(a)}
\lbl[tl]{47,37;(b)}
\lbl[tr]{9,26;$A$}
\lbl[b]{1,30;$F$}
\lbl[lb]{59,15;$F$}
\lbl[l]{16,34;$v$}
\lbl[l]{65.5,12;$v$}
\lbl[b]{32,1;{\footnotesize опора}}
\lbl[b]{40,11;\parbox{22mm}{\footnotesize\centering  шаровой\\шарнир}}
\end{lpic}
\caption{(a) Для поддержания неизменного направления оси требуется постоянное усилие в перпендикулярном направлении.
(b) Постоянное уклонение волчка не даёт ему упасть.}
\label{pic:9.3}
\end{figure}

\paragraph{Задача.}
\label{Антигравитационное велоколесо:Задача}
\rindex{колесо}
Предположим, что я двигаю конец $A$ оси вращающегося волчка с постоянной скоростью.
Куда направлена сила с которой я действую на $A$?
%В каком направлении нужно толкать конец $A$ оси вращающегося волчка, чтобы перемещать $A$ с постоянной скоростью? ???

\paragraph{Решение.}
Сила направлена перпендикулярно направлению движения (рисунок \ref{pic:9.3})!
Объяснение уже было дано при обсуждении вращающегося и поворачивающего велосипедного колеса.
Игра с настоящим вращающимся колесом даёт странное ощущение: если толкнуть ось, то она уйдёт под прямым углом к толчку.
Осознав это поведение, становится легко переориентировать колесо в любом направлении почти без усилий.

Похоже действует магнитное поле на движущийся заряд, эта сила направлена перпендикулярно скорости заряда.

Немного отвлечёмся.
Часто реакция людей ортогональна приложенному стимулу,
и это очень напоминает гироскопический/магнитный эффект.
Однако с магнитами сравнивают людей совсем другого типа.

\paragraph{Устойчивость за счёт уклонения.}
Волчок удерживается не потому, что сопротивляется гравитации, а более хитрым способом.
Любое движение оси порождает гироскопическую силу%
\footnote{Надо разъяснить, что эта сила фиктивная.
Когда я говорю «сила», я имею в виду, что волчок ведёт себя так, как если бы на него действовала внешняя сила.}%
, перпендикулярную этому движению, как показано на рисунке \ref{pic:9.3}.
На рисунке видно, что волчок может сначала начинать падать%
\footnote{В этот момент ось ускоряется, поэтому расссуждение из предыдущей задачи не применимо.\pr}%
, но затем отклоняется от падения вниз.
Постоянное действие этой уклоняющей силы приводит к траектории, изображённой на рисунке \ref{pic:9.3}.
Такой механизм можно назвать «уклончивой устойчивостью».

\paragraph{Энергетические соображения.}
То, что ось реагирует силой, \emph{перпендикулярной} навязанному движению (рисунок \ref{pic:9.3}), можно объяснить через закон сохранения энергии.
Действительно, если я перемещаю конец оси с постоянной скоростью $v$, то я не изменяю энергию вращения гироскопа.
Ведь если подшипники идеальны, я никак не могу влиять на скорость вращения.
Следовательно, я не совершаю работы, а значит, силе действия моей руки придётся быть перпендикулярной к скорости её движения.

\section{Гироскоп в велосипеде}
\rindex{велосипед}
\rindex{колесо}

Современный велосипед прошёл долгий путь дарвиновскотехнологической эволюции.
Он идеален, настолько насколько таковым может быть творение цивилизации.
Научиться ездить на велосипеде гораздо легче, чем объяснить физику этого действия.%
\footnote{Можно сказать, что тело умнее головы.}
Следующие две задачи посвящены этому весьма сложному процессу.

\begin{figure}[ht!]
\centering
\begin{lpic}[t(8mm),b(2mm),r(0mm),l(0mm)]{pics/9.4}
\lbl[tl]{-10,37;(a)}
\lbl[tl]{28,37;(b)}
\lbl[b]{9,31;\parbox{22mm}{\footnotesize\centering  вид сверху на\\переднее колесо}}
\lbl[b]{47,31;\parbox{32mm}{\footnotesize\centering вид спереди без\\поступательного движения}}
\lbl[l]{19,15;\parbox{22mm}{\footnotesize  направление\\движения}}
\lbl[bl]{14,4;$1$}
\lbl[l]{10,15.5;$2$}
\lbl[tl]{14,27;$3$}
\lbl[r]{46,15.5;$1$}
\lbl[lr]{42,27;$2$}
\lbl[t]{0,14;$F$}
\lbl[t]{55,14;$F$}
\end{lpic}
\caption{Гироскопический эффект переднего колеса.
(a) Поворот колеса вправо наклоняет велосипед влево.
(b) Наклон велосипеда влево также поворачивает колесо влево.}
\label{pic:9.4}
\end{figure}

\paragraph{Вопрос.}
Я еду на велосипеде прямо и слегка повeрнул руль вправо.
Как на меня повлияет гироскопический эффект от переднего колеса?

\paragraph{Ответ.}
Влево; это объясняется на рисунке \ref{pic:9.4}a.

\paragraph{Вопрос.}
Теперь я еду на велосипеде без рук по прямой и наклоняю раму влево (скажем, согнувшись в сторону в поясе).
В какую сторону гироскопический эффект будет пытаться повернуть переднее колесо?

\paragraph{Ответ.}
Тоже влево, как объясняется на рисунке \ref{pic:9.4}b.%
\footnote{То, что гироскопический эффект пытается повернуть колесо влево не означает, что оно туда и повернётся, ведь есть и другие силы, влияющие на колесо.\pr}

\section{Как катится монета?}
\rindex{колесо}

Устойчивость катящейся монеты похожа на чудо.
Создаётся впечатление, что монета разумна или, по крайней мере, обладает рефлексом опытного моноциклиста (уж точно она справляется лучше неопытного).
Случайные неровности поверхности ей не помеха --- она легко под них подстраивается;
она умудриться устоять даже если её слегка толкнуть.
Без каких-либо подвижных частей монета воплощает предельную простоту; но как же объяснить её рефлексы?
Как безмозглый кусок метала умудряется управлять своим движением, удерживаясь на тонкой грани между падением влево и вправо?

Следующая задача чуть приблизит нас к ответу.
Но даже разобравшись, как всё устроено, меня не перестаёт удивлять ловкость и естественность, с которой монета катится по столу.
Несмотря на объяснение (которое скоро будет дано), кажется счастливой случайностью, что гироскопический эффект \emph{помогает} монете устоять, а не упасть.
Более всего восхищает контраст между устойчивостью монеты и отсутсвием у неё мозгов.%
\footnote{В этом смысле некоторые люди тоже могут восхищать.}

\begin{figure}[ht!]
\centering
\begin{lpic}[t(2mm),b(2mm),r(0mm),l(0mm)]{pics/9.5}
\lbl[rw]{38,27;{\footnotesize 1: наклон вправо}}
\lbl[r]{25,21;\parbox{13mm}{\footnotesize 2: момент силы}}
\lbl[tl]{39,21;\parbox{16mm}{\footnotesize 3: поворот направо не даёт упасть}}
\end{lpic}
\caption{Гироскопический эффект подправляет движение катящейся монеты, предотвращая её падение.
Из-за наклона (1) сила тяжести создаёт гироскопический момент (2), который заставляет монету поворачивать вправо (3).}
\label{pic:9.5}
\end{figure}

\paragraph{Задача.}
Если монета катится вперёд с уклоном вправо, как показано на рисунке \ref{pic:9.5}, то её траектория поворачивает тоже вправо.
Из-за этого счастливого совпадения монета не падает.
Похоже, что монета умна --- наклонившись вправо, она поворачивает вправо, как это делает наклонившийся велосипедист, тем самым избегая падения.
Почему же монета поворачивает в сторону своего наклона?

\paragraph{Решение.}
Когда монета катится вперёд с наклоном вправо,
она начинает падать (рисунок \ref{pic:9.5}).
Это падение --- то есть рост её наклона ---
вызывает гироскопический момент (объясняется в следующем предложении;
см. также страницу \pageref{Антигравитационное велоколесо:Ответ}),
который подворачивает монету, а значит и её траекторию, вправо; эта поправка и предотвращает падение.
Для понимания механизма происходящего, вообразим, что монета движется прямо, продолжая всё больше наклоняться.
В системе отсчёта, движущейся вместе с монетой, мы бы наблюдали изгиб траекторий частиц на её ободе, как показано
на рисунке \ref{pic:9.2}.
Возникающая центробежная сила будет пытаться повернуть монету вправо.

Приведённое рассуждение говорит лишь, что монета может не упасть, оно вовсе не доказывает, что монета не упадёт.
Откуда нам знать, например, что этого самокорректирующего эффекта хватит, чтобы удерживать монету?
А может и наоборот, этот эфект окажется слишком сильным, вызывая нестабильный колебательный процесс?
Конечно же, чтобы сохранять равновесие, монета должна катиться достаточно быстро.
Подробное обсуждение устойчивости катящейся монеты приводится на страницах 58---67 \emph{«Динамики неголономных систем»} Неймаркa и Фуфаева.

\section{Как удержаться на скользком куполе?}\label{Как удержаться на скользком куполе?}

\paragraph{Задача.}
Может ли твёрдое кольцо удержаться от падения на идеально скользком сферический куполе, даже если разрешается слегка подтолкнуть кольцо (рисунок \ref{pic:9.6})?
Установить кольцо точно на верху не получиться, ведь малейший толчок заставит его соскользнуть.
Использование внешних опор, включая магнитные или тому подобные устройства, не разрешается.

\begin{figure}[ht!]
\centering
\begin{lpic}[t(2mm),b(2mm),r(0mm),l(0mm)]{pics/9.6}
\end{lpic}
\caption{Как не сокользнуть с идеально скользкого купола?}
\label{pic:9.6}
\end{figure}

\paragraph{Решение.}
Надо поместить кольцо на купол,
хорошо раскрутить быстро вокруг его центральной оси и отпустить.
Если кольцо вращается достаточно быстро и оно расположено не слишком далеко от вершины, то оно не соскользнёт, а будет двигаться так, как показано на рисунке \ref{pic:9.7}d.%
\footnote{Мы считаем, что трения нет вовсе. Иначе кольцо стало бы  замедлятся и соскользывать.}

\paragraph{Почему это сработает.}
Можно думать, что наше кольцо на сфере это волчок: колесо с длинной осью, вращающееся на шарнире (рис. \ref{pic:9.3}a, страница \pageref{pic:9.3}).
Представьте себе, что вместо того чтобы строить идеально скользкую сферу (что очень не просто), мы прикрепим кольцо к концу стержня невесомыми спицами, получится велосипедное колесо на длинной оси.
Конец оси опирается на стол или прикреплён к столу с помощью шарнира без трения.
Таким образом, положение кольця будет ограничено невидимой сферой.
Итак, кольцо на сфере --- это волчок.%
\footnote{Чтобы аналогия с волчком стала полной, придётся предположить, что кольцо не может отрываться от поверхности купола (хотя на самом деле такое может произойти).
Для этого можно считать, что кольцо удерживается на сфере какой-то силой, например, магнитной.}
Остаётся вспомнить, что волчок не падает, если его раскрутить достаточно быстро. Читатель может обратиться к объяснению на странице \pageref{Антигравитационное велоколесо:Ответ} или прочитать следующий абзац.

\begin{figure}[ht!]
\centering
\begin{lpic}[t(7mm),b(2mm),r(0mm),l(0mm)]{pics/9.7}
\lbl[bl]{-5,75;(a)}
\lbl[bl]{47,75;(b)}
\lbl[bl]{-5,34;(c)}
\lbl[bl]{47,34;(d)}
\lbl[lt]{79,50;$1$}
\lbl[lt]{73.5,47.5;$2$}
\lbl[tl]{69,43;$3$}
\lbl[br]{59,58;$1'$}
\lbl[br]{65,60;$2'$}
\lbl[br]{69,65;$3'$}
\lbl[bl]{70,54;$F$}
\lbl[tr]{67,54;$F'$}
\lbl[bl]{5,32;$F$}
\lbl[tr]{14.5,18;$F'$}
\lbl[tl]{23,9;\parbox{12mm}{\footnotesize центр сферы}}
\lbl[b]{24,13,39.5;{\footnotesize ось момента $F$ и $F'$}}
\lbl[t]{69,10;{\footnotesize траектория центра кольца}}
\end{lpic}
\caption{Почему кольцу удаётся не соскользнуть вниз.}
\label{pic:9.7}
\end{figure}

\paragraph{Прямое объяснение.}
Вскоре после того как кольцо отпускают, оно начинает скользить вниз.
Поэтому, если смотреть из системы отсчёта, связанной с центром кольца, оно как бы поворачивается, как показано на рисунке \ref{pic:9.7}b.
Этот поворот вызывает искривление траекторий частиц кольца --- например, траектории 1–2–3.
В силу инерции частица сопротивляется этому искривлению центробежной силой $F$, перпендикулярной к плоскости кольца.
Диаметрально противоположная частица кольца создаёт такую же по величине, но противоположно направленную силу $F$.
Эти две силы создают момент (рисунок \ref{pic:9.7}c).
(Я рассмотрел только две частицы, но остальные суммарно создают такой же эффект.)
Этот гироскопический момент заставляет кольцо отклоняться от направления своего движения, и кольцо будет двигаться так, как показано на рисунке \ref{pic:9.7}d.

Именно так это и работает: не сопротивление падению, а уклонение от него!

Чтобы развить физическую интуицию
представьте себя скользящим на пузе по большой сфере, и при этом вы быстро крутитесь колесом (как при выполнении акробатического колеса).

\section{Как найти север с помощью гироскопа}

\paragraph{Вопрос.}
Как найти направление на географический север, используя гироскоп?
Считаем, что у вас идеальный гироскоп, без трения, способный вращаться вечно.

\paragraph{Ответ.}
Если установить гироскоп горизонтально на плотик, который плавает в ёмкости с водой,
то ось гироскопа медленно повернётся точно по меридиану!
При этом направление на север будет то, в которое гироскоп вкручивается по правилу правого винта, то есть если думать, что ось это винт, то она будет вкручиваться в направлении севера.
Иными словами, гироскоп пытается как можно лучше выровнять своё вращение с вращением Земли, принимая во внимание, что его ось должна остоваться горизонтальной.

\paragraph{Почему же гироскоп ищет север?}
Для простоты разместим нашу установку на экваторе (рисунок \ref{pic:9.8}).
Вращение Земли заставляет вращаться нашу установку, и можно считать, что она вращается вокруг пунктирной линии север---юг.
Предположим, что ось гироскопа изначально направлена в каком-то другом направлении --- скажем, восток---запад, как на рисунке.
Вращения Земли толкает ось гироскопа вдоль стрелок ($A$);
гироскоп же отвечает поворотом в направлении ($B$).
Гироскоп ведёт себя ровно так, как объяснялось на странице \pageref{Антигравитационное велоколесо:Задача}: если толкнуть его ось, то он отреагирует движением в перпендикулярном направлении.
В итоге ось плавающего гироскопа ориентируется вдоль меридиана, как показано на рисунке.

\begin{figure}[ht!]
\centering
\begin{lpic}[t(7mm),b(2mm),r(0mm),l(0mm)]{pics/9.8}
\lbl[r]{0,86;$A$}
\lbl[t]{18,69;$A$}
\lbl[t]{13,71;$B$}
\lbl[b]{6,83;$B$}
\lbl[tl]{-5,97;{\footnotesize НАЧАЛО:}}
\lbl[tl]{-5,34;{\footnotesize КОНЕЦ:}}
\lbl[b]{18,88,50;{\scriptsize меридиан}}
\lbl[b]{21,28,60;{\scriptsize меридиан}}
\lbl[b]{43,30,64;{\scriptsize меридиан}}
\lbl{47.5,40.5;{\footnotesize экватор}}
\lbl[tl]{19,76;\parbox{16mm}{\scriptsize с земным\\ вращением}}
\lbl[t]{10,64;\parbox{20mm}{\footnotesize\centering ответ на\\ вращение}}
\lbl[l]{43,22;\parbox{16mm}{\footnotesize земнoe\\ вращение}}
\lbl[]{46,58;\parbox{18mm}{\footnotesize\centering плотик\\ на воде}}
\lbl[lt]{53,76.5;
\begin{tikzpicture}[
  decoration={
    text effects along path,
    text/.expanded=\bracetext{выстраивается по вращению земли },
    text effects/.cd,
    text along path,
    character count=\i,
    character total=\n,
    characters={scale=.7}
    }
]
\draw [decorate] (0,.6) .. controls (1.6,.9) and (3.8,.1) .. (2.7,-1.2);
\end{tikzpicture}
}
\end{lpic}
\caption{Как работает гирокомпас.}
\label{pic:9.8}
\end{figure}

\paragraph{Больше свободы.}
Вместо того чтобы помещать гироскоп на плотик, можно оставить его ось несвязанной, установив гироскоп на карданном подвесе или погрузив его в жидкость так, чтобы он имел нейтральную плавучесть.
Тогда гироскоп будет постепенно выравниваться по оси Земли. А угол его оси с горизонтом укажет широту!

В итоге, независимо от способа подвеса, гироскоп пытается по возможности согласовать своё вращение с вращением Земли.

\paragraph{Задача.} Почему подвешенный гирокомпас выравнивает свою ось по оси Земли?
(Подсказка: трение заставляет гироскоп переориентироваться.)

Полезность гирокомпаса в его независимости от магнитных аномалий.
Это очень полезно на стальном корабле.
Кроме того он указывает на географический, а не магнитный полюс.

\paragraph[Немного истории.]%
{Немного истории.%
\footnote{Больше подробностей можно найти о гирокомпасе в статье Википедии.}
}
Гирокомпас был запатентован в 1908 году Э. А. Сперри, автором множества других изобретений.
На счету Сперри более 400 патентов, но гирокомпас, пожалуй, самый известный.
Его гирокомпас сыграл важную роль в Первой мировой войне.
После смерти Сперри в его честь был назван корабль ВМС США (плавбаза подводных лодок).
Этот корабль был спущен на воду через десять дней после нападения на Перл-Харбор, а после долгой службы, завершившейся в 1982 году, его перевели в музей.
Гирокомпас Сперри используется на кораблях и по сей день.

Ещё в 1916 году, в разгар Первой мировой войны,
Сперри и Питер Хьюитт изобрели БПЛА --- самолёт-дрон.
Сперри пророчески назвал его «бомбой будущего».
В конеце войны Сперри и Хьюитт в длинной череде проб и ошибок пытались сделать из этой идеи надёжное оружие.
Сын изобретателя, Лоуренс Сперри, учавствовал в опасных испытаниях, и несколько раз чуть не погиб.
