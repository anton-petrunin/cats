\chapter{Парадоксы гироскопа}

\section{Как волчок уклоняется от гравитации?}

Что удерживает вращающийся волчок в вертикальном положении, так это не сила, противодействующая гравитации.
Наоборот, это странная отклоняющая сила --- сила, которая всегда остаётся перпендикулярной направлению движения оси волчка.
Эта отклоняющая сила подрывает неустойчивость: волчок начинает падать, но затем отклоняется в сторону, и в результате движется так, как показано на рисунке \ref{pic:9.3}.
В следующем абзаце я постараюсь показать, как эта странная «гироскопическая сила» возникает из второго закона Ньютона.

\paragraph{Велосипедное колесо, противостоящее гравитации.}
Будем использовать велосипедное колесо в качестве нашего волчка.
Подвесьте его на двух верёвках, как показано на рисунке \ref{pic:9.1}, и раскрутите быстро.
Теперь перережьте одну из верёвок.
Удивительно, но не поддерживаемый конец оси не упадёт вниз.%
\footnote{Предполагаем, что колесо хорошо раскручено.}
Вместо этого он начнёт медленно поворачивать.
Причём чем быстрее вращается колесо, тем медленнее будет идти поворачивание.

\begin{figure}[ht!]
\centering
\begin{lpic}[t(2mm),b(2mm),r(0mm),l(0mm)]{pics/9.1}
\lbl[b]{15,31;$F$}
\lbl[b]{57,20;$F$}
\lbl[lb]{71,8;$1$}
\lbl[l]{67,20;$2$}
\lbl[lt]{71,30;$3$}
\lbl[b]{60,35;$1$}
\lbl[b]{65.3,36;$2$}
\lbl[b]{71,35;$3$}
\lbl[b]{10,8,95;{\footnotesize вращение}}
\lbl[r]{4,33;\parbox{22mm}{\footnotesize\raggedleft  поворачивание\\ оси}}
\lbl[l]{38,21;{\footnotesize разрыв}}
\lbl[t]{66,3;{\footnotesize вид сверху}}
\end{lpic}
\caption{Как инерция частиц гироскопа не даёт ему упасть после разрыва верёвки.}
\label{pic:9.1}
\end{figure}

\paragraph{Вопрос.}
Что же мешает колесу упасть --- как ему удаётся продиводействовать силе тжести?


\paragraph{Ответ.}
Если коротко, то за это отвечает определённая центробежная сила, но не та, что первой приходит на ум, а другая --- перпендикулярная ей!

\begin{figure}[ht!]
\centering
\begin{lpic}[t(2mm),b(2mm),r(0mm),l(0mm)]{pics/9.2}
\lbl[t]{3,3;\parbox{22mm}{\footnotesize\centering  момент\\ силы}}
\lbl[]{17,38.6,-20;\parbox{22mm}{\footnotesize\centering  центробежная\\ сила}}
\lbl[b]{67,20;{\footnotesize вид сверху}}
\lbl[tl]{21,16;$L$}
\end{lpic}
\caption{Разбор гироскопического эффекта.}
\label{pic:9.2}
\end{figure}

Давайте посмотрим, что происходит с вращающимся колесом, когда его ось поворачивает.
На рисунке \ref{pic:9.2} показана траектория одной частицы обода, когда она проходит вблизи верхней точки колеса.
Эта траектория искривлена из-за поворота оси.
Повенуясь инерции, частица старается двигаться как можно прямей, сопротивляется отклонению с некоторой центробежной силой $F$, как показано.
Подобная сила $-F$ действует на частицу вблизи нижней точки.
Совместный эффект таков, как если бы невидимый момент сил скручивал колесо вокруг линии $L$; именно этот момент не даёт колесу упасть.

\paragraph{Странная сила.}
Вращающееся колесо демонстрирует пример очень странной силы, похожей на магнитную силу, действующую на движущийся заряд: в отличие от трения, эта сила направлена перпендикулярно движению оси.
Чтобы выразиться точнее, превратим наше колесо в волчок, поставив один конец оси на землю (так, чтобы он не мог скользить, но мог поворачиваться), как показано на рисунке \ref{pic:9.3}.
Давайте посмотрим, что мы почувствовали бы, если бы попытались сдвинуть свободный конец $A$ оси; чтобы не запутывать картину, пренебрежём действием силы тяжести.

\begin{figure}[ht!]
\centering
\begin{lpic}[t(2mm),b(2mm),r(0mm),l(0mm)]{pics/9.3}
\lbl[tl]{-4,37;(a)}
\lbl[tl]{47,37;(b)}
\lbl[b]{1,30;$F$}
\lbl[lb]{59,15;$F$}
\lbl[l]{16,34;$v$}
\lbl[l]{65.5,12;$v$}
\lbl[b]{32,1;{\footnotesize опора}}
\lbl[b]{40,11;\parbox{22mm}{\footnotesize\centering  шаровой\\шарнир}}
\end{lpic}
\caption{(a) Для поддержания неизменного направления оси требуется постоянное усилие в перпендикулярном направлении.
(b) Постоянное отклоняющее воздействие гироскопического эффекта не даёт волчку упасть.}
\label{pic:9.3}
\end{figure}

\paragraph{Задача.}
В каком направлении нужно толкать конец $A$ оси вращающегося волчка, чтобы перемещать $A$ с постоянной скоростью?

\paragraph{Решение.}
Следует приложить силу перпендикулярно желаемому направлению движения (рисунок \ref{pic:9.3})!
Объяснение уже было дано при обсуждении вращающегося и поворачивающего велосипедного колеса.
Игра с настоящим вращающимся колесом даёт странное ощущение: когда вы толкаете ось, она отклоняется под прямым углом к вашему толчку.
Осознав это поведение, становится легко переориентировать колесо в любом направлении без особых усилий.

Подобным образом действует магнитная сила на движущийся заряд --- в направлении перпендикулярном скорости заряда.

Лирическое отступление:
гироскопический/магнитный эффект имеет аналогию в человеческом поведении: часто реакция людей ортогональна приложенному стимулу, а при этом с магнитами сравнивают людей совсем другого типа.

\paragraph{Устойчивость за счёт отклонения.}
Волчок удерживается не потому, что сопротивляется гравитации, а более тонким образом.
Любое движение оси порождает гироскопическую силу%
\footnote{Надо разъяснить, что эта сила фиктивная.
Когда я говорю «сила», я имею в виду, что волчок ведёт себя так, как если бы на него действовала внешняя сила.}%
, перпендикулярную этому движению, как показано на рисунке \ref{pic:9.3}.
На рисунке видно: волчок может сначала начинать падать, но затем отклоняется от падения вниз.
Постоянное действие этой отклоняющей силы приводит к траектории, изображённой на рисунке \ref{pic:9.3}.
Такой механизм можно назвать «устойчивостью за счёт отклонения».

\paragraph{Энергетические соображения.}
Тот, что ось реагирует силой, перпендикулярной навязанному движению (рисунок \ref{pic:9.3}), можно объяснить законом сохранения энергии следующим образом.
Если я перемещаю конец оси с постоянной скоростью $v$, то я не изменяю энергию вращения гироскопа.
Действительно, подшипники идеальны, поэтому я не ускоряю и не замедляю вращение гироскопа.
Следовательно, я не совершаю работы, а значит, сила моей руки должна быть перпендикулярна скорости движения моей руки.

\section{Гироскоп в велосипеде}

Современный велосипед прошёл долгой путь дарвиновскотехнологической эволюции и похоже, что он совершенен, насколько таковым может быть творение цивилизации.
Научиться ездить на велосипеде гораздо легче, чем объяснить физику этого действия.%
\footnote{Похоже, что тело умнее головы.}
Следующие две задачи посвящены этому весьма сложному вопросу.

\paragraph{Вопрос.}
Я еду на велосипеде прямо и слегка повeрнул руль вправо.
Как на меня повлияет гироскопический эффект от переднего колеса?

\begin{figure}[ht!]
\centering
\begin{lpic}[t(7mm),b(2mm),r(0mm),l(0mm)]{pics/9.4}
\lbl[tl]{-10,37;(a)}
\lbl[tl]{34,37;(b)}
\lbl[b]{9,31;\parbox{22mm}{\footnotesize\centering  вид сверху на\\переднее колесо}}
\lbl[b]{47,31;{\footnotesize вид спереди}}
\lbl[l]{19,15;\parbox{22mm}{\footnotesize  направление\\движения}}
\lbl[bl]{14,4;$1$}
\lbl[l]{10,15.5;$2$}
\lbl[tl]{14,27;$3$}
\lbl[r]{46,15.5;$1$}
\lbl[lr]{42,27;$2$}
\lbl[t]{0,14;$F$}
\lbl[t]{55,14;$F$}
\end{lpic}
\caption{Гироскопический эффект переднего колеса.
(a) Поворот колеса вправо наклоняет велосипед влево.
(b) Наклон велосипеда влево также поворачивает колесо влево.}
\label{pic:9.4}
\end{figure}

\paragraph{Ответ.}
Рама наклонится влево; это объясняется на рисунке \ref{pic:9.4}a.

\paragraph{Вопрос.}
Теперь я еду на велосипеде без рук по прямой, я наклоняю раму влево --- скажем, согнувшись в сторону в поясе.
В какую сторону гироскопический эффект будет пытаться повернуть переднее колесо?

\paragraph{Ответ.}
Тоже влево, как объясняется на рисунке \ref{pic:9.4}b.

\section{Как катится монета?}

Устойчивость катящейся монеты похожа на чудо.
Создаётся впечатление, что монета обладает умом или, по крайней мере, рефлексом опытного моноциклиста (уж точно она справляется лучше неопытного).
Случайные неровности поверхности не имеют для неё значения --- монета легко  под них подстраивается;
она умудриться устоять даже если её слегка толкнуть.
Без ких-либо подвижных частей монета воплощает предельную простоту; но как же объяснить её рефлексы?
Как безмозглый кусок метала умудряется управлять своим движением, удерживаясь на тонкой грани между падением влево или вправо?

Следующая задача чуть приблизит нас к пониманию.
Но даже разобравшись, как всё устроено, я остаюсь под впечатлением от ловкости, простоты и естественности, с которым монета катится, не падая.
Несмотря на объяснение (которое скоро будет дано), кажется счастливой случайностью, что гироскопический эффект удерживает монету от падения, а не, наоборот, помогает ей упасть.
Восхищает контраст между устойчивостью монеты и отсутсвием у неё мозгов.%
\footnote{В этом смысле некоторые люди тоже меня поражают.}

\begin{figure}[ht!]
\centering
\begin{lpic}[t(2mm),b(2mm),r(0mm),l(0mm)]{pics/9.5}
\lbl[rw]{38,27;{\footnotesize 1: наклон вправо}}
\lbl[r]{25,21;\parbox{13mm}{\footnotesize 2: момент силы}}
\lbl[tl]{39,21;\parbox{16mm}{\footnotesize 3: поворот направо не даёт упасть}}
\end{lpic}
\caption{Гироскопический эффект подправляет движение катящейся монеты, предотвращая её падение. Из-за наклона (1) сила тяжести создаёт гироскопический момент (2), который заставляет монету поворачивать вправо (3).}
\label{pic:9.5}
\end{figure}

\paragraph{Задача.}
Если монета катится вперёд с уклоном вправо, как показано на рисунке \ref{pic:9.5}, то её траектория поворачивает тоже вправо.
Из-за этого счастливого совпадения монета не падает.
Похоже, что монета умна --- наклонившись вправо, она поворачивает вправо, как это делает наклонившийся велосипедист, тем самым избегая падения.
Почему же монета поворачивает в сторону своего наклона?

\paragraph{Решение.}
Когда монета катится вперёд с наклоном вправо,
она начинает падать (рисунок \ref{pic:9.5}).
Это падение --- то есть рост её наклона ---
вызывает гироскопический момент (объясняется в следующем предложении;
см. также страницу???), который поворачивает монету вправо,
заставляя её траекторию поворачивать вправо; эта коррекция и предотвращает падение.
Для понимания гироскопического механизма происходящего, вообразим, что бы произошло, если бы монета двигалась прямо, продолжая всё больше наклоняться.
В системе отсчёта, движущейся вместе с монетой, мы бы наблюдали изгиб траекторий частиц на её ободе, как показано
на рисунке \ref{pic:9.2}.
Возникающая центробежная сила будет стремиться повернуть монету вправо.

Приведённое рассуждение говорит лишь, что монета может не упасть --- оно вовсе не доказывает, что монета не упадёт.
Откуда нам знать, например, что этот самокорректирующий эффект достаточно силён, чтобы удерживать монету?
А может и наоборот, этот эфект окажется чрезмерно сильным, вызывая нестабильный колебательный процесс?
На самом деле, чтобы сохранять равновесие, монета должна катиться достаточно быстро.
Подробное обсуждение устойчивости катящейся монеты
можно найти на страницуах ???---??? «Динамики неголономных систем» Неймаркa и Фуфаева.

\section{Как удержаться на скользком куполе?}

\paragraph{Задача.}
Можно ли поместить твёрдое кольцо на идеально гладкий полусферический купол (рис. 9.6) так, чтобы оно не соскользнуло, даже если его слегка подтолкнуть?
Удержать кольцо точно на верху не получиться, ведь малейший толчок заставит его соскользнуть.
Внешние опоры, включая магнитные или тому подобные устройства, не допускаются.

\begin{figure}[ht!]
\centering
\begin{lpic}[t(2mm),b(2mm),r(0mm),l(0mm)]{pics/9.6}
\end{lpic}
\caption{Как не сокользнуть с идеально скользкого купола?}
\label{pic:9.6}
\end{figure}

\paragraph{Решение.}
Надо поместить кольцо на купол,
хорошо раскрутить быстро вокруг его центральной оси и отпустить.
Если кольцо вращается достаточно быстро и оно расположено не слишком далеко от вершины, то оно не соскользнёт, а будет двигаться так, как показано на рисунке \ref{pic:9.7}d.%
\footnote{Мы считаем, что трения нет вовсе. Иначе кольцо стало бы  замедлятся и соскользывать.}

\paragraph{Почему это сработает.}
Наше кольцо на сфере --- это по сути волчок: колесо с длинной осью, вращающееся на шарнире (рис. \ref{pic:9.3}a, стр. ???).
Чтобы пояснить эту аналогию: вместо того чтобы строить идеально скользкую сферу (что очень не просто), можно просто прикрепить кольцо к одному концу стержня с помощью невесомых спиц --- по сути, получится велосипедное колесо с длинной осью.
Конец оси опирается на стол или прикреплён к столу с помощью шарнира без трения.
Таким образом, кольцо фактически будет ограничено невидимой сферой, без трения.
Итак, кольцо на сфере --- это волчок.%
\footnote{Чтобы аналогия с волчком стала полной, придётся предположить, что кольцо не может отрываться от поверхности купола (хотя на самом деле такое может произойти).
Для этого можно считать, что кольцо удерживается на сфере какой-то силой, например, магнитной.}
А волчок не падает, если его раскрутить достаточно быстро. Читатель может обратиться к объяснению на стр. ??? или прочитать следующий абзац.

\begin{figure}[ht!]
\centering
\begin{lpic}[t(2mm),b(2mm),r(0mm),l(0mm),draft]{pics/9.7}
\lbl[bl]{-5,75;(a)}
\lbl[bl]{47,75;(b)}
\lbl[bl]{-5,34;(c)}
\lbl[bl]{47,34;(d)}
\end{lpic}
\caption{Почему кольцу удаётся не соскользнуть вниз.}
\label{pic:9.7}
\end{figure}

\paragraph{Прямое объяснение.}
Вскоре после того как кольцо отпускают, оно начинает скользить вниз.
Поэтому, если смотреть из системы отсчёта, связанной с центром кольца, оно как бы поворачивается, как показано на рисунке \ref{pic:9.7}b.
Этот поворот вызывает искривление траекторий частиц кольца --- например, траектории 1–2–3.
В силу инерции частица сопротивляется этому искривлению центробежной силой $F$, перпендикулярной к плоскости кольца.
Диаметрально противоположная частица кольца создаёт такую же по величине, но противоположно направленную силу $F$.
Эти две силы действуют на кольцо с моментом сил (рис. 9.7c).
(Я рассмотрел только две частицы, но остальные в совокупности создают такой же эффект.)
Этот гироскопический момент заставляет кольцо отклоняться от направления своего движения, и кольцо будет двигаться так, как показано на рисунке \ref{pic:9.7}d.

Именно так это и работает: не сопротивление падению, а отклонение от него!

Чтобы лучше развить физическую интуицию попробуйте представить, что вы лежите на скользком куполе и вас раскрутили с большой скоростью колесом (как при колесе в гимнастике).

\section{Как найти север с помощью гироскопа}

\paragraph{Вопрос.}
Как найти направление на истинный север, используя гироскоп?
Считаем, что у вас идеальный гироскоп, не имеет трения и может вращаться вечно.

\paragraph{Ответ.}
Установите ваш гироскоп горизонтально на плотик, который плавает в ёмкости с водой.
Ось гироскопа будет медленно повернётся точно по меридиану!
А направление на север --- это то, в которое гироскоп вкручивается по правилу правого винта:
если думать, что ось это винт, то он будет вращаться так, чтобы двигаться на север.
Иными словами, гироскоп стремится как можно сильнее выровняться с вращением Земли, при том, что его ось остаётся в горизонтальной плоскости.

\paragraph{Почему же гироскоп ищет север?}
Для простоты разместим механизм на экваторе (рисунок \ref{pic:9.8}).
Из-за вращения Земли вся установка фактически вращается вокруг пунктирной линии север–юг.
Предположим, что ось гироскопа изначально направлена в каком-то другом направлении --- скажем, восток–запад, как на рисунке.
Вследствие вращения Земли на ось действует толчок вдоль стрелок (A);
гироскоп отвечает на это, переориентируясь в направлении (B).
Здесь гироскоп ведёт себя именно так, как объяснялось на странице ???: когда ось толкают в одном направлении, он реагирует движением в перпендикулярном направлении.
В итоге ось плавающего гироскопа ориентируется вдоль меридиана, как показано на рисунке.

\begin{figure}[ht!]
\centering
\begin{lpic}[t(2mm),b(2mm),r(0mm),l(0mm)]{pics/9.8}
\end{lpic}
\caption{}
\label{pic:9.8}
\end{figure}

\paragraph{Больше свободы.}
Вместо того чтобы помещать гироскоп на плавающую платформу, можно оставить его ось несвязанной, установив гироскоп на карданном подвесе или погрузив его в жидкость так, чтобы он имел нейтральную плавучесть.
Тогда гироскоп будет постепенно выравниваться по оси Земли. А угол его оси с горизонтом укажет широту!

В итоге, независимо от способа подвеса, гироскоп старается согласовать своё вращение с вращением Земли настолько, насколько это позволяют ограничения.

\paragraph{Задача.} Почему подвешенный гирокомпас выравнивает свою ось по оси Земли?
(Подсказка: трение с жидкостью заставляет гироскоп переориентироваться.)

Прекрасное свойство гирокомпаса заключается в его независимости от магнитных аномалий --- большое преимущество на стальном корабле --- а также в его способности указывать на географический, а не магнитный полюс.

\paragraph{Немного истории.}
Гирокомпас был запатентован в 1908 году Э. А. Сперри, автором множества других изобретений, на его счету более 400 патентов.
Гирокомпас является, пожалуй, самым известным его изобретением, он сыграл важную роль в Первой мировой войне.
После смерти Сперри в его честь был назван корабль ВМС США (плавбаза подводных лодок).
Этот корабль был спущен на воду всего через десять дней после нападения на Перл-Харбор, а после долгой службы, завершившейся в 1982 году, его перевели в музей.
Гирокомпас Сперри используется на кораблях и по сей день.

Сперри вместе с Питером Хьюиттом также изобрели БПЛА --- самолёт-дрон --- ещё в 1916 году, в разгар Первой мировой войны.
Сперри пророчески назвал дрон «бомбой будущего».
В конеце войны Сперри и Хьюитт в длинной череде проб и ошибок пытались сделать эту идею более надёжной.
Во время этих испытаний сын изобретателя, Лоуренс Сперри учавствовал в опасных пробных полётах, и  несколько раз едва не погиб.
