\chapter{Парадоксы в невесомости}

\section{Гелиевый шар}

\begin{thm}{Задача}
Два космонавта, Андрей и Боря, пристёгнуты к противоположным концам космической капсулы, как показано на рисунке \ref{pic:2.1}.
В начале всё находится в покое и Андрей держит в руках большой воздушный шарик наполненный гелием.
Он толкает шарик, и тот начинает двигаться в сторону Бори.
В каком направлении начнёт двигаться сама капсула с точки зрения наблюдателя, парящего в открытом космосе вне капсулы?
Поскольку Андрей и Боря пристёгнуты к стенкам, их можно считать частью капсулы.
\end{thm}

\begin{figure}[ht!]
\centering
\begin{lpic}[t(2mm),b(2mm),r(0mm),l(0mm)]{pics/2.1(1)}
\lbl[b]{29,24;куда?}
\lbl{29,20;воздух}
\lbl{29,14;{\footnotesize гелий}}
\lbl[lt]{0,-.5;Андрей}
\lbl[rt]{58,-.5;Боря}
\end{lpic}
\caption{Куда двинется капсула когда Андрей толкнёт шар?}
\label{pic:2.1}
\end{figure}

\paragraph{Првдоподобное рассуждение.}
Когда Андрей толкает шарик вправо, шарик отталкивает Андрея влево, ведь по третьему закону Ньютона «действие равно противодействию».
А раз шарик толкает Андрея влево, то он и вся капсула начнут двигаться влево.
Похоже на правду?

\paragraph{Ответ.}
На самом деле --- нет: капсула тоже будет двигаться вправо!

\paragraph{Объяснение через центр масс.}
Центр масс всей системы (капсулы и её содержимого) остаётся неподвижным, поскольку на систему не действуют внешние силы (все понятия из этого предложения объясняются в приложении, стр. \pageref{???}).

Рассмотрим движение внутри капсулы с точки зрения Андрея, как показано на рисунке~\ref{pic:2.2}.
Шарик имеет гораздо меньшую массу, чем вытесняемый им воздух.
Значит, с точки зрения Андрея, центр масс смещается влево.
И поскольку центр масс всей системы без внешних сил оставаётся неподвижным,
и Андрей и капсула движутся вправо с точки зрения внешнего наблюдателя.

Наша ошибка состояла в том, что мы слишком много думали о шарике и недостаточно о более массивном воздухе, который перемещается влево, занимая его место.

\begin{figure}[ht!]
\centering
\begin{lpic}[t(2mm),b(2mm),r(0mm),l(0mm)]{pics/2.2(1)}
\lbl[tl]{22,20;воздух}
\lbl[tl]{22,3.5;воздух}
\lbl{21.5,12.5;{\footnotesize гелий}}
\lbl[bl]{58,12;капсула}
\end{lpic}
\caption{Движение с относительно капсулы (и Андрея).}
\label{pic:2.2}
\end{figure}
