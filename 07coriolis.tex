\chapter{Парадоксы силы Кориолиса}

\section{Что такое сила Кориолиса}

\paragraph{Вопрос.}
Представим себе, что вы играете в мяч на платформе крытой карусели, так что вам ничего не видно снаружи.
Стоя в центре, вы хотите попасть мячом в цель на краю.
Вы кидаете мяч прямо в цель, но промахиваетесь, он отклоняется вправо, как показано на рисунке \ref{pic:7.1}.
Почему же это происходит?
Забудем пока о силе тяжести и будем считать, что вращение происходит против часовой стрелки.

\begin{figure}[ht!]
\centering
\begin{lpic}[t(7mm),b(2mm),r(30mm),l(0mm)]{pics/7.1}
\lbl[b]{50,28;$A$}
\lbl[t]{52,13;$A'$}
\lbl[l]{61,30;$B$}
\lbl[tl]{58,7;$B'$}
\lbl[r]{33,12;\parbox{22mm}{\footnotesize\raggedleft  след мяча\\на платформе}}
\lbl[b]{35,38;\parbox{42mm}{\footnotesize\centering  прямолинейное движение\\мяча относительно земли}}
\lbl[l]{61,39;\parbox{41mm}{\footnotesize
когда мяч долетит до точки
$A$, он окажется над точкой $A'$, а позже, точка $B'$ окажется под мячом в точке $B$.}}
\end{lpic}
\caption{Объяснение силы Кориолиса.}
\label{pic:7.1}
\end{figure}

\paragraph{Ответ.}
Самый простой ответ такой:
«На самом деле мяч летит прямо, но из-за вращения платформы, смещается сама цель.
Поэтому мяч и пролетает правее.
Просто в закрытом вращающемся помещении создаётся впечатление, будто это мяч отклонился вправо».

Чтобы получше в этом разобраться, вообразите, что мяч отмечает свой путь на платформе, выпуская вниз струйку чернил.%
\footnote{Напомним, что мы пренебрегаем силой тяжести. Соответсвенно мяч летит по горизонтальной прямой.}
Хотя мяч летит прямо, из-за вращения платформы, след, который он оставит, окажется изогнутым как показано на рисунке.
Для нас, находящихся на земле, в таком искривлении нет ничего загадочного.
Но наблюдатель на платформе%
\footnote{ Земля — это пример такой платформы, хотя на протяжении большей части человеческой истории люди не осознавали её вращения.}%
, который воспринимает её неподвижной, испытывает иллюзию действия невидимой силы.
Эта мнимая сила и называется \emph{силой Кориолиса}.

Мы с вами живём во вращающемся мире, где сила Кориолиса проявляется повсюду.
Она вызывает вращение циклонов и антициклонов, а также влияет на океанические течения.
Сила Кориолиса обсуждается во многих книгах, например у Арнольда, Гольдштейна и Ландау --- Лифшица.%
\footnote{Марк, по-моему здесь стоит сказать что она обсуждается в приложении.}

\paragraph{Вопрос.}\label{Гудзон}
Река Гудзон течёт на юг.
В каком направлении сила Кориолиса действует на текущую в ней воду?

\paragraph{Ответ.}
Сила отклоняет воду на запад.
Действительно, представте порцию воды, движущуюся на юг вдоль меридиана.
Из-за вращения всё на Земле движется на восток, причём чем дальше от северного полюса, тем быстрее.
Поэтому, когда порция воды в Гудзоне удаляется от полюса, её скорость в восточном направлении возрастает.
Сопротивляясь этому возрастанию по инерции, порция воды будет прижиматься к западному берегу реки.
Соответственно, самой воде будет казаться, что какая-то сила толкает её на восток.
Объясняет ли это то, что западный берег Гудзона (напротив Манхэттена, со стороны Нью-Джерси) крутой, а восточный (манхэттенский) берег пологий?
Скорее всего, нет.

\section{Кориолис в самолёте}\label{Кориолис в самолёте}

\paragraph{Вопрос.}
Как велика сила Кориолиса, действующая на человека в реактивном самолёте (обычная скорость около $250$ м/с)?

\paragraph{Ответ.}
Чтобы упростить вычисления, будем считать, что самолёт летит над Северным полюсом.
В этом случае можно думать, что Земля это плоская платформа --- огромная карусель, вращающуюся вокруг своей оси.
Тогда пассажир испытывает силу Кориолиса%
\footnote{Эту формулу можно найти в упомянутых выше книгах. На странице ??? я «выведу» эту формулу без множителя 2 и предложу найти ошибку.}
\begin{equation}
F = 2 m \omega v,
\label{eq:7.1}
\end{equation}
где $m$ --- его масса,
$\omega$ --- угловая скорость вращения Земли,
а $v$ — скорость самолёта.
Округлим человека до $m = 70$ кг (извините за каламбур),
$\omega = \tfrac{2\pi}{24 \cdot 3600}\,\text{рад/с}$ и
$v = 250 \,\text{м/с}$.
Подставив всё это в \eqref{eq:7.1}, получим
$F/g \approx 240 \,\text{г}$ --- \emph{этой силы хватило бы чтобы поддерживать чашку с водой!}

Другой способ почувствовать величину этой силы — посмотреть на угол~$\theta$,
на который она отклонила бы подвешенный маятник.
Этот угол (в радианах) близок к отношению силы Кориолиса к весу $mg$:
\[
\theta \;\approx\; \tan\theta \;=\; \frac{2 \omega v}{g}.
\]

Получается примерно $1/600$??? радиана, или около
$0{,}1$°???.
Именно на такой угол самолёт теоретически должен накрениться, чтобы избежать бокового сноса.
Что это означает в терминах разницы высот между концами крыльев?
Примерно $1/600$??? от размаха крыла.
Размах крыла боинга 747 около $60$~м, а значит разница высот порядка $10$??? см.
Немного, но заметить можно.

\section{Кориолис в канализации}

\paragraph{Вопрос.}
Часто можно услышать, что из-за силы Кориолиса вода в северном полушарии стекает в слив по часовой стрелке, но правда ли это?

\paragraph{Ответ.}
Нет, не правда.
Сила Кориолиса действует и в унитазе и в ванне, но она ничтожно мала.
Эту силу непросто заметить даже в самолёте (см. страницу \pageref{Кориолис в самолёте}), и ещё труднее в воде, которая движется в тысячи раз медленнее.
Вода закручивается при сливе по другим причинам.
В некоторых унитазах, например, вода подаётся под углом, так что она уже закручена.
В ванне вращение может возникнуть из-за того, что воду взолтали и она обладает небольшой завихренностью%
\footnote{Завихренность определена на странице \pageref{def:завихренность}.}%
, которая становится заметной лишь при подходе к сливному отверстию.
Ещё одна причина, по которой вода может начать вращаться при сливе даже из состояния покоя, — это сочетание асимметрии ванны и вязкости воды.
\begin{figure}[ht!]
\centering
\begin{lpic}[t(2mm),b(2mm),r(0mm),l(0mm)]{pics/7.2}
\lbl[t]{18,17;{\footnotesize глубоко}}
\lbl[b]{18,1;{\footnotesize мелко}}
\lbl[lt]{5,9;{\footnotesize слив}}
\end{lpic}
\caption{Закручивание воды при сливе может быть вызвано асимметрией ванны в сочетании с вязкостью воды.}
\label{pic:7.2}
\end{figure}
На рисунке \ref{pic:7.2} приведён пример ванны, которая будет сливаться против часовой стрелки и в Бостоне, и в Буэнос-Айресе.

\section{Высокое давление и хорошая погода}

\paragraph{Вопрос.}
Массы воздуха высокого давления, называемые антициклонами.%
\footnote{Приствка \emph{анти-} указывает на вращение противположное вращению Земли.}
В северном полушарии они вращаются по часовой стрелке.
Почему же высокое давление сопровождается вращением по часовой стрелке?

\paragraph{Ответ.}
Это объясняется силой Кориолиса.
Представьте, что воздух сначала расходится от центра (рисунок \ref{pic:7.3}a).
Тода сила Кориолиса будет отклонять каждую частицу воздуха вправо от направления потока.
\footnote{Как и порцию воды в Гудзоне на странице \pageref{Гудзон}.}
Это приведёт к тому, что частицы будут отклоняться от радиального пути, закручиваясь по часовой стрелке.
Можно представить себе установившуюся вращение воздушных масс, при котором сила Кориолиса движущихся по кругу частиц как бы сдерживает высокое давление в центре антициклона, похоже на то как овчарки бегают вокруг стада овец, собирая его вместе (рисунок \ref{pic:7.3}).

\begin{figure}[ht!]
\centering
\begin{lpic}[t(7mm),b(8mm),r(0mm),l(0mm)]{pics/7.3}
\lbl[tl]{0,34;(a)}
\lbl[tl]{42,34;(b)}
\lbl[tl]{22,10.5;{\footnotesize $f$}}
\lbl[b]{24.5,14.5;{\footnotesize $v$}}
\lbl{13,13;\parbox{22mm}{\footnotesize\centering высокое\\давление}}
\lbl{62,13;\parbox{22mm}{\footnotesize\centering высокое\\давление}}
\lbl[b]{21,28;\parbox{22mm}{\footnotesize\centering движение\\наружу}}
\lbl[l]{28,20;\parbox{22mm}{\footnotesize сила\\Кориолиса}}
\lbl[l]{75,22;\parbox{22mm}{\footnotesize сила\\Кориолиса}}
\lbl[b]{55,27,25;\parbox{22mm}{\footnotesize\centering движение\\воздуха}}
\lbl[t]{13,-1;\parbox{42mm}{\footnotesize\centering высокое давление\\
запускает вращение\\
силой Кориолиса}}
\lbl[t]{62,-1;\parbox{42mm}{\footnotesize\centering
сила Кориолиса\\
удерживает\\
высокое давление}}
\end{lpic}
\caption{Как сила Кориолиса связывает высокое давление с антициклонами.}
\label{pic:7.3}
\end{figure}

\paragraph{Как антициклоны связаны с хорошей погодой?}
Благодаря «подушке» повышенного давления в центре воздух движется вниз, нагревается от сжатия%
\footnote{Награвание от сжатия объясняется на странице ???.}%
, и облака «растворяются».
В циклонах происходит противоположное: воздух поднимается, охлаждаясь при расширении, и влага конденсируется, образуя облака.

\section{Что вызывает пассаты?}

Пассаты образуют тропический пояс ветров, который постоянно дует с востока на запад.
Откуда они берутся?

Причина в сочетании атмосферной циркуляции и силы Кориолиса.
Ниже приведено сильно упрощённая схема, которая всё же даёт представление о том, что происходит.
\begin{enumerate}
\item Более холодный воздух течёт, с высоких широт в сторону экватора.
Это происходит в нижних слоях атмосферы и напоминает то как из открытой двери холодный воздух растекается по полу комнаты зимой.
\item Теперь вступает в игру сила Кориолиса.
Она отклоняет этот поток к западу, как показано на рисунке \ref{pic:7.4}.
\item Ближе к экватору, воздух нагревается, поднимается вверх и движется обратно к полюсям, но уже в верхних слоях атмосферы.
\end{enumerate}

\begin{figure}[ht!]
\centering
\begin{lpic}[t(2mm),b(2mm),r(0mm),l(0mm)]{pics/7.4}
\lbl[t]{19.5,42;\parbox{22mm}{\footnotesize\centering
холодней}}
\lbl[b]{19.5,18;\parbox{22mm}{\footnotesize\centering
теплей}}
\lbl[bl]{60,50;\parbox{22mm}{\footnotesize\centering
вращение}}
\lbl{19.5,29;\parbox{22mm}{\footnotesize\centering
циркуляция\\
без вращения\\
Земли}}
\lbl{63,29;\parbox{22mm}{\footnotesize\centering
Отклонение\\
из-за силы\\
Кориолиса}}
\end{lpic}
\caption{Пассаты возникают в результате действия силы Кориолиса на потоки воздуха.}
\label{pic:7.4}
\end{figure}

Атмосфера подобна двигателю на солнечной энергии.
Она питается энергией излучения Солнца, и возвращает её обратно в космос через излучение.
Небольшая часть солнечного тепла заставляет атмосферу двигаться, преодолевая трение.
Трение превращает эту энергию в тепло, которое и излучается наружу.
То есть солнечная энергия встряхивает атмосферу Земпли по пути от Солнца в космос.
Земля и всё, что на ней, включая нас с вами, подобна организму, который поглощает энергию от Солнца, и выделяет её в том же количестве, но другой форме, а точнее, в другой части спектра излучения.
