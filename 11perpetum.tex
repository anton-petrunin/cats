\chapter{Пара вечных двигателей}

Вечный двигатель — это утопическая мечта, а мечты тянут к себе чудаков.
К счастью, в отличие от многих социальных утопистов прошлого (да и настоящего), эти не опасны; они редко убивают ради идеи.
Общее для всех утопий — это попытка нарушить какой-нибудь закон, будь то закон сохранения энергии, закон экономики, закон человеческой психологии или закон общества.

Изобретателям вечного двигателя нужно обладать безграничным умом, ведь надо справиться с бесконечно трудной задачей — изобрести невозможное.
Такие люди бывают умны, но мудрых среди них немного.

Два вечных двигателя этой главы, — это головоломки, в каждой из которых требуется найти изъян.%
\footnote{Вскоре после Галилея разоблачать ложные теории физики  стало довольно безопасно.
В экономике это случилось гораздо позже.
В Советском Союзе, приблизительно в 1947 году, мой знакомый получил 12 лет за устные рассуждение в классе экономики о том, что стремление к прибыли может быть необходимым условием хорошо работающей экономики.}

\section{Вечный двигатель на капелярной тяге}\label{Вечный двигатель на капелярной тяге}

Вода поднимается в тонкой трубочке благодаря капиллярному эффекту, как показано на рисунке \ref{pic:11.1}.
Можно ли как-то использовать работу, поднимающейся воды?
она могла бы тянуть что-то за собой, но, к сожалению, этот двигатель останавливается, как только вода достигает определённой высоты.
Но вот способ обойти эту проблему.
\begin{figure}[ht!]
\centering
\begin{lpic}[t(2mm),b(2mm),r(0mm),l(0mm)]{pics/11.1}
\lbl[r]{6,25;\parbox{20mm}{\footnotesize\raggedleft сила\\капиляра}}
\lbl{43,17;\parbox{20mm}{\footnotesize\centering вид\\сверху}}
\lbl[b]{72.8,18.1,10;{\tiny $+--+$}}
\end{lpic}
\caption{Вечный двигатель на капелярной тяге}
\label{pic:11.1}
\end{figure}
Разместим поршень внутри тонкой замкнутой трубки и добавим каплю воды вплотную к поршню, как показано на рисунке \ref{pic:11.1}, так чтоб не было пузырьков между водой и поршнем.
Поршень может скользить практически без трения.
Положим трубку горизонтально на стол, чтобы не приходилось бороться с силой тяжести.
Теперь капиллярный эффект будет тянуть поршень по трубке, и нет ничто, что остановило бы его движение;
то есть, ему придётся двигаться бесконечно.
Чтобы всё работало, надо только добиться, чтобы сила трения поршня была меньше, тянущей его капиллярной силы.
Так мы получим бесконечный источник энергии, не требующий топлива, ведь при малом, но ненулевом трении выделяется тепло.

\paragraph{Вопрос.}
Если это не первый в мире работающий вечный двигатель, то в рассуждении должна быть ошибка, но где?
То есть где именно ошибка в моём «доказательстве», что поршень будет двигаться вечно?
(Она вовсе не в реализации поршня с малым трением.)

\paragraph{Ответ.}
Проблема не в отсутсвии хорошей смазки, она серьёзнее.
Сначала разберёмся, как вода поднимается в капиляре.
Это происходит по двум причинам:
(1) из-за поверхностного натяжения:
особенности расположения молекул воды заставляют её поверхность вести себя подобно натянутой резиновой плёнке;
(2) наличие электростатического притяжения воды к стенкам трубки.
Электростатическое притяжение притягивает воду к внутренней стенке трубки, из-за чего поверхность воды принимает вогнутую форму мениска.
Тут же включается поверхностное натяжение: стремясь выпрямить вогнутую поверхность, оно тянет за собой водяной столб.
Эти два процесса — растекание вдоль стенки и подтягивание — происходят одновременно, и вода поднимается.
В какой-то момент сила тяжести уравновесит капилярный эффект, и подъём прекратится.

Теперь вернёмся к нашему двигателю.
Тот же самый эффект, который поднимал воду в трубочке, здесь тоже присутствует: вогнутый мениск по-прежнему тянет воду, пытаясь увлечь за собой поршень.
Но что происходит рядом с поршнем?
Сейчас мы выясним, что там возникает противоположный эффект, сводящий на нет всю идею.
Из-за электростатического притяжения давление воды у стенок оказывается выше, и, в частности, оно выше возле поршня.
Это дополнительное давление толкает поршень в противоположном направлении.
Именно этот противодействующий эффект и был упущен.

\section{Вечный двигатель из эллиптического отражателя}

Эту задачу, я узнал от Петера Унгара в середине
1970-х годов.
Она основана на прекрасном свойстве эллиптических зеркал:
любой луч света, исходящий из одного фокуса эллипса,
после отражения%
\footnote{Предполагается, что поверхость отражает идеально.}
пройдёт через другой его фокус.
Следующий вечный двигатель использует это свойство.

\paragraph{Идея теплопередачи.}
Внутренняя поверхность эллипсоидальной оболочки (рисунок \ref{pic:11.2}) представляет собой идеальное зеркало.
\begin{figure}[ht!]
\centering
\begin{lpic}[t(2mm),b(2mm),r(0mm),l(0mm)]{pics/11.2}
\lbl[t]{29,42.5;\parbox{20mm}{\footnotesize\centering идеальное\\зеркало}}
\end{lpic}
\caption{Излучение шаров в фокусах идеально отражается эллипсоидным зеркалом.}
\label{pic:11.2}
\end{figure}
В двух фокусах помещены два шара разного радиуса, находящиеся при одной и той же температуре $T$.
Оба наших шара излучают, (как и любое тело с температурой выше абсолютного нуля).
Из двух шаров при одинаковой температуре больше энергии излучает больший (если предположить, что они сделаны из одного и того же материала и имеют одинаковый цвет).
Так как каждый луч, исходящий из одного фокуса, после отражения проходит через другой, все лучи от б\'{о}льшего шара попадают на меньший, и наоборот.
Отсюда следует, что при равной температуре большой шар отдаёт больше тепла, чем получает, и поэтому маленький будет нагреваться, а большой — охлаждаться.
Разность температур можно использовать чтобы приводить в действие двигатель%
\footnote{Например, она может вызывать движение воздуха, которое можно использовать для вращения колеса.}%
, так что наше устройство обеспечивает вечный источник энергии.

\paragraph{Вопрос.} Где ошибка?

\paragraph{Ответ.} Не всякий луч, исходящий от большего шара, попадёт на маленький.
Некоторые вернутся обратно к большому;
это относится, например, к лучам, выходящим из большого шара влево как на рисунке \ref{pic:11.2}.
Но ещё важнее то, что лучи исходят с поверхности тела во всех направлениях, а не только радиально, и такие нерадиальные лучи не проходят через фокусы.%
\footnote{Интенсивность излучения в данном направлении прямо пропорциональна элементу площади поверхности тела и косинусу угла между направлением и нормалью.
Более того, если бы нашёлся материал для которого интенсивность излучения описывалась другим законом, то вечный двигатель стал бы реальностью. \pr}
Это и разрушает исходное рассуждение.



