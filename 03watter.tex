\chapter{Во вращающейся воде}

Архимед открыл свой знаменитый закон: сила, с которой вода вытесняет погружённое в неё тело (сила Архимеда), равна весу вытесненной этим телом воды.%
\footnote{Следующий мысленный эксперимент объяснит, почему этот закон верен. Я хочу показать, почему на боулинговый шар, лежащий на дне бассейна, действует сила Архимеда, равная весу вытесненной им воды. В качестве мысленного эксперимента представим, что мы заменили шар на шар из воды той же формы. Этот водяной шар останется неподвижным, так как предполагается, что вода в бассейне находится в покое. Мы заключаем, что сила тяжести в точности уравновешивается силой Архимеда для водяного шара. То есть, по крайней мере для водяного шара, сила Архимеда равна весу вытесненной воды. Но эта сила зависит только от формы тела, а значит, будет такой же и для боулингового шара. Это и доказывает закон Архимеда.
Вкратце, закон сводится к двум вещам:
(1) неподвижная вода остаётся неподвижной пока нет внешнего воздействия;
(2) сила Архимеда, действующая на тело, зависит только от формы тела, а не от материала, из которого оно сделано.}

Однако во вращающемся мире, таком как Земля, закон Архимеда приобретает неожиданный поворот (без игры слов), порождая удивительные явления.
Один из них — парадокс с плавающей пробкой, который будет рассмотрен далее.
Схожий пример того же явления — парадокс айсберга (страница \pageref{sec:iceberg}).

\section{Плавающая пробка}\label{sec:cork}

\paragraph*{Эксперимент.}
Парк аттракционов с вращающимся бассейном --- мечта любого ребёнка, даже если все думают, что он уже дедушка.
С такими мыслями я готовил эксперимент для своей лекции по анализу, показывающий, что поверхность воды во вращающейся чаше принимает форму параболоидa.
Большая салатница с водой водружённая на проигрыватель, отлично для этого подхошла.
Я ждал с минуту пока вода не стала вращаться вместе с салатницей как единое целое со скоростью 33 оборота в минуту и увидел гладкую идеально параболическую поверхность воды.

Из чистого любопытства, я положил пробку на наклонную поверхность.
Я ожидал, что пробка останется на склоне, мне будет прикольно на это смотреть, а ещё лучше представить себя плавающим во вращающемся бассейне!
Однако пробка повела себя неожиданно:
она медленно поплыла вниз по поверхности параболоида и остановилась в самой нижней его точке.
«Возможно, это из-за сопротивления воздуха», — подумал я.
Чтобы это проверить, я накрыл салатницу прозрачной плёнкой.
Пробка плавала у самой стенки, я снова включил проигрыватель, и снова произошло то же самое!
Значит, воздух здесь ни при чём, ведь он входит во вращение даже быстрее, чем вода.
Затухание внутреннего движения в воздухе происходит быстрее, чем в воде.
Вязкость воздуха выше, чем у воды если её мерить относительно плотности.%
\footnote{Такая относительная вязкость называется кинематической.
Она определяется как отношение обычной (динамической) вязкости к плотности, что обычно записывается как $\nu=\mu/\rho$ ($\nu$ читается как английское «new»).
Джо Келлер, oбучавший меня гидродинамике в аспирантуре, сказал, что на приветствие «what’s new?» (что нового?), теперь можно отвечать: «mu over rho» (мю на ро).}

\paragraph*{Вопрос.} Почему же пробка плывёт вниз?

\paragraph*{Объяснение.}
Давайте перейдём в систему отсчёта, вращающейся вместе с салатницей.
На рисунке~\ref{pic:3.1} показана область воды $B$, ровно той же формы как вода вытесненная пробкой.
В нашей вращающейся системе область $B$ находится в покое.
Это означает, что центробежная сила%
\footnote{Центробежная сила обсуждается в приложении на странице ??? — это фиктивная сила, возникающая из-за того, что система отсчёта не инерциальна.}
уравновешивает горизонтальную составляющую силы Архимеда.

\begin{figure}[ht!]
\centering
\begin{lpic}[t(2mm),b(2mm),r(0mm),l(0mm)]{pics/3.1}
\lbl[b]{38,24,59.3;пробка}
\lbl[ur]{4,21;$B$}
\end{lpic}
\caption{Архимедовы силы для пробки (справа) и области $B$ (слева), одинаковы.
Но у пробки центробежная сила слабей, поскольку её центр ближе к оси.
Это и заставляет её двигаться.
}
\label{pic:3.1}
\end{figure}

Теперь представим, что область $B$ начинает разбухать, превращаясь в пробку, при этом сохраняя ту же подводную форму.
В этом процессе частицы в области $B$ будут постепенно приближаться к оси вращения.
Следовательно, его центробежная сила будет ослабевать, тогда как сила Архимеда останется прежней.
Это несоответствие приводит к тому, что пробку начинает толкать в сторону оси.%
\footnote{Попробуйте убедиться, что частицы области $B$ обязаны (в среднем) обязаны приблизиться к оси.
Для этото придётся воспользоваться двумя наблюдениями
(1) вертикальная составляющая силы Архимеда области $B$ уровновешивается весом воды в $B$ и
(2) то, что пробка однородна.
Пробка со смещённым центром тяжести, может наоборот будет плыть к краю;
см. раздел~\ref{sec:sails}.
\pr}

\section{Параболические зеркала и две кухонные головоломки}

\paragraph*{Параболические зеркала.}
Зеркала телескопов имеют форму параболоидов (параболоид — это поверхность, образуемая вращающейся вокруг своей оси симметрии параболой, как показано на рисунке 3.1).
Параболоиды столь полезны потому, что собирают пучок лучей, идущих параллельно оси, в одну точку.%
\footnote{По этой же причине тарелки микрофонов для подслушки, спутниковые и другие антенны также имеют форму параболоидов.
Датчик размещается в фокусе параболоида, чтобы собирать все «лучи», отражающиеся от антенны.}
Среди всех поверхностей только параболоиды обладают этим фокусирующим свойством.
И вот по замечательному совпадению природа предоставляет простой способ получить параболоидную форму: поверхность вращающейся жидкости, как на рисунке 3.1, автоматически принимает форму параболоида.

Если дать расплавленному стеклу в вращающемся контейнере медленно остыть, получится отличный параболоид — без всякой механической обработки.
В случае с параболоидом природа выступает одновременно и как аналоговый компьютер, и как токарный станок.

\paragraph*{Съедобная головоломка.}
Налейте жидкий желатин в миску, поставьте её на проигрыватель и крутитите его пока желатин не застынет. Поверхность застывшего желатина будет представлять собой аккуратный параболоид.
Ваши друзья могут озадачиться (а возможно, и немного встревожиться), решив, что вы провели часы, кропотливо вырезая углубление до такой невероятной гладкости.
Позже можно будет их угостить, наполнив параболоид взбитыми сливками.

\paragraph*{Кулинарное приложение теоремы Тейлора --- Праудмена.}
А вот способ сделать интересные цветные узоры в желатине.
Поставьте стакан с жидким желатином на проигрыватель, дайте ему несколько секунд, чтобы он стал вращаться вместе со столом, и влейте туда столовую ложку или две желатина другого цвета.
Эти две жидкости смешаются неожиданным образом: добавленный желатин сформирует завесу в виде закрученного рулона.
Подождите, пока этот удивительный узор застынет.
Спросите свох друзей как такое сделать
(будем надеятся, что от этого у вас не убавится друзей).
Это может их озадачить, хотя скорее всего, они догадаются, что здесь не обошлось без вращения.

Это необычное перемешивание происходит из-за гироскопического эффекта в жидкостях.
Явление описывается теоремой Тейлора --- Праудмена, которая, грубо говоря, утверждает, что быстро вращающаяся жидкость приобретает направленную «жёсткость» и ведёт себя так, будто состоит из «зубочисток», параллельных оси вращения.
Чем быстрее вращение (по сравнению с внутренним движением жидкости), тем больше эта жёсткость.

Этот эффект играет важную роль в движении атмосферы и океанов.
Подробное обсуждение (интуитивно понятная часть которого не требует знания математического анализа) можно найти в классической книге Дж. Бэтчелора «Введение в динамику жидкостей».
%??? а картинки есть???

\section{Параболическая тарелка}

\parbf{Задача.}
Вообразите сосуд с водой, стоящий на плавно вращающейся платформе (см. рисунок 3.2).
\begin{figure}[ht!]
\centering
\begin{lpic}[t(2mm),b(2mm),r(0mm),l(0mm)]{pics/3.2}
\lbl[t]{32.5,8;$x$}
\lbl[tl]{2,6;$S=$ поперечная площадь}
\lbl[l]{42.8,19;$y$}
\lbl[l]{26,40;$\omega$}
\lbl[tr]{22.5,10;$A$}
\lbl[tl]{42.5,10;$B$}
\lbl[bl]{42.5,28;$C$}
\lbl[b]{35,12;$F_c$}
\end{lpic}
\caption{Почему поверхность принимает параболическую форму?
}
\label{pic:3.2}
\end{figure}
Её поверхность принимает форму параболоида — как показано на рисунке.
Если разсечь эту поверхность вертикальной плоскостью, проходящей через ось вращения, то получится парабола \(y = kx^2\).
Её крутизна определяется параметром \(k\), который можно выразить через угловую скорость \(\omega\) и ускорение свободного падения \(g\):
\begin{equation}
k = \frac{\omega^2}{2g}.
\label{eq:3.1}
\end{equation}
На Луне, где \(g_{\text{лун}} \approx g/6\), этот параболоид будет в шесть раз куче.
А на Юпитере раза в 2{,}5 более пологим.
Если же сменить скорость вращения с 33 до 78 оборотов в минуту — чуть больше, чем в два раза — то \(k\) вырастет почти в шесть раз.
Можно добиться того же же, что на Луне, но дешевле.

Почему же вода выбирает форму параболоида из бесконечного множества возможных форм?
Вот простое объяснение без использования анализа.
Единственное, что нужно знать, — это то, что центробежная сила, действующая на материальную точку массой \(m\), вращающуюся по окружности радиуса \(r\), равна:
\begin{equation}
m\omega^2 r,
\label{eq:3.2}
\end{equation}
всё это объясняется в приложении на странице ???.

Задача — найти глубину на любом расстоянии \(x\) от оси вращения (см. рисунок 3.2). Вот простой ключ к решению.

Центробежная сила, действующая на горизонтальный столбик воды \(AB\), создаёт дополнительное давление в точке \(B\).
Это давление поддерживает вертикальный столб воды \(BC\).
То есть:
\[
p_{AB} = p_{BC}.
\label{eq:3.3}
\]

С одной стороны, давление \(p_{AB}\) в точке \(B\) равно центробежной силе \(F_c\), действующую на трубку \(AB\), делённую на её поперечную площадь \(S\):
$p_{AB} = F_c/S$.
Центробежная сила равна \(F_c = m\omega^2 r\); это объясняется на странице ???.
Здесь масса трубки: $m = \text{плотность} \cdot \text{объём}\z= \rho x S$,
а расстояние от центра масс трубки до оси равно \(r = \frac{x}{2}\).
Значит
\[
p_{AB} = \frac{F_{AB}}S= \frac{\rho x S \cdot \omega^2 \cdot x/2}S
= \frac{1}{2} \rho \omega^2 x^2.
\]
С другой стороны, давление от вертикального столба $BC$:
\[
p_{BC} = \rho g y,
\]
где \(y\) — глубина.
Подставив всё это в уравнение \ref{eq:3.3}, получим
\[
\frac{1}{2} \rho \omega^2 x^2 = \rho g y,
\quad\text{или}\quad
y = \frac{\omega^2 x^2}{2g},
\]
что и требовалось.

Заметим, что в формуле плотность \(\rho\) сократилась, то есть на форму поверхности влияют только \(\omega\) и \(g\).
Значит, при прочих равных, у ртути будет та же поверхность что и у воды.
Так что при изготовлении зеркала телескопа, вращением расплавленного материала, о плотности можно не беспокоиться — одной заботой меньше.

\section{Моторная лодка на склоне}

\paragraph*{Вопрос.}
Представьте снова воду, вращающуюся в миске, как на рисунке 3.3.
Игрушечная лодка с дистанционным управлением плавает на поверхности.
Оператор хочет, чтобы лодка оставалась в фиксированной точке относительно земли в стороне от центра, как показано на рисунке.
В каком направлении он должен повернуть нос лодки?
И в какую сторону он должен ею управлять: прямо, вправо или влево?

\begin{figure}[ht!]
\centering
\begin{lpic}[t(2mm),b(2mm),r(0mm),l(0mm)]{pics/3.3}
\lbl[b]{36,19;$1$}
\lbl[b]{33,25;$2$}
\lbl[b]{27,28;$3$}
\end{lpic}
\caption{Куда должна направляться лодка во вращающейся воде, чтобы оставаться в неподвижном положении относительно земли?
}
\label{pic:3.3}
\end{figure}

\paragraph*{Ответ.}
Нос лодки должен быть указывать в направлении $2$.
Действительно, лодке необходимо иметь скорость, противоположную направлению потока, а также некоторую тягу от центра, чтобы не скатываться вниз по склону: центробежной силы, удерживающей лодку на склоне, уже нет, ведь лодка покоится относительно земли.

\section{Без вёсел и парусов}\label{sec:sails}

\paragraph*{Вопрос.} Вы плывёте на лодке во вращающемся бассейне.
Сможете ли вы управлять движением лодки без вёсел, гребного винта и парусов?
Сопротивлением воздуха можно пренебречь. %??? паруса не рефмуются с тем, что воздухом можно пренебречь.

\paragraph*{Ответ.}
Чтобы двигаться к центру, следует \emph{встать} во весь рост.
Однако «вверх» во вращающейся системе отсчёта не то же самое, что «вверх» относительно земли:
вставая, вам придётся придвинуться к оси.
Центробежная сила при этом уменьшается, и лодка может начать дрейфовать к оси.
Чтобы удалиться от оси, надо лечь на дно лодки, максимально увеличив расстояние до оси, тем самым усилив дрейф к краю.

\paragraph*{Вопрос.}
Моя плавучесть чуть выше нуля. Смогу ли я плавать во вращающемся бассейне?

\paragraph*{Ответ.}
Не обязательно.
Ноги имеют большую плотность, чем грудная клетка, из-за воздуха в лёгких, поэтому я обычно держусь вертикально в воде (педпочитаю держать голову сверху).
Теперь представьте, что я нахожусь ближе к краю вращающегося бассейна, где поверхность воды имеет крутой уклон. Если моё тело перпендикулярно поверхности, то оно лежит почти горизонтально.
В этом положении мои ноги окажутся дальше от оси, и центробежная сила, потянет их вниз;%
\footnote{Моя голова будет ближе к оси вращения, чем остальное тело --- я буквально почувствую лёгкость в голове и тяжесть в ногах.}
это может превзойти мою плавучесть и я утону.

Однако есть несколько способов спастись.
Во-первых, можно попробовать держаться на воде так, чтобы ноги были ближе к поверхности и направлены вниз по склону.
А ещё можно добраться до стенки басейна, опуститься до дна и ползти по дну к центру (стараясь двигаться ногами вперёд).
В центре центробежная сила исчезнет, и можно всплывать.

\section{Айсберг}\label{sec:iceberg}

\paragraph*{Вопрос.} Чувствуют ли айсберги вращение Земли?
(Разумеется, на айсберги влияют течения и ветры, которые сами по себе зависят от вращения Земли, но я спрашиваю о более прямом воздействии.)

\paragraph*{Ответ.}
Вращение Земли создаёт силу, тянущую айсберги к экватору.
Это уже объяснялось в задаче о плавающей пробке (страница \pageref{sec:cork}),
но я повторю его для айсберга.
Представьте, что айсберг ни откуда не приплыл, а намёрз из воды, которую он и вытеснил.
Вода расширяется, превращаясь в айсберг, часть которого поднимается над поверхностью.
Это увеличивает среднее расстояние этой воды до оси вращения Земли.
Поскольку центробежная сила возрастает с удалением от оси, айсберг испытывает б\'{о}льшую центробежную силу — в направлении от оси!
Эта сила и тянет айсберг к экватору, как показано на рисунке \ref{pic:3.4}.

\begin{figure}[ht!]
\centering
\begin{lpic}[t(2mm),b(2mm),r(0mm),l(0mm)]{pics/3.4}
\lbl[t]{12.5,21,-14;{\footnotesize экватор}}
\end{lpic}
\caption{Из-за вращения Земли айсберги тянуться к экватору.}
\label{pic:3.4}
\end{figure}

А насколько эта сила велика?
Грубая оценка, которой я не хочу вас утомлять (а может, и раздражать), показывает, что для айсберга протяжённостью около \(\approx 10\) км и толщиной \(0{,}2\) км скорость устойчивого дрейфа, вызываемого этой силой, будет порядка \(1\) м/с:
при этой скорости сопротивление уравновешивает центробежную силу.

Метр в секунду — это скорость пешехода, и она не такая уж малая даже по сравнению со скоростью некоторых океанических течений.
Однако есть одно «но»: чтобы достичь этой скорости, начиная с нуля, потребуется примерно год — настолько мало ускорение.
С другой стороны, некоторые айсберги не тают год и два, так что времени вполне достаточно.
Действительно, если принять среднюю скорость за \(0{,}5\) м/с, то за 1 год $\approx 3{,}15\cdot10^7\approx  \pi\cdot10^7$%
\footnote{Это забавное приближение $1$ год $\approx\pi\cdot10^7$ секунд я узнал от Тадаси Токиэды.}
секунд мы проделаем расстояние порядка $(\pi/2)\cdot10^7$ м, или около $15\,000$ км — примерно от полюса до экватора!
Это меня поразило.
Разумеется, я пренебрегал гораздо более сильным влиянием ветров и течений,
но центробежный эффект хоть и слаб, но действует постоянно;
а ветры и течения --- сильны, но переменчивы.
Неясно (по крайней мере мне), приводит ли эта слабая сила к тому, что айсберги действительно дрейфуют в сторону экватора.%
\footnote{Чтобы разобраться в этом вопросе, нужно больше информации о течениях и ветрах.
Теоретически можно представить как течения, при которых слабый дрейф не оказывает существенного влияния,
так и течения при которых влияние дрейфа огромно.
При наличии более полной информации о течениях и ветрах, сделав некоторые естественные допущения
задача может быть решена использованием теории динамических.
Компьютерный эксперимент может сильно помочь, особенно в случае течений, с которыми теоретические методы не справляются.
Течения океанов, вероятно, как раз относятся к такому типу.}
