\documentclass[twoside]{article}
\usepackage{bm}


\begin{document}
\pagestyle{empty}

\section*{Questions and corrections to\\
``Why cats land on their feet, and 76 other physical paradoxes
and puzzles''}

\today

\

Mark, could you write ``approved'' OR ``not approved'' after each proposed change.
(We did not start appendix.)

\

\noindent
\textbf{?!} = we changed it, but want an approval;\\
\textbf{??} = we did not change it, but would like to;\\
\textbf{!!} = mistake --- change has been made;\\
\textbf{!?} = mistake --- change has not been made;\\
$>\!>\!>$ = change to.

\

\subsection*{Chapter 1}

\paragraph{?? page 2, lines 5--6:} Do you want to explain why “some ‘mistakes’ can be beneficial, at least temporarily”?
\vskip 0.1 in {\it  Mistakes can teach us what not to do next time. } 

\paragraph{?? page 2, line 10--...:} There are some repetitions here; perhaps you could make it a bit shorter.
\vskip 0.1 in {\it  probably true, i will ask you for more specifics.} 

\paragraph{?! page 3, footnote 3:} “I am referring to” $>\!>\!>$ “For example,”.
\vskip 0.1 in {\it I agree. "For example, in the sling problem on page 93, where the rock reaches infinite speed after one second, some calculus is used." Please feel free to edit.} 

\paragraph{?! page 4, line 6:} “Literature” $>\!>\!>$ “Library”.
\vskip 0.1 in {\it  I think "literature" is right, but correct me if you still don't think so.}

\subsection*{Chapter 2}

\paragraph{?? page 10, Figure 2.5:} Figure 2.5 should appear after Figure 2.6.
\vskip 0.1 in {\it  Agreed.
} 
\paragraph{?? page 10, lines $\bm{(-14)}$--$\bm{(-13)}$:} It seems that you refer to the caption of Figure 2.5 — the picture itself is not directly related. I would suggest copying the text from the caption and removing the reference to the figure.
\vskip 0.1 in {\it  Agreed, thanks!} 

\paragraph{?! page 11, footnote 3:} The phrase “increasing kinetic energy” might be misleading in this context — I would say “pumping energy into the system” instead.
\vskip 0.1 in {\it Agree, this is better. } 

\paragraph{?? page 13, Figure 2.7:} It seems that parts (a) and (b) of the figure appear in the wrong order, and the labels (a) and (b) are missing.
\vskip 0.1 in {\it  You are exactly right. } 

\paragraph{?! page 14, Problem 2.4:} Nearly the same problem appears in Makovetsky’s book (Problem 22).
\vskip 0.1 in {\it I just looked-you are right! Please feel free to mention it in the translation, of course. } 

\paragraph{?! page 16, lines 1--2:} $G$ was defined a couple of lines above.
\vskip 0.1 in {\it Agreed, should be deleted. } 
\subsection*{Chapter 3}

\paragraph{?? page 20, A culinary application of Taylor--Proudman’s theorem:} Do you have an illustration for this part? 
\vskip 0.1 in {\it   I have a photo but its blurry. It looks much sharper when you watch it; the camera was not very good, the exposure was too long. I did it about 20 years ago when cameras were not as good. If you think it's worth it I can repeat the experiment quite easily.  } 

\paragraph{?! page 21, sec. 3.3, line 1:} I would move the word “Problem.” to line 10 on the next page, and mark the next sentence with “Solution.”
\vskip 0.1 in {\it Much better. } 

\paragraph{?! page 24, Answer:} Should you add a comment that it is not always possible?
\vskip 0.1 in {\it  I think so, otherwise someone may think of being at the very center when it won't work. } 

\subsection*{Chapter 4}

\paragraph{?! page 29, line 10:} “as Figure 4.1 illustrates,” but it does not illustrate anything.
\vskip 0.1 in {\it  Agreed, probably drop the ref to the figure.} 

\paragraph{?? page 32, sec. 4.3:} This section does not seem to belong in this chapter.
\vskip 0.1 in {\it  Agreed. Maybe chapter 6? Or 14?} 

\paragraph{?! page 32, sec. 4.3:} In this section, the text switches from “you” to “I” — I would use “I” consistently thruout.
\vskip 0.1 in {\it Agree } 

\subsection*{Chapter 5}

\paragraph{?? page 39, sec. 5.1, line 5:} “nonviscous water” — do you want to say something about viscosity?
\vskip 0.1 in {\it Sure, maybe in a footnote? Something like "Viscosity of a fluid is, roughly speaking, internal friction: to pump honey through a pipe is harder than it is to pump water through the same pipe because honey's viscosity is higher. Ideal fluid has zero viscosity. Here we neglect water's viscosity."} 

\paragraph{?! page 41, line 7:} Remove “and $A$ is the area of the piston’s face” — it is already defined and can be omitted.
\vskip 0.1 in {\it  Agreed} 

\paragraph{?! page 41, lines 10--11:} Remove “where $a$ is the cross-sectional area of the exit tube” — it is already defined and can be omitted.
\vskip 0.1 in {\it  Agreed} 

\paragraph{?! page 41, formula (5.2):} A square and a period are missing; it should be
\[
F = \frac{1}{2} \rho A v^{2} \left[ \left( \frac{A}{a} \right)^{2} - 1 \right].
\]
\vskip 0.1 in {\it  Agreed} 

\paragraph{?? page 42, Figure 5.3:} I don’t think it works with air — the man is also under atmospheric pressure.
\vskip 0.1 in {\it As the piston moves to the right, total kinetic energy of the air decreases (since faster parts slow down), and to remove this kinetic energy the guy must pull back. Unless I am making a mistake somewhere. We can discuss this.  Maybe we can discuss this in person?} 


\paragraph{?! page 42, sec. 5.2, line 4:} “gravity plays” $>\!>\!>$ “gravity and viscosity play”.
\vskip 0.1 in {\it  Agreed} 

\paragraph{?! page 47, line 5:} “because the arm is a semicircle” — I would write the following:

“Indeed, the radial component of this velocity is the same as in the reference frame rotating together with the sprinkler.
And in this frame, the water jet must be directed tangentially to the tube, that is, perpendicular to the direction toward $P$; in other words, it has zero radial component.”
\vskip 0.1 in {\it  Agreed} 

\paragraph{!? page 47, Figure 5.9:} This figure is misleading — since the sprinkler rotates, the water will look more like in Figure 5.8; viewed from above, it should form an Archimedean spiral.
\vskip 0.1 in {\it  I agree it's misleading. It should be adjusted, maybe we can discuss how. One possibility is to draw a small piece of Archimedean spiral but indicate that all points have radial velocities?} 


\paragraph{!! page 52, Figure 5.12:} The picture on the left does not agree with the picture on the right — the horizontal bar in the second “+” should not be that horizontal.

\paragraph{?! page 53, Figure 5.14:} Add labels (a) and (b).
\vskip 0.1 in {\it  Agreed} 

\paragraph{?! page 53, line $\bm{(-2)}$:} “simple” — did you mean to say that “it has a fixed velocity field”?
\vskip 0.1 in {\it  Yes,  this is better.} 

\paragraph{!! page 54, line 3:} In the figure, the fixed point is marked by $S$, but here it is referred to as $P$.
\vskip 0.1 in {\it  Thanks. This of course should be changed.} 

\paragraph{?? page 54, line 3:} Do you want to comment on where exactly this point lies on the segment?
\vskip 0.1 in {\it  I actually don't know. It is the minimum of a solution of Laplace's equation $ \Delta u = \omega $ with Dirichlet boundary conditions.} 

\paragraph{?? page 54, sec. 5.8:} I would add some references here.
\vskip 0.1 in {\it  Do you mean about breakup of jets? I can find some. } 

\paragraph{?? page 54, Figure 5.15:} The droplets must have a larger diameter than the tube!
\vskip 0.1 in {\it  Agreed! An embarrassing oversight.} 

\paragraph{?? page 55, Figure 5.16:} From the picture, one might incorrectly conclude that the front lines are straight.
\vskip 0.1 in {\it  Agreed. The easiest thing is probably to make a comment that these are only schematic. } 

\subsection*{Chapter 6}

\paragraph{?? page 59, Figure 6.2:} The man does not look like stone.
\vskip 0.1 in {\it  Maybe the caption should be something like "The further one drops, the less speed one gains per meter traveled."} 

\paragraph{?! page 60, Figure 6.3:} Reflect (a) and (c) so that they have the same orientation as (b).
\vskip 0.1 in {\it  Agreed} 

\paragraph{!! page 65, Figure 6.5:} The second bike from the left leans to the wrong side.
\vskip 0.1 in {\it  Agreed} 

\paragraph{?! page 64, sec. 6.5, line 12:} Add a footnote after “counterturn.”:

“This may depend on your bicycle riding style. Try turning by moving only the handlebars, not your body.”
\vskip 0.1 in {\it  Agreed} 

\paragraph{?! page 65, line $\bm{(-2)}$:} I would add a footnote: “For the same reason as in 6.2.”
\vskip 0.1 in {\it  Agreed} 

\paragraph{?! page 66, lines 7--8:} You are cheating here. I would write the following:

“Cancelling $m$, we obtain
\[
V^2 - v^2 = 2gh,
\]
and after a few algebraic manipulations,
\[
V - v = \frac{2gh}{v + V} = \frac{2gh}{v + \sqrt{2gh + v^2}}.
\]
In particular, as the initial velocity $v$ increases, its increment $V - v$ decreases.”
\vskip 0.1 in {\it  Agreed} 

\paragraph{?! page 72, sec. 6.11, line 1:} “top.” $>\!>\!>$ “top (Figure 6.2).”
\vskip 0.1 in {\it  Sorry, I don't understand - can we discuss?} 

\paragraph{!! page 75, Figure 6.9:} Change $v$ to $V$ once and $v_1$ to $V_1$ twice.
\vskip 0.1 in {\it  Agreed} 

\paragraph{?! page 76, line 8:} $\Delta K$ $>\!>\!>$ $\Delta K_{\mathrm{total}}$.
\vskip 0.1 in {\it  Agreed} 

\subsection*{Chapter 7}

\paragraph{!! page 79, sec. 7.2, line 9 in Answer:}
“$F_{\mathrm{coriolis}}\approx 240$ g” $>\!>\!>$
“$F_{\mathrm{coriolis}}/g\approx 240$ g” or, better, “$F_{\mathrm{coriolis}}/g\approx 0.24$ kg” (so $g \ne$ g).
\vskip 0.1 in {\it  Agreed. } 

\paragraph{?! page 79, sec. 7.2, line 11 in Answer:} Do you want to move “This force can hold up a cup of water!” to the previous paragraph?
\vskip 0.1 in {\it  Good idea.} 

\paragraph{!! page 79, line $\bm{(-3)}$:} Instead of $1/600$, it should be $1/300$, and $0{.}1$° should be $0{.}2$°, and 10 cm as well... (But maybe I’m mad.)
\vskip 0.1 in {\it  Agreed (not with the mad part, the mad one was me.)} 


\paragraph{?! page 81, Answer, lines 2--3:} “Each particle...” $>\!>\!>$ “Assuming that we are in the Northern Hemisphere, each particle...”.
\vskip 0.1 in {\it  Agreed} 

\paragraph{?? page 82, lines 1--3:} “With the ‘pillow’ of higher pressure in the center, the air moves downward...” — is it because of higher density? If so, should you mention that? 

\vskip 0.1 in {\it  Maybe something along the lines "The air spiralling inward in upper layers of the atmosphere ``piles up" in the center and causes the air to sink, and to thus to compress; compression warms the air and causes the clouds to melt." ? } 

\subsection*{Chapter 8}

\paragraph{?? page 84, line $\bm{(-6)}$:} “And a lighter plane uses less fuel.” Maybe this needs some explanation — most of the fuel is used to gain speed, altitude, and to overcome air resistance. So do we care only about the very first part of the flight?
\vskip 0.1 in {\it  Maybe adding something along the lines "A heavier plane consumes more fuel even in level flight. This is because the plane holds itself aloft   by pushing down the air by keeping positive angle of attack of its wings. And the heavier the plane is, the more air it has to accelerate downward; it does so by increasing the angle of attack. This in turn requires a greater engine thrust."  }?  

\paragraph{?! page 86, line $\bm{(-4)}$:} “The velocity shown in Figure 8.1 is wrong.” The picture actually shows the tangential component of velocity correctly. Maybe write instead: “Figure 8.1 does not show everything”?
\vskip 0.1 in {\it  Yes, much better. } 

\paragraph{?! page 87, further in the text:} “component $v \sin(\omega \,\Delta t) \approx \omega v \,\Delta t$” $>\!>\!>$ “term $v \sin(\omega \,\Delta t) \approx \omega v \,\Delta t$ in the expression for the change in velocity perpendicular to the \emph{initial} radius.”
\vskip 0.1 in {\it  Agreed} 

\paragraph{!! page 87, Figure 8.2:} $v\omega\Delta t\cos(\omega t)$ $>\!>\!>$ $v\omega\Delta t\cos(\omega \Delta t)$.
\vskip 0.1 in {\it  Agreed} 

\paragraph{?! page 88, line 18 (right above Question):} “in (b)” $>\!>\!>$ “in Figure 8.3b”.
\vskip 0.1 in {\it  Agreed} 

\paragraph{!! page 95, line 1:} “nonzero” $>\!>\!>$ “positive”.
\vskip 0.1 in {\it  Agreed} 

\paragraph{?! page 95, line 5:} “$a$” $>\!>\!>$ “$a_{\mathrm{tan}}$”.
\vskip 0.1 in {\it  Probably better;  then $a$  should also be replaced in the rest of page 95: lines 5, 13 and (8.4). } 

\paragraph{?! page 95, line 13:} “$a=a_c$” $>\!>\!>$ “$a_{\mathrm{tan}} = a_{\mathrm{central}}$”; and further,
“$a_c$” $>\!>\!>$ “$a_{\mathrm{central}}$”. (This is more consistent with the other notations.)
\vskip 0.1 in {\it  Agreed (addressed in preceding comment).  } 

\paragraph{!! page 100, line 12:} “... and $r$ ...” $>\!>\!>$ “..., $u$ is the velocity of the center of mass, and $r$ ...”.
\vskip 0.1 in {\it  Agreed} 

\paragraph{!! page 102, Figure 8.11:} “$r(s)$” $>\!>\!>$ “$r(s,t)$”,
and “$r(s+ds)$” $>\!>\!>$ “$r(s+ds,t)$”.
\vskip 0.1 in {\it  Agreed} 


\paragraph{?? page 102, line $\bm{(-2)}$:} Do you want to change “$ds$” $>\!>\!>$ “$\Delta s$”?
\vskip 0.1 in {\it I think I agree, although it would require changing a few more places on pp. 102-103.} 

\subsection*{Chapter 9}

\paragraph{?? page 104, line $\bm{(-4)}$:} Do you want to define precession?
\vskip 0.1 in {\it  Something like this, perhaps?: ``...precession: the wheel will remain nearly vertical while slowly turning around the vertical axis as illustrated in Figure 9.2" } 

\paragraph{?! page 105, Figure 9.1 (left):} Move the arrow around the left rope downward,
and either extend the right rope upward or remove the left rope above the upper bar.
\vskip 0.1 in {\it  Agreed} 

\paragraph{!! page 105, Figure 9.2 (left):} Label the torque axis as $L$ — it is referred to as $L$ on the next page, line 7.
\vskip 0.1 in {\it  Agreed} 

\paragraph{!! page 106, Problem:} The formulation is unclear — do you mean that the end moves at constant speed, or that it is known to move at constant speed? I would write the following:

“Assume I push the end $A$ of the axis of a spinning top so that it moves at constant speed. In which direction do I push the end?”
\vskip 0.1 in {\it  Very good.} 


\paragraph{!! page 105, Figure 9.3a:} Missing label $A$.
\vskip 0.1 in {\it  Agreed} 

\paragraph{?! page 107, Stability by deflection, line 5:} Add the following footnote after “falling initially”:

“At that moment it accelerates; therefore, the solution of the previous problem cannot be applied.”
\vskip 0.1 in {\it  Agreed} 

\paragraph{?! page 108, Figure 9.4b:} “frontal view” $>\!>\!>$ “frontal view with subtracted shift”.
\vskip 0.1 in {\it  Agreed} 

\paragraph{?! page 115, Problem:} Remove “with the fluid” — above you mentioned that it might be mounted on gimbals.
\vskip 0.1 in {\it  Agreed} 

\subsection*{Chapter 10}

\paragraph{?! page 118, caption in Figure 10.1:} Replace ``$1/(1+1/N)$'' $>\!>\!>$ ``$1+1/N$''. (I think ``decrease by a factor of 2'' means ``divided by 2''.)
\vskip 0.1 in {\it  Agreed} 

\paragraph{!! page 118, line $\bm{(-6)}$:} ``perfect internal temperature'' $>\!>\!>$ ``recommended internal temperature''. (This temperature is recommended by the FDA, but the optimal temperature may be lower.)
\vskip 0.1 in {\it  Agreed} 

\paragraph{!! page 118, line $\bm{(-3)}$:} ``best temperature'' $>\!>\!>$ ``best water temperature''.
\vskip 0.1 in {\it  Agreed} 

\paragraph{!! page 119, line $\bm{(-3)}$:} ``$N/(N+1)=1/(1+1/N)$'' $>\!>\!>$ ``$(N+1)/N = 1 + 1/N$''.
\vskip 0.1 in {\it    (you meant line 9) - agreed.} 

\paragraph{?! page 120, lines 1--2:} ``here is another one'' $>\!>\!>$ ``let me remind''.
\vskip 0.1 in {\it  Sorry, I didn't get it, let's discuss it? } 

\paragraph{!! page 120, Figure 10.2:} This figure is completely wrong!
\vskip 0.1 in {\it  You are right, one strip must be flipped left to right, including labels. Basically top bar can be copied with only the direction of the flow reversed. And shading should be more gradual to reflect linear change.} 


\paragraph{?! page 121, sec. 10.2, line 2:} ``or to'' $>\!>\!>$ ``only, or there is''.
\vskip 0.1 in {\it  Much better.} 

\paragraph{?! page 123, Answer:} Remove the first and the last sentences.
\vskip 0.1 in {\it  Agreed} 

\paragraph{?! page 123, Answer, line 16:} Start a new line before ``In actual''.
\vskip 0.1 in {\it  Agreed} 

\paragraph{?! page 124, sec. 10.4:} The title is not relevant to the problem. Maybe ``Two Rooms''?
\vskip 0.1 in {\it  Agreed} 

\paragraph{?! page 124, footnote 3:} Don’t you want to state that the air is assumed to be an ideal gas?
\vskip 0.1 in {\it  Agreed. Maybe, as you said, adding a sentence ``Assume the air to be an ideal gas."} 

\paragraph{!! pages 125--126:} The numbers in the answer and the calculations do not agree; the numbers given in the answer are incorrect.
\vskip 0.1 in {\it  Agreed} 


\subsection*{Chapter 11}

\paragraph{?! page 128, Question:} There are too many repetitions.
\vskip 0.1 in {\it  Agree completely. Maybe just "Where is the mistake in my ``proof" that the piston will circulate indefinitely? } 

\paragraph{!! page 131, Answer, line 4:} ``Furthermore'' $>\!>\!>$ ``More importantly''. (Otherwise, the reader might think that by fixing the first issue, they could build a perpetual motion machine.)
\vskip 0.1 in {\it  Great point, thank you.} 

\paragraph{?! page 131, Answer, line 6:} Add the following footnote after ``the foci.'':

``The intensity of radiation in a given direction is directly proportional to the surface area element of the body and to the cosine of the angle between the direction and the normal. Moreover, if a material were found for which the intensity of radiation followed a different law, a perpetual motion machine could become a reality.''
\vskip 0.1 in {\it  Let's discuss this. I know what you mean, but it may not be understood the way you mean it.} 


\subsection*{Chapter 12}

\paragraph{?! page 133, Figure 12.1:} The triangle between the sail and the taut line looks like a sail, which might be misleading. Also, I would align the arrow labeled ``direction of sailing'' with the sail.
\vskip 0.1 in {\it  Agreed} 

\paragraph{?! page 135, Figure 12.3:} Missing period in the caption.
\vskip 0.1 in {\it  Agreed} 

\paragraph{?! page 135, footnote 3:} Replace the dead link with the following reference:

``Wind-propeller sails proposed for liners,'' Modern Mechanix and Inventions, January 1935, p. 49.
\vskip 0.1 in {\it  Agreed} 

\paragraph{?! page 137, footnote 4:} This might be misleading — the windmill produces energy, but it cannot be perfect.
\vskip 0.1 in {\it  Maybe just deleting the footnote? It's not really doing much.} 

\subsection*{Chapter 14}

\paragraph{?! page 151, sec. 14.4:} It would be more appropriate to restate the introductory problem as ``man on a platform'' — the boat is too complicated, and the solution given in the footnote does not work for the boat. For the platform, it seems acceptable; moreover, it may be reasonable to assume that the drag force of the air is proportional to the velocity.
\vskip 0.1 in {\it  Agreed} 

\paragraph{?! page 151, footnote 5:} In case you want to kee the boat: ``displacement'' $>\!>\!>$ ``shift'' (displacement has a different meaning in this context).

\paragraph{?! page 156:} Add the following footnote after ``below the surface'':

``The gravitational field will look extravagant, but this is OK.''
\vskip 0.1 in {\it  Sorry, I could not find the place it refers to.} 

\end{document}
