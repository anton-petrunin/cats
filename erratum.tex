\documentclass[twoside]{article}
\usepackage{bm}


\begin{document}
\pagestyle{empty}

\section*{Questions and corrections to\\
Why cats land on their feet, and 76 other physical paradoxes
and puzzles}

\today

\

\noindent
\textbf{?} = we changed it, but want and approval;\\
\textbf{??} = we did not changed it, but want to;\\
\textbf{!} = mistake;\\
$>\!>\!>$ = change to.

\

\subsection*{Chapter 1}

\paragraph{? page 2, line 5--6:} What do you mean by ``some ‘mistakes’ can be beneficial, at least temporarily''?

\paragraph{? page 2, line 10--...:} There some repetitions there, maybe you want to make it bit shorter.

\paragraph{? page 3, footnote 3:} ``I am refering to'' $>\!>\!>$ ``For example,''.

\paragraph{? page 4, line 6:} ``Literature'' $>\!>\!>$ ``Library''

\subsection*{Chapter 2}

\paragraph{? page 10, Figure 2.5:} Figure  2.5 should be after Figure 2.6.

\paragraph{? page 10, line $\bm{(-14)}$--$\bm{(-13)}$:} It seems that you refer to the caption of the figure 2.5 --- the picture is not quite related. I would just copy text from the cauption and remove ref to the figure.

\paragraph{? page 11, footnote 3:} The words ``increasing kinetic energy'' might be misleading in this context --- I would say ``pumping energy into the system''.

\paragraph{? page 13, Figure 2.7:} It seems that (a) and (b) parts of the figure uppear in wrong order + there are no labels (a) and (b).

\paragraph{? page 14, Problem 2.4:} Nearly the same problem appers in Makovetsky’s book (Problem 22).

\paragraph{? page 16, line 1--2:} $G$ was defined couple of lines above.

\subsection*{Chapter 3}

\paragraph{? page 20, A culinary application of Taylor--Proudman’s theorem:} Do you have an illustrations for this part?

\paragraph{? page 21, sec 3.3 line 1:} I would move word ``Problem.'' to line 10 on the next page and  mark by ``Solution.'' the next sentence.

\paragraph{? page 22, sec 3.5 line 2:} ``sails'' is a bit strange together with ``Air resistance
is to be ignored''.

\paragraph{? page 24, Answer:} Should you make a comment that it is not always possible?

\subsection*{Chapter 4}

\paragraph{? page 29, line 10:} ``as Figure 4.1 illustrates'', but it does not illustrates.

\paragraph{? page 32, sec 4.3:} This section does not seem to belong to this chapter.

\paragraph{! page 32, sec 4.3:} In this section ``you'' switches to ``I'' --- I would write all with ``I''.

\subsection*{Chapter 5}

\paragraph{? page 39, sec 5.1 line 5:} ``nonviscous water'' --- do you want to say something about viscosity?

\paragraph{? page 41, line 7:} Remove ``and $A$ is the area of the piston’s face'' --- it is already defined and can be removed.

\paragraph{? page 41, line 10--11:} Remove ``, where $a$ is the cross-sectional area of the exit tube'' --- it is already defined and can be removed.

\paragraph{? page 41, formula (5.2):} Forgotten square and period, it should be
\[F = \frac12\rho A v^{2} \left[ \left(\frac{A}{a}\right)^2  - 1 \right].\]

\paragraph{? page 42, Figure 5.3:} I do not think it works with air --- the men is also under atmospheric pressure.

\paragraph{? page 42, sec 5.2 line 4:} ``gravity plays'' $>\!>\!>$ ``gravity and viscosty play''.

\paragraph{? page 47, line 5:} ``because the arm is a semicircle'' --- I would the following:

``Indeed, the radial component of this velocity is the same as in the reference frame rotating together with the sprinkler.
And in this frame, the water jet must be directed tangentially to the tube, that is, perpendicular to the direction toward $P$; in other words, it has zero radial component.''

\paragraph{! page 47, Figure 5.9:} This figure is misleading, since the sprinkler rotates, the water will look a bit like on 5.8, if you watch it from above it should be Archimedean spiral.

\paragraph{! page 52, Figure 5.12:} The picture on the left does not agree with picture on the right --- the horisontal bar in the second ``+'' should not be that horizontal.

\paragraph{? page 53, Figure 5.14:} Add labeles (a) and (b).

\paragraph{? page 53, line $\bm{(-2)}$:} ``simple'' did you want to say that ``it has fixed velocity field''?

\paragraph{! page 54, line 3:} One the picture the fixed point is marked by $S$ and now it is $P$.

\paragraph{? page 54, line 3:} Do you want to comment on where exactly this point lies on the segment?

\paragraph{? page 54, sec 5.8:} I would either add some references here or exchange it to another problem.

\paragraph{?? page 54, Figure 5.15:} The droplets must have bigger diameter than the tube!

\paragraph{?? page 55, Figure 5.16:} By looking at the picture one may arrive at a wrong conclusion that the front lines are straight.

\subsection*{Chapter 6}

\paragraph{? page 59, Figure 6.2:} The men does not look like a stone.

\paragraph{? page 60, Figure 6.3:} Better to reflect (a) and (c) so it will have the same orientation as (b).

\paragraph{! page 65, Figure 6.5:} The second bike from the left leans to wrong side.

\paragraph{? page 64, sec 6.5, line 12:} Add a footnote after ``counterturn.'':

``This may depend on your bicycle riding style. Try turning by moving only the handlebars, not your body.''

\paragraph{? page 65, line $\bm{(-2)}$:} I would add footnote ``The same reason as in 6.2''.

\paragraph{? page 66, line 7--8:} You are cheating here. I would write the following:

``Cancelling $m$, we obtain
\[
V^2 - v^2 = 2gh,
\]
and after a few algebraic manipulations,
\[
V - v = \frac{2gh}{v+V} = \frac{2gh}{v+\sqrt{2gh+v^2}}.
\]
In particular, as the initial velocity $v$ increases, its increment $V - v$ decreases.''

\paragraph{? page 72, sec 6.11, line 1:} ``top.'' $>\!>\!>$  ``top (Figure 6.2).''

\paragraph{! page 75, Figure 6.9:} Change $v$ to $V$ once and $v_1$ to $V_1$ twice.

\paragraph{? page 76, line 8:} $\Delta K$ $>\!>\!>$ $\Delta K_{\mathrm{total}}$.

\subsection*{Chapter 7}

\paragraph{! page 79, sec 7.2 line 9 in Answer:}
``$F_{\mathrm{coriolis}}\approx 240$ g'' $>\!>\!>$
\\
``$F_{\mathrm{coriolis}}/g\approx 240$ g'' or better ``$F_{\mathrm{coriolis}}/g\approx .24$ kg'' (so $g\ne$ g)

\paragraph{? page 79, sec 7.2 line 11 in Answer:} Do you want to move ``This force can hold up a cup of water!'' to the previous par?

\paragraph{! page 79, line $\bm{(-3)}$:} Instead of $1/600$ it should be $1/300$ and $0{.}1$° should be $0{.}2$° and 10 cm as well... (But maybe I am mad.)

\paragraph{? page 81, Answer, line 2-3:} ``Each particle...'' $>\!>\!>$ ``Assuming that we are in the northern hemisphere. Each particle...''

\paragraph{? page 82, line 1-3:} ``With the
“pillow” of higher pressure in the center, the air moves
downward...'' is it because of higher density? If yes, should you say it?

\subsection*{Chapter 8}

\paragraph{? page 84, line $\bm{(-6)}$:} ``And a lighter plane uses less fuel.'' Maybe it has to be explained --- most of fuel goes to get speed, hight, and work against air resistance, so do we care only about very first part of the trip?

\paragraph{? page 86, line $\bm{(-4)}$:} ``The velocity shown in Figure 8.1 is wrong'' --- the picture correctly shows tangential component of velocity. Maybe ``Figure 8.1 does not show everything''?

\paragraph{? page 87, further in the text:} ``component $v \sin(\omega \,\Delta t) \approx \omega v \,\Delta t$'' $>\!>\!>$ ``term $v \sin(\omega \,\Delta t) \approx \omega v \,\Delta t$ in the formula for change of velocity in perpendicular to the \emph{initial} radius''.

\paragraph{! page 87, Figure 8.2:} $v\omega\Delta t\cos(\omega t)$ $>\!>\!>$ $v\omega\Delta t\cos(\omega \Delta t)$.

\paragraph{? page 88, line 18 (right above Question):} ``in (b)'' $>\!>\!>$ ``in Figure 8.3b''

\paragraph{! page 95, line 1:} ``nonzero'' $>\!>\!>$ ``positive''.

\paragraph{? page 95, line 5:} ``$a$'' $>\!>\!>$ ``$a_{\mathrm{tan}}$''.

\paragraph{? page 95, line 13:} ``$a=a_c$'' $>\!>\!>$ ``$a_{\mathrm{tan}}=a_{\mathrm{central}}$'' and further\\
``$a_c$'' $>\!>\!>$ ``$a_{\mathrm{central}}$''. (It is more consistent with the other notations.)

\paragraph{! page 100, line 12:} ``... and $r$ ...'' $>\!>\!>$ ``..., $u$ is the velocity of the center of mass, and $r$ ...''

\paragraph{! page 102 Figure 8.11:} ``$r(s)$'' $>\!>\!>$ ``$r(s,t)$''\\ and ``$r(s+ds)$'' $>\!>\!>$ ``$r(s+ds,t)$''.

\paragraph{? page 102, line $\bm{(-2)}$:} Do you want to change ``$ds$'' $>\!>\!>$ ``$\Delta s$''?

\subsection*{Chapter 9}

\paragraph{? page 104, line $\bm{(-4)}$:} Do you want to define precession?

\paragraph{? page 105, Figure 9.1 left:} Move the arrow around left rope down.
\\ + extend the right rope up or remove the left rope above the upper bar.

\paragraph{! page 105, Figure 9.2 left:} Label the axis of torque by $L$ --- it is called $L$ on the next page line 7.

\paragraph{! page 106, Problem:} The formulation is unclear, do you want to move at constant speed, or it is known that it moves with constant speed. I woulde say the following:

``Assume I push the end $A$ of the axis of a spinning top in such a way that it moves  constant speed. In which direction do I push the end?''

\paragraph{! page 105, Figure 9.3a:} Missing $A$.

\paragraph{? page 107, Stability by deflection, line 5:} Add the following footnote after ``falling initially'':

``At that moment it accelerates, therefore the solution of previous problem cannot be applied.''

\paragraph{? page 108, Figure 9.4b:} ``frontal view'' $>\!>\!>$ ``frontal view with subtracted shift''.

\paragraph{? page 115, Problem:} Remove ``with the fluid'' --- above you said that it might be mounted on gimbals.

\subsection*{Chapter 10}

\paragraph{? page 118, caption in Figure 10.1:} ``$1/(1+1/N)$'' $>\!>\!>$ ``$1+1/N$''. (I think ``decrease by factor 2'' means divided by 2.)

\paragraph{! page 118, line $\bm{(-6)}$:}  ``perfect internal temperature'' $>\!>\!>$ ``recommended internal temperature''. (This temperature is recommended by FDA, but perfect temperature is lower.)

\paragraph{! page 118, line $\bm{(-3)}$:} ``best temperature'' $>\!>\!>$ ``best water temperature''.

\paragraph{! page 119, line $\bm{(-3)}$:} ``$N/(N+1)=1/(1+1/N)$'' $>\!>\!>$ ``$(N+1)/N=1+1/N$''.

\paragraph{? page 120, lines 1--2:} ``here is another one'' $>\!>\!>$ ``let me remind''. (It was mentioned already.)

\paragraph{! page 120, Figure 10.2:} This figure is completely wrong!

\paragraph{? page 121, sec 10.2, line 2:} ``or to'' $>\!>\!>$ ``only, or there is''.

\paragraph{? page 123, Answer:} Remove the first and last sentences.

\paragraph{? page 123, Answer, line 16:} Start new line before ``In actual''

\paragraph{? page 124, sec 10.4:} The title is not relevant to the problem. Maybe ``Two Rooms''?

\paragraph{? page 124, footnote 3:} Don't you want to say that the air is assumed to be ideal gas?

\paragraph{! page 125--126:} The numbers in answer and calculation do not agree.
They are not correct in the answer.

\subsection*{Chapter 11}

\paragraph{? page 128, Question:} Too many repetitions.

\paragraph{! page 131, Answer, line 4:} ``Furthermore'' $>\!>\!>$ ``More importantly''. (Otherwise reader might think that fixing the first issue he can build a p. m. machine.)

\paragraph{? page 131, Answer, line 6:} Add the following footnote after ``the foci.'':

``The intensity of radiation in a given direction is directly proportional to the surface area element of the body and the cosine of the angle between the direction and the normal. Moreover, if a material were found for which the intensity of radiation followed a different law, a perpetual motion machine would become a reality.''

\subsection*{Chapter 12}

\paragraph{? page 133, Figure 12.1:} The triangle betewen sail and taut line looks like a sail, which might be misleading. Also I would align the arrow with ``direction of sailing'' with the sail.

\paragraph{? page 135, Figure 12.3:} Missing period in the caption.

\paragraph{? page 135, footnote 3:} Change dead link to the following reference:

``Wind-propeller sails proposed for liners'' Modern Mechanix and Inventions, January 1935, p. 49.

\paragraph{? page 137, footnote 4:} It might be misleading since --- the windmil produces energy, it just cannot be perfect.

\subsection*{Chapter 14}

\paragraph{? page 151, sec 14.4:} It is more adequate to restate the intro problem as ``man on a platform'' --- the boat is too complicated and the solution given in the footnote does not work for boat. For platform it seems to be OK, moreover, it might be OK to think that the drag force of the air is proprtionaltothe velocity.

\paragraph{? page 151, footnote 5:}  ``displacement'' $>\!>\!>$ ``shift'' (displacement has a different meaning in this context)





\end{document}
