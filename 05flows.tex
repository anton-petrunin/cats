\chapter{Потоки и струи}

\section{Закон Бернулли и шприц}

\paragraph{Вопрос.}
Представим, что вы брызгаетесь водой из шприца, нажимая на поршень.
Вспомните о первом законе Ньютона (движение сохраняется равномерным, если на тело не действуют силы) и ответьте, надо ли прикладывать какое-то усилие, чтобы перемещать поршень с постоянной скоростью при условии, что поршень скользит без трения, а вода имеет нулевую вязкость?
Другими словами, что случится если я толкну поршень, сообщив ему некоторую скорость и отпущу;
будет ли он продолжать двигаться с той же скоростью по инерции?

\begin{figure}[ht!]
\centering
\begin{lpic}[t(2mm),b(2mm),r(0mm),l(0mm)]{pics/5.1}
\end{lpic}
\caption{Будет ли поршень двигаться равномерно по инерции при отсутствии вязкости и трения?}
\label{pic:5.1}
\end{figure}

\paragraph{Ответ.}
Чтобы двигать поршень с постоянной скоростью, придётся применить силу, даже в идеальном мире без трения и вязкости.
Первый закон Ньютона о равномерном движении по инерции здесь нельзя применять, так как \emph{некоторая часть воды ускоряется}.
А именно, вода ускоряется при подходе к выходу из цилиндра шприца; см. рисунок \ref{pic:5.2}.
\begin{figure}[ht!]
\centering
\begin{lpic}[t(2mm),b(2mm),r(0mm),l(0mm)]{pics/5.2}
\lbl[rb]{3,15;{\footnotesize медленное}}
\lbl[rt]{3,13;{\footnotesize движение\ }}
\lbl{23,14;{\footnotesize ускорение}}
\lbl[b]{57.5,17;{\footnotesize быстрое}}
\lbl[t]{57.5,11;{\footnotesize движение}}
\end{lpic}
\caption{Каким-то частицы воды приходится ускоряться, и для этого необходима разница давлений.}
\label{pic:5.2}
\end{figure}
Причина в том, что частицы жидкости подталкиваются сзади, то есть давление сзади частицы выше, чем впереди неё.
В этом и состоит \label{эффект Бернулли}\emph{закон Бернулли}, который обычно формулируют как «чем ниже давление, тем выше скорость».
Из-за этого может показаться, что увеличение скорости вызывает понижение давления.
На самом деле всё наоборот: «жидкость ускоряется в направлении понижения давления».
В частности, если давление вдоль потока понижается, то повышается скорость.

Мы одновременно сформулировали закон Бернулли, и объяснили его.
Таким образом, закон Бернулли --- это частный случай второго закона Ньютона.

Закон Бернулли наводит на аналогию с падающим камнем: чем ниже уровень/потенциальная энергия камня, тем быстрее он движется.
Более того, про закон Бернулли можно думать и как о частном случае закона сохранения энергии.%
\footnote{Эта связь ещё более очевидна, если всё же записать закон Бернулли формулой: \[\tfrac12\rho v^2 + p = \text{const},\]
здесь $v$ --- скорость устоявшегося потока несжимаемой жидкости, $\rho$ --- плотность, а $p$ --- давление в той же точке; силой тяжести пренебрегли.\pr}

\paragraph{Задача.}
Какую силу нужно приложить, чтобы двигать поршень с постоянной скоростью $v$?
Площадь поршня равна $A$, а площадь выходного отверстия — $a$.

\paragraph{Решение.}
Когда мы прикладываем силу $F$, чтобы переместить поршень на расстояние $D$, мы совершаем работу $W = F D$.
Вся эта работа тратится на увеличение кинетической энергии воды (ведь мы предположили отсутствие трения и вязкости):
\begin{equation}
F \cdot D = \frac{m v_{\text{вых}}^{2}}{2} - \frac{m v^{2}}{2},
\label{eq:5.1}
\end{equation}
где $m$ — масса вытесненной воды.
Остаётся выразить $F$ через $v$, $a$ и $A$.
Сначала заметим, что $m = \rho A D$, где $\rho$ — плотность воды.

Кроме того, так как вода несжимаема, объём, вытесняемый поршнем в секунду ($vA$), равен выходящему объёму:
\[
vA = v_{\text{вых}} a
\qquad\text{или}\qquad
v_{\text{вых}} = \frac{A}{a} v,
\]
о чём вы наверно сразу догадались.
Подставляя всё это в~\eqref{eq:5.1}, получаем
\begin{equation}
F = \frac12\rho A v^{2} \left[ \left(\frac{A}{a}\right)^2  - 1 \right].
\label{eq:5.2}
\end{equation}

Приведём пару интересных следствий этой формулы.
Давайте посмотрим, что будет при разных отношениях площадей $A/a$ и фиксированной скорости поршня $v$.
\begin{enumerate}
\item
Предположим, что выходное отверстие узкое, то есть отношение $A/a$ велико.
Тогда, согласно~\eqref{eq:5.2}, требуемая сила $F$ также велика.
Проталкивать воду через узкое отверстие трудно, но не из-за вязкости, как можно было бы подумать --- трудно постоянно ускорять частицы.
Совершаемая нами работа идёт не в тепло (кинетическую энергию хаотического движения), а в кинетическую энергию упорядоченного движения выбрасываемой воды.
\item
Теперь предположим, что трубка расширяется, а не сужается; то есть $A/a < 1$, как на рисунке~\ref{pic:5.3}.
Тогда, согласно~\eqref{eq:5.2}, сила $F$ отрицательна.
Это означает, что, для поддержания постоянной скорости, поршень придётся придерживать!
Опять же, это можно увидеть и без формулы, ведь вода выходит с меньшей скоростью, чем скорость внутри цилиндра.
Это значит, что придётся замедлять частицы воды, придерживая поршень.
\end{enumerate}

\begin{figure}[ht!]
\centering
\begin{lpic}[t(7mm),b(2mm),r(0mm),l(0mm)]{pics/5.3}
\lbl[b]{20,17.5;\parbox{45mm}%
{\centering\footnotesize  Оттягиваем поршень\\для поддержания\\постоянной скорости.}}
\lbl[t]{18,3;\parbox{12mm}%
{\centering\footnotesize низкое\\давление}}
\lbl[t]{27,11;\parbox{12mm}{\centering\footnotesize быстрое\\движение}}
\lbl[b]{46,15.5,7;{\footnotesize замедление}}
\lbl[t]{46,12.5,-7;{\footnotesize движения}}
\lbl[b]{63,13,90;\parbox{20mm}{\centering\footnotesize медленное\\движение}}
\lbl[t]{73,13,90;\parbox{20mm}{\centering\footnotesize атмосферное\\давление}}
\lbl[b]{48,26,40;\parbox{30mm}{\centering\footnotesize снижение давления\\тормозит частицы}}
\end{lpic}
\caption{Поддержание постоянной скорости в расширяющейся трубке требует постоянного придерживания (торможения) поршня.}
\label{pic:5.3}
\end{figure}

\section[Коктейльная трубочка]{Коктейльная трубочка и\\ необратимость времени}

\paragraph{Вопрос.}
Требуется ли больше усилий, чтобы втянуть, или выдуть воду из коктейльной трубочки (см. рисунок \ref{pic:5.4})?
Предполагается, что вода в трубочке движется с постоянной скоростью и что сила тяжести не играет заметной роли.

\begin{figure}[ht!]
\centering
\begin{lpic}[t(2mm),b(2mm),r(0mm),l(0mm)]{pics/5.4}
\end{lpic}
\caption{Одинаковые ли усилия нужны, чтобы поддерживать скорость постоянной?}
\label{pic:5.4}
\end{figure}


\paragraph{Ответ.}
Втягивать трудней; причина объясняется на рисунке \ref{pic:5.5}.
При всасывании вода поступает в трубочку со всех направлений, как на части (b) рисунка.
В среднем, частицы воды заметно увеличивает скорость, подходя к отверстию.
Это ускорение как раз и обеспечивается всасыванием.%
\footnote{Тот самый эффект Бернулли со страницы \pageref{эффект Бернулли}.}
Иными словами, мы должны затратить энергию на разгон жидкости, а это требует усилий.

\begin{figure}[ht!]
\centering
\begin{lpic}[t(2mm),b(2mm),r(0mm),l(0mm)]{pics/5.5}
\lbl[tl]{0,47;(a)}
\lbl[tl]{40,47;(b)}
\end{lpic}
\caption{Что трудней всасывать или выдувать воду (поддерживая скорость постоянной)?}
\label{pic:5.5}
\end{figure}

С другой стороны, как показано на рисунке \ref{pic:5.5}a, вытекающая вода образует струю.
Поскольку струя расширяется медленно, давление тоже меняется медленно вдоль потока.

\paragraph{Направление времени.}
Можно было бы подумать, что, изменив направление потока в трубочке, мы просто изменим направление движения воды повсюду;
но с детства всем известно, что так не получится.
Задуть свечу легко, а погасить её, втягивая воздух (с безопасного расстояния, не обжигая губ), невозможно.
Я был бы рад услышать подробности от тех, кому это удалось (но не от их адвокатов).

\begin{figure}[hb!]
\centering
\begin{lpic}[t(2mm),b(2mm),r(0mm),l(0mm)]{pics/5.6}
\end{lpic}
\caption{Поток, начинающийся как входящая струя, нестабилен и через мгновение начнёт выглядеть так, как на рисунке \ref{pic:5.5}.}
\label{pic:5.6}
\end{figure}

Вопрос предпочтительного направления времени много лет занимал учёных.
Его суть в следующем кажущемся противоречии.
В классической механике, законы Ньютона обратимы во времени.
И всё же, если рассматривать систему с большим числом классических частиц, например идеальный газ, то кажется, что обратимость времени исчезает.
Разрешение этого противоречия состоит в том, что ситуация, показанная на рисунке \ref{pic:5.6} (та же, что на рисунке \ref{pic:5.5}a, но с обращённым временем), хоть и возможна теоретически, но крайне неустойчива: даже если искусственно заставить жидкость двигаться как на рисунке \ref{pic:5.6}, то через мгновение всё сломается и жидкость начнёт двигаться как на рисунке \ref{pic:5.5}b.

\section{Как двигаться в космическом корабле}\label{Как двигаться в космическом корабле}

\paragraph{Вопрос.}
Представьте, что вы зависли посредине кабины космического корабля.
Вы уже отдохнули и хотели бы добраться до стены.
Можно чего-нибудь выбросить (например, ботинок или ремень%
\footnote{Они ведь вообще не нужны в невесомости --- ходить негде и штаны не спадают.}) и начать движение в противоположную сторону, но предположим, что кидаться предметами запрещено.
Как же добраться до стены?

\begin{figure}[ht!]
\centering
\begin{lpic}[t(2mm),b(2mm),r(0mm),l(0mm)]{pics/5.7}
\lbl[b]{8,41;вдох}
\lbl[b]{27.5,41;выдох}
\lbl[b]{50,41;вдох}
\lbl[b]{73,40;выдох}
\end{lpic}
\caption{Движение в невесомости за счёт дыхания.}
\label{pic:5.7}
\end{figure}

\paragraph{Ответ.}
Просто дышите.
При вдохе вы втягиваете воздух со всех сторон, а при выдохе он выходит струёй, как показано на рисунке~\ref{pic:5.7}.
В совокупности один цикл вдох-выдох выбрасывает воздух в направлении этой струи, это заставит вас двигаться в противоположную сторону.
Вы начнёте передвигаться как крайне малоэффективный кальмар.%
\footnote{Кальмары движутся по тому же принципу, только выбрасывают воду сзади.}
Если дышать ртом, то движение будет медленнее (что неудивительно), ведь воздух выталкивается через рот с меньшей скоростью, чем через нос.

\section{Загадка о садовой поливалке}

\paragraph{Вопрос.}
Садовая поливалка состоит из $S$-образной трубки, вращающаяся вокруг точки $P$, как показано на рисунке \ref{pic:5.8}.
Вода подаётся через шланг, и сила струи заставляет поливалку вращаться.
В каком направлении будет вылетать вода относительно наблюдателя на земле?
Считайте, что вращение происходит без трения и с постоянной скоростью.


\begin{figure}[ht!]
\centering
\begin{lpic}[t(2mm),b(2mm),r(0mm),l(0mm)]{pics/5.8}
\lbl[bl]{0,35;(a)}
\lbl[bl]{50,35;(b)}
\lbl[br]{20,0;шланг}
\lbl[br]{67,0;шланг}
\lbl[b]{23,28;вращение}
\lbl[b]{22,23;$P$}
\lbl[b]{77,33.5;полуокруность}
\end{lpic}
\caption{В каком направлении струя выходит из трубок?}
\label{pic:5.8}
\end{figure}

\paragraph{Ответ.}
Вода будет выходить в радиальном направлении, то есть прямо от точки $P$; направление, показанное на рисунке \ref{pic:5.8}a, неверно!
Касательная составляющая скорости воды равна нулю, как на рисунке 5.9.

\paragraph{Объяснение.}
Вода, подаваемая через шланг, не вращается вокруг вертикальной оси.
Единственное, что могло бы заставить её вращаться, --- это трение в шарнире, но по условию оно отсутствует.
Поэтому вода выходит с тем же нулевым вращением, с каким вошла.
Значит ей придётся вылетать в строго радиальном направлении.%
\footnote{Чтобы сделать это рассуждение совсем точным, достаточно заменить расплывчатый термин \emph{вращение} на точный термин \emph{момент импульса} и сказать, что момент импульса не меняется из-за отсутствия момента силы.}

\paragraph{Ещё вопрос.}
Что если переделать поливалку, придав каждому плечу форму полуокружности, как показано на рисунке \ref{pic:5.8}b.
Хорошая ли это идея?

\paragraph{Ответ.}
Идея любопытна, но поливалка станет непригодной для полива.
Вода будет выходить с нулевой скоростью относительно земли, то есть литься аккуратно вниз!
Действительно, как уже установлено, вода обязана выходить в радиальном направлении.
Но радиальная составляющая скорости на выходе должна быть равна нулю, потому что плечо имеет форму полуокружности.%
\footnote{Действительно, радиальная составляющая этой скорости та же что в системе отсчёта крутящейся вместе с поливалкой.
А в этой системе струя воды должна быть направлена по касательной к трубке, то есть перпендикуляно к направлению к $P$; то есть у неё нулевая радиальная составляющая. \pr}
Получится довольно странная поливалка: вода будет входить с положительной скоростью, а выходить с нулевой.

\paragraph{Вопрос.}
Итак, в нашу странную поливалку вода втекает с положительной кинетической энергией, а вытекает с нулевой скоростью и, следовательно, с нулевой кинетической энергией.
Куда же девается кинетическая энергия?

\begin{figure}[ht!]
\centering
\begin{lpic}[t(2mm),b(2mm),r(0mm),l(0mm)]{pics/5.9}
\lbl[l]{33,26;\parbox{32mm}%
{\centering\footnotesize  поршень движется с\\ постоянной скоростью}}
\lbl[r]{18,8;\parbox{22mm}%
{\centering\footnotesize  я придерживаю\\ поршень}}
\end{lpic}
\caption{Если поршень движется вверх, то его надо придерживать, чтобы скорость оставалась постоянной.}
\label{pic:5.9}
\end{figure}

\paragraph{Ответ.}
Энергию высасывает поливалка --- в самом прямом смысле;
это не фигура речи, а объяснение физического процесса.
Поливалка сосёт в том смысле, что давление в шланге отрицательно%
\footnote{Точнее, оно ниже атмосферного давления.}
(рисунок \ref{pic:5.9}).
Это происходит потому, что вода во вращающейся трубе выбрасывается наружу центробежным эффектом, создавая разрежение.

\paragraph{Водяной кнут.}\label{Водяной кнут}
Что же будет с поливалкой, если поршень на рисунке \ref{pic:5.9} не удерживать?

Вращающиеся плечи создают центробежное разрежение, которое ускоряет движение воды.
Поливалка начнёт вращаться быстрее и быстрее, пока вся вода не выльется.
Это поведение схоже, как это ни странно, со щелчком кнута.
Если послать волну вдоль кнута, то двигаясь к кончику, она укорачивается%
\footnote{Так как толщина кнута уменьшается от начала к концу.\pr}%
, и та же энергия концентрируется в очень короткой волне у конца.
При правильном движении концентрация энергии может достичь того, что кончик кнута превысит скорость звука.
Похожая, хотя и менее впечатляющая, концентрация энергии происходит и в нашем мысленном эксперименте с поливалкой.

\section{Быстрый слив с нулевой скоростью}

\paragraph{Вопрос.}
На рисунке \ref{pic:5.10} показан бак с присоединённым резиновым шлангом.
Можно ли сливать воду из шланга так, чтобы она вытекала с нулевой скоростью?%
\footnote{Скорость меряется относительно земли.}

\begin{figure}[ht!]
\centering
\begin{lpic}[t(2mm),b(2mm),r(0mm),l(0mm)]{pics/5.10}
\end{lpic}
\caption{Можно ли сливать воду из бака так, чтобы вода при выходе из шланга имела (почти) нулевую скорость?}
\label{pic:5.10}
\end{figure}

\paragraph{Ответ.}
Надо просто двигать конец шланга со скоростью, противоположной скорости выходящей воды;
тогда вода будет выходить с нулевой скоростью.
Это то же, что бросить яблоко назад из движущегося автомобиля со скоростью, равной скорости автомобиля.
В момент броска яблоко будет иметь нулевую скорость относительно земли.

Странная поливалка, показанная на рисунке \ref{pic:5.8}b, как раз может сливать воду с нулевой скоростью, если подсоединить её к баку.
Более того, эта поливалка идеально подходит для опорожнения бака: она позволяет очень быстро сливать жидкость из одного бака в другой, без брызг и без насоса.
Действительно, поскольку поливалка создаёт разрежение (как объясняется на странице \pageref{Водяной кнут}), она сама работает как насос!%

\section{Загадка о замороженной струе}

Следующий вопрос пришёл ко мне в голову, пока я мыл посуду и наблюдал, как струя воды из крана ударяется о дно раковины.

\paragraph{Вопрос.}
Вода равномерно льётся из банки на плоскую платформу весов, и разлетается в стороны, как показано на рисунке \ref{pic:5.11}.
Показания весов измеряют силу удара струи.
Что больше: сила удара струи или вес падающей струи?
Или же они примерно равны?
\begin{figure}[ht!]
\centering
\begin{lpic}[t(2mm),b(2mm),r(0mm),l(0mm)]{pics/5.11}
\lbl[ul]{19,9;$W_1$}
\lbl[ul]{61,9;$W_2$}
\lbl[l]{54,30;\parbox{32mm}%
{\footnotesize  замороженная\\ водяная струя}}
\end{lpic}
\caption{Как соотносится сила удара падающей воды с её весом?}
\label{pic:5.11}
\end{figure}
Будем игнорировать сопротивление воздуха, поверхностное натяжение и другие относительно несущественные эффекты.
Вертикальной скоростью воды у горлышка банки и внутри неё следует пренебречь.

\paragraph{Ответ.}
В каждый момент времени о платформу бьётся лишь небольшая часть воды.
Это может навести на мысль, что сила удара меньше веса всей струи, но на самом деле эти силы примерно равны!

\paragraph{Неформальное объяснение.}
Импульс%
\footnote{Импульс объясняется в приложении, на странице ???, здесь говорится об импульсе в вертикальном направлении.}
всей воды не меняется, пока струя льётся равномерно.
Действительно, у струи импульс не меняется, так как сама струя не меняется в процессе течения, а импульс остальной воды равен нулю.

Как объяснялось в конце раздела~\ref{sec:A.4}, постоянство импульса означает, что сумма сил, действующих на воду, равна нулю:
\[
W - R = 0,
\]
где $W$ — вес воды, а $R$ — сумма реакций со стороны банки, весов и нулевого уровня.
Соотношение $W = R$ даёт
\begin{equation}
W_{\text{банка}} + W_{\text{струя}} + W_{\text{ноль}}
= R_{\text{банка}} + R_{\text{весы}} + R_{\text{ноль}}.
\label{eq:5.3}
\end{equation}
Но $W_{\text{банка}} = R_{\text{банка}}$ и $W_{\text{ноль}} = R_{\text{ноль}}$, ведь можно считать, что вода в банке и на нулевом уровне находится в покое.
Выбросив эти слагаемые из~\eqref{eq:5.3}, получаем:
\[
W_{\text{струя}} = R_{\text{весы}},
\]
что и требовалось доказать.

\section{Загадка о завихрении}

\paragraph{Краткая справка.}
Для этой задачи нам потребуется следующее утверждение;
оно будет расшифровано чуть ниже.
\begin{quote}
\emph{Зав\'{и}хренность невязкой жидкости остаётся равной нулю, если она была равна нулю изначально}.
\end{quote}
Это частный случай теоремы Кельвина.%
\footnote{Формулировку и доказательство этой теоремы можно найти, например, в учебнике Бэтчелора.
Отмечу, что для двумерных жидкостей, есть другое более наглядное доказательство, которое почти сразу вытекает из следующих двух фактов: (1) круглый сгусток жидкости не испытывает крутящего момента (относительно своего центра) из-за отсутствия вязкости и (2) площадь сгустка не изменяется при его переносе потоком жидкости (при этом сгусток не обязан оставаться круглым; предполагается лишь, что в один из моментов времени он был круглым).}

\begin{figure}[ht!]
\centering
\begin{lpic}[t(7mm),b(2mm),r(0mm),l(0mm)]{pics/5.12}
\lbl[b]{58,34;\parbox{32mm}%
{\footnotesize\centering  вертикальная\\составляющая\\скорости}}
\lbl[b]{85,34;\parbox{32mm}%
{\footnotesize\centering  горизонтальная\\составляющая\\скорости}}
\lbl{70,7; $\mathrm{curl}=\omega_1+\omega_2$ %??? или $\mathrm{rot}=\omega_1+\omega_2$???
}
\lbl[t]{60,19;$\omega_1$}
\lbl[r]{73.5,25;$\omega_2$}
\end{lpic}
\caption{Завихренность жидкости есть сумма угловых скоростей двух бесконечно малых штрихов, сходящихся в данной точке, в тот момент когда они перпендикулярны друг другу.}
\label{pic:5.12}
\end{figure}

\paragraph{Что такое зав\'{и}хренность?}\label{def:завихренность}
Название подсказывает, что завихренность%
\footnote{Ударение на первый слог.\pr}
мерит как вращается жидкость.
Сейчас я точно как её мерить для двумерных жидкостей.
Представим, что в какой-то момент я добавил к жидкости краситель, нарисовав на ней плюс из двух коротких перпендикулярных штрихов, как показано на рисунке~\ref{pic:5.12}.
Этот плюс станет поворачиваться по мере переноса потоком (он также будет растягиваться, но это нас не интересует);
давайте измерим угловые скорости $\omega_1$ и $\omega_2$ этих штрихов в начальный момент, когда они ещё перпендикулярны.
По определению, завихренность жидкости в точке $p$ равна сумме $\omega_1+\omega_2$.
Если бы жидкость, вращалась как твёрдое тело, то её завихренность равнялась бы удвоенной угловой скорости твёрдого вращения.

\paragraph{Загадка о развороте жидкости.}

На рисунке \ref{pic:5.13} показана идеальная жидкость, заполняющая кольцо с поршнем внутри.
Всё находится в покое.
Затем мы проталкиваем поршень по всему кольцу и останавливаем его.
Согласно теореме Кельвина, на протяжении всего процесса завихренность должна оставаться нулевой.
Но ведь вся вода при этом разворачивается, как же это возможно?

\begin{figure}[ht!]
\centering
\begin{lpic}[t(7mm),b(2mm),r(0mm),l(0mm)]{pics/5.13}
\lbl{33,19;{\scriptsize поршень}}
\lbl[l]{41,19;\parbox{22mm}%
{\footnotesize  движение\\поршня}}
\end{lpic}
\caption{Как может жидкость вращаться, сохраняя нулевую угловую скорость
(точнее, зав\'{и}хренность)?}
\label{pic:5.13}
\end{figure}

\paragraph{Решение.}
Когда поршень движется против часовой стрелки, вода не перемещается как твёрдое тело; если бы это было так, то завихренность действительно была бы ненулевой.
Чтобы сохранять нулевую завихренность, воде приходится двигаться быстрее у внутренней стенки, чем у внешней, как показано на рисунке \ref{pic:5.14}a.
Жидкость одновременно совершает два движения:
(1) вращается против часовой стрелки вместе с поршнем и
(2) циркулирует по часовой стрелке.
В системе отсчёта, вращающейся вместе с поршнем, течение выглядит как на рисунке \ref{pic:5.14}b.

\begin{figure}[ht!]
\centering
\begin{lpic}[t(2mm),b(5mm),r(0mm),l(0mm)]{pics/5.14}
\lbl[tl]{0,40;(a)}
\lbl[tl]{48,40;(b)}
\lbl[t]{54,18.5;$S$}
\lbl[b]{41,22,-85;\parbox{22mm}%
{\footnotesize\centering  движение\\поршня}}
\lbl{33,19;{\scriptsize поршень}}
\lbl{81,19;{\scriptsize поршень}}
\lbl[t]{20,-1;{\scriptsize относительно земли}}
\lbl[t]{67,-1;\parbox{32mm}%
{\footnotesize\centering  относительно поршня\\ точка $S$ фиксирована}}
\end{lpic}
\caption{(a) Относительно земли поток быстрее на внутренней стороне.
(b) Поток в системе отсчёта, вращающейся вместе с поршнем.}
\label{pic:5.14}
\end{figure}

\paragraph{Вопрос.} Есть ли в жидкости точка, которая возвращается на место после одного оборота поршня?

\paragraph{Ответ.}
Точка $S$, %???где на отрезке она находится???S=P???
расположенная диаметрально противоположно поршню, возвращается в исходное положение. Более того, эта точка остаётся неподвижной относительно поршня на протяжении всего оборота.%Почему??? видимо здесь подразумевается, что линии потока не зависят от времени???
\footnote{Существование такой точки следует из теоремы Боля --- Брауэра о неподвижной точке, формулировку и доказательство которой можно найти в любом учебнике по топологии для студентов, например, в книге Дж. Манкриза.}

\section{Вопрос о струйном принтере}\label{Вопрос о струйном принтере}

Струйные принтеры работают, выбрасывая тонкие струи чернил на бумагу.

\paragraph{Вопрос.}
Вода (или чернила) выпрыскиваются из тонкой трубки и из-за поверхностного натяжения распадается на капли (рисунок \ref{pic:5.15}).
Летят ли капли с той же скоростью $v$ с которой вода выходит из трубки?
Силой тяжести и сопротивлением воздуха следует пренебречь.

\begin{figure}[ht!]
\centering
\begin{lpic}[t(2mm),b(5mm),r(0mm),l(0mm)]{pics/5.15}
\lbl[l]{10,2.5;{\footnotesize $v_1$}}
\lbl[l]{48,2.5;{\footnotesize $v_2$}}
\lbl[t]{23,0;$J$}
\lbl[b]{35,9;$T$}
\end{lpic}
\caption{Будут ли капли продолжать лететь с той же скоростью, с какой вылетает струя?
Сопротивлением воздуха следует пренебречь.}
\label{pic:5.15}
\end{figure}

\paragraph{Ответ.}
Капли движутся медленнее, чем струя в трубке.
Поверхностное натяжение заставляет струю $J$ стягиваться подобно резиновой ленте.
Это натяжение тянет кончик $T$ влево, к выходу из трубки, замедляя его.
Позднее этот кончик отрывается и превращается в каплю, которая теперь движется медленнее, чем чернила в трубке.%???непонятно.

\section{Загадка о водопаде}

\paragraph{Вопрос.}
Рассмотрите горизонтальную трубу на рисунке \ref{pic:5.16}.
Из трубы вытекает вода, слой $A$, выйдя из трубы, начинает падать,
слой $B$, всё ещё находящийся в трубе, пока ещё не падает.
То есть наш поток сдвигается вниз, и значит завихривается по часовой стрелке.
Значит по выходе из трубы вода приобретает ненулевую завихренность.
Но, как мы знаем, это противоречит теореме Кельвина (сформулированной выше), согласно которой завихренность меняться не может.

Где ошибка рассуждения?

\begin{figure}[ht!]
\centering
\begin{lpic}[t(7mm),b(5mm),r(0mm),l(0mm)]{pics/5.16}
\lbl[tl]{0,32;(a)}
\lbl[tl]{30,32;(b)}
\lbl[tl]{61,32;(c)}
\lbl[lb]{22,25;{\footnotesize завихривается}}
\lbl[rt]{12,11;{\footnotesize не завихривается}}
\lbl[l]{22,15;{\footnotesize $A$}}
\lbl[l]{17,18;{\footnotesize $B$}}
\lbl[b]{74,22;\parbox{42mm}%
{\footnotesize\raggedleft  река или водопроводная\\ труба}
}
\end{lpic}
\caption{Слои воды начинают опускаться, по выходу из трубы, приобретая завихренность --- или всё же нет?}
\label{pic:5.16}
\end{figure}

\paragraph{Ответ.}
Ошибка в неправильно нарисованном слое $A$.
На самом деле слои не остаются вертикальными, а наклоняются, как показано на рисунке \ref{pic:5.16}b.
Этот наклон компенсирует и вращение в моём исходном рассуждении.

\paragraph{Поток в трубе.}
Подобное противовращение наблюдается в воде, проходящей через изгиб трубы или реки.
На рисунке \ref{pic:5.16}c показано, как линия красителя движется по изогнутой трубе.
Когда труба поворачивает направо, линия красителя поворачивается влево так, чтобы сохранить нулевую завихренность. На втором изгибе трубы происходит обратное: труба поворачивает налево, линия — направо.
