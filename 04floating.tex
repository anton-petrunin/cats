\chapter{Плававание и дайвинг}

\section{Ванна на колесиках}\label{Ванна на колесиках}

Вот вариант задачи о шарике с гелием, но на Земле.

\paragraph{Вопрос.}
Лодочка плавает у конца ванны.\rindex{лодка}
Ванна установлена на колёсиках способных без трения ездить по полу.
В начале всё находится в состоянии покоя.
С помощью пульта управления вы заставляете лодочку переместиться с одного конца ванны в другой и ждёте пока ванна и её содержимое опять не перейдут в состояние покоя.
В какую сторону сместилась ванна?

Будем считать, что $m$ и $M$ --- массы лодочки и ванны соответственно, а $L$ --- длина ванны.

\begin{figure}[ht!]
\centering
\begin{lpic}[t(2mm),b(2mm),r(0mm),l(0mm)]{pics/4.1}
\end{lpic}
\caption{Насколько сместится ванна после того, как лодочка проплывёт слева направо?}
\label{pic:4.1}
\end{figure}

\paragraph{Ответ}
\[
\text{расстояние}=\frac{m}{m + M}L
\]
неверен.
На самом деле ванна окажется там же, где была вначале.

\paragraph{Объяснение.}\rindex{лодка}
По закону Архимеда, масса лодочки равна массе вытесненной ею воды.
Поэтому перемещение лодочки с одного конца в другой эквивалентно перестановке двух равных масс.
Но такая перестановка не сдвинет центр масс системы вода$+$лодка относительно ванны.

Центр масс также не может сдвинуться относительно пола, поскольку к ванне не приложены внешние силы.
Следовательно, ванна не передвинулась.

\paragraph{Вопрос.}
Круглое блюдо, наполненное водой, уравновешено на узкой опоре (рисунок \ref{pic:4.2}a).
Резиновая уточка плавает у края блюда.
Вы медленно вытаскиваете уточку.
В какую сторону опрокинется блюдо, если вообще опрокинется?

\begin{figure}[ht!]
\centering
\begin{lpic}[t(2mm),b(2mm),r(0mm),l(0mm)]{pics/4.2}
\lbl[bl]{0,22;(a)}
\lbl[bl]{38,22;(b)}
\end{lpic}
\caption{Опрокинется ли блюдо, если осторожно вытащить уточку?}
\label{pic:4.2}
\end{figure}

\paragraph{Ответ.}
Блюдо не опрокинется; оно останется в равновесии.
Это видно из следующего мысленного эксперимента.
Во-первых, по закону Архимеда, уточку можно заменить вытесненной ей водой --- это влияет на равновесие.
Теперь вытаскивание уточки эквивалентно высасыванию этой области воды.
Поскольку всё это делается медленно, вода успевает перераспределиться, и можно считать, что высасывается слой воды равной толщины, а это не нарушает равновесия.

\paragraph{Вопрос.}
Изменится ли ответ, если блюдо не было бы симметричным, как например на рисунке~\ref{pic:4.2}b?

\paragraph{Ответ.}
Равновесие может нарушиться.
Например, если удалять воду выше пунктирной линии на рисунке \ref{pic:4.2}b, то центр масс воды будет смещаться влево, соответсвенно и блюдо опрокинется влево.

\section{Задача с углублением}\label{Углублённая задача}

Не начинайте эту задачу, не разобравшись с предыдущей.

\paragraph{Загадка.}
На рисунке \ref{pic:4.3} показана лодочка\rindex{лодка} из предыдущей задачи, но теперь она подцеплена снизу натянутым тросиком, который прикреплённ к колесику, катящемуся по подводной рельсе, прикреплённой к ванне.
Как и в предыдущей задаче,
в начале всё покоится,
далее лодочка плывёт к противоположному концу ванны,
останавливается,
и мы ждём некоторое время пока ванна и её содержимое не придут в состояние покоя.
Вопрос тот же: сдвинется ли ванна? и если да, то в каком направлении?

\begin{figure}[ht!]
\centering
\begin{lpic}[t(2mm),b(2mm),r(0mm),l(0mm)]{pics/4.3}
\lbl[r]{1,2.3;{\footnotesize колёсико}}
\lbl[t]{37,5.5;{\footnotesize рельса}}
\end{lpic}
\caption{В каком направлении сдвинется ванна, если тросик подтягивет лодочку вниз?}
\label{pic:4.3}
\end{figure}

\paragraph{Кое-какие соображения.}
В предыдущей задаче, где троссик отсутствовал, мы установили, что ванна останется на месте.
Но ведь вертикальная сила, создаваемая тросиком, не может повлиять на горизонтальное движение ванны и лодочки.
Значит, ванна останется неподвижной, как и в предыдущей задаче.

Верно ли сказанное, а если нет, то где ошибка?

\paragraph{Ответ (теперь правильный).}
Ванна сдвинется в том же направлении, что и лодочка.\rindex{лодка}
Объяснение аналогично задаче про шарик с гелием на странице~\pageref{Гелиевый шар} --- по сути, это та же задача.

Поскольку троссик тянет лодочку вниз, масса вытесненной ею воды превысит массу самой лодочки, то есть лодочка вытесняет больший объём, при той же массе --- как и шарик с гелием в задаче на странице~\pageref{Гелиевый шар}.
То есть, менее массивная лодочка поменялась местами с более массивной водой.
Значит, относительно ванны, центр масс смещается влево.
Но центр масс обязан оставаться неподвижным относительно земли, и, значит, сама ванна смещается вправо.

Иными словами, раз центр масс движется влево относительно ванны, сама ванна обязана двигаться вправо, ведь импульс всей системы должен оставаться нулевым.

\paragraph{Вопрос.}
А где же обман в доказательстве, что ванна не сдвинется?

\paragraph{Ответ.}
Обман в заявлении, что трос не влияет на горизонтальные силы, действующие на ванну.
Когда лодочка движется, она передвигает воду, а вода взаимодействует со стенками ванны.\rindex{лодка}
Таким образом, вертикальное натяжение троса оказывает горизонтальное воздействие (хоть и непрямое).%
\footnote{Представим себе, что теперь у нас появилась возможность дистнационно подтягивать и осаблять торсик. Как мы только-что выяснили, если перегонать лодочку вправо, то и ванна перевинется вправо. Далее расслабим тросик и перегоним лодочку влево.
Согласно предыдущей задаче ванна не сдвинется.
Похоже, что можно повторять этот цикл заставляя ванну двигаться всё дальше и дальше.
Но, тогда и центр масс ванны (с её содеримым) будет двигаться вправо --- что-то не так. Где же ошибка? \pr}

\paragraph{Отрицательные массы.}
Поскольку троссик тянет лодочку вниз, её масса меньше массы вытесненной ею воды.\rindex{лодка}
Поэтому можно думать, что вместо лодочки мы передвигаем в ванне отрицателную массу.
Именно эта отрицательность и приводит к неожиданному результату.

\section{Как сбросить вес за долю секунды}

\paragraph{Загадка.}
Стоя на напольных весах, смогу ли я сделать так, чтобы они показали меньший вес (ни на что не опираясь и не снимая одежду)?

\paragraph{Ответ.} Вопрос с подвохом: это можно сделать, но только на короткое время.
Для этого достаточно подогнуть колени.
Если сделать это \emph{очень} быстро, то мои ноги могут на какое-то мгновение оторваться от опоры, и весы покажут нулевой вес.
Если подгибать колени помедленней, то разница будет не столь заметной, но уменьшение веса всё равно произойдёт.
Довольно скоро придётся прекратить ускорение вниз и начать ускоряться вверх;
весы начнут показывать больший вес, пока движение не прекратится.

Всё это результат работы закона Ньютона:%
\footnote{Объясняется в приложении на странице\pageref{Законы Ньютона}.}
\[
ma=F,
\]
где $a$ --- ваше ускорение, а $F=S-W$ --- сила внешних воздействий;
здесь $S$ --- сила, с которой весы толкают меня вверх, а $W$ --- мой вес в покое.
Весы всегда показывают силу $S$.
Если я сгибаю колени, то ускоряюсь вниз: $a<0$, и, следовательно,
\[
ma=S - W < 0,
\qquad\text{то есть,}\qquad
S < W,
\]
и показания весов меньше моего веса.
Если я стою неподвижно, то $a=0$ и $S-W=0$; весы показывают печальную правду.
Если же я начну прыжок, то $a>0$, значит, $S-W>0$ и весы показывают больше моего веса.

\medskip

Следующая задача потребует немного математического анализа.

\paragraph{Задача.}
Докажите, что независимо от того, как я прыгаю на весах, среднее значение, которое они покажут будет стремиться к моему истинному весу, предполагая, что я готов ждать.

\paragraph{Решение.}
Пусть $T$ --- достаточно большое время ожидания.
Интегрируя уравнение $ma(t)=S(t) - W$, получаем
\[
\int\limits_0^T ma(t)\, dt
=
\int\limits_0^T(S(t)-W)\, dt.
\]
По основной теореме анализа (см. страницу~\pageref{Основная теорема анализа}),
\[
\int\limits_0^T a(t)\, dt
=
\int\limits_0^T v'(t)\, dt
=
v(T)-v(0).
\]
Значит
\[
m\big(v(T) - v(0)\big)
=
\int\limits_0^T S(t)\, dt - W T.
\]
Поделив на $T$, получим
\[
\frac mT \big(v(T) - v(0)\big) =
\!\!\!\!\underbrace{\frac1T \int\limits_0^T S(t)\, dt}_{\tiny \text{среднее значение}\ S}\!\!\!\!- W.
\]
Пусть $T\to\infty$ (будем считать, что я доживу до этого момента),
тогда левая часть стремится к нулю, так как $v$ ограничена
в силу моих человеческих возможностей.
Следовательно, и правая часть тоже стремится к нулю,
то есть, среднее значение
\[
\frac1T\int\limits_0^T S(t)\, dt
\]
стремится к $W$, что и требовалось доказать.

\section{Шарик под водой}\label{Шарик под водой}

\paragraph{Вопрос.}
Шарик, надутый воздухом, удерживается под водой нитью, привязанной ко дну банки, а сама банка стоит на весах.
Что произойдёт с показаниями весов сразу после того как нить разорвётся?

\begin{figure}[ht!]
\centering
\begin{lpic}[t(2mm),b(2mm),r(0mm),l(0mm)]{pics/4.4}
\end{lpic}
\caption{Что произойдёт с показаниями весов (увеличиваться, уменьшаться или не изменятся) сразу после разрыва нити?}
\label{pic:4.4}
\end{figure}

\paragraph{Правда или нет?}
До разрыва нить тянула дно банки вверх.
После разрыва эта сила исчезает, от этого банка становится как бы тяжелее.
Поэтому сразу после разрыва нити весы покажут больший вес.

\paragraph{Теперь правда.}
На самом деле всё наоборот: весы вначале покажут \emph{меньший вес}, то есть банка покажется легче.
Чтобы это понять, проследим за центром масс всей системы (банки с её содержимым).
Как только нить рвётся, шарик ускоряется вверх, и тот же объём воды ускоряется вниз.
Поскольку вода значительно плотнее воздуха, центр масс содержимого банки ускоряется вниз.
А это и означает, что сила, действующая на весы, уменьшается --- так же, как в предыдущей задаче, где я подгибал колени, уменьшая этим показания весов.

Более формально, по второму закону Ньютона (страница \pageref{Законы Ньютона}), ускорение $a$ центра масс определяется всеми силами, действующими на банку, их всего две --- реакция опоры (весов) и сам вес; то есть,
\[\text{реакция}-\text{вес}=ma.\]
Когда нить рвётся, начальное ускорение направлено вниз, то есть $a\z<0$, это означает, что
\[
\text{реакция}-\text{вес}<0
\qquad\text{или}\qquad
\text{реакция}<\text{вес},
\]
то есть сила реакции опоры (её и показывают весы) меньше веса.

\paragraph{А где была ошибка?}
В предыдущем рассуждении я сказал правду, но не всю.
Не была упомянута сила \emph{воды}, действующая на банку.
Действительно, в момент разрыва нити вода начинает опускаться, из-за чего давление на дно банки уменьшается, и банка кажется легче.
Утверждение, что разрыв нити делает банку тяжелее верно, но в то же самое время банка становится и легче за счёт уменьшения давления на дно.
Как показывает рассуждение с центром масс, облегчение побеждает утяжеление.

\paragraph{Мораль.}
Эта задача подтверждает, что внешность обманчива.
Шар хорошо виден, но лёгок.
Вода же гораздо массивнее и, следовательно, куда важнее, но её легко упустить из вида (может, потому что она прозрачная).
Я потратил время на видимое и несущественное, а упустил существенное и невидимое.
В физике как и в жизни, пустышки могут привлечь больше внимания, чем того заслуживают.

\section{Аквалангист в цистерне}

\paragraph{Вопрос.}
Аквалангист со слегка положительной плавучестью плавает на глубине большой прямоугольной цистерны; ему приходиться работать ластами чтобы не всплыть.
Рядом с этой цистерной стоит такая же, заполненная водой до того же уровня, но без аквалангиста.%
\footnote{Иными словами, объём воды вместе с аквалангистом в первой цистерне равен объёму воды во второй.}
Какая из цистерн тяжелее?

\paragraph{Парадокс.} Вот два противоречащих друг другу рассуждений:

(A): С одной стороны, первая цистерна явно легче, поскольку объём её содержимого совпадает с объёмом содержимого второй цистерны, а плотность аквалангиста меньше плотности воды.

(B): С другой стороны, так как глубина воды в цистернах одинакова, давление воды на их дно также одинаково.
Значит, цистерны и весят одинаково.

Где же ошибка?

\paragraph{Ответ.}
Верно рассуждение (A) --- против алгебры не попрёшь.
Остаётся найти ошибку в (B).
Плавучий аквалангист должен работать ластами, чтобы оставаться под водой;
в движущейся воде давление не обязано быть таким же, как в неподвижной воде на той же глубине.%???воде-воде
Более того, само понятие глубины плохо определено, ведь поверхность воды не вполне плоская.
Когда аквалангист работает ластами он направляет воду вверх, чтобы удержаться под водой,
и над ним образуется небольшой водяной холм.
Из рассуждения (A) мы заключаем, что среднее давление воды на дно цистерны с аквалангистом меньше, чем в другой цистерне.

Следующий вопрос я оставлю читателю:

\paragraph{Задача.}
Вертолёт завис над водой.
Поток воздуха от его лопастей создаёт на поверхности воды небольшое углубление.
Выполняется ли для него закон Архимеда?
Иначе говоря, равен ли вес вертолёта приблизительно весу вытесненной воды?
Будем считать, что движение воздуха и поверхность воды установились.

\section{Проблема с весом}\label{Проблема с весом}

Два сосуда на рисунке \ref{pic:4.5} наполнены водой до одного уровня.
Днища обоих имеют одинаковую форму и размер.

\paragraph{Вопрос.}
Верно ли, что сила давления воды на дно в левом сосуде больше, чем в правом?

\begin{figure}[ht!]
\centering
\begin{lpic}[t(2mm),b(2mm),r(0mm),l(0mm)]{pics/4.5}
\end{lpic}
\caption{Верно ли, что вода в левом сосуде оказывает большее давление на дно?}
\label{pic:4.5}
\end{figure}

\paragraph{Ответ.}
Нет, неверно: на оба дна действует одинаковая сила.
Причина в том, что давление (то есть сила, приходящаяся на единицу площади) зависит только от глубины и не зависит от формы сосуда: будь то кружка или озеро, давление на одной и той же глубине одинаково!
Поскольку дно обоих сосудов находятся на одной глубине, давления там равны.
А поскольку площадь дна та же, то и силы, действующие на них, равны.

\paragraph{Вопрос.}
А что не так вот с таким рассуждением: «Поскольку вода в правом сосуде весит меньше, она давит на дно с меньшей силой»?

\paragraph{Ответ.} На рисунке \ref{pic:4.6} показано, что дно сосуда с горлышком испытывает силу, б\'{о}льшую, чем вес самой воды.
Эта избыточная сила равна направленной вниз силе от свода сосуда.
Действительно, согласно первому закону Ньютона, силы, действующие на воду, находятся в равновесии: силы, направленные вниз, уравновешивают силу, направленную вверх:
\[\text{вес} + \text{сила вниз}=\text{сила вверх}
\qquad\text{и}\qquad
\text{вес}<\text{сила вверх}.
\]
Итак, вес воды меньше, чем сила, с которой дно действует на воду снизу.

Узкое горлышко создаёт то же давление, что и широкое!

\begin{figure}[ht!]
\centering
\begin{lpic}[t(2mm),b(2mm),r(0mm),l(0mm)]{pics/4.6}
\end{lpic}
\caption{Сила, с которой вода действует на дно, зависит только от площади дна и глубины, но не от веса воды.}
\label{pic:4.6}
\end{figure}

