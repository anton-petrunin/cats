\chapter{О движениях кошки и Земли}

\section{Как кошка разворачивает лапы вниз?}\rindex{кошка}

Если держать кошку лапами вверх и отпустить, то за долю секунды она развернётся лапами вниз.
Как у неё это может получиться без всякой опоры?%
\footnote{Попробуйте сами развернуться на вращающемся стуле, не касаясь пола.}
Некоторые утверждают, что для этого кошка крутит хвостом.
Но если внимательно рассмотреть её движения, можно убедиться, что это неверно.
Известно например, что бесхвостые кошки переворачиваются не хуже хвостатых.
Кроме того, расчёты показывают, что для выполнения переворота на 180° за долю секунды кошке пришлось бы вращать хвост слишком быстро, так что его кончик должен был бы преодолеть звуковой барьер или как минимум приблизиться к нему.
Это должно сопровождаться грохотом или, как минимум, громким свистом.
К тому же чудовищная центробежная сила оторвала бы хвост, сделав его смертельно опасным снарядом, подобным пуле.
Так что хвостатая теория отметается.

Удивительней то, что вращение%
\footnote{Стого говоря, вместо слова вращения, надо говорить \emph{момент импульса}.
При отсутствии внешнего момента сил на летящую кошку её момент импульса должен сохраняться, то есть оставаться нулевым.
Дополнительные сведения о моменте импульса можно почерпнуть на странице \pageref{Момент импульса}.}%
, начавшись с нуля, должно оставаться нулевым, ведь на кошку не действуют никакие моменты сил, пока она падает.
Так как же кошке удаётся обойти условие нулевого вращения и перевернуться?

На рисунке \ref{pic:13.1} показано как действует идеальная кошка: два цилиндра, соединённые невесомой гибкой талией.
Начав с положения лапами вверх, кошка сгибается в талии.
Затем она изворачивает талию так, чтобы цилиндры поворачивались вокруг своих осей в противоположных направлениях, пока лапы не окажутся внизу.
Заметим, что цилиндры вращаются в разные стороны, так что суммарное вращение во время переворота остаётся нулевым --- этого и требовал упомянутый закон сохранения!
В этом суть кошачьей гениальности.
Наконец, кошка выпрямляет талию и готова к приземлению.

\begin{figure}[ht!]
\centering
\begin{lpic}[t(4mm),b(2mm),r(0mm),l(0mm)]{pics/13.1}
\lbl[b]{39,17;{\footnotesize сгибается, изворачивается, изворачивается и разгибается}}
\end{lpic}
\caption{Основные этапы: (1) сгиб; (2, 3) изворот; (4) разгиб.
Гениальность движений кошки при нулевом моменте импульса: после сгиба, вращения двух половинок тела компенсируют друг друга, таким образом, общее вращение остаётся нулевым.}
\label{pic:13.1}
\end{figure}

Кстати, изворачивание талии (шаги 2 и 3 на рисунке) не означает скручивания.%
\footnote{Ведь, держа в руках резиновый шланг в форме буквы U,
можно одновременно поворачивать концы шланга по и против часовой стрелки, не скручивая сам шланг.}
Если бы кошка не согнулась, а держала тело прямо, то поворот был бы невозможен.
Чтобы повернуть свой перед вокруг оси, прямой кошке пришлось бы повернуть зад в противоположную сторону, тем самым сохранив вращение нулевым.
Это скрутило бы кошачью талию в штопор.
В принципе, такая кошка смогла бы приземлиться на лапы, но с талией перекрученной на полный оборот.

Всё становится ещё ясней, если подумать о сороконожке на рисунке \ref{pic:13.2}.

\begin{figure}[ht!]
\centering
\begin{lpic}[t(2mm),b(2mm),r(0mm),l(0mm)]{pics/13.2}
\lbl[t]{17,1;{\footnotesize сгиб}}
\lbl[t]{40,1;{\footnotesize изворот}}
\lbl[b]{66,7;{\footnotesize разгиб}}
\end{lpic}
\caption{Как сороконожка могла бы поворачивать ножки вниз. Ножек больше, но принцип тот же.}
\label{pic:13.2}
\end{figure}

Идеальная кошка помогла нам понять суть дела, однако она не совсем кошка.\rindex{кошка}
Настоящие кошки сгибают тело, но не на 180°, а примерно на 45°.
Затем они изворачиваются, подобно нашей идеальной кошке.
Из-за того, что сгиб меньше, ей приходится больше изворачиваться до нужного переворота.

\section{Могут ли пассаты замедлить вращение Земли?}\label{Могут ли пассаты замедлить вращение Земли?}

\paragraph{Вопрос.}
Пассаты, дующие на восток, создают трение на поверхности океана.
Это трение действовало против вращения Земли на протяжении многих миллионов лет.
Могло ли оно замедлить вращение Земли?

\paragraph{Ответ.}
Нет, поскольку суммарный момент импульса Земли и атмосферы сохраняется.
Это означает, что общий эффект движения атмосферы на вращение Земли равен нулю.
Существуют и другие ветры (например, западные), действующие в противоположном направлении.
На самом деле Земля не ускоряет, а замедляет своё вращение --- главным образом из-за приливного торможения Луны.
По оценкам, земные сутки составляли часов шесть с половиной и за 4 миллиарда лет назад сократились до нынешних 24 часов.
Разумеется, за время человеческой истории это изменение не заметно.

По мере того как Земля передаёт свой момент импульса Луне%
\footnote{Я не учитываю влияние Солнца и гораздо более слабое влияние других планет.}%
, Луна постепенно удаляется от Земли, это напоминает манёвры длинного спутника на странице \pageref{Управление спутником}.
