\chapter{О движении кошки и Земли}

\section{Как кошка разворачивает лапы вниз?}

Если отпустить кошку лапами вверх, то за долю секунды она сможет  развернуться лапами вниз.
Как ей это удаётся сделать без всякой опоры?%
\footnote{Попробуйте сами развернуться на вращающемся стуле, не касаясь пола.}
Некоторые утверждают, что для этого кошка крутит хвостом.
Но если внимательно рассмотреть её движения, можно убедиться, что это неверно.
Известно например, что бесхвостые кошки переворачиваются не хуже хвостатых.
Кроме того, расчёты показывают, что для выполнения переворота на 180° за долю секунды кошке пришлось бы вращать хвост слишком быстро, так что его кончик должен был бы преодолеть звуковой барьер или как минимум приблизиться к нему.
Это должно сопровождаться грохотом или, как минимум, громким свистом.
К тому же чудовищная центробежная сила оторвала бы часть хвоста, превратив её в смертельно опасный снаряд, подобный пуле.
Так что хвостатая теория отметается здравым смыслом.

Что делает достижение кошки удивительным, так это то, что её вращение%
\footnote{Вместо слова вращения, правильней говорить \emph{момент импульса}. При отсутствии внешнего момента сил на летящую кошку её момент импульса должен сохраняться, то есть оставаться нулевым. Дополнительные сведения о моменте импульса можно почерпнуть на странице ???.}%
, начавшись с нуля, должно оставаться нулевым, поскольку на неё не действуют никакие крутящие моменты, пока она в воздухе.
Так как же кошке удаётся перевернуться, обойдя условие нулевого вращения?

На рисунке \ref{pic:13.1} показано как действовала бы идеальная кошка: два цилиндра, соединённые невесомой гибкой талией.
Начав с положения лапами вверх, кошка сгибается в талии.
Затем она изгибает талию так, чтобы цилиндры поворачивались вокруг своих осей в противоположных направлениях, пока лапы не окажутся внизу.
При этом цилиндры вращаются в разные стороны, так что суммарное вращение во время переворота остаётся нулевым — как того требует упомянутый закон сохранения!
В этом и заключается кошачья гениальность.
Наконец, кошка выпрямляет талию и готова к приземлению.

\begin{figure}[ht!]
\centering
\begin{lpic}[t(4mm),b(2mm),r(0mm),l(0mm)]{pics/13.1}
\lbl[b]{7,17;{\footnotesize сгибается}}
\lbl[b]{36,17;{\footnotesize изгибается . . . изгибается}}
\lbl[b]{65,17;{\footnotesize разгибается}}
\end{lpic}
\caption{}
\label{pic:13.1}
\end{figure}

Кстати, изгибание талии (шаги 2 и 3 на рисунке) не означает её скручивания.%
\footnote{Ведь, держа в руках резиновый шланг в форме буквы U,
можно одновременно поворачивать концы шланга по и против часовой стрелки, не закручивая сам шланг.}
Если бы кошка не согнулась, а держала тело прямо, то поворот был бы невозможен.
Чтобы повернуть перед вокруг её оси, прямой кошке пришлось бы повернуть зад в противоположную сторону, тем самым сохранив вращение нулевым.
Это скрутило бы кошачью талию в штопор.
В принципе, такая кошка смогла бы приземлиться на лапы, но с талией перекрученной на полный оборот.

Всё становится ещё ясней, если подумать о сороконожке на рисунке \ref{pic:13.2}.

\begin{figure}[ht!]
\centering
\begin{lpic}[t(2mm),b(2mm),r(0mm),l(0mm)]{pics/13.2}
\lbl[t]{17,1;{\footnotesize сгиб}}
\lbl[t]{40,1;{\footnotesize изгиб}}
\lbl[b]{66,7;{\footnotesize разгиб}}
\end{lpic}
\caption{Как сороконожка могла бы поворачивать ножки вниз. Ножек больше, но принцип тот же.}
\label{pic:13.2}
\end{figure}

Наша идеальная кошка передаёт суть дела хоть она и не совсем кошка.
Настоящие кошки сгибают тело, но не на 180°, а примерно на 45°.
Затем они изгибаются, подобно нашей идеальной кошке.
Из-за того, что изгиб меньше, ей приходится больше изгибаться до нужного переворота.

\section{Могут ли пассаты замедлить вращение Земли?}

\paragraph{Вопрос.}
Пассаты, дующие на восток, создают трение на поверхности океана.
Это трение действовало против вращения Земли на протяжении многих миллионов лет.
Могло ли оно замедлить вращение Земли?

\paragraph{Ответ.}
Нет, поскольку суммарный угловой момент Земли и атмосферы сохраняется.
Это означает, что общий эффект движения атмосферы на вращение Земли равен нулю.
Существуют и другие ветры (например, западные), действующие в противоположном направлении.
На самом деле Земля не ускоряет, а замедляет своё вращение — главным образом из-за приливного торможения Луны.
По оценкам, когда-то земные сутки составляли часов шесть с половиной и за примерно 4 миллиарда лет увеличились до нынешних 24 часов.
Разумеется, за время человеческой истории это изменение пренебрежимо мало.

По мере того как Земля передаёт свой угловой момент Луне%
\footnote{Я не учитываю влияние Солнца и гораздо более слабое влияние других планет.}%
, Луна постепенно удаляется от Земли, это напоминает манёвры длинного спутника на странице \pageref{Управление спутником}.
