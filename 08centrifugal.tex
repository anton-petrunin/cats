\chapter{Центробежные парадоксы}

\section{Куда дешевле лететь, на запад или на восток?}

\paragraph{Задача.}
Известно, что на полёт из Бостона в Лондон расходуется меньше топлива, чем на обратный рейс.
Это происходит потому, что ветер дует примерно в восточном направлении.
Но представим себе, что ветер волшебным образом исчез.
Исчезло бы тогда и различие в расходе топлива?
Чтобы не отвлекаться на мелочи, давайте заменим Бостон с Лондоном парой точек $A$ и $B$ на экваторе и спросим: при отсутствии ветра будет ли перелёт на восток из $A$ в $B$ требовать столько же топлива, сколько перелёт на запад из $B$ в $A$?

\paragraph{Решение.}
Из-за вращения Земли полёт на восток потребует меньше топлива.
Действительно, каждая точка экватора совершает орбитальное движение вокруг центра Земли.
Если самолёт движется на восток, то это \emph{увеличивает} его орбитальную скорость.
Значит, увеличивается центробежная сила, и самолёт становится чуть легче.
Ну а если самолёт легче, то и расход топлива меньше.%
\footnote{Тяжёлый самолёт расходует больше топлива даже при горизонтальном полёте.
Ведь самолёт удерживается в воздухе, отталкивая воздух вниз за счёт положительного угла атаки своих крыльев.
Чем тяжелее самолёт, тем больше воздуха ему приходится ускорять вниз.
Для этого приходится увеличивать угол атаки.
Как следствие, требуется большая тяга двигателей.}

Ну а насколько легче?
При скорости полёта 250 м/с разница в весе составит около $
\tfrac23 \%$ (две трети процента).%
\footnote{Чуть ниже мы увидим, что отношение разности весов к истинному весу вычисляется по формуле
$4v\omega/g$,
где $v$ --- скорость самолёта,
$\omega$ --- угловая скорость вращения Земли,
а $g$ --- ускорение свободного падения.}
Загруженный Boeing 747 может весить 300 тонн.
Значит, разница составит тонны две, то есть человек 30 без багажа!

Можно думать, что самолёт, по сути, спутник, только очень медленный:
б\'{о}льшую часть веса держат его крылья, но немного веса приходится на центробежную силу.

Однако напомню, что влияние ветра всё же сильней центробежной силы.

\paragraph{Задача.}
А как, зная вес самолёта в покое, подсчитать, насколько он станет легче за счёт центробежной силы?

\paragraph{Решение.} Разница между весом при полёте на восток и на запад равна разнице центробежных сил%
\footnote{См. страницу \pageref{sec:A.9}.}%
:
\[\Delta W
=
\frac{m v_{\text{восток}}^{2}}{R}-\frac{m v_{\text{запад}}^{2}}{R}
=
\frac mR\left((\omega R+v)^{2}-(\omega R-v)^{2}\right),
\]
где $\omega$ --- угловая скорость вращения Земли, $R$ --- радиус Земли, а $v$ --- скорость самолёта.
Раскрыв скобки и сократив, получим
\[
\Delta W=4 m \omega v.
\]
Значит, отношение к весу равно
\[
\frac{\Delta W}{W}=\frac{4 m \omega v}{mg}=\frac{4 \omega v}{g}.
\]

\section{Парадокс с силой Кориолиса}\label{Парадокс с Кориолисом}

Человек, идущий с (постоянной) скоростью $v$ по платформе,
вращающейся с (постоянной) угловой скоростью $\omega$, испытывает
ускорение Кориолиса:
\begin{equation}
a=2 \, \omega v.
\label{eq:8.1}
\end{equation}
Эта формула приводится в нескольких учебниках механики, упомянутых
на странице \pageref{Арнольд-Лифшиц}.
Чуть ниже я «выведу» похожую формулу, но без множителя $2$:
\begin{equation}
a=\omega v.
\label{eq:8.2}
\end{equation}

\paragraph{Вопрос.} Попробуйте найти, где потерялась половина силы Кориолиса в следующем рассуждении.

\begin{figure}[ht!]
\centering
\begin{lpic}[t(2mm),b(2mm),r(0mm),l(0mm)]{pics/8.1}
\lbl[tr]{10,19;$v\,\Delta t$}
\lbl[br]{3,23;$\omega v\,\Delta t$}
\end{lpic}
\caption{Что не так с этим «доказательством»?}
\label{pic:8.1}
\end{figure}

\paragraph{«Вывод» формулы \eqref{eq:8.2}.}
Предположим, что я иду по радиусу вращающейся платформы
(см. рисунок~\ref{pic:8.1}) со скоростью $v$.
За время $\Delta t$ я отойду на расстояние $r=v\,\Delta t$ от центра.
Из-за вращения платформы, моя скорость, перпендикулярная радиусу, будет равна
\[\omega r=\omega v\,\Delta t.\]
Таким образом, за время $\Delta t$ моя скорость в направлении, перпендикулярном радиусу, изменилась на величину
$\Delta v=\omega v\,\Delta t$;
следовательно, ускорение равно
\[\Delta v/\Delta t
= \omega v\,\Delta t/\Delta t
= \omega v.
\]
Итак, я получил равенство \eqref{eq:8.2}.
Где же ошибка?

\begin{figure}[ht!]
\centering
\begin{lpic}[t(2mm),b(2mm),r(0mm),l(0mm)]{pics/8.2}
\lbl[l]{12,37;\parbox{32mm}{\centering
забытый член:\\
$v\sin(\omega\,\Delta t)\approx v\omega\,\Delta t$}}
\lbl[r]{-1,25;$v\omega\,\Delta t$}
\lbl[r]{-1,17;$v\omega\,\Delta t\cos(\omega\,\Delta t)\approx v\omega\,\Delta t$}
\end{lpic}
\caption{Место пропажи множителя 2.}
\label{pic:8.2}
\end{figure}

\paragraph{Решение.}
Рисунок~\ref{pic:8.1} не вполне верен; его надо заменить на рисунок~\ref{pic:8.2}.
Я упустил из виду то, что радиус, по которому я шёл, повернулся на угол $\omega\Delta t$,
а вместе с ним повернулся и вектор моей скорости.
Это добавляет слагаемое $v\sin(\omega\,\Delta t)\approx\omega v\,\Delta t$ к изменению скорости в перпендикулярном направлении к \emph{изначальному} радиусу --- та самая потерянная половинка!

Итак, множитель $2$ в формуле \eqref{eq:8.1} складывается из двух частей:
(1) из различия скоростей между точками платформы и
(2) из изменения направления движения вследствие поворота платформы.


\section{Что держит стоячий маятник?}

\paragraph{Вопрос.}
Маятник --- это грузик на стержне.
У него есть два положения равновесия: висячее устойчиво, а стоячее --- неустойчиво.
Если установить маятник стоя на опоре, то малейшее движение заставит его упасть.
Разумеется, можно удерживать маятник в равновесии, как метлу на ладони,
что требует осмысленной реакции на её движение.
А что произойдёт, если мы просто будем трясти точку его подвеса (скажем, вверх-вниз)?

\paragraph{Ответ.}
Если точку подвеса достаточно быстро трясти в вертикальном направлении, то стоячее положение окажется устойчивым.
Это поразительное явление известно больше сотни лет \cite{stephenson1908}.
При этом равновесие удерживается совсем не так, как при использовании обратной связи:
трясущемуся подвесу наплевать на то, что делает маятник, и он никак не реагирует на его движение.
Удивительно, что тряска может оказывать столь умное действие:
в конце концов, почему бы быстрым движениям вверх-вниз просто не компенсировать друг друга?
И почему тряска помогает устойчивости, а не неустойчивости?

\begin{figure}[ht!]
\centering
\begin{lpic}[t(7mm),b(10mm),r(0mm),l(0mm)]{pics/8.3}
\lbl[tl]{0,95;(a)}
\lbl[tl]{0,52;(b)}
\lbl[t]{25,96;\parbox{32mm}{\footnotesize\centering
стоячий маятник\\
нестабилен}}
\lbl[rb]{5,88;{\footnotesize грузик}}
\lbl[t]{9,75,-70;{\footnotesize стержень}}
\lbl[tr]{36,79;$\theta$}
\lbl[tl]{37,70;$mg$}
\lbl[tl]{43,86;$mg\sin\theta$}
\lbl[t]{13,63;\parbox{32mm}{\footnotesize\centering
упор\\
(стационарный)}}
\lbl[b]{21,70,-90;{\footnotesize притяжение}}
\lbl[lb]{15,50;{\footnotesize стабилен!}}
\lbl[r]{7,24;{\footnotesize шарнир}}
\lbl[b]{18,24,-90;{\footnotesize быстро елозит}}
\lbl{12,4;\parbox{22mm}{\footnotesize\centering электро-\\лобзик}}
\lbl[b]{42,30,37;{\footnotesize направление тряски}}
\lbl[l]{68,24;\parbox{22mm}{\footnotesize
сила\\
тяжести}}
\lbl[t]{11,-1;\parbox{42mm}{\footnotesize\centering
стоячий маятник стабилен если опора трясётся достаточно часто}}
\end{lpic}
\caption{(a) Стоячий маятник нестабилен, но (b) он становится стабильным, если опору сильно трясти.}
\label{pic:8.3}
\end{figure}

\paragraph{Эксперимент.}
На рисунке \ref{pic:8.3} показан алюминиевый стержень, который нежёстко приделан к лезвию электролобзика.%
\footnote{Это демонстрируется в моём видеоролике \cite{LeviInvertedPendulum}.}
Когда я включаю лобзик, лезвие начинает быстро ходить вверх-вниз --- примерно 30 раз в секунду.
При этом возникает ощущение, словно пружина удерживает стержень параллельно движению лезвия.
Эта воображаемая пружина достаточно сильна: она способна держать маятник почти горизонтально, если лобзик повернуть, как показано на рисунке \ref{pic:8.3}b, справа.

\paragraph{Вопрос.}
Как же тряске удаётся стабилизировать маятник?
(Только без формул!)

\paragraph{Ответ.}
Вместо того чтобы рассматривать стержень, как на рисунке
\ref{pic:8.3} или в моём видеоролике, будем думать о грузике, закреплённом на конце невесомого стержня.
Ускорение точки подвеса настолько велико, что гравитацией (малой по сравнению с ним) можно пока пренебречь.
Со стороны стержня грузик испытывает сильное притягивание и отталкивание поочерёдно.
Поскольку это толкательно-тянущие усилия направлены \emph{точно вдоль стержня}, грузик старается двигаться в направлении стержня.
Таким образом, грузик стал бы двигаться по кривой траектории, как на рисунке \ref{pic:8.4}a.
Такая траектория называется \emph{кривой погони} или \emph{трактрисой}.%
\footnote{Трактриса определяется как кривая, для которой данная прямая отсекает от касательных отрезки равных длин.
Например, если катить переднее колесо велосипеда по прямой, то его заднее колесо будет двигаться по трактрисе.}
Давайте временно считать, что движение грузика ограничено этой трактрисой.
Это довольно безобидное ограничение, ведь оно не мешает сильным толчкам и рывкам со стороны стержня.
При таком ограничении грузик будет совершать быстрые колебания взад-вперёд по короткой дуге $AB$.
Так как дуга изогнута, грузик будет испытывать центробежную силу (см. рисунок \ref{pic:8.4}b), и грузик захочет двигаться в направлении этой центробежной силы!
Если теперь снять ограничение, то грузик подчинится своему желанию.
Если тряска достаточно сильна, эта сила превзойдёт дестабилизирующее действие силы тяжести%
\footnote{Более подробное обсуждение даётся в \cite[стр. 158]{levi1999}.
Удивительно, но для того, чтобы превратить интуитивную идею в критерий устойчивости, требуется всего пара строк: стоячее положение устойчиво, если  $
\langle v^2 \rangle \ge g \ell$,
где $v$ --- скорость точки подвеса, $\langle \cdot \rangle$ обозначает среднее за период работы электролобзика, а $\ell$ --- длина маятника.
Наше физическое объяснение заменяет длиннющее вычисление и при этом оно объясняет, что происходит на самом деле.}%
, и маятник будет стоять вертикально, что и завершает объяснение.

\begin{figure}[ht!]
\centering
\begin{lpic}[t(7mm),b(2mm),r(0mm),l(0mm)]{pics/8.4}
\lbl[tl]{0,37;(a)}
\lbl[tl]{35,37;(b)}
\lbl[t]{8,12,75;{\footnotesize трактриса}}
\lbl[tl]{11,23;$A$}
\lbl[tl]{18,30;$B$}
\lbl[br]{40,14;$v$}
\lbl[bl]{41.8,9;$\theta$}
\lbl[bl]{61,31;$u=v\sin\theta$}
\lbl[tl]{60,17;$g\sin\theta$}
\lbl[b]{44,31;$ku^2$}
\end{lpic}
\caption{Неочевидная центробежная сила отвечает за устойчивость стоячего маятника.}
\label{pic:8.4}
\end{figure}

\paragraph{Ловушка Паула.}
Описанное явление известно не меньше века;
самое раннее упоминание, которое мне удалось найти, содержится в указанной выше статье Стивенсона 1908 года.
Тот же самый эффект стабилизации с помощью тряски, но в иной форме, используется в так называемой ловушке Паула --- устройстве для удерживания заряженных частиц в вакууме с помощью вибрирующих электрических полей.%
\footnote{Если бы я оказался в системе отсчёта стержня в эксперименте с лобзиком, то мне бы казалось, что вибрирует сила тяжести.}
За это изобретение Вольфганг Пауль был удостоен Нобелевской премии.
Объяснение Паула основано на дифференциальных уравнениях.
Но я не советую рассказывать его случайному прохожему (даже не пытайтесь, особенно если вы не в своём районе).

\section{Антигравитационная патока}

\paragraph{Вопрос.}
Банка с крышкой наполовину наполнена патокой или другой густой тяжёлой жидкостью вроде мёда.
Если перевернуть банку вверх дном, патока, разумеется, начнёт перетекать вниз.
А можно ли двигать банку так, чтобы патока не вытекала даже тогда, когда банка перевёрнута (рисунок \ref{pic:8.5})?
Иными словами, можно ли удержать патоку в перевёрнутой банке?

\begin{figure}[ht!]
\centering
\begin{lpic}[t(2mm),b(2mm),r(0mm),l(0mm)]{pics/8.5}
\lbl{10,11;\textcolor{white}{\footnotesize патока}}
\end{lpic}
\caption{Тряска удерживает патоку от перетекания вниз.}
\label{pic:8.5}
\end{figure}

\paragraph{Ответ.}
Если банку быстро трясти в направлении её оси, то патока не будет вытекать даже после переворачивания.%
\footnote{Этот эксперимент обсуждается в статье Михаэлиса и Вудварда
\cite{michaelis1991}.
Теоретическое обоснование приводит Вольф \cite{wolf1970}.}
Я воспроизвёл этот опыт, воспользовавшись подручными средствами из мастерской и кухни, сделав так, чтобы банка при включённом электролобзике тряслась вдоль оси.
Если всю эту конструкцию перевернуть вверх дном, то патока не вытечет, а чудесным образом остаётся наверху, как будто сила тяжести изменила своё направление.
Удивительным образом тряска удерживает поверхность патоки ровной и не даёт ей течь.
Ещё больше впечатляет следующее: если повернуть банку вбок (при работающем лобзике), то поверхность патоки будет оставаться вертикальной, будто это стена воды в расступившемся Красном море.

\section{Почему праща не может работать}\label{Почему праща не может работать}

\paragraph{Парадокс.}
Представим себе, что я раскручиваю камень на верёвке,
верёвка натягивается и действует на камень силой $T$.
Эта сила направлена прямо к опоре --- точке, скажем $P$, где мои пальцы держат верёвку.
Следовательно, момент силы%
\footnote{Обсуждается на странице \pageref{sec:A.5}.}
$T$ относительно точки $P$ равен нулю.
Но нулевой момент силы означает, что момент импульса камня не меняется.
То есть $Lv=\mathrm{const}$, где
$L$ --- длина верёвки, а $v$ --- скорость камня, в направлении перпендикулярном верёвке.
Значит, и $v$ не меняется.
Однако уже Голиаф убедился, что это не так --- где же ошибка?

\begin{figure}[ht!]
\centering
\begin{lpic}[t(2mm),b(2mm),r(0mm),l(0mm)]{pics/8.6}
\lbl[br]{12,42;$B$}
\lbl[l]{19,11;\parbox{32mm}{\footnotesize траектория\\ руки}}
\lbl[l]{28,36;\parbox{28mm}{\footnotesize камень ускоряется к убегающему равновесию в точке $B$}}
\lbl[r]{-1,0;\parbox{32mm}{\footnotesize\raggedleft  ускорение\\ руки}}
\lbl[b]{8,27,70;\parbox{32mm}{\footnotesize\centering ощущаемая\\ гравитация}}
\end{lpic}
\caption{Праща --- это маятник, бегущий к постоянно ускользающему положению равновесия.}
\label{pic:8.6}
\end{figure}

\paragraph{Ответ.} То, что
\[
\text{момент силы}=0 \quad\Rightarrow\quad \text{момент импульса}=\text{const}
\]
справедливо в инерциальных системах отсчёта.
А система отсчёта, связанная с моими пальцами,
вовсе не инерциальна ---  мне приходится ускорять пальцы при вращении.

\paragraph{Как же работает праща?}
Ответ можно увидеть на рисунке~\ref{pic:8.6}.
По сути, мы создаём маятник, который всё время скользит вниз по наклонной, \emph{в погоне за ускользающим от него положением равновесия~$B$}.
Моя рука~--- точка подвеса маятника~$P$~--- движется (скажем) по окружности,
всё быстрее и быстрее.
Наблюдатель, связанный с точкой~$P$, будет ощущать перегрузку, показанную на рисунке.
В частности, он увидит, что камень ускоряется как бы вниз к положению равновесия~$B$,
так же как обычный маятник ускоряется вниз к своей нижней точке.
При этом точка $B$ постоянно ускользает от камня, двигаясь против часовой стрелки, и камню приходится гнаться за~$B$, всё время разгоняясь.

Следующая задача раскрывает другой неожиданный аспект физики пращи.

\section{Задача Давида и Голиафа}\label{Задача Давида и Голиафа}

Напомним из предыдущей задачи, что праща --- это камень на верёвке, который раскручивают, а затем отпускают.%
\footnote{Здесь и далее мы пренебрегаем гравитацией.}
Следующий парадокс возник при попытке разобраться в физике пращи.
В ответ может быть трудно поверить.

\paragraph{Задача о праще.}
Я веду конец верёвки по окружности так, что камень на другом её конце движется
по большей концентрической окружности, и при этом угол опережения верёвки по отношению к скорости камня остаётся постоянным, скажем $\theta=45$°.
Камень будет вращаться всё быстрее, благодаря касательной составляющей
$T_{\text{кас}}$ силы натяжения верёвки.
(Мне придётся вращать пальцы также быстрее, чтобы обеспечить постоянный угол 45°).
Предположим, что камень движется по окружности радиуса $1 \,\text{м}$.
Попробуйте прикинуть, сколько понадобится времени,
чтобы разогнать его с начальной скорости $1 \,\text{м/с}$
до скорости звука $(330 \,\text{м/с})$?
До скорости света $(300{,}000{,}000 \,\text{м/с})$?%
\footnote{Притворимся, что ньютоновская механика применима при любых скоростях, так что разрешается двигаться быстрее скорости света.
Также не будем учитывать гравитацию и сопротивление воздуха, а также ограничения на прочность верёвки и человеческие возможности.}

\begin{figure}[ht!]
\centering
\begin{lpic}[t(2mm),b(2mm),r(0mm),l(0mm)]{pics/8.7}
\lbl[r]{10,8;{\footnotesize 45°}}
\lbl[r]{56.7,6.6;{\footnotesize 45°}}
\lbl[rb]{14,10;{\footnotesize 1\,м}}
\lbl[b]{23,13,45;{\footnotesize тяга}}
\lbl[tl]{15,1;{\footnotesize ракета}}
\lbl[t]{62,1;{\footnotesize камень}}
\lbl[l]{30,12;\parbox{32mm}{\footnotesize путь\\ ракеты}}
\lbl[br]{67,10;$T$}
\end{lpic}
\caption{Ускорение прямо пропорционально скорости в квадрате.}
\label{pic:8.7}
\end{figure}

\paragraph{Задача о ракете.}
Игрушечная ракета летит по окружности радиусом $1 \,\text{м}$,
её двигатели постоянно направлены с опережением под фиксированным углом
$\alpha=45$° к центру (см. рисунок~\ref{pic:8.7}).%
\footnote{Это требует постоянно увеличивающейся тяги.}
Сколько времени понадобится ракете,
чтобы увеличить свою скорость с $1 \,\text{м/с}$ до скорости звука?
До скорости света?

\paragraph{Решение.}
Камень (как и ракета) \emph{превысит скорость света --- не говоря уже о скорости звука --- менее чем за секунду!}
Скорость стремится к бесконечности, когда время приближается к отметке в одну секунду.
Это попросту означает, что в принципе невозможно продолжительно раскручивать камень по окружности, поддерживая угол опережения 45° (или любой другой положительный угол).
Вот тому объяснение.

\paragraph{Объяснение.}
Я покажу, что касательное ускорение камня $a_{\text{кас}}$ прямо пропорционально квадрату скорости $v$ ---
а именно,
\[
a_{\text{кас}}=v^{2}.
\]
В частности скорость изменения $v$ прямо пропорциональна $v^2$.
И, как покажут выкладки чуть ниже, такая величина подойдёт к бесконечности за конечное время.

\paragraph{Подробные расчёты.}
Так как угол между силой натяжения $T$ и касательной равен 45°,
из рисунка~\ref{pic:8.7} следует, что касательная и радиальная компоненты силы $T$ равны.
То же самое справедливо и для касательного и центростремительного ускорений:
$a_{\text{кас}}=a_{\text{цен}}$.
Но центростремительное ускорение (см.~страницу~\pageref{sec:A.9}) задаётся формулой
\[
a_{\text{цен}}=\frac{v^2}{r}=v^2 \quad (\text{ведь}\  r=1\,\text{м}).
\]
Значит,
\begin{equation}
a_{\text{кас}}=v^2. \label{eq:8.3}
\end{equation}
Теперь начинается матанализ.
Сейчас мы покажем, что в силу этого соотношения
скорость $v$ достигнет бесконечности за конечное время.
Уравнение~\eqref{eq:8.3} можно переписать как
\begin{equation}
\frac{1}{v^2}\,\frac{dv}{dt}=1. \label{eq:8.4}
\end{equation}
Взяв первообразную, получим
\[
-\frac{1}{v}=t+c,
\]
где $c=-1$, ведь $v=1$ при $t=0$.
Значит
\[
\frac{1}{v}=1-t.
\]
При $t=0.9999 \,\text{с}$ получаем $v=10{,}000 \,\text{м/с}$ ---
достаточно, чтобы запустить камень на орбиту Земли и почти достаточно,
чтобы преодолеть её гравитацию.
При $t=0.999999$ скорость превысит скорость света.
За какое-то время до 1 секунды кинетическая энергия камня
превысит суммарную энергию Солнца и других
звёзд Вселенной.
Вот такими могут оказаться вполне реалистичные предположения.

\paragraph{Вопрос.}
При $t > 1$ мы получаем
\[
v=\frac{1}{1-t} < 0,
\]
то есть камень будет двигаться назад.
Как объяснить такую нелепость?

\paragraph{Ответ.}
При $t > 1$ формула $v=1/(1-t)$ попросту не применима.\footnote{Напомним, что $v=1/(1-t)$ решает уравнение $\tfrac{dv}{dt}=v^2$ при $t<1$ и $t>1$, но в момент $t=1$ правая часть не определена и формула не даёт решения в окрестности~$1$.}

\paragraph{Чудесный банковский счёт.}
Мы увидели, что если скорость изменения $\tfrac{dv}{dt}$
некоторой величины $v$ прямо пропорциональна её квадрату $v^{2}$,
то $v$ подходит к бесконечности за конечное время.
Вообразим на минуту, что банк решает начислять проценты по такому принципу,
позволяя балансу $v$ изменяться по этому закону%
\footnote{То есть $\tfrac{dv}{dt}=kv^2$ вместо зависимости $\tfrac{dv}{dt}=kv$, используемой при обычном непрерывном начислении процентов.}%
, то есть начисляемые проценты прямо пропорциональны квадрату текущего баланса.
Для клиента это было бы прекрасно (а для банка --- ужасно).
В частности, баланс достиг бы бесконечности за конечное время.
Однако если клиент проворонит определённый момент (например, $t=1$ в нашем предыдущем примере), то баланс станет отрицательным.%
\footnote{Если считать, что наше решение
$v=\frac{1}{1-t}$ продолжается за точку разрыва. Вопрос применимости этой формулы с момента $t=1$ спорный и должен решаться в суде.}
В математике, прямо как в жизни --- внезапно огромное состояние может превратиться в огромный долг,
ведь если что-то убегает на $+\infty$ за конечное время, то часто возвращается из $-\infty$.

При этом странном начислении процентов клиенты станут выигрывать от объединения своих счетов.
Например, если два равных счёта объединяются в один,
то проценты увеличиваются в четыре раза, ведь
$(2v)^2=4v^2$,
то есть доход \emph{каждого} человека удвоится.
Это приблизит момент, когда клиенты станут бесконечно богатыми.
Обычное,
применяемое в банках, экспоненциальное начисление процентов
$\frac{dv}{dt}=kv$, является единственно разумным ---
в частности клиенты не выигрывают и не проигрывают
от объединения своих счетов.%
\footnote{Иными словами, дифференциальное уравнение, описывающее баланс, является линейным.
Для такого уравнения сумма двух решений также является решением.
Это означает, что объединённый счёт будет иметь тот же баланс, что и сумма двух счетов, если их вести раздельно.}
И ещё одно замечание: при экспоненциальном начислении ваша прибыль будет той же, независимо от того, в чём мерить баланс: в рублях, копейках или долларах.
Но это вовсе не так в случае начисления по закону $\tfrac{dv}{dt}=v^2$:
как только вы убедите банк мерить ваше богатство в копейках, скорость обогащения возрастёт в сотню раз!

\section{Вода в трубе}

\paragraph{Вопрос.}
На рисунке \ref{pic:8.8} вода течёт по изогнутой трубе.
Подойдя к изгибу, она пытается продолжать двигаться прямо
и давит на трубу, как показано на рисунке,
в том же направлении, в котором двигалась до поворота.
Верно ли указано направление силы?

\begin{figure}[ht!]
\centering
\begin{lpic}[t(5mm),b(2mm),r(10mm),l(0mm)]{pics/8.8}
\lbl[tl]{-10,41;(a)}
\lbl[tl]{35,41;(b)}
\lbl[tl]{-10,17;(c)}
\lbl[r]{-1,35;{\footnotesize ток}}
\lbl[b]{28,36;{\footnotesize сила}}
\lbl[br]{47,36;$v$}
\lbl[l]{52,29;$v'$}
\lbl[b]{71,37;$\Delta v=v'-v$:}
\lbl[l]{77,29;$v'$}
\lbl[tr]{73,23;$-v$}
\lbl[b]{71,31,45;$\Delta v$}
\lbl[t]{8,4, 45;\parbox{22mm}{\footnotesize\centering  сила на\\воду}}
\lbl[t]{44,4, 45;\parbox{22mm}{\footnotesize\centering  сила на\\трубу}}
\end{lpic}
\caption{В каком направлении вода действует на трубу в изгибе?}
\label{pic:8.8}
\end{figure}

\paragraph{Ответ.}
Нет, сила на рисунке показана неверено.
На самом деле она направлена вверх и вправо, под углом 45° к обоим прямым участкам трубы.
Нельзя забывать, что вода поворачивает вниз, и, значит, она обязана толкать трубу вверх.
Точнее говоря, посмотрим на изменение импульса частицы воды при прохождении изгиба.
Скорость частицы изменилась с $v$ на $v'$ (см. рисунок~\ref{pic:8.8}).
Приращение скорости $\Delta v=v'-v$ идёт по биссектрисе прямого угла, как показано на рисунке.
Согласно второму закону Ньютона, средняя сила, действующая на частицу, направлена вдоль приращения скорости,
а согласно третьему закону Ньютона, вода прикладывает к трубе равную по величине и противоположно направленную силу.

\paragraph{Ещё раз чуть по-другому.}
А вот ещё способ увидеть, что ответ на рисунке~\ref{pic:8.8}a неверен.
Хоть это не вполне строго, но можно считать, что сила, действующая на трубу, складывается из центробежных сил всех частиц на изгибе.
Но центробежная сила, действующая на частицу, равна $mv^{2}/r$.
Тут важно, что скорость $v$ возводится в квадрат;
в частности, замена $v$ на $-v$ ничего не меняет.
Но если рассуждать как на рисунке~\ref{pic:8.8}a, то при обратном токе воды сила была бы иной.
Значит, вектор скорости на рисунке был указан неверено.

\section{Натяжение колец}\label{Что сильней натягивается?}

Следующая задача имеет неожиданный ответ, простое решение и ещё более удивительное следствие, описанное на странице \pageref{Скользящие тросики в невесомости}.

\paragraph{Задача.}
Два кольца разного радиуса вращаются с той же (линейной) скоростью.
Они сделаны из тросиков разной длины, но идентичных во всём остальном.
Центробежная сила растягивает оба кольца.
Какое кольцо больше натягивается?
Будем считать, что тросик идеально гибок и нерастяжим, а внешние силы, включая силу тяжести, отсутствуют.

\begin{figure}[ht!]
\centering
\begin{lpic}[t(2mm),b(2mm),r(0mm),l(0mm)]{pics/8.9}
\lbl[lb]{1,1;{\footnotesize картонка}}
\lbl[tr]{21,6;$A$}
\lbl[br]{21,38.5;$B$}
\lbl[b]{15,7.7;$F$}
\lbl[t]{15,35;$F$}
\lbl[b]{8,38;$v$}
\lbl[t]{37,6;$v$}
\end{lpic}
\caption{Натяжение тросика.}
\label{pic:8.9}
\end{figure}

\paragraph{Ответ.}
Тросики натягиваются одинаково.
Чтобы это увидеть (почти без вычислений), рассмотрим половину кольца (см. рисунок~\ref{pic:8.9}) — закроем левую часть картонкой, просто чтобы её не было видно.
Мы увидим, что тросик растёт из точки $A$ и исчезает в точке $B$ с одной и той же скоростью $v$.
Между входом и выходом каждая частица изменяет свою скорость на $2v$.
Это изменение вызывается силой натяжения $F$ в точках $A$ и $B$.
Чтобы найти $F$, подождём время $\Delta t$ (скоро оно сократится).
За время $\Delta t$ некоторая масса $\Delta m$ выросла из точки $A$ и та же масса $\Delta m$ исчезла в точке $B$.
То есть масса $\Delta m$ изменила скорость на $2v$ за время $\Delta t$.
По второму закону Ньютона ($F=ma=m \Delta v / \Delta t$) получаем:
\[(2F)\Delta t=\Delta m \cdot (2v),
\quad\text{или}\quad
F=\frac{\Delta m}{\Delta t} v.
\]
Теперь заметим, что $\Delta m=\rho \cdot (v \Delta t)$, где $\rho$ --- линейная плотность тросика (масса на единицу длины).
Подставив это в последнее выражение, получим
\[
F=\rho v^2,
\]
так что натяжение действительно не зависит от радиуса окружности --- только от $v$ и $\rho$.

А вот более короткое доказательство того, что натяжение $F$ не зависит от радиуса.
Рассмотрим полуокружность и заметим, что сила $2F$ удерживает её центр масс на круговой орбите:
\[2F=\frac{mu^{2}}{r},\]
где $m$ --- масса полуокружности,
$r$ --- расстояние от центра масс полукольца до центра кольца,
а $u$ --- скорость движения центра масс.
Теперь отметим, что
(1) отношение $m/r$ не зависит от радиуса кольца $R$, ведь и $m$, и $r$ пропорциональны $R$;
(2) скорость $u$ не зависит от $R$ (а только от $v$).
Следовательно, и $F$ не зависит от $R$.

\section{Скользящие тросики в невесомости}\label{Скользящие тросики в невесомости}

Тросики могут вести себя неожиданно.
Вот например, если тросик замкнуть в кольцо%
\footnote{Будем считать, что наш тросик идеален: он нерастяжим, не сопротивляется изгибу и очень тонкий.
Можно думать о цепочке из маленьких шариков, как на офисных авторучках.}
и раскрутить
(рисунок~\ref{pic:8.9}), то в условиях невесомости будет сохраняться его круглая форма.


\begin{figure}[ht!]
\centering
\begin{lpic}[t(2mm),b(2mm),r(0mm),l(0mm)]{pics/8.10}
\end{lpic}
\caption{Какие из тросиков (если таковые найдутся) не будут меняться двигаясь в невесомости, если придать их частицам равные начальные скорости вдоль стрелок?}
\label{pic:8.10}
\end{figure}

\paragraph{Вопрос.}
Есть ли другие кривые, кроме окружности, вдоль которых тросик может скользить в невесомости, оставаясь неподвижным как целое?
Например, будут ли обладать этим свойством какие-нибудь из кривых на рисунке~\ref{pic:8.10}, если придать частицам тросика равную начальную скорость вдоль кривой?

Иначе говоря, представьте себе абсолютно гибкий шланг,
наполненный водой, в котором вода циркулирует без трения.
Какие из начальных положений шланга на рисунке~\ref{pic:8.10} не будут меняться во времени?

\paragraph{Ответ.}
В это трудно поверить, но при вышеописанных условиях \emph{любая} гладкая кривая будет сохранять свою форму \cite{routh1884}.

\paragraph{Объяснение.}
Всё станет ясно, если вспомнить (раздел \ref{Что сильней натягивается?}), что натяжение \emph{кругового} тросика не зависит от его радиуса.
Заметим, что любую кривую можно приблизить кривой, составленной из дуг окружностей разного радиуса.
Поскольку натяжение в каждой из этих дуг будет одинаковым, тросик охотно сохранит свою форму.

\paragraph{Более строгое объяснение.}
Сейчас потребуется немного векторного анализа.
Нам надо будет применить второй закон Ньютона к движущемуся тросику.
Для этого запараметризуем тросик его длиной $s$ от отмеченной точки.
Обозначим через $\mathbf{r}(s,t)$ радиус-вектор частицы с параметром $s$ в момент времени $t$ (рисунок~\ref{pic:8.11}).

\begin{figure}[ht!]
\centering
\begin{lpic}[t(2mm),b(2mm),r(0mm),l(0mm)]{pics/8.11}
\lbl[r]{12,10;$0$}
\lbl[b]{32,19,20;$\mathbf{r}(s,t)$}
\lbl[t]{39,12,7;$\mathbf{r}(s+\Delta s,t)$}
\lbl[b]{38,31,-22;$T \mathbf{r}'(s,t)$}
\lbl[b]{68,8,-50;$T \mathbf{r}'(s+\Delta s,t)$}
\end{lpic}
\caption{Второй закон Ньютона для движущегося тросика.}
\label{pic:8.11}
\end{figure}

Чуть ниже я покажу, что второй закон Ньютона можно записать как
\begin{equation}
\rho\, \ddot{\mathbf{r}}=(T \mathbf{r}')',
\label{eq:8.5}
\end{equation}
где точка (~$\dot{}=\partial/\partial t$) обозначает производную по времени,
штрих (~$'\z=\partial/\partial s$) --- производную по $s$,
а $T=T(s,t)$ --- натяжение тросика.
Поскольку тросик не растяжим,
\begin{equation}
\|\mathbf{r}'\|=1,
\label{eq:8.6}
\end{equation}
где $\|\cdot\|$ обозначает длину вектора.
Уравнения \eqref{eq:8.5}–\eqref{eq:8.6} образуют полную систему для неизвестных функций
$\mathbf{r}$ и $T$.
Теперь уже несложно показать, что любое скользящее движение тросика
сохраняет форму.
Пусть $\mathbf{R}=\mathbf{R}(s)$ --- параметризация кривой длиной её дуги.
Если частица начинает движение в точке $\mathbf{R}(s)$
и скользит вдоль кривой со скоростью $v$, то в момент $t$
она будет находиться в точке $\mathbf{R}(s+vt)$;
то есть $\mathbf{r}(s,t)=\mathbf{R}(s+vt)$.
Положим $T=\rho v^2$.
Тогда, подставив это в \eqref{eq:8.5}–\eqref{eq:8.6}, получим равенства (убедитесь в этом сами).
Это и доказывает наше утверждение.

Остаётся вывести \eqref{eq:8.5}.
На дугу $(s, s+\Delta s)$ действуют только две силы натяжения на концах.
Их равнодействующая равна
$(T \mathbf{r}')_{s+\Delta s}-(T \mathbf{r}')_s$,
здесь я не указал зависимость от $t$.
Центр масс дуги находится в точке $
\tfrac{1}{\Delta s} \int_s^{s+\Delta s} \mathbf{r}(\sigma, t) d\sigma$.
Ускорение центра масс --- это вторая производная по времени:
$\mathbf{a}=\tfrac{1}{\Delta s} \int_s^{s+\Delta s} \ddot{\mathbf{r}}(\sigma, t) d\sigma$.
Масса дуги вычисляется как $m \z= \rho \Delta s$, где $\rho$ --- линейная плотность, то есть масса тросика на единицу длины.
Таким образом, закон Ньютона: $m\mathbf{a}=\mathbf{F}$, принимает вид
\begin{equation}
(\rho \Delta s)
\cdot
\frac{1}{\Delta s}
\int\limits_s^{s+\Delta s}
\rho\,\ddot{\mathbf{r}}(\sigma, t)\, d\sigma
= (T \mathbf{r}_s)_{s+\Delta s}-(T \mathbf{r}_s)_s .
\end{equation}
Разделив обе части на $\Delta s$ и устремив $\Delta s \to 0$, получаем \eqref{eq:8.5}.

\medskip

Вот ещё несколько любопытных тем/задач для читателей, знакомых с векторным анализом:

\begin{enumerate}
\item Покажите, что момент импульса скользящего движения плоского тросика равен
\begin{equation}
L=\rho v A ,
\end{equation}
где $v$ --- скорость, а $A$ --- площадь, охватываемая тросиком.
\item Покажите, что $z$-координата момента импульса скользящего движения пространственного тросика равна
\begin{equation}
L_z=\rho v A_{xy},
\end{equation}
где $A_{xy}$ --- (ориентированная) площадь проекции тросика на плоскость $xy$.
Тоже верно и для любой прямой и плоскости, перпендикулярной ей.
\item Определим \emph{циркуляцию} тросика как интеграл
$\mathcal{C}=\int v_{\text{кас}} ds$,
то есть интеграл касательной скорости $v_{\text{кас}}$ по длине кривой $s$.
Покажите, что при \emph{любом} (не только скользящем) свободном движении тросика в невесомости
его циркуляция не меняется.
\end{enumerate}

