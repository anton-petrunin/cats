\chapter*{Приложение}
\addcontentsline{toc}{chapter}{Приложение}
%\addtocontents{toc}{\protect\contentsline{chapter}{\protect\numberline{}Приложение}{}{}}
\addtocounter{chapter}{1}
\setcounter{equation}{0}
\setcounter{figure}{0}

Ниже привдится краткий обзор понятий, упоминаемых в книге.

\section{Законы Ньютона}

Все законы Ньютона формулируются в инерциальной системе отсчёта, то есть в системе, которая движется без ускорения и без вращения.

\paragraph{Первый закон Ньютона.}
Тело продолжает двигаться с постоянной скоростью по прямой или остаётся в покое, пока сумма всех сил, действующих на него, остаётся равной нулю.

\paragraph{Второй закон Ньютона.}
Силы, действующие на тело, вызывают ускорение,
и это ускорение $\mathbf{a}$ прямо пропорционально сумме приложенных сил
$\mathbf{F}$:
\begin{equation}
m\mathbf{a}=\mathbf{F},
\label{eq:A.1}
\end{equation}
коэффициент пропорциональности $m$ называется \emph{массой}.
Здесь $a$ и $\mathbf{F}$ --- это векторы (см. рисунок~\ref{pic:A.1}).
Часто рассматривают проекцию равенства \eqref{eq:A.1} на какое-либо направление;
например, при изучении прямолинейного движения нас интересует только направление линии.
В таких случаях ускорение и силы рассматриваются как скалярные (в отличие от векторных) величины.

\begin{figure}[ht!]
\centering
\begin{lpic}[t(2mm),b(2mm),r(0mm),l(0mm)]{pics/A.1}
\lbl[lb]{14,13;{\footnotesize результирующая сила}}
\lbl[r]{-.5,7;$m$}
\lbl[l]{23,7;$\mathbf{a}$}
\lbl[lb]{16,8;$\mathbf{F}$}
\lbl[t]{11,1;$\mathbf{F}_1$}
\lbl[b]{6,13;$\mathbf{F}_2$}
\lbl[rb]{45,4;$\mathbf{F}$}
\lbl[lb]{50,4.7;$\mathbf{a}$}
\lbl[lb]{60,4.5;$\mathbf{v}$}
\end{lpic}
\caption{Второй закон Ньютона}
\label{pic:A.1}
\end{figure}

Скалярный варинат закону Ньютона можно переписать как $m\z=F/a$.
В частности, если ускорение $a=1$, то $m=F$;
то есть масса --- это сила, необходимая для сообщения телу единичного ускорения.
Именно так мы получаем интуитивное представление об инерции: мы чувствуем, какая сила нужна, чтобы ускорить тело.

Первый закон является частным случаем второго.
На мой взгляд, он сформулиран отдельно лишь потому, что это очень важный частный случай.

Типичная ошибка при применении законов Ньютона заключается в том, что при вычислении равнодействующей силы $F$ в \eqref{eq:A.1} некоторые силы не учитываются.
Несколько парадоксов в книге (например, 2.1, 4.2 и 4.4) основаны именно на этой ошибке.

\paragraph{Третий закон Ньютона.}
Два взаимодействующих тела действуют друг на друга с равными по величине и противоположно направленными силами:
если тело $A$ действует на тело $B$ силой $\mathbf{F}$,
то тело $B$ действует на тело $A$ силой $-\mathbf{F}$ (см. рисунок \ref{pic:A.2}).


\begin{figure}[ht!]
\centering
\begin{lpic}[t(2mm),b(2mm),r(0mm),l(0mm)]{pics/A.2}
%\lbl[t]{17,1;{\footnotesize сгиб}}
\end{lpic}
\caption{Третий закон Ньютона.}
\label{pic:A.2}
\end{figure}

\paragraph{Задача.}
Когда я тащу ящик по полу, сила, с которой я тяну ящик вперёд, равна силе, с которой ящик тянет меня назад.
Почему же выигрываю я, а не ящик?

\paragraph{Решение.}
Скрытая ошибка в неправильном расчёте сил, действующих на меня (и на ящик).
Если я тащу ящик с постоянной скоростью, то трение моих ног о землю уравновешивает силу, с которой ящик тянет меня назад.
Когда я начинаю тащить ящик, моё трение больше, чем сила тяги ящика, и я ускоряюсь.
В то же время, для ящика моя сила тяги больше его трения о пол, и он тоже ускоряется.
Обратите внимание, что нам ни разу не пришлось сравнивать силу, с которой я действую на ящик, с силой, с которой ящик действует на меня.

\section[Энергия и работа]{Кинетическая энергия, потенциальная энергия и работа}

Поскольку и кинетическая, и потенциальная энергии определяются через работу, определим сначала работу.

\subsection{Работа}

Рассмотрим постоянную силу $F$, перемещающую тело на расстояние $D$ (см. рисунок~\ref{pic:A.3}).
Определим работу $W$, совершаемую этой силой, как
\begin{equation}
    W = F D.
    \label{eq:A.2}
\end{equation}
Но что делать, если сила не направлена вдоль линии движения?
\begin{figure}[ht!]
\centering
\begin{lpic}[t(2mm),b(2mm),r(0mm),l(0mm)]{pics/A.3}
%\lbl[t]{17,1;{\footnotesize сгиб}}
\end{lpic}
\caption{Определение работы.}
\label{pic:A.3}
\end{figure}
В этом случае придётся изменить приведённое определение, взяв вместо силы её проекцию на направление движения:
\begin{equation}
W = F_1 D = (F \cos \theta) D.
\label{eq:A.3}
\end{equation}

\paragraph{Пример}\label{Работа:Пример}
Чтобы поднять груз веса $F = mg$ вертикально вверх на высоту $H$, нужно приложить силу $mg$ на протяжении вего пути длинной $H$, тем самым совершив работу $mgH$, см. рисунок~\ref{pic:A.4}.
Если же тащить тот же груз по наклонной плоскости на ту же высоту, то нужна меньшая сила: $F_1 = mg \sin \theta$,
но на большем расстоянии $D_1 = H / \sin \theta$;
работа в этом случае равна
$F_1 D_1$.
В этой формуле $\sin \theta$ сокращается.
Поэтому работа окажется той же, что и при вертикальном подъёме:
$W_1 = W = mgH$.
А ведь если бы было не так, то у нас была бы вечный двигатель.
Действительно, вообразите на мгновение, что $W_1 > W$.
Тогда мы могли бы поставить гибридный автомобиль в точке $C$,
скатить его вниз в $A$, тем самым зарядив батарею энергией $W_1$ (при отсутствии потерь),
затем проехать из $A$ в $B$ (что не требует работы, так как дорога горизонтальна),
а потом подняться на автомобильном лифте из $B$ в $C$, затратив энергию $W < W_1$.
Завершив цикл, мы получили бы избыток энергии $W_1 - W$ — слишком хорошо, чтобы быть правдой.


\begin{figure}[ht!]
\centering
\begin{lpic}[t(2mm),b(2mm),r(0mm),l(0mm)]{pics/A.4}
%\lbl[t]{17,1;{\footnotesize сгиб}}
\end{lpic}
\caption{Работа, необходимая для перемещения массы из точки $A$ в точку $C$ в гравитационном поле, не зависит от пути.}
\label{pic:A.4}
\end{figure}

А как определить работу в общем случае, когда сила меняется вдоль пути и сам путь не прямой?
В этом случае, путь разрезается на малые отрезки.
Каждый отрезок почти прямой, и сила на нём почти постоянна, так что определение \eqref{eq:A.3} применимо к каждому такому отрезку с хорошей точностью.
Затем результаты для всех отрезков следует сложить.
Чем мельче разбиение пути, тем ближе сумма к тому, что мы называем истинной работой.%
\footnote{Формально говоря, работа определёется как интеграл
\begin{equation}
    W = \int\limits_C \mathbf{F} \cdot d\mathbf{r}
      = \int\limits_C F \cos \theta  ds,
    \label{eq:A.4}
\end{equation}
где $s$ --- это длина дуги вдоль пути $C$;
буква $C$ призвана напонить слово «curve» (кривая).}

\subsection{Кинетическая энергия}

Кинетическая энергия $K$ материальной точки, движущейся со скоростью $v$, определяется как работа, необходимая для её разгона из состояния покоя до скорости $v$.%
\footnote{Это определение молчаливо предполагает, что работа не зависит от того, как она совершается: слабой силой за долгое время, или сильной силой за короткое время, или даже переменной силой.
В данном случае это предположение верно: работа не зависит от способа её выполнения.
Это можно увидеть, разбив время на короткие интервалы и просуммировав работу, совершаемую на каждом из них.
Члены в этой сумме сократятся, и получится то же выражение, как если бы приложенная сила была постоянной.}

Из приведённого определения, выведится, что $K = m v^2/2$, где $m$ обозначает массу материальной точки.
Действительно, давайте разгонять массу $m$ от покоя до скорости $v$, прикладывая некоторую постоянную силу $F$ (величина $F$ скоро сократится).
Кинетическая энергия определяется как совершённая работа:
\begin{equation}
    K = F \cdot D,
    \label{eq:A.5}
\end{equation}
где $D$ --- пройденное расстояние, а $F$ --- постоянная приложенная сила.
(Напоним, что нам надо выразить $K$ только через $m$ и $v$,
так что $F$ и $D$ должны будут сократиться.)
Заметим, что
\[F = ma = m \frac{v}{T},\]
где $T$ --- время, за которое достигается скорость $v$.
Кроме того,
\[D = v_{\text{сред}} T = \frac{0 + v}{2}  T = \frac{v}{2} T.\]
Подставлив полученное в \eqref{eq:A.5}, получим:
\[K = F \cdot D= \left(m \frac{v}{T}\right) \cdot\left(\frac{v}{2} T\right)= \frac{m v^2}{2} .\]
Теперь ясно, почему $v$ возводится в квадрат --- обе величины сила $F = m(v/T)$ и расстояние $D = (v/2)T$ в \eqref{eq:A.5} пропорциональны $v$.
При данном $T$ б\'{о}льшая скорость $v$ требует большей силы $F$ и сопровождается б\'{о}льшим расстоянием $D$.
Этот двойной удар и объясняет квадрат.
Половинка в $mv^2/2$ возникает из-за того, что $v_{\text{сред}} = v/2$.

\subsection{Потенциальная энергия}

Потенциальная энергия объекта, находящегося в точке $A$ в силовом поле, определяется как работа, необходимая для перемещения объекта в точку $A$ из некоторой выбранной заране точки $O$.
Иными словами, это работа, которую нужно совершить против силового поля, перемещаясь из $O$ в $A$.

\paragraph{Пример 1.}
Выберем точку $O$ на уровне земли.
Потенциальная энергия массы, находящейся в точке $A$ на высоте $H$,
согласно приведённому выше определению, равна работе,
необходимой для перемещения массы из $O$ в $A$.
Эта работа равна $mgH$, как было объяснено на странице~\pageref{Работа:Пример}.

\paragraph{Пример 2 (потребует знания матанализа).}
Для кометы в гравитационном поле Солнца выберем точку $O$ на бесконечности.
Предположим, что сама комета находится на расстоянии $r$ от центра Солнца.
Потенциальная энергия кометы в этом случае равна работе,
которую нужно совершить против гравитационной силы
\[F = \frac{k}{x^2},\]
(здесь $x$ --- расстояние до центра Солнца)
при изменении $x$ от $\infty$ до $r$:
\begin{equation}
    P(r) = \int\limits_\infty^r \frac{k}{x^2}dx = -\frac{k}{r}.
    \label{eq:A.6}
\end{equation}
Минус в этой формуле говорит, что при перемещении массы с $\infty$ в $r$ нам надо прикладывать силу, противоположную направлению движения.
Иными словами, при движении с бесконечности гравитационное поле
совершает работу за нас.
(Чтобы связать это с предыдущим примером, знак минус в \eqref{eq:A.6}
аналогичен утверждению, что потенциальная энергия массы ниже уровня пола отрицательна.)

На рисунке \ref{pic:A.5} показан воронкообразный график потенциальной энергии кометы.

\begin{figure}[ht!]
\centering
\begin{lpic}[t(2mm),b(2mm),r(0mm),l(0mm)]{pics/A.5}
%\lbl[t]{17,1;{\footnotesize сгиб}}
\end{lpic}
\caption{Потенциальная энергия кометы в гравитационном поле звезды.}
\label{pic:A.5}
\end{figure}

Потенциальная энергия определяется с точностью до произвольной добавочной константы, что связано со свободой выбора опорной точки $O$.
Например, мы можем вычислить потенциальную энергию шара в комнате относительно уровня пола или, если захотим, относительно уровня поверхности стола и так далее; результаты будут отличаться на константу.

\paragraph{Консервативные поля.}
В нашем определении потенциальной энергии предполагалось, что работа не зависит от выбора пути между точками $O$ и $A$.
Это предположение выполняется для гравитационного и электростатического полей.
Такие поля называются \emph{консервативными}.
Но существуют и другие силовые поля, в которых работа зависит от пути; простой пример приведён на рисунке \ref{pic:A.6}.%
\footnote{На самом деле, с математической точки зрения, консервативные поля редки: это поля с нулевым ротором.}
Для таких неконсервативных полей понятие потенциальной энергии теряет смысл.
Из неконсервативного силового поля можно извлечь энергию.
Например, работа, совершаемая полем по замкнутому пути
$ABCDA$ на рисунке \ref{pic:A.6}, положительна.
Если бы такое силовое поле могло быть создано каким-либо фиксированным расположением зарядов, у нас оказался бы неисчерпаемый источник энергии.

\begin{figure}[ht!]
\centering
\begin{lpic}[t(2mm),b(2mm),r(0mm),l(0mm)]{pics/A.6}
%\lbl[t]{17,1;{\footnotesize сгиб}}
\end{lpic}
\caption{Пример неконсервативного поля.}
\label{pic:A.6}
\end{figure}

\subsection{Сохранение энергии}

Рассмотрим частицу массы $m$, движущуюся под действием консервативного силового поля (например, комету вокруг Солнца или снаряд на Земле при пренебрежении сопротивлением воздуха).
При движении частицы её кинетическая энергия $K$ и потенциальная энергия $P$ изменяются каждая по отдельности; однако их сумма остаётся постоянной во времени:
\begin{equation}
    K + P = \text{const}.
    \label{eq:A.7}
\end{equation}
Это является следствием второго закона Ньютона вместе с предположением,
что силовое поле консервативно.%
\footnote{Вот быстрый вывод тождества \eqref{eq:A.7} в скалярном случае с применением матанализа.
Умножив обе стороны во втором законе Ньютона $ma = F$ на скорость $v = \dot{x}$
(точка обозначает производную по времени),
получим
\[m \ddot{x}\dot{x} = F(x)\dot{x},\]
что равносильно
\[\frac{d}{dt}\left(m \frac{\dot{x}^2}2 - \int\limits_{0}^{x} F(s)ds\right)
=
0.\]
Заметим, что $-\int_{0}^{x} F(s)ds=\int_{0}^{x} (-F(s))ds$ --- это как раз потенциальная энергия,
ведь это работа, которую необходимо совершить против силы $F$, чтобы
переместить массу из $0$ в $x$;
знак минус появился от слова «против».
Итак, получаем
\[\frac{m v^2}{2} + \int\limits_{0}^{x} \big(-F(s)\big)ds = K + P = \text{const},\]
что и требовалось.}

Для кометы закон сохранения энергии принимает вид
\[\frac{m v^2}{2} - \frac{k}{r} = E = \text{const}.\]
В частности, если $r$ уменьшается, то $v$ увеличивается,
что согласуется с интуицией.

\section{Центр масс}

Понятие центра масс было известно и использовалось ещё 2400 лет назад Архимедом и, возможно, раньше.
Центр масс можно определить как точку равновесия тела, подвешенного в постоянном гравитационном поле.%
\footnote{Если гравитационное поле непостоянно, то
точка равновесия тела зависит от ориентации тела и приведённое определение центра масс перестаёт работать.}
Однако центр масс является чисто геометрическим понятием; его можно определить как среднее положение частиц тела без обращения к физике.

Для примера рассмотрим гантель, состоящую из двух масс $m$ и $M$, расположенных на оси $x$ в точках с координатами $x$ и $X$ соответственно.
Пусть $m$ и $M$ --- целые числа;
тогда можно думать что $m$ монеток лежит в точке $x$,
и $M$ монеток в точке $X$.
Чтобы найти среднюю координату, надо сложить координаты всех монет и разделить на их число:
\[
\text{Ц.М.} =
\frac{\overbrace{x + \dots + x}^{m} + \overbrace{X + \dots + X}^{M}}{\underbrace{m+M}_{\text{число монеток}}}
=\frac{mx + MX}{m+M}.
\]
В общем случае $N$ масс $m_i$, $1 \le i \le N$, каждая из
которых находится в точке $\mathbf{x}_i$ пространства, радиус-вектор
центра масс выражается похожей формулой:
\begin{equation}
    \mathbf{x} = \frac{1}{m} \sum m_i  \mathbf{x}_i,
    \qquad\text{где}\qquad m = \sum m_i.
    \label{eq:A.8}
\end{equation}
Выражение \eqref{eq:A.8} можно переписать как
\[
\mathbf{x} = \sum \frac{m_i}{m}  \mathbf{x}_i.
\]
Это даёт другое понимание центра масс, как взвешенное среднее радиус-векторов частиц, где веса пропорциональны вкладу каждой массы в общую массу.

\section{Импульс}

\paragraph{Одна частица.}
Постоянная сила $\mathbf{F}$, приложенная
к массе $m$, вызывает постоянное ускорение $\mathbf{a}$, и выполняется
\begin{equation}
    m \mathbf{a} = \mathbf{F}.
    \label{eq:A.9}
\end{equation}
За время $\Delta t$ масса изменит свою скорость на
$\Delta \mathbf{v} = \mathbf{a}\Delta t$, ведь ускорение определяется как изменение скорости за единицу времени.%
\footnote{Для непостоянного ускорения $\mathbf{a}= \mathbf{a}(t)$ приведённая выше формула должна быть изменена: $\Delta v = \bar{\mathbf{a}}  \Delta t,$
где $\bar{\mathbf{a}}$ --- среднее ускорение, определяемое как
\[
\bar{\mathbf{a}} = \frac{1}{\Delta t} \int\limits_{t_1}^{t_2} \mathbf{a}(t) dt,
\qquad \Delta t = t_2 - t_1.
\]
}
Умножив обе части закона Ньютона \eqref{eq:A.9} на $\Delta t$
и воспользовавшись равенством $\mathbf{a}\Delta t = \Delta \mathbf{v}$, получаем
\begin{equation}
m \Delta \mathbf{v} = \mathbf{F}\Delta t,
\qquad \text{или} \qquad
m \mathbf{v}_2 - m \mathbf{v}_1 = \mathbf{F}\Delta t.
\label{eq:A.10}
\end{equation}
Вектор $m\mathbf{v}$ называется \textit{импульсом} частицы.
Интуитивно $m\mathbf{v}$ показывает величину и направление движения.

В большинстве примеров мы будем рассматривать движение вдоль прямой.
В этих случаях можно думать, что импульс это скалярная величина.

\paragraph{Задача.}
Даже если дверь слегка приоткрыта, пуля, выпущенная
в дверь, почти не сдвинет её, несмотря на огромную силу, с которой пуля действует на дерево продырявливая его.
Однако лёгкое нажатие пальцем откроет дверь.
Как это объяснить?

\paragraph{Решение.}
Палец сообщает двери больший импульс
из-за гораздо большего времени контакта:
\[
F_{\text{палец}}  \Delta t_{\text{палец}}
>
F_{\text{пуля}}  \Delta t_{\text{пуля}}.
\]
В этом примере направление импульса не имеет значения, поэтому мы рассматриваем импульс как скаляр.
Подсознательно мы используем тот же эффект, когда отрываем бумагу от рулона туалетной бумаги: резкий рывок (в отличие от мягкого потягивания) отрывает бумагу, почти не заставляя
рулон вращаться.
Дети, которые ещё не освоили этот важный навык, могут размотать весь рулон, пытаясь оторвать совсем чуть-чуть.

\paragraph{Несколко частиц.}
До сих пор мы обсуждали законы Ньютона
только для одной материальной точки.
Но любая сложная система, например гантель с двумя массами, космический шаттл или даже кошка,
может рассматриваться как совокупность многих материальных точек.
Каждая масса может взаимодействовать с другими, а также подвергаться
действию внешней силы.
Теперь центр масс любой совокупности частиц ведёт себя как одна
материальная точка в том смысле, что он подчиняется второму закону Ньютона:
\begin{equation}
    \mathbf{F} = m \mathbf{a},
    \label{eq:A.11}
\end{equation}
где $m$ — это полная масса, $\mathbf{F}$ — сумма всех внешних сил,
а $\mathbf{a}$ — ускорение центра масс.
Важно, что $\mathbf{F}$ не включает внутренние силы, то есть силы
взаимодействия между частицами системы; как оказывается, эти силы
взаимно сокращаются в силу третьего закона Ньютона.

\paragraph{Вывод равенстрва \eqref{eq:A.11}.}
Надо записать закон Ньютона для каждой частицы системы, просуммировать и воспользоваться
третьим законом Ньютона чтобы сократить силы взаимодействия между частицами внутри системы.
$i$-тая частица испытывает действие внешней силы, а также сумму сил от всех остальных частиц, кроме самой себя:
\[
m_i \mathbf{a}_i = \mathbf{F}_i^{\text{внеш}} + \sum_{j \neq i} \mathbf{F}_{ij},
\]
где $\mathbf{F}_{ij}$ — это сила, действующая на $i$-тую частицу
со стороны $j$-той частицы.
По третьему закону Ньютона $\mathbf{F}_{ij} = -\mathbf{F}_{ji}$.
Следовательно, при сложении всех этих уравнений взаимные силы
сокращаются, так как каждая пара $\mathbf{F}_{ij}$ и $\mathbf{F}_{ji}$
встречается ровно один раз в общей сумме.
В результате остаётся:
\[
\sum_i m_i \mathbf{a}_i = \sum_i \mathbf{F}_i^{\text{внеш}} = \mathbf{F},
\]
что совпадает с нашим утверждением \eqref{eq:A.11}, используя \eqref{eq:A.12} ниже.

\paragraph{Замена частиц на их центр масс.}
Импульс любой системы частиц равен импульсу её центра масс, наделённого суммарной массой всех частиц.

\paragraph{Доказательство.}
Определение центра масс \eqref{eq:A.8}, немного
переписанное, имеет вид
\begin{equation}
    m \mathbf{x} = \sum_i m_i \mathbf{x}_i.
    \label{A.12}
\end{equation}
Отсюда следует, что
\[
m \mathbf{v} = \sum_i m_i \mathbf{v}_i.
\]
Это и доказывает утверждение: левая часть представляет собой импульс центра масс, наделённого полной массой,
а правая часть — импульс всей системы.

Второй закон Ньютона для системы из многих частиц имеет непосредственное следствие:

\paragraph{Закон сохранения линейного импульса.} \emph{Если сумма внешних сил, действующих на систему частиц, равна нулю ($\mathbf{F} = 0$), то суммарный импульс системы остаётся постоянным.}
В частности, центр масс системы либо покоится, либо движется с постоянной скоростью.

\section{Момент силы}

Рассмотрим силу $F$, приложенную к точке $A$.
Пусть также выбрана точка вращения $O$.

\emph{Момент силы \(F\) относительно точки \(O\) определяется как произведение расстояния \(OA\) на составляющую силы \(F\) в направлении перпендикулярном к \(OA\):
\begin{equation}
 T = OA \cdot F_{\perp} = OA \cdot F \sin \theta.
\label{eq:A.13}
\end{equation}
}
Момент силы измеряет интенсивность вращения.
Заметим, что $OA \sin \theta = D$ есть расстояние от точки \(O\) до линии действия силы.
Поэтому момент силы можно записать как $T = F (OA \sin \theta)$,
или
\begin{equation}
T = F D,
\label{eq:A.14}
\end{equation}
где \(D\) показано на рисунке \ref{pic:A.7}.
Другими словами, момент силы есть произведение силы \(F\) на расстояния \(D\) от линии действия силы до $O$.
До сих пор мы говорили о скалярном моменте, но на самом деле, момент силы можно определить и как вектор следующим образом.

\begin{figure}[ht!]
\centering
\begin{lpic}[t(2mm),b(2mm),r(0mm),l(0mm)]{pics/A.7}
%\lbl[t]{17,1;{\footnotesize сгиб}}
\end{lpic}
\caption{Определение момента силы.}
\label{pic:A.7}
\end{figure}

Существует естественная \emph{ось вращения} — прямая, перпендикулярная плоскости чертежа, то есть плоскости, определяемой вектором \(\overline{OA}\) и вектором силы \(\mathbf{F}\).
Вдоль этой прямой выбирается предпочтительное направление — а именно то, в котором будет продвигаться гайка с правой резьбой при закручивании силой.
Векторный крутящий момент определяется как вектор в этом направлении, величина которого задаётся формулой \eqref{eq:A.13}.
Иными словами, крутящий момент определяется как векторное произведение плеча и силы:
\begin{equation}
\mathbf{T} = \overline{OA} \times \mathbf{F}.
\label{eq:A.15}
\end{equation}
Приведённое рассуждение и мотивирует определение векторного произведения.

\section{Момент импульса}

Момент импульса $M$ является вращательным аналогом
ипульса.
Для материальной точки массы $m$ в точке $P$, момент импульса относительно точки $O$ определяется как $r (m v_{\perp})$, где $r$ — расстояние до $O$, а $v_{\perp}$ — проекция скорости
в направлении, перпендикулярном $OP$.

На самом деле угловой момент обладает направлением и, следовательно, является вектором; то, что мы определяли до сих пор, — это его величина. Направление этого вектора принимается как «ось вращения», то есть прямая, перпендикулярная как радиус-вектору $r$, так и вектору скорости $v$, как показано на рисунке \ref{pic:A.8}.

Строго говоря,
\[M = r \times m v,\]
где $r$ — радиус-вектор, $v$ — вектор скорости, а символ $\times$ обозначает векторное произведение.
На самом деле, два предыдущих абзаца объясняют, почему векторное произведение определяется именно так в курсах анализа и аналитической геометрии.
Заметим, что второй множитель в приведённом выражении — это сам импульс.
Таким образом, момент импульса является векторным произведением радиус-вектора и вектора импульса.

Для системы из многих частиц угловой момент определяется как сумма угловых моментов отдельных частиц:
\begin{equation}
M = \sum_i r_i \times m_i v_i.
\label{eq:A.16}
\end{equation}

\paragraph{Закон сохранения углового момента.}
\emph{Если сумма внешних моментов сил, действующих на систему частиц, равна нулю, то угловой момент системы остаётся постоянным.}

В частности, если не трогать кошку, пока она падает, её момент импульса не будет меняться, как бы она ни извивалась во время полёта (пренебрегая сопротивлением воздуха).

Прежде чем перейти к доказательству, я хотел бы подчеркнуть, что
внутренние моменты сил — то есть моменты, создаваемые частицами системы друг на друга, — взаимно уничтожаются,
и поэтому при сложении всех моментов сил, действующих на все частицы, в сумме остаются только внешние моменты.
Взаимное уничтожение внутренних моментов видно из рисунка~\ref{pic:A.9}.
Рассмотрим плоскость, проходящую через начало координат $O$ и две массы $m_i$ и $m_j$.
Силы $F_{ij} = -F_{ji}$ лежат в этой плоскости.
Направления моментов, создаваемых этими силами относительно $O$, очевидно противоположны
(оба они перпендикулярны плоскости рисунка).
Чтобы показать, что эти моменты взаимно уничтожаются, нужно лишь доказать, что их модули равны.

Согласно \eqref{eq:A.14}, величины моментов равны $T_{ji} = D F_{ji}$ и $T_{ij} = D F_{ij}$;
важно отметить, что расстояние $D$ (см. рисунок~~\ref{pic:A.9}) одинаково для обеих сил.
Так как $F_{ij} = F_{ji}$ по третьему закону Ньютона, модули моментов равны.
Это доказывает, что внутренние моменты сил взаимно уничтожаются.

Теперь перейдём к доказательству закона сохранения углового момента.
Желая показать, что $M$ является постоянным вектором, продифференцируем его по времени в \eqref{eq:A.16}:
\[
\frac{dM}{dt}
= \left( v_i \times m_i v_i + r_i \times m_i a_i \right)
= r_i \times F_i,
\tag{A.17}
\]
где $F_i$ обозначает сумму всех сил, действующих на $i$-ю частицу, как внешних, так и со стороны других частиц системы.
Таким образом,
\[
r_i \times F_i = r_i \times F_i^{\text{ext}} + \sum_{j \ne i} r_i \times F_{ij}.
\]
Последний член представляет собой сумму всех внутренних моментов сил, действующих на $i$-ю частицу.
Подставляя это выражение в \eqref{eq:A.17}, получаем, что внутренние моменты складываются и взаимно уничтожаются (см. предыдущий абзац).
В результате остаётся
\[
\frac{dM}{dt} = \sum_i r_i \times F_i^{\text{ext}} = T,
\tag{A.18}
\]
где $T$ — сумма моментов сил, создаваемых внешними силами.
В частном случае, когда $T = 0$, мы заключаем, что $M = \text{const}$.
Тем самым мы доказали, что угловой момент сохраняется, если сумма внешних моментов сил равна нулю.

Уравнение \eqref{eq:A.18} является вращательным аналогом второго закона Ньютона.

\begin{figure}[ht!]
\centering
\begin{lpic}[t(2mm),b(2mm),r(0mm),l(0mm)]{pics/A.8}
%\lbl[t]{17,1;{\footnotesize сгиб}}
\end{lpic}
\caption{Определение момента импульса.}
\label{pic:A.8}
\end{figure}

\begin{figure}[ht!]
\centering
\begin{lpic}[t(2mm),b(2mm),r(0mm),l(0mm)]{pics/A.9}
%\lbl[t]{17,1;{\footnotesize сгиб}}
\end{lpic}
\caption{Взаимоуничтожение моментов сил внутри системы.}
\label{pic:A.9}
\end{figure}

\section{Угловая скорость и центростремительное ускорение}

\paragraph{Угловая скорость.}
Для точки, движущейся по окружности с центром в точке $O$, угловая скорость $\omega$ определяется как скорость изменения угла $\theta$, образованного радиус-вектором точки и фиксированным направлением (см. рисунок \ref{pic:A.10}).

\begin{figure}[ht!]
\centering
\begin{lpic}[t(2mm),b(2mm),r(0mm),l(0mm)]{pics/A.10}
%\lbl[t]{17,1;{\footnotesize сгиб}}
\end{lpic}
\caption{Определение угловой скорости.}
\label{pic:A.10}
\end{figure}

\paragraph{Угловая скорость и линейная скорость.}
Скорость точки, движущейся по окружности с угловой скоростью~$\omega$, выражается формулой
\begin{equation}
v = \omega r,
\label{eq:A.19}
\end{equation}
где $r$~--- радиус окружности.
Это следует из определения угловой меры: напомним, что радианная мера~$\theta$ угла, соответствующего дуге окружности, равна длине~$s$ этой дуги, делённой на радиус окружности,
\[\theta = \tfrac{s}{r},\]
откуда $s=\theta r$.
Так как $s$ и $\theta$ находятся в прямой пропорциональной зависимости, то и их скорости изменения пропорциональны с тем же коэффициентом; это и доказывает формулу~\eqref{eq:A.19}.

\paragraph{Центростремительное ускорение.}
Имеет ли точка, движущаяся с постоянной скоростью, нулевое ускорение?
Ответ --- нет, если только точка не движется по прямой.
Ускорение может быть вызвано изменением направления движения.
Строго говоря, ускорение --- это скорость изменения вектора скорости,
и, следовательно, само является вектором.
Для точки, движущейся по окружности с постоянной скоростью, ускорение направлено к центру
и имеет величину
\begin{equation}
a_c = \omega v.
\label{eq:A.20}
\end{equation}
Это соотношение имеет следующее объяснение, прекрасное своей краткостью.
Идея состоит в том, чтобы применить формулу $v = \omega r$ не к физической окружности,
а к окружности, описываемой концом вектора скорости
(параллельно перенесённого так, что его начало всё время находится в начале координат).
Подробности на рисунке~\ref{pic:A.11}.

\begin{figure}[ht!]
\centering
\begin{lpic}[t(2mm),b(2mm),r(0mm),l(0mm)]{pics/A.11}
%\lbl[t]{17,1;{\footnotesize сгиб}}
\end{lpic}
\caption{Нахождение центростремительного ускорения.}
\label{pic:A.11}
\end{figure}

Если все векторы скорости~$v$ перенести так, чтобы их начала совпадали с началом координат,
то их концы~$Q$ будут описывать окружность радиуса~$v$ с угловой скоростью~$\omega$
(той же, что и угловая скорость точки, поскольку $v$ и $r$ всегда перпендикулярны).
Следовательно, применив формулу~\eqref{eq:A.19} к окружности скоростей, получаем:
\[
\text{(скорость точки $Q$)} = \omega \cdot
\text{(радиус окружности скоростей)},
\]
то есть $a_c = \omega v$, что и доказывает формулу~\eqref{eq:A.20}.

\paragraph{Эквивалентные формулы.}
Подставив формулу~\eqref{eq:A.19} в~\eqref{eq:A.20}, получаем следующие выражения:
\begin{equation}
a_c = \omega^2 r = \frac{v^2}{r}.
\label{eq:A.21}
\end{equation}
В большинстве учебников используются именно эти выражения, хотя формула~\eqref{eq:A.20} проще и в некотром смысле основная.

\paragraph{Задача.}
Когда машина движется по кругу, шины сцепляются с дорогой с некоторой силой, удерживающей машину на дороге.
Как изменится эта силя, если удвоить скорость?

\paragraph{Решение.}
Согласно формуле~\eqref{eq:A.21}, сила увеличивается в четыре раза.

\paragraph{Вопрос.}
А можно объясненить это, без формул?

\paragraph{Ответ.}
Здесь срабатывает двойной удар:
при удвоении скорости вдвое увеличивается длина вектора скорости, и кроме того, вдвое возрастает скорость его поворота.
В результате скорость движения конца этого вектора увеличивается в четыре раза, а эта скорость и есть центростремительное ускорение.

\section{Центробежная и центростремительная силы}

Представьте, что вы сидите на карусели, двигаетесь по окружности или едете в автомобиле по закруглённому съезду с шоссе.
Кажется, будто какая-то невидимая сила тянет вас прочь от центра окружности.
Это мнимая сила — в том смысле, что на самом деле никто не тянет вас от центра; это лишь иллюзия, вызванная вашей инерционной склонностью двигаться прямо, которая противоречит повороту автомобиля.
Эта мнимая сила называется центробежной силой.
Однако ничуть не мнимо усилие, которое пассажир прикладывает к автомобилю в направлении от центра.
Это усилие вполне правомерно можно назвать центробежной силой — хотя само оно не действует на тело пассажира.

Сила, которую автомобиль прикладывает к вашему телу, заставляет вас двигаться по окружности.
Эта сила направлена к центру окружности.
Эта реальная сила называется \emph{центростремительной}.
Согласно второму закону Ньютона, эта сила равна $ma_{c}$, где $a_{c}$ — это центростремительное ускорение, задаваемое формулой \eqref{eq:A.21}.
Таким образом, центростремительная сила выражается как
\[F_{c} = ma_{c} = m\frac{v^{2}}{r}.\]

\section{Кориолисова и центробежная силы через комплексную экспоненту}

\paragraph{Предварительные замечания.}
Сейчас я расскажу, как с помощью комплексных чисел и нехитрого матанализа выводятся и кориолисова, и центробежная силы.
Всё необходимое знание анализа и комплексных чисел приведено ниже.

\begin{enumerate}
\item Комплексное число $a + ib$ — это просто точка $(a, b)$ на плоскости;
точки на оси $x$ отождествляются с вещественными числами, так что вещественные числа являются подмножеством комплексных.
Таким образом, мы пишем $(a, 0) = a$.

\item Угол между положительным направлением оси $x$ и радиус-вектором точки $(a, b)$ называется
\textit{аргументом} числа $a + ib$; расстояние $\sqrt{a^{2} + b^{2}}$ до начала координат называется
\textit{модулем} числа $a + ib$.

\item По определению, два комплексных числа умножаются путём сложения их аргументов
и умножения их модулей. В частности, $i = (0, 1)$ имеет аргумент $\tfrac{\pi}{2}$
и модуль $1$; следовательно, аргумент $i^{2} = i \cdot i$ равен
$\tfrac{\pi}{2} + \tfrac{\pi}{2} = \pi$, а модуль равен $1 \times 1 = 1$.
Таким образом, $i^{2} = (-1, 0) \equiv -1$.

\item Комплексная экспонента $e^{is}$ определяется как точка $P(s)$ на единичной окружности
с центром в начале координат, отстоящая на дуге на расстояние $s$ от положительного направления оси $x$ (см. рисунок~\ref{pic:A.12}).
Напомним, что $\sin$ и $\cos$ определяются как координаты точки $P(s)$.
Таким образом, по определению
\[
e^{is} = \cos s + i \sin s,
\]
что представляет собой знаменитую формулу Эйлера (он открыл и множество других).

\end{enumerate}

\paragraph{Как повернуть точку?}
Если $Z$ — это точка на плоскости, то
$e^{i\theta} Z$ есть точка, получающаяся из $Z$ поворотом на угол $\theta$ вокруг начала координат.
Действительно, длина $e^{i\theta}$ равна $1$, а его аргумент равен $\theta$.
Следовательно, умножение $Z$ на $e^{i\theta}$ не изменяет длину $Z$ и прибавляет $\theta$ к его аргументу,
то есть поворачивает $Z$ на угол $\theta$.
Вскоре нам это пригодится.

\begin{figure}[ht!]
\centering
\begin{lpic}[t(2mm),b(2mm),r(0mm),l(0mm)]{pics/A.12}
%\lbl[t]{17,1;{\footnotesize сгиб}}
\end{lpic}
\caption{Комплексные числа и комплексная экспонента.}
\label{pic:A.12}
\end{figure}

\begin{figure}[ht!]
\centering
\begin{lpic}[t(2mm),b(2mm),r(0mm),l(0mm)]{pics/A.13}
%\lbl[t]{17,1;{\footnotesize сгиб}}
\end{lpic}
\caption{Кориолисова и центробежная силы используя комплексные числа.}
\label{pic:A.13}
\end{figure}
