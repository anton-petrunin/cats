\chapter{Велосипеды, гимнасты и ракеты}

\section{Как качаться на качелях?}\label{Как качаться на качелях?}

\paragraph{Вопрос.}
Иногда сказать легче, чем сделать.
Но бывает, что легче сделать, чем сказать.
Обычные качели --- хороший тому пример.
Объясните, как именно ребёнок переводит энергию своих мышц в раскачивание качелей.
Ответ совсем не простой.%
\footnote{Однажды внук спросил деда:
«Ты спишь с бородой над одеялом или под?», после этого дед не мог заснуть, клал бороду и так, и эдак; оба варианта казались неудобными.}

\paragraph{Ответ (анатомия резонанса).}
Представим себе, что вы качаетесь на качелях.
Наибольшую $g$-силу вы ощущаете в самой нижней точке,
а наименьшую --- в самых высоких (крайних) точках траектории.%
\footnote{Меньшая $g$-сила в верхней точке обусловлена сразу двумя причинами: (1) центробежная сила меньше и (2) проекция силы тяжести тоже меньше.}
Теперь вообразим, что у вас в руках груз.
Внизу траектории вы поднимаете его к плечам, держите до самой верхней точки, а там быстро опускаете на колени.
Далее всё повторяется: внизу поднимаем, наверху опускаем и так далее.

\begin{figure}[ht!]
\centering
\begin{lpic}[t(2mm),b(2mm),r(0mm),l(0mm)]{pics/6.1}
\lbl[r]{19.5,4.2,-25;{\scriptsize поднимаем}}
\lbl[r]{-0.2,10.2;{\scriptsize опускаем}}
\lbl[l]{48.2,10.2;{\scriptsize опускаем}}
\end{lpic}
\caption{Движение центра масс, раскачивающее качели.}
\label{pic:6.1}
\end{figure}

Эти действия будут раскачивать качели; давайте поймём почему.
Заметим, что вы поднимаете груз, когда он тяжелей, а опускаете, когда он лёгче.
То есть суммарно вы совершаете положительную работу, и она идёт на раскачивание.

Конечно, вовсе не обязательно держать в руках груз: можно использовать голову (в самом прямом смысле), торс или ноги.
Именно это и делают дети:
внизу они выпрямляют колени и спину (поднимая вес),
а наверху сгибают колени и откидываются назад (опуская вес).

Ребёнком я всё это делал, но объяснить не мог.
Теперь --- наоборот.

\section{Почему дорожает энергия?}\label{Почему дорожает энергия?}

\paragraph{Вопрос.}
Камень падает с постоянным ускорением (сопротивлением воздуха пренебрегаем).
На рисунке \ref{pic:6.2} показана скорость камня после каждого пройденного метра.
Почему прирост скорости уменьшается с каждым следующим метром?

\begin{figure}[ht!]
\centering
\begin{lpic}[t(2mm),b(2mm),r(0mm),l(0mm)]{pics/6.2}
\lbl[l]{25,8;{\scriptsize 1 м \quad $\Delta E$}}
\lbl[l]{25,18.2;{\scriptsize 1 м \quad $\Delta E$}}
\lbl[l]{25,28.4;{\scriptsize 1 м \quad $\Delta E$}}
\lbl[l]{25,38.6;{\scriptsize 1 м \quad $\Delta E$}}
\lbl{15,38.6;{\footnotesize $16$ км/ч}}
\lbl{15,28.4;{\footnotesize $6{,}6$ км/ч}}
\lbl{15,18.2;{\footnotesize $5{,}1$ км/ч}}
\lbl{15,8;{\footnotesize $4{,}3$ км/ч}}
\lbl{5,24,90;прирост скорости}
\end{lpic}
\caption{Чем ниже падает ныряльщик, тем меньше прирост его скорости на каждый пройденный метр.}
\label{pic:6.2}
\end{figure}

\paragraph{Ответ.}
По мере того как камень ускоряется, он тратит всё меньше времени на прохождение каждого следующего метра и, соответственно, имеет всё меньше времени, чтобы увеличить скорость.

Вот более строгое объяснение.
Если камень падает с высоты $h$, то в начале его потенциальная энергия%
\footnote{Потенциальная энергия, по определению, есть работа, необходимая для поднятия массы на высоту $h$.
Эта работа равна: сила $\cdot$ высоту $=$ вес $\cdot\  h=mgh$, так как вес равен $mg$.}
равна $mgh$
(мы считаем, что он отпущен без начальной скорости, то есть его кинетическая энергия равна нулю).
Непосредственно перед ударом о землю вся энергия становится кинетической:
$\tfrac{mv^2}2$.
Приравняв эти два выражения и сократив $m$, получим
\[
v^2=2gh
\qquad\text{или}\qquad
v=\sqrt{2gh}.
\]
Таким образом, зависимость $v$ от $h$ задаётся параболой, с рогами в бок; наклон её касательной убывает по $h$.
В частности, по мере увеличения $h$, равные приращения $h$ дают меньшие приращения~$v$.

\paragraph{Вопрос.} А откуда мы знаем, что вес тела массы $m$ равен $mg$?

\paragraph{Ответ.}
По определению, $g$ --- это ускорение, вызываемое силой тяжести $W$ (то есть весом) действующей на свободно падающую массу.
Следовательно, по второму закону Ньютона ($F=ma$, стрaница~\pageref{Законы Ньютона}), имеем $W=mg$.

\section[Большие обороты на перекладине]{Большие обороты на перекладине\\и хомячок в колесе}\label{Большие обороты на перекладине}

Большие обороты --- это базовый гимнастический элемент на перекладине, гимнаст стоит на руках, совершает оборот, прокручиваясь под перекладиной, снова поднимается в стойку на руках, и так далее (рисунок \ref{pic:6.3}a).

\begin{figure}[ht!]
\centering
\begin{lpic}[t(2mm),b(2mm),r(0mm),l(0mm)]{pics/6.3}
\lbl[tl]{0,77;(a)}
\lbl[tl]{46,77;(b)}
\lbl[tl]{30,31;(c)}
\lbl[b]{47,51,90;{\scriptsize начало}}
\lbl[rb]{60,64;\parbox{22mm}{\footnotesize\raggedleft  ваша\\сила}}
\lbl[lb]{63,64;\parbox{22mm}{\footnotesize  сила\\на вас}}
\lbl[lt]{85,39;{\footnotesize центр масс}}
\end{lpic}
\caption{(a) Выполнение больших оборотов на перекладине.
(b) Как используется момент веса.
(c) Как подкачивать энергию в большие обороты.}
\label{pic:6.3}
\end{figure}

\paragraph{Вопрос.}
Гимнаст неподвижно висит на перекладине.
Будем считать, что между его руками и перекладиной нет трения: хват абсолютно скользящий --- запястиями невозможно создать момент силы.
Сможет ли он выполнить большие обороты?

Давайте игнорировать прогиб перекладины, сопротивление воздуха и другие малозначительные факторы.

Многие физики и математики отвечают приблизительно следующее:
\begin{quote}
\emph{«Без трения нет момента силы, а без момента силы невозможно создать момент импульса,
и значит, вращение невозможно.
В частности, без трения гимнаст не сможет выполнить большие обороты.»
\footnote{Моменты силы и импульса обсуждаются в приложении.}}
\end{quote}

Это рассуждение неверно.
К счастью, юные гимнасты, которые выполняют большие обороты, не знают физики достаточно хорошо, чтобы их остановила подобная логика.
Они не слыхали ни о моменте силы, ни о моменте импульса ---
зачастую всех прожитых ими лет не хватило бы для подготовки диссертации.
Иногда меньшие знания дают преимущество.

\paragraph{Где же ошибка?}
Неверно уже предположение об отсутствии момента силы: сила тяжести может создавать момент.
Этот момент равен нулю, пока гимнаст висит неподвижно --- это правда --- но, сгибая тело, он может сделать этот момент ненулевым.
Давайте разберёмся как.

Представим себе, что вы висите на перекладине (рисунок \ref{pic:6.3}b) и сгибаетесь в поясе, как будто пытаетесь дотянуться руками до пальцев ног.
Напрягая мышцы живота, вы тянете руки вперёд, тем самым толкая перекладину вперёд (стрелка влево на рисунке).
Перекладина в ответ толкает вас вправо --- по третьему закону Ньютона (действие равно противодействию).
Значит, ваш центр масс тоже смещается вправо --- по второму закону Ньютона.
Теперь центр масс уже не совсем под перекладиной, и сила тяжести начинает менять момент импульса, увлекая ваше тело под перекладину, как в маятнике.
Итак, сгибая тело, вы заставляете силу тяжести создавать ненулевой момент.

\paragraph{Вопрос.}
Предположим, что гимнаст уже выполняет большие обороты.
Что ему (в принципе) следует делать, чтобы их ускорить?

\paragraph{Ответ.}
Принцип точно такой же, как на качелях: нужно совершать положительную суммарную работу.
Для этого следует приближать центр масс к перекладине тогда, когда это труднее (внизу), и отдалять его от перекладины тогда, когда это легче (вверху).

Грубая пародия на это показана на рисунке \ref{pic:6.3}c.

\paragraph{Несбалансированное колесо.}
А вот другой способ понять движения гимнаста.
На рисунке \ref{pic:6.3}c показана траектория его центра масс.
Представим, что масса распределена по всей траектории --- будем думать, что она лежит на ободе колеса, с осью на перекладине.
Так как колесо смещено влево, момент силы тяжести будет направлен против часовой стрелки.
Такое смещение можно поддерживать, постоянно подстраивая спицы колеса;
это сделает момент силы тяжести постоянным.
Именно это позволит набирать скорость и компенсировать трение.

Заметим, что хомяк в колесе делает то же самое:
он смещает центр масс и тем самым заставляет момент силы тяжести, раскручивать колесо.
(Хотя сам хомяк, вряд ли, думает в этих терминах.)

\paragraph{Задача.}
Изменение длины спиц в регулируемом колесе требует работы.
Объясните поподробней, как нужно менять длины спиц, чтобы поддерживать смещение обода.

Обратите внимание на то, что натяжение у укорачивающихся спиц в среднем должно быть больше, чем у удлиняющихся.
Как и раньше, длина спицы определяет расстояние от центра масс гимнаста до перекладины.

\section{Машина на льду}

Следующую задачу я узнал от Энди Руины %???
из Корнеллского университета.

\paragraph{Задача.}
Представьте, что вы за рулём машины;
вы едете по прямой по замёрзшему озеру и надавили тормоз.
Разумеется, лучше всего, чтобы колёса продолжали крутиться,
но если всё-таки не повезло, какие бы колёса вы предпочли заблокировать: передние или задние?
Ваша цель --- остановиться, двигаясь прямо без заноса.

\paragraph{Решение.}
Как это ни странно, лучше, если заблокируются передние колёса.%
\footnote{В этом случае рулить бессмысленно, но ваша цель лишь двигаться прямо.}
В этом случае машина продолжит двигаться прямо.
Если же заблокируются задние, то машина развернётся и до полной остановки будет  двигаться задом на перёд (при условии, что руль фиксирован).
Резко вдавив задний тормоз на велосипеде, можно заметить схожий эффект:
если удалось заблокировать заднее колесо, то оно начнёт скользить в сторону.

\begin{figure}[ht!]
\centering
\begin{lpic}[t(2mm),b(2mm),r(0mm),l(0mm)]{pics/6.4}
\lbl[t]{7.5,0,9.8;{\footnotesize крутится}}
\lbl[t]{42,5,9.8;{\footnotesize скользит}}
\lbl[b]{5,15,9.8;{\footnotesize стабилизатор}}
\end{lpic}
\caption{Как оперение удерживает стрелу прямо, так и катящиеся задние колёса удерживают прямо машину (с заблокированными передними колёсами).}
\label{pic:6.4}
\end{figure}

\paragraph{Объяснение.}
Очень помогает сравнение с движением стрелы.
Стрела летит прямо, потому что оперение не даёт её хвосту уходить в сторону (см. рисунок \ref{pic:6.4}).
Тоже происходит и с машиной, когда передние колёса заблокированы, катящиеся задние колёса играют ту же роль стабилизирующего оперения.
Поэтому если задние колёса продолжают катиться, то машина приобретает устойчивость, но теряет её, если катятся только передние.

Однажды вечером, в снежную погоду, я провёл этот эксперимент на пустой заснеженной парковке.
Блокировка задних колёс (ручным тормозом), легко разворачивала машину на 180°.

\section{Как поворачивать на велосипеде}\rindex{велосипед}

\paragraph{Задача.}
Велосипедист едет прямо и внезапно решает свернуть налево.
Что при этом он делает с рулём?

\paragraph{Решение.}
Чтобы поворачивать налево, нужно чтобы велосипед наклонялся влево;
наклон нужен чтобы скомпенсировать центробежную силу.
Для создания этого наклона велосипедист на мгновение подворачивает руль вправо, при этом колёса сдвигаются из-под него вправо, а тело по инерции продолжает двигаться прямо (рисунок \ref{pic:6.5}).
Добившись так нужного наклона влево, следует поворачивать руль влево, вписываясь желаемый поворот.
Если вы проверите это на себе (как сделал я), то заметите, что руки подсознательно делают первоначальный обратный поворот.%
\footnote{Это может зависеть от стиля вождения велосипедом, попробуйте повернуть двигая только рулём, но не телом. \pr}
Наши рефлексы отлично разбираются в механике.%
\footnote{Те читатели, которые умеют управлять велосипедом не держась за руль,
могут попытаться разобраться как это у них получается.
Лучше всего сначала привести доказательство того, что это невозможно, а потом найти в нём ошибку.
Сравнение с большими оборотами (раздел~\ref{Большие обороты на перекладине}) может оказаться полезным. \pr}

\begin{figure}[ht!]
\centering
\begin{lpic}[t(2mm),b(2mm),r(0mm),l(0mm)]{pics/6.5+}
%\lbl[b]{32,8,-3;\parbox{32mm}{\footnotesize\centering  руль слегка вправо,\\ а наклон влево}}
%\lbl[tr]{72,15,13;{\footnotesize создав наклон,}}
%\lbl[tl]{72,15,56;{\footnotesize поворачиваем}}
\lbl[bl]{-4,-5;
\begin{tikzpicture}
\path [
    decorate,
    decoration={
        text along path,
        text/.expanded=\bracetext{Сначала прямо, руль чуть вправо,},
        text align=left,
    }
](0,-.8) .. controls (3,-.0) and (8,-3) .. (8.3,2.5);
\end{tikzpicture}
}
\lbl[bl]{-4,-10;
\begin{tikzpicture}
\path [
    decorate,
    decoration={
        text along path,
        text/.expanded=\bracetext{а наклон влево. Создав наклон, поворачиваем.},
        text align=right,
    }
](0,-.8) .. controls (3,-.0) and (8,-3) .. (8.3,2.5);
\end{tikzpicture}
}
\end{lpic}
\caption{Чтобы свернуть влево, велосипедист сначала чуть поворачивает руль вправо, создавая наклон влево.}
\label{pic:6.5}
\end{figure}

В дополнение к сказанному, у быстро движущегося велосипеда с массивными шинами
наклон усиливается за счёт гироскопического эффекта.
Повернув руль вправо, вы поворачиваете переднее колесо  вызываете заметный наклон влево.%
\footnote{Марк, а можно (и нужно ли) здесь сказать, «... поворачиваете переднее колесо, создавая момент импульса относительно оси в направлении движения, который должен компенсироваться наклоном велосипеда влево.»?\pr}

\section{Разгон одним наклоном}\label{Разгон одним наклоном}\rindex{велосипед}

\paragraph{Вопрос.}
Можно ли изменить скорость велосипеда, используя только руль?
Не разрешается крутить педали и двигать телом.

\paragraph{Ответ.}
Предположим, что велосипед движется по прямой.
Тогда при входе в поворот с наклоном скорость автоматически увеличится.%
\footnote{Как наклоняется велосипед при входе в поворот, описано в предыдущей задаче.}
Причина: от наклона уменьшается потенциальная энергия.
Следовательно, должна увеличиться кинетическая, а вместе с ней и скорость.%
\footnote{Строго говоря, часть кинетической энергии превращается в кинетическую энергию вращения --- ведь велосипедист теперь вращается.
Тем не менее можно показать, что остаётся достаточно энергии, чтобы вызвать увеличение скорости.}

А как же зависит прирост скорости от начальной скорости (при одном и том же угле наклона)?
Удивительно, но прирост оказывается больше при меньших скоростях.%
\footnote{По той же причине, что прыжок с в четыре раза большей высоты даёт прирост скорости лишь в два раза.}%
\footnote{Марк, а можно (и нужно ли) сравнить это с \ref{Почему дорожает энергия?}?\pr}
Один и тот же угол наклона при движении со скоростью $2$ км/ч даст больший прирост скорости, чем при $20$ км/ч.
Вот объяснение.
Когда я наклоняюсь и тем самым опускаю свой центр масс на
величину $h$, я увеличиваю кинетическую энергию, на столько же на сколько я уменьшил потенциальную:
\[
\frac{mV^2}2 - \frac{mv^2}2=mgh,
\]
где $V$ --- новая скорость, а $v$ --- начальная скорость.%
\footnote{Замечу, что я поворачиваю, и часть кинетической энергии ушла на вращение, но я этим пренебрёг.}
Сократив $m$, получим
\[
V^2 - v^2=2gh,\]
и после пары алгебраический манипуляций
\[V - v=\frac{2gh}{v+V}=\frac{2gh}{v+\sqrt{2gh+v^2}}.\]
В частности,
с ростом начальной скорости $v$
её прирост $V - v$ уменьшается.

\section{Как разогнаться на велосипеде, двигая только тело?}\label{Как разогнаться на велосипеде, двигая только тело?}\rindex{велосипед}

Как мы только что выяснили, велосипедист может разогнаться зайдя в поворот.
Но это даёт лишь небольшой, однократный прирост скорости --- его нельзя повторять, разгоняя велосипед.

\paragraph{Задача.}
Может ли велосипедист (теоретически) увеличивать свою скорость бесконечно, не крутя педали, а только двигая телом?

Чтобы исключить возможные лазейки, считаем, что ветра нет, нельзя пользоваться двигателями и так далее.

\paragraph{Подсказка.}
Колесо напоминает конёк на льду: и то и другое легко движется куда направлено и не хочет двигаться вбок.

\paragraph{Решение.}
Я начинаю движение по прямой, сидя прямо.
Моя цель --- оказаться в том же положении, но с большей скоростью.
Добиться этого можно в три шага:
\begin{enumerate}
\item Наклониться вперёд к рулю, опустив тем самым центр масс.
\item Войдя в крутой поворот, выпрямиться, тем самым поднимая центр масс.
\item Снова начать ехать прямо.
\end{enumerate}

Почему же эти действия увеличат скорость?
Заметим, что при движении по окружности, перегрузка ($g$-сила) становится больше из-за дополнительного центробежного эффекта.%
\footnote{Это подробно объясняется в следующей задаче.}
Когда я выпрямляюсь против этой большей силы тяжести, я совершаю больше работы, чем ту, которую «получаю обратно», опуская центр масс.%
\footnote{Когда я говорю, что «получаю обратно» энергию, я имею в виду использование гравитационной энергии для зарядки батареи, подобно тому, как гибридный автомобиль делает это при торможении.}
Разность этих энергий переходит в приращение моей кинетической энергии.
Тот же принцип использует ребёнок, раскачивая качели, или гимнаст, выполняющий большие обороты.

\paragraph{Другое объяснение.}
Прирост скорости можно объяснить через сохранение момента импульса.
При движении по окружности мой момент импульса относительно центра окружности остаётся постоянным (так как момент силы, действующей на меня относительно центра окружности, равен нулю --- закон сохранения момента импульса обсуждается на странице~\pageref{Закон сохранения момента импульса}).
Когда я выпрямляюсь, центр масс приближается к центру окружности, и чтобы сохранить момент импульса, должна возрасти скорость.%
\footnote{Строго говоря, велосипед с человеком не образуют замкнутую систему, ведь они взаимодействуют с землёй. Значит надо дополнительно убедиться, что импульс не вкачивается и не выкачивается из системы.\pr}


\section{Как набрать вес на мопеде}\rindex{велосипед}

\paragraph{Вопрос.}
Вы едете на мопеде ровно по окружности с постоянной скоростью, наклоняясь в сторону поворота под фиксированным углом.
Какую перегрузку ($g$-силу) вы испытываете?
Другими словами, какой вы ощущаете вес?

\paragraph{Ответ.}
На рисунке~\ref{pic:6.6} показаны две силы, действующие на мопед:
(1) реакция опоры $R$ со стороны земли и (2) его вес $W$.
Сила реакции $R$ и есть ощущаемый вес.
Равнодействующая этих двух сил --- это центростремительная сила, которая заставляет мопед двигаться по окружности.
Следовательно, эта равнодействующая направлена к центру окружности, а значит горизонтально, как это и показано на рисунке~\ref{pic:6.6}.
Треугольник $ABC$ прямоугольный, и угол при $A$ совпадает с углом наклона~$\theta$.

\begin{figure}[ht!]
\centering
\begin{lpic}[t(2mm),b(2mm),r(0mm),l(0mm)]{pics/6.6}
\lbl[br]{3,18;$B$}
\lbl[r]{29,17;$B$}
\lbl[b]{25,34;$R$}
\lbl[bl]{31,4;$\theta$}
\lbl[bl]{31,21;$\theta$}
\lbl[t]{29,-.5;$A$}
\lbl[bl]{51,18;$C$}
\lbl[br]{40,26;$R$}
\lbl[tl]{33,19;\parbox{22mm}{\footnotesize центростре-\\мительная\\сила }}
\lbl[t]{81,15;\parbox{22mm}{\footnotesize\centering центр\\ пути\\ мопеда}}
\end{lpic}
\caption{Прирост ощущаемого веса при движении по окружности.}
\label{pic:6.6}
\end{figure}

Из $\triangle ABC$ получаем
\begin{equation}
\frac{W}{R}=\cos \theta.
\label{eq:6.1}
\end{equation}
Поскольку $\cos \theta < 1$ получаем, что $R > W$ --- при равномерном повороте всегда будет перегрузка.
При $\theta=30\degree$ ощущаемый вес увеличивается на $15\%$.
Если удастся удерживать наклон в $45\degree$, то вес увеличится на $41\%$: 70-килограммовый человек будет чувствовать себя как 100-килограммовый.
При $60\degree$ (огромном наклоне) вес будет удваиваться.

Та же формула работает и для самолёта в равномерном повороте.
Например, чтобы выдерживать постоянную перегрузку $2g$,
пилот должен наклонить самолёт на $60\degree$
(так что крылья будут составлять $30\degree$ с вертикалью).
Это также показывает, что войдя в поворот на машине, вы становитесь тяжелее.
Можно определить, насколько именно, измерив угол~$\theta$ с помощью подвешенного на нитке груза
и подставив этот $\theta$ в~\eqref{eq:6.1}.

\section[Как почувствовать квадрат в mv²/2]{Как почувствовать квадрат в $\tfrac{mv^2}{2}$ через велосипедные педали?}\rindex{велосипед}

Кинетическая энергия $K$ определяется как работа, необходимая для разгона массы $m$ из состояния покоя до заданной скорости $v$.
Как мы знаем, $K=mv^2/2$;
это объясняется на странице~\pageref{Кинетическая энергия}
 в приложении.

\paragraph{Вопрос.}
Как почувствовать квадрат в $\tfrac{mv^2}{2}$  через велосипедные педали?
Будем считать, что вы едете по ровной дороге, без сопротивления воздуха и трения качения.%

\paragraph{Ответ.}\label{Ответ:Как почувствовать квадрат}
Так как $v$ возводится в квадрат, то чем быстрее вы едете,
тем больше топлива требуется, чтобы прибавить $1$ км/ч.
Действительно, чтобы разогнаться от $v$ до $v+1$, нужна энергия
\[
\frac{m(v+1)^2}{2} - \frac{mv^2}{2}
  =\frac{m(2v+1)}{2}
  =mv + \tfrac{m}{2} > mv,
\]
и эта энергетическая цена ростёт с увеличением $v$.

Чтобы интуитивно ощутить эту растущую цену, представьте, что вы нажимаете на педали с постоянной силой, тем самым ускоряясь с постоянным ускорением.
То есть на то, чтобы прибавить $1$ км/ч, уходит та же секунда, независим от того, быстро или медленно вы движетесь.
А тут возникает неприятность:
при быстром движении, педали приходится крутить тоже быстро, и за ту же секунду ваши ноги должны пройти большее расстояние.
Это означает, что за секунду быстрой езды вы совершаете больше работы, чем за секунду медленной.
Итак, чтобы поддерживать постоянное ускорение, ваш двигатель должен работать всё быстрее и быстрее, сохраняя ту же силу.
Иначе говоря, его мощность должна постоянно расти.
Так что спешка изнуряет даже без трения, а с трением дела ещё хуже.

\paragraph{Задача.}
На сколько больше топлива требуется,
чтобы разогнаться до $70$ км/ч, чем до $10$ км/ч
(пренебрегая всеми потерями на трение и предполагая идеальный двигатель со 100 \% КПД)?

\paragraph{Решение.}
Почти в $50$ раз!
Действительно,
\[
\frac{K_{70}}{K_{10}}
 =\frac{m \cdot 70^2/2}{m \cdot 10^2/2}
 =\frac{70^2}{10^2}
 =49.
\]

\section{Парадокс с ракетами}\label{Парадокс с ракетами}

\paragraph{Ускоряющаяся ракета.}
Когда ракета сжигает единицу топлива, её скорость увеличивается на определённую величину.
\emph{Эта величина не зависит от того, с какой скоростью ракета двигалась до начала сжигания топлива.}%
\footnote{Всё происходит в невесомости; скорости меряются в инерциальной системе.}
Действительно, ускорение ракеты не зависит от её текущей скорости;
в этом ракета отличается от велосипеда: чем быстрее движется велосипед, тем труднее его разгонять.%
\footnote{Дополнительные пояснения на странице \pageref{Ответ:Как почувствовать квадрат}.}
Это различие приводит к следующему любопытному выводу.

\begin{figure}[ht!]
\centering
\begin{lpic}[t(2mm),b(2mm),r(0mm),l(0mm)]{pics/6.7}
\lbl[r]{1,5;$v$}
\lbl[l]{32,30;$v$}
\lbl[t]{16,16;$O$}
\lbl[tl]{41,8;\parbox{22mm}{\footnotesize струя\\двигателя}}
\end{lpic}
\caption{Ракеты получают больше кинетической энергии, чем её было в топливе, но разве такое возможно?}
\label{pic:6.7}
\end{figure}

\paragraph{Парадокс.}
На рисунке~\ref{pic:6.7} изображены две ракеты, закреплённые на стержне, который может свободно вращаться вокруг точки $O$.
Сообщив ракетам начальное вращение со скоростью $v$, мы включаем их двигатели.
После сгорания топлива скорость каждой ракеты увеличивается на $1\ \text{м/с}$.
Это приращение одинаково независимо от начальной скорости $v$.
Теперь скорость равна $v+1$, суммарная кинетическая энергия имеет вид
\[
E_{\text{после}}=m (v+1)^2,
\]
а прирост кинетической энергии равен
\[
\Delta K=\underbrace{m(v+1)^2}_{\text{после}} - \underbrace{\phantom{(}m v^2\phantom{)}}_{\text{до}}=2mv + m.
\]
Согласно этой формуле, выбрав $v$ достаточно большим, приращение энергии $\Delta K$ можно сделать произвольно большим.
Получается, что ракеты могут приобрести больше кинетической энергии, чем содержится в их топливе!
Неужели это правда?

\paragraph{Ответ.}
Это может прозвучать неожиданно, но ответ на последний вопрос утвердительный --- ракеты могут получить за время работы двигателей больше энергии, чем содержалось в топливе.
Однако это не нарушает закон сохранения энергии, потому что есть ещё один участник процесса --- выброшенное топливо, которое теряет значительную часть своей кинетической энергии.
При больших скоростях ракеты эта потеря особенно велика.
Если учесть эту энергию, то парадокс исчезнет.

Более подробное обсуждение схожей задачи приведено на страницах~\pageref{Мячик из машины}---\pageref{end:Мячик из машины}.

\section{Ракета-кофеварка}

Некоторые кофеварки снабжены рычагом, нажимая на который, вы перекачиваете кофе в чашку (рисунок~\ref{pic:6.8}).
Обратите внимание, что струя кофе, бьющая вниз, создаёт подъёмную  реактивную силу, действующую на кофеварку.
Обычно эта сила слишком мала, чтобы поднять кофеварку, не говоря уже о преодолении давления руки, нажимающей на рычаг,
но давайте поймём может ли подобная кофеварка взлететь в принципе.

\begin{figure}[ht!]
\centering
\begin{lpic}[t(2mm),b(2mm),r(0mm),l(0mm)]{pics/6.8}
\lbl[l]{34,48;$f$}
\lbl[l]{34,2;$F$}
\lbl[l]{45,25;$\ell$}
\lbl[lb]{45,49;$L$}
\end{lpic}
\caption{Можно ли заставить что-то взлететь, надавив сверху?}
\label{pic:6.8}
\end{figure}

\paragraph{Вопрос.}
Можно ли, хотя бы в принципе, сконструировать кофеварку с такими пропорциями, чтобы она поднималась со стола, когда нажимают на рычаг?

\paragraph{Ответ} (прыгающая кофеварка).
Удивительно, но это можно сделать.
Если отношение плеч рычага $L/\ell$ на рисунке~\ref{pic:6.8} очень велико,
то сила $F$ на поршне будет огромной.
Следовательно, можно достичь произвольно реактивной тяги, прикладывая лишь слабую силу $f$
(заглавные буквы $F, L$ и их строчные варианты $f, \ell$ указывают кто из них больше, а кто меньше).
Так, что реактивная сила может превысить сумму веса и приложенной силы $f$.
К сожалению рычаг в такой кофеварке придётся двигать очень быстро, ведь для поддержания высокого давления требуется достаточно большая скорость (закон Бернулли\rindex{закон Бернулли} объясняется на странице \pageref{eq:5.2}).
Так что, к сожалению, у вас не получится разыграть кого-нибудь прыгающей кофеваркой.
А это печально, ведь если бы идея сработала, то кофе давило бы на чашку с силой превышающей вес кофеварки в сумме с силой давления на рычаг.

Теперь более подробно.
Чтобы кофеварка поднялась, реактивная сила струи $F_J$ должна превысить сумму веса%
\footnote{Вес кофеварки уменьшается из-за выливающегося кофе, но не будем обращать внимание на эти мелочи.
Для тех же, кто настаивает на большей строгости, можно сказать так: пусть $W$ --- это начальный (наибольший) вес кофеварки, до того как кофе начал выливаться.}
$W$ и силы $f$:
\begin{equation}
F_J > W + f.
\label{eq:6.2}
\end{equation}
Для определённости будем считать, что $f=W/2$.
Тогда условие отрыва примет вид
\begin{equation}
F_J > \tfrac32W.
\label{eq:6.3}
\end{equation}
Должно быть ясно, что если сила поршня $F$ достаточно велика,
то и реактивная сила струи $F_J$ окажется достаточно большой;
так что для $F_J$ будет выполняться условие~\eqref{eq:6.3}.
Таким образом, всё, что нужно, --- это создать очень большую силу поршня.
А этого можно добиться, выбрав отношение плеч рычага достаточно большим.
Действительно, по закону рычага
\[
F=\frac{L}{\ell} f,
\]
и, следовательно, силу $F$ можно сделать сколь угодно большой,
увеличивая отношение $L/\ell$.

\paragraph{Задача.}
Можно ли достичь отрыва при $f > W$; то есть нажимая с силой превышающей вес кофеварки?

\paragraph{Задача.}
Вода под давлением вытекает из одного сосуда в другой сосуд, стоящий под ним, с постоянной скоростью.
Оба сосуда находятся на платформе весов.
Что будут показывать весы по сравнению с суммарным весом сосудов и воды?

\section{Мячик из машины}\label{Мячик из машины}

\paragraph{Ситуация.}
Двигаясь в машине, я бросаю вперёд мячик, сообщая ему дополнительную кинетическую энергию.
Однако прирост энергии мячика (с точки зрения наблюдателя на земле) может превысить ту энергию, которую затратили мои мышцы при броске, не правда ли, странно?
Следующий абзац проясняет в чём дело.

\paragraph{Подробности.}
Изначально мячик двигался вместе с машиной со скоростью $V$.
Я бросил мячик вперёд со скоростью $v=1$, и его новая скорость теперь равна $V+1$.
Изменение кинетической энергии мячика равно
\begin{equation}
\Delta K
=
\underbrace{\frac{m(V+1)^2}{2}}_{\text{после}}
-
\underbrace{\phantom{(}\frac{mV^2}{2}\phantom{)}}_{\text{до}}
= \frac{m}{2}+ mV.
\label{eq:6.4}
\end{equation}

\paragraph{Парадокс.}
Согласно \eqref{eq:6.4}, чем быстрее движется машина, тем больше кинетической энергии приобретает мячик (при той же скорости броска $v=1$)!
Ещё удивительнее то, что энергия, приобретённая мячиком, может превышать энергию, затраченную моими мышцами, если машина движется достаточно быстро.
Как это объяснить?

\paragraph{Решение.}
Хотя в \eqref{eq:6.4} содержится ошибка (см. следующий абзац), странный вывод остаётся в силе: мячик действительно может получить больше энергии, чем произвела моя рука.
Однако этот прирост происходит за счёт потери кинетической энергии машины,
Нельзя игнорировать то, что происходит с машиной, как это было сделано в \eqref{eq:6.4}, даже если этот эффект кажется малым,
ведь когда я бросаю мячик вперёд, машина получает толчок назад.
Поэтому её кинетическая энергия уменьшается, и с учётом этого уменьшения получается верное значение общего прироста кинетической энергии.

\begin{figure}[ht!]
\centering
\begin{lpic}[t(2mm),b(2mm),r(0mm),l(0mm)]{pics/6.9}
\lbl[b]{14,14;$m$}
\lbl[b]{77,14;$m$}
\lbl{14,7;$M$}
\lbl{55,7;$M$}
\lbl[b]{34,10;$V$}
\lbl[b]{71,7;$V_1$}
\lbl[t]{85,11;$V_1+v$}
\lbl[t]{13,0;{ до броска}}
\lbl[t]{54,0;{ после броска}}
\end{lpic}
\caption{Как перераспределяется кинетическая энергия.}
\label{pic:6.9}
\end{figure}

\paragraph{Баланс энергии без жульничества.}
Вот точное решение парадокса (предполагается отсутствие трения, сопротивления воздуха
и других факторов, отвлекающих от основной идеи).
Сначала найдём скорость машины после броска.
Акт броска не изменяет импульса системы (см. рисунок~\ref{pic:6.9}):
\begin{equation}
(M+m)V=M V_1 + m(V_1+v),
\label{eq:6.5}
\end{equation}
где $M$ --- масса машины, $m$ --- масса мячика, $V_1$ --- новая скорость машины,
а $v$ --- скорость мячика относительно машины в момент броска.

Общее изменение кинетической энергии равно
\begin{equation}
\Delta K_{\text{общ}}
=
\underbrace{\frac{MV_1^{2}}{2}}_{\text{машина после}}
+
\underbrace{\frac{m(V_1+v)^{2}}{2}}_{\text{мячик после}}
-
\underbrace{\frac{(m+M)V^{2}}{2}}_{\text{машна и мячик до}},
\label{eq:6.6}
\end{equation}
Воспользовавшись \eqref{eq:6.5} и перескочив через алгебраические выкладки, получаем
\begin{equation}
\Delta K_{\text{общ}}=\frac{m v^{2}}2\cdot\frac M{m+M}.
\label{eq:6.7}
\end{equation}

\paragraph{Обсуждение.}
\begin{enumerate}
\item Согласно \eqref{eq:6.7}, $\Delta K_{\text{общ}}$ не зависит от начальной скорости машины $V$, как и ожидалось.

\item Если $M \gg m$, как в случае машины и мячика, то из \eqref{eq:6.7} получаем
$\Delta K_{\text{общ}} \approx  m v^2/2$,
как если бы машина была неподвижно закреплена на земле --- тоже согласуется с ожиданиями.

\item Разница энергий $\Delta K_{\text{общ}}$ разбивается на две части:
\[
\Delta K_{\text{общ}}=\Delta K_{\text{мячик}} + \Delta K_{\text{машина}}
= \frac{m v^{2}}2\cdot\frac M{m+M}.
\]
Если машина движется быстро, то $\Delta K_{\text{мячик}}$ будет сильно положительной,
а $\Delta K_{\text{машина}}$ --- сильно отрицательной.
То есть мячик ворует много энергии у машины.
В то время как общая разница $\Delta K_{\text{общ}}$ не зависит от скорости машины,
то как $\Delta K_{\text{общ}}$ разбивается на $\Delta K_{\text{мячик}}$ и $\Delta K_{\text{машина}}$ очень даже зависит.
\end{enumerate}
\label{end:Мячик из машины}
