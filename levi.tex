\documentclass[twoside]{book}
\usepackage{book-ru}



\geometry%{top=0.9in, bottom=0.9in,left=0.9in, right=0.9in, paperwidth=6in, paperheight=9in}
{top=0.9in, bottom=0.9in,inner=0.9in, outer=0.7in, paperwidth=6in, paperheight=9in}


\def\thetitle{Почему кошки падают на лапы}
\def\theauthor{Марк Леви (перевод и редакция А. M. Петрунина)}

\hypersetup{
pdftitle={\thetitle},
pdfauthor={\theauthor},
pdfsubject={Математика}}

\makeatletter
\newcommand{\rindex}[2][\imki@jobname]{%
  \index[#1]{\detokenize{#2}}%
}
\makeatother


\begin{document}
%\pagestyle{empty}

\title{\thetitle\\
и ещё 76 парадоксов и головоломок}
\author{Марк Леви
\\
\\
перевод Р. Г. Матвеева и А. М. Петрунина}
\date{}
\maketitle

\thispagestyle{empty}

Предварительное издание, предназначенное исключительно для отлова ляпов. 

Исправления слать по адресу 
\url{petrunin@math.psu.edu}.

\vfill

\pagebreak

\thispagestyle{empty}
\chapter*{Благодарности}
Я признателен Полу Нахину за несколько полезных замечаний, в частности за ссылки, касающиеся парадокса Бресса,
Карол Швагер — за многочисленные улучшения изложения,
и
Вики Керн, чья поддержка придавала мне силы продолжать работу.
Я также с благодарен поддержке Национального научного фонда США, грант 1009130.
\chapter[Парадоксы, головоломки, задачи]{Физические парадоксы, головоломки и задачи}

\section{Введение}

Хороший физический парадокс --- это (1) неожиданность, (2) головоломка и (3) урок и всё это в прикольной обёртке.
Парадокс часто основан на убедительном рассуждении, которое приводит либо к ошибочному, но правдоподобному выводу, либо к верному но неожиданному выводу, который кажется ошибочным;
так, что трудно не поддаться искушению разобраться что к чему.
В годы холодной войны ходила байка, что ЦРУ мешало работе советских военных НИИ, подбрасывая листовки с головоломками и логическими задачами.
Времена изменились, теперь те же задачи используются при приёме на работу, советский пропагандист сказал бы, что они остались орудиями капитализма.

Парадоксы не только увлекательны, но и полезны.
Они развивают интуицию, логическое мышление и критический подход, так, что
у человека развивается внутренний детектор лжи.
Хороший парадокс также учит скромности и осторожности, показывая, как легко ошибиться даже в элементарных вопросах физики.
Успокаивает то, что даже очень умные люди допускают ошибки в, казалось бы, очевидных вопросах.
А ведь объекты, с которыми имеют дело другие области --- астрономия, биология, медицина, экономика, климатология, политика и СМИ посложней%
\footnote{Я не пытаюсь сравнивать сложности наук, просто хочу сказать, что типичный объект в физике (например, кристалл) проще типичного объекта в других областях (например, клетки в биологи).}%
, чем в физике, а значит, там ещё легче ошибиться.
Кроме того, некоторые «ошибки» могут приносить пользу, по крайней мере, некоторое время.

В этой книге я хотел поделиться прикольными размышлениями о том, как устроен мир.
Надеюсь они помогут вам понять суть некоторых физических явлений, и при этом не измучают%
\footnote{Я говорю о «мучениях» с иронией --- математика, разумеется, незаменима и прекрасна, по крайней мере для меня, уже потому, что это моя профессия.}
математической.

Эта книга по физике — науке, которая ходит на двух ногах, одна нога --- это математика, а другая --- физическая интуиция.
К сожалению, школьная физика часто выходит храмоногой.

\paragraph{Сравнение с музыкой.}
Если бы музыке учили так же, как зачастую учат физике, то мы знали бы отдельные ноты, но не мелодии, которые из них получаются.
Увы, слишком много учеников видят в физике набор формул, которые надо лишь применить в подходящем случае, и,
как следствие, много способных учеников теряют к физике интерес.

\paragraph{Сначала интуиция.}
Эта книга призвана натренировать вашу физическую интуицию.
Слишком часто курсы физики пренебрегают интуицией, делая упор на подбор формулы, подходящей в конкретном случае.
В этой книге всё наоборот:
я хотел добиться максимума интуитивного понимания при минимуме формул.
Обсуждение волчка --- хороший тому пример;
без каких-либо формул я объясню, почему вращающийся волчок остаётся в вертикальном положении.
Нужны годы освоения математики и физики в том объёме, чтобы научиться записывать дифференциальные уравнения движения волчка и понять, как из этих уравнений следует его устойчивость.
Пройдя весь этот путь, лишь немногие придут к интуитивному пониманию, почему волчок не падает.
Обидно, когда всё это время мощнейший инструмент --- физическая интуиция --- оказывается не у дел.

\section{Предварительные сведения}

Б\'{о}льшая часть книги (хоть и не вся) должна быть доступна читателям без специальной подготовки в физике;
все необходимые понятия объяснены в приложении.
Обычно математика остаётся в рамках алгебры, но изредка используется математический анализ.
Но даже в этих местах читатель, готовый принять на веру кое-какие математические выкладки, должен сносно себя чувствовать.%
\footnote{Речь идёт, например, о задаче с пращой на странице \pageref{???}, где камень достигает бесконечной скорости за секунду.}

Тяга к новому — основной инстинкт большинства живых существ, или, по крайней мере, млекопитающих.
Побуждая нас к исследованию, этот инстинкт помогает выживать — за исключением некоторых случаев, вроде лауреатов «Премии Дарвина» или героев шоу «Чудаки» (также известного как «Придурки», англ. Jackass).
Тот же самый инстинкт, который привёл Эйнштейна к его великим открытиям, толкает любопытного ребёнка разобрать механические часы и заглянуть внутрь.
Он же побуждает щенков и котят исследовать окружающий мир,
а у некоторых людей этот инстинкт настолько силён, что способен противостоять системе школьного образования.

\section{Источники}

Собирать подобные задачи посоветовал мне отец после того, как увидел одну мою головоломку;
она пришла мне в голову после школьного урока о капиллярном эффекте (см. страницу \pageref{???}).
Из этой коллекции и выросла данная книга.
Несмотря на то, что многие задачи книги моего собственного сочинения%
\footnote{Например, \ref{Гелиевый шар}, \ref{Парадокс с кометой}, \ref{Хочешь медленнее}, 3.1, 3.2, 3.5, 3.6, 4.1, 4.2, 4.4---4.6, 5.3---5.8, 6.6, 6.7, 6.10---6.12, 8.2, 8.5, 8.6, 9.4, 11.1, 12.3, 13.2, 14.6, 14.8.???
}, \emph{я уверен, что кто-то уже задумывался над ними или чем-то похожем задолго до моего рождения.}
Если мне известен автор или источник, то я его указываю.

\paragraph{Книжная полка.}
К счастью, %???
многое из основ физики можно понять, получив от этого удовольствие и (почти) без формул ---
несколько замечательных научно-популярных книг тому подтверждение;
среди них — \emph{The Flying Circus of Physics} Уолкера,
\emph{Thinking Physics} Эпштейна,
\emph{Mad about Physics} Яргодзки и Поттера,
а также классическая «Занимательная физика» Перельмана.
К сожалению, шикарная книга Маковецкого \emph{Смотри в корень}, которая разошлась тиражом более миллиона экземпляров в Советском Союзе, похоже, так и не была переведена на английский.
Книга Миннарта \emph{The Nature of Light and Color in the Open Air} (посвящёна оптическим явлениям в природе) никогда не устареет и доставит радость любому любознательному человеку, которому посчастливится её открыть.
\chapter{В открытом космосе}

\section{Шарик с гелием}\label{Гелиевый шар}

\paragraph{Задача}
Два космонавта, Андрей и Боря, пристёгнуты к противоположным концам космической капсулы, как показано на рисунке \ref{pic:2.1}.
В начале всё находится в покое и Андрей держит в руках большой шарик надутый гелием.
Он толкает шарик, и тот начинает двигаться в сторону Бори.
В каком направлении начнёт двигаться вся капсула с точки зрения наблюдателя, парящего в открытом космосе снаружи?
Поскольку Андрей и Боря пристёгнуты к стенкам, их можно считать частью капсулы.


\begin{figure}[ht!]
\centering
\begin{lpic}[t(2mm),b(2mm),r(0mm),l(0mm)]{pics/2.1(1)}
\lbl[b]{29,24;куда?}
\lbl{29,20;воздух}
\lbl{29,14;{\footnotesize гелий}}
\lbl[lt]{0,-.5;Андрей}
\lbl[rt]{58,-.5;Боря}
\end{lpic}
\caption{Куда двинется капсула когда Андрей толкнёт шар?}
\label{pic:2.1}
\end{figure}

\paragraph{Правдоподобное рассуждение.}\label{Првдоподобное рассуждение}
Поскольку Андрей толкает шарик вправо, шарик отталкивает Андрея влево, ведь по третьему закону Ньютона «действие равно противодействию».
А раз шарик толкает Андрея влево, то он и вся капсула начнут двигаться влево.

Похоже ли это на правду?

\paragraph{Ответ.}
На самом деле --- нет: капсула тоже будет двигаться вправо!

\paragraph{Объяснение через центр масс.}
Центр масс всей системы (капсулы и её содержимого) остаётся неподвижным, поскольку на систему не действуют внешние силы (все понятия этого предложения объяснены в приложении, страница \pageref{???}).

Рассмотрим движение внутри капсулы \emph{с точки зрения Андрея}, как показано на рисунке~\ref{pic:2.2}.
Шарик имеет гораздо меньшую массу, чем вытесняемый им воздух.
Значит, с точки зрения Андрея, центр масс смещается \emph{влево}.
И поскольку центр масс всей системы без внешних сил остаётся неподвижным,
Андрей и вся капсула движутся вправо с точки зрения внешнего наблюдателя.

Ошибка состояла в том, что мы слишком много думали о шарике и недостаточно о более массивном воздухе, который перемещается влево, занимая его место.

\begin{figure}[ht!]
\centering
\begin{lpic}[t(2mm),b(2mm),r(0mm),l(0mm)]{pics/2.2(1)}
\lbl[tl]{22,20;воздух}
\lbl[tl]{22,3.5;воздух}
\lbl{21.5,12.5;{\footnotesize гелий}}
\lbl[bl]{58,12;капсула}
\end{lpic}
\caption{Движение с относительно капсулы (и Андрея).}
\label{pic:2.2}
\end{figure}

\paragraph{То же самое через импульс.}
В приложении (страница \pageref{???}) объяснятся, что неподвижность центра масс эквивалентна тому, что суммарный импульс остаётся равным нулю.

С точки зрения Андрея, вытесненный воздух движется влево.
Это означает, что сам Андрей (вместе с капсулой) обязан двигаться вправо, чтобы скомпенсировать движение воздуха и держать суммарный импульс нулевым.

Всё станет совсем очевидным, если довести соотношение масс до крайности, как на рисунке~\ref{pic:2.3}, где вместо пары гелий-воздух берётся гелий-вода.
Поскольку почти вся масса приходится на воду, она практически не двигается.
А значит, когда шарик с гелием перемещается вправо, оболочка капсулы (чьей массой мы пренебрегаем) тоже движется вправо, уступая место гелию.

\begin{figure}[ht!]
\centering
\begin{lpic}[t(2mm),b(2mm),r(0mm),l(0mm)]{pics/2.3}
\lbl[r]{-1,36.5;до}
\lbl{7,37;гелий}
\lbl{36,36.5;вода}
\lbl[r]{12,10;после}
\lbl{64,11;гелий}
\lbl{36,10.5;вода}
\end{lpic}
\caption{Вода остаётся на месте, а почти невесомая оболочка капсулы сдвигается вправо.}
\label{pic:2.3}
\end{figure}

\paragraph{Назойливое сомнение.}
Приведённый выше ответ верен.
Но разве я не доказал противоположное на странице~\pageref{Првдоподобное рассуждение}, в подразделе с правдоподобным рассуждением?
В чём же ошибка того рассуждения?

\paragraph{Ответ.}
Ошибка в том, что не были учтены \emph{все} силы, действующие на Андрея.
Мы забыли о силе со стороны оболочки капсулы!
Когда Андрей толкает шарик, эта сила передаётся через воздух к оболочке, и та в свою очередь толкает Андрея в спину.
Удивительно, что этот толчок оказывается сильнее толчка шарика.
По сути, ударив шарик, он ещё сильнее ударил себя в спину!
Довольно удивительно (особенно для Андрея).
А как же это происходит?
Интуитивное объяснение этому даст следующий параграф.

\paragraph{Как пнуть себя в спину своим же коленом?}
Этот параграф объяснит, как может капсула толкать Андрея сильнее, чем он толкает шарик.
Чтобы легче это прочувствовать, упростим ненадолго задачу:
будем считать, что шарик надут не гелием, а воздухом.
Значит, когда Андрей толкает шарик, он лишь перераспределяет воздух внутри капсулы.%
\footnote{Предполагается, что оболочка шарика не имеет массы.}
Перемещения воздуха внутри капсулы не меняет его центр масс, а значит ни капусла, ни Андрей не будут двигаться.
По первому закону Ньютона, равнодействующая сила на Андрея будет равна нулю: \emph{его ладони и спина ощутят равные противоположнонаправленные силы}.

А что изменится, если воздух в шарике заменить гелием?
У шарика уменьшится инерция, то есть его будет легче ускорять.
Поэтому при той же силе в спину ладони Андрея ощутят меньшую силу.

\paragraph{Детское воспоминание.}
Ту же ошибку, я допустил в детстве, реализуя свою мечту летать.
В один прекрасный день меня осенило --- я забрался на стул, схватился за сиденье и начал изо всех сил тянуть его вверх.
Я думал, что сиденье потащит меня вверх, а сам я превращусь в человека-ракету (предполагалось, что ножки стула разойдутся в стороны, как антенны спутника).
Но взлёт не удался, ведь я упустил из виду, что мои руки (косвенно) соединены с моим задом
(это стечение обстоятельств сейчас воспринимается удачным и логичным),
таким образом мой зад толкал сиденье вниз, сводя на нет подъёмную силу рук.
Я давил на стул двумя частями своего тела, но одну не учёл.

Сходство с разобранным парадоксом должно быть очевидно.
Как и в первом рассуждении, я не учёл все силы.
Андрей связан с шариком не только руками, но и спиной — через оболочку и воздух;
эту вторую связь мы как раз и упустили в первом рассуждении.

\section[Управление спутником]{Управление спутником без реактивных двигателей}
\label{Управление спутником}

\paragraph{Задача.}
Может ли спутник изменить свою орбиту вокруг Земли, не пользуясь реактивными двигателями, солнечным ветром, и тому подобными средствами тяги?
Разрешается использовать солнечные панели для сбора энергии.

\paragraph{Подсказка.}
Восспользуйтесь тем, что сила тяжести зависит от расстояния до Земли; спутник не обязан быть материальной точкой.

\paragraph{Ответ.}
Самый простой спутник, с которым это сработает, состоит из двух масс, соединённых тросом.
Мотор, питающийся солнечной энергией, будет менять длину троса.
Давайте считать, что спутник движется по орбите, вращаясь, как показано на рисунке~\ref{pic:2.4}.

\begin{figure}[ht!]
\centering
\begin{lpic}[t(2mm),b(2mm),r(40mm),l(40mm)]{pics/2.4}
\lbl{4.7,17;{\footnotesize Земля}}
\lbl[bl]{23,28;орбита}
\end{lpic}
\caption{Гантелеобразный спутник регулируемой длины.}
\label{pic:2.4}
\end{figure}

Орбиту нашего спутника можно поднять или опустить, если надлежащим образом регулировать длину троса!
Будем считать, что спутник изначально вращается и движется по орбите, как показано на рисунке~\ref{pic:2.4}.
Допустим мы хотим понизить орбиту.
Для этого будем следовать инструкциям на рисунке~\ref{pic:2.5}:
подтягивать трос, когда он направлен к Земле, и ослаблять его, когда он почти перпендикулярен направлению к Земле. %??? я не вижу там инструкций!!!
Повторяя это раз за разом на каждом обороте, мы заставим спутник снижаться.
Если же надо поднять орбиту, то следует делать всё наоборот.

\paragraph{Суть.}
Из-за вращательного (кувыркательного) движения трос испытывает центробежное натяжение.
Но не только центробежное, натяжение слегка меняется из-за приливного эффекта, это объясняет рисунок~\ref{pic:2.6}.

\begin{figure}[ht!]
\centering
\begin{lpic}[t(2mm),b(2mm),r(0mm),l(0mm)]{pics/2.5}
\lbl{4.7,17.5;{\footnotesize Земля}}
\lbl[l]{47,17;\parbox{20mm}{движение\\ по орбите}}
\end{lpic}
\caption{Для понижения орбиты, укорачивайте трос, когда он направлен к Земле.}
\label{pic:2.5}
\end{figure}

\begin{figure}[ht!]
\centering
\begin{lpic}[t(2mm),b(2mm),r(0mm),l(0mm)]{pics/2.6}
\lbl{4.7,4.5;{\footnotesize Земля}}
\lbl[t]{32,3;$A$}
\lbl[t]{47,3;$B$}
\end{lpic}
\caption{Тросс натягивается потому, что притяжение в точке $A$ сильнее, чем в точке $B$.}
\label{pic:2.6}
\end{figure}

Как показано на рисунке~\ref{pic:2.6}, натяжение троса большее, когда он направлен к Земле,
ведь массы $A$ и $B$ в этот момент находятся на разных расстояниях от Земли,
и разница сил притяжения в $A$ и $B$, растягивает трос.%
\footnote{Тот же эффект отвечает за приливное вытягивание Земли вдоль прямой Луна---Земля.}
Эту переменную силу натяжения можно использовать для ускорения вращения.
Натягивая трос, когда натяжение больше, и ослабляя его, когда оно меньше, мы совершаем работу.
Эта работа идет на увеличение скорости вращения.%
\footnote{По этому же принципу раскачиваются качели.
Mы поднимаем часть тела, когда $g$-сила больше, и опускаем её, когда $g$-сила меньше.
Так мы совершаем работу, которая переходит в движение качелей
(подробнее это обсуждается на страницах \pageref{Как качаться на качелях?}---\pageref{Почему дорожает энергия?}).
Такой приём «покупай дорого, продавай дёшево» прекрасно работает для подкачки энергии в систему,
но привёл бы к разорению при игре на бирже.}
Поскольку суммарный момент импульса%
\footnote{Справка о моменте импульса дана на странице \pageref{???}.} сохраняется увеличивая вращательный момент, мы уменьшаем орбитальный момент.
А уменьшение орбитального углового момента означает, что спутник \emph{понижает} свою орбиту --- это будет объяснено чуть ниже.

\paragraph{Объяснение размазыванием.}
Есть ещё один способ понять, как увеличивается момент импульса спутника.
Представьте, что массы двух шариков, составляющих гантель, размазаны вдоль их траекторий,
как на рисунке~\ref{pic:2.5}.
То есть мы заменили две массы на что-то вроде обруча из провода.
Суть в том, что этот обруч наклонён по отношению к направлению на Землю, и поэтому на него действует момент  силы, вызванный тем, что ближе к Земле притяжение сильней.
Этот момент пытается повернуть обруч против часовой стрелки,
то есть \emph{увеличивает} его момент импульса — в точности как мы утверждали в исходном объяснении.
(Та же идея с размазыванием используется при объяснении гимнастического элемента на странице~\pageref{Большие обороты на перекладине}.)

\paragraph{Уточнения.}
Вот пара уточнений к предыдущим объяснениям, включая то, что чем выше орбита, тем больше момент вращения.

Поскольку сила тяжести направлена строго к центру Земли, она не может менять полный момент импульса спутника.%
\footnote{Это не совсем правда, поскольку Земля не совсем круглая, но не будем обращать на это внимание.}
Значит полный момент импульса $M$ спутника относительно центра Земли остаётся постоянным, независимо от того, что мы делаем с тросом; то есть,
\[
M = M_{\text{вращ}} + M_{\text{орб}} = \mathrm{const}.
\]
Таким образом, увеличивая вращательный момент $M_{\text{вращ}}$, мы уменьшаем орбитальный момент $M_{\text{орб}}$.
А этот момент связан с радиусом орбиты $r$ формулой
\begin{equation}
\sqrt{M_{\text{орб}}} = k \sqrt{r}, \label{eq:2.1}
\end{equation}
где $k = m \sqrt{G M}$; здесь $m$ --- масса спутника, $G$ --- гравитационная постоянная, а $M$ --- масса Земли.%
\footnote{Действительно, при движении по круговой орбите радиуса $r$, непосредственно из определения момента импульса получаем, что
\begin{equation}
M_{\text{орб}} = m v r. \label{eq:2.2}
\end{equation}
Здесь $v$ — это скорость и она связана с $r$ вторым законом Ньютона.
Согласно этому закону, центростремительное ускорение $\frac{v^2}{r}$ определяется силой притяжения:
\[
\frac{m  v^2}{r} = \frac{G m M}{r^2},
\]
и значит
$v = m  \sqrt{G M}/\sqrt{r}$.
Подставив это выражение в \eqref{eq:2.2}, получим \eqref{eq:2.1}.}
По этой формуле, уменьшение $M_{\text{орб}}$ влечёт уменьшение радиуса орбиты $r$.
Это подтверждает ранее сделанное утверждение: ускоряя вращение (кувыркание), мы уменьшаем орбитальный момент, и значит спутник снижается.

\paragraph{Другие манёвры.}
Можно добиться и большего.
Например, можно менять эксцентриситет орбиты,
но как это делать предоставляется выяснить читателю.

\section{Парадокс с кометой}\label{Парадокс с кометой}

Обсуждение движения пушечных ядер часто начинают с того, что горизонтальная скорость ядра не меняется, поскольку в горизонтальном направлении не действуют никакие силы (сопротивлением воздуха пренебрегаем).
Вот попытка применить то же рассуждение для движения в открытом космосе.

\begin{figure}[ht!]
\centering
\begin{lpic}[t(2mm),b(2mm),r(0mm),l(0mm)]{pics/2.7}
\lbl[lb]{6,60.5;Солнце}
\lbl[lt]{6,27;Солнце}
\lbl[lt]{11,-.5;Солнце}
\lbl[lb]{10,3.5;$S$}
\lbl[l]{50,12;\parbox{40mm}{в этих направлениях\\силы не действуют}}
\lbl[lt]{29,34;притяжение}
\lbl[rb]{23,40;$v$}
\lbl[b]{36,44;$v_\perp$}
\lbl[rt]{42.5,9;$C$}
\lbl[t]{2,53;\parbox{25mm}{быстро\\вращается}}
\lbl[tl]{49,52;\parbox{40mm}{медленно\\вращается}}
\end{lpic}
\caption{(a) Никакая сила не действует перпендикулярно к линии $SC$.
Остаётся ли $v_\perp$ постоянной?
(b) Линия Солнце---комета поворачивается быстрее, когда комета ближе к Солнцу.
%??? не надо ли поменять местами катинки (или подписи)
}
\label{pic:2.7}
\end{figure}

Солнце может только притягивать к себе комету, и не может раскручивать её вокруг себя,
ведь сила притяжения всегда направлена точно в сторону Солнца (см. рисунок~\ref{pic:2.7}).
Иными словами, сила солнечного притяжения не имеет компоненты
в направлении, перпендикулярном прямой, соединяющей Солнце и комету: \( \bm{F}_\perp = 0 \).
Поскольку сила в перпендикулярном направлении равна нулю,
по первому закону Ньютона (если сила нулевая, то скорость не меняется), получаем,
что, компонента скорости \(v_\perp\) в этом направлении не меняется.

Но тут появляются сомнения.
Момент импульса%
\footnote{Определения и пояснения даны на странице ???.}
кометы обязан сохраняться:
\[
M = mv_\perp \cdot r = \text{const},
\]
где \(r\) — расстояние от кометы до центра Солнца.
Поскольку \(r\) меняется по мере движения кометы по эллиптической орбите,
компоненте скорости \(v_\perp\) придётся меняться, чтобы оствить произведение \(v_\perp \cdot r\) неизменным.
Что же из этого верно (если верно хоть что-то)?

\parbf{Решение.}
Ошибка в первом рассуждении: я неверно использовал первый закон Ньютона.
Этот закон справедлив только в инерциальной системе отсчёта,%
\footnote{То есть в системе отсчёта, которая движется без ускорения и без вращения.
Подробности по закону Ньютона даны в приложении на странице ???.}
то есть я неявно и неверно предположил, что прямая $SC$ есть ось координат инерциальной системы.
Но ось $SC$ вращается, а значит эта система не инерциальна.

\section[Хочешь медленнее --- разгоняйся!]{Хочешь медленнее --- разгоняйся!\footnote{Эта же задача обсуждается в книге «Смотри в корень!» Маковетского \cite[Задачa 22 «Хочешь быстрее --- тормози»]{makovetskij}. \pr}}
\label{Хочешь медленнее}




\paragraph{Вопрос.}
Космический корабль движется по круговой орбите вокруг планеты.
Желая повысить орбиту, космонавт включает двигатели, разгоняя корабль вперёд.
После выхода на новую круговую орбиту, двигатели выключаются.
Пока работали двигатели кораблю сообщалась ускорение примерно в направлении движения; движется ли теперь он быстрее, чем раньше?

\paragraph{Ответ.}
На самом деле корабль сбавил скорость.

\paragraph{Обяснение.}
Ответ покажется не столь странным, если подумать о езде на велосипеде;
ведь въезжая на горку можно замедлиться даже сильно давя на педали.
То же самое происходит и с космическим кораблём: подъём орбиты — это та же горка.
Энергия двигателей тратится не на ускорение, а на преодоление тяготения.
Корабль получает потенциальную энергию, но теряет кинетическую;
при этом прирост потенциальной энергии превышает потерю кинетической.

\begin{figure}[ht!]
\centering
\begin{lpic}[t(2mm),b(2mm),r(0mm),l(0mm)]{pics/2.8}
\lbl[t]{16,8;толчок}
\lbl[b]{16,33;толчок}
\lbl[ul]{35,21;траектория}
\lbl[r]{9.5,16;$1$}
\lbl[l]{27,16;$2$}
\lbl[rb]{4,27;$3$}
\end{lpic}
\caption{Разгон замедляет.
Два коротких импульса в направлении движения прикладываются в отмеченных точках траектории.
В результате орбита поднялась, но корабль сбавил скорость.
}
\label{pic:2.8}
\end{figure}

А как именно зависит орбитальная скорость \(v\) от радиуса орбиты~\(r\)?
Оказывается, что:
\[
v = \frac{k}{\sqrt{r}},
\]
где \(k = \sqrt{GM}\),
\(G\) — универсальная гравитационная постоянная,
а \(M\) — масса планеты.
Когда спутник находится на круговой орбите, второй закон Ньютона \(ma = F\) говорит, что центростремительное ускорение%
\footnote{То есть ускорение по направлению к центру; подробности и пояснения на странице ???.}
$a = \frac{v^2}{r}$ обеспечивается силой притяжения $F = {GmM}/{r^2}$, и, значит,
\[
m \frac{v^2}{r} = \frac{GmM}{r^2}.
\]
где \(m\) — масса спутника.
Отсюда получаем
\[
v = \sqrt{\frac{GM}{r}}.
\]
Согласно этой формуле при подъёме (то есть при увеличении \(r\)) спутник замедляется.
%дважды сказано про G???
\chapter{Во вращающейся воде}

Архимед открыл свой знаменитый закон: сила, с которой вода вытесняет погружённое в неё тело (сила Архимеда), равна весу вытесненной этим телом воды.%
\footnote{Следующий мысленный эксперимент объяснит, почему этот закон верен. Я хочу показать, почему на боулинговый шар, лежащий на дне бассейна, действует сила Архимеда, равная весу вытесненной им воды. В качестве мысленного эксперимента представим, что мы заменили шар на шар из воды той же формы. Этот водяной шар останется неподвижным, так как предполагается, что вода в бассейне находится в покое. Мы заключаем, что сила тяжести в точности уравновешивается силой Архимеда для водяного шара. То есть, по крайней мере для водяного шара, сила Архимеда равна весу вытесненной воды. Но эта сила зависит только от формы тела, а значит, будет такой же и для боулингового шара. Это и доказывает закон Архимеда.
Вкратце, закон сводится к двум вещам:
(1) неподвижная вода остаётся неподвижной пока нет внешнего воздействия;
(2) сила Архимеда, действующая на тело, зависит только от формы тела, а не от материала, из которого оно сделано.}

Однако во вращающемся мире, таком как Земля, закон Архимеда приобретает неожиданный поворот (без игры слов), порождая удивительные явления.
Один из них — парадокс с плавающей пробкой, который будет рассмотрен далее.
Схожий пример того же явления — парадокс айсберга (стр. ???).

\section{Плавающая пробка}

\paragraph*{Эксперимент.}
Парк аттракционов с вращающимся бассейном --- мечта любого ребёнка, даже если все думают, что он уже дедушка.
С такими мыслями я готовил эксперимент для своей лекции по анализу, показывающий, что поверхность воды во вращающейся чаше принимает форму параболоидa.
Большая салатница с водой водружённая на проигрыватель, отлично для этого подхошла.
Я ждал с минуту пока вода не стала вращаться вместе с салатницей как единое целое со скоростью 33 оборота в минуту и увидел гладкую идеально параболическую поверхность воды.

Из чистого любопытства, я положил пробку на наклонную поверхность.
Я ожидал, что пробка останется на склоне, мне будет прикольно на это смотреть, а ещё лучше представить себя плавающим во вращающемся бассейне!
Однако пробка повела себя неожиданно:
она медленно поплыла вниз по поверхности параболоида и остановилась в самой нижней его точке.
«Возможно, это из-за сопротивления воздуха», — подумал я.
Чтобы это проверить, я накрыл салатницу прозрачной плёнкой.
Пробка плавала у самой стенки, я снова включил проигрыватель, и снова произошло то же самое!
Значит, воздух здесь ни при чём, ведь он входит во вращение даже быстрее, чем вода.
Затухание внутреннего движения в воздухе происходит быстрее, чем в воде.
Вязкость воздуха выше, чем у воды если её мерить относительно плотности.%
\footnote{Такая относительная вязкость называется кинематической.
Она определяется как отношение обычной (динамической) вязкости к плотности, что обычно записывается как $\nu=\mu/\rho$ ($\nu$ читается как английское «new»).
Джо Келлер, oбучавший меня гидродинамике в аспирантуре, сказал, что на приветствие «what’s new?» (что нового?), теперь можно отвечать: «mu over rho» (мю на ро).}

\paragraph*{Вопрос.} Почему же пробка плывёт вниз?

\paragraph*{Объяснение.}
Давайте перейдём в систему отсчёта, вращающейся вместе с салатницей.
На рисунке~\ref{pic:3.1} показана область воды $B$, ровно той же формы как вода вытесненная пробкой.
В нашей вращающейся системе область $B$ находится в покое.
Это означает, что центробежная сила%
\footnote{Центробежная сила обсуждается в приложении на странице ??? — это фиктивная сила, возникающая из-за того, что система отсчёта не инерциальна.}
уравновешивает горизонтальную составляющую силы Архимеда.

\begin{figure}[ht!]
\centering
\begin{lpic}[t(2mm),b(2mm),r(0mm),l(0mm)]{pics/3.1}
\lbl[b]{38,24,59.3;пробка}
\lbl[ur]{4,21;$B$}
\end{lpic}
\caption{Архимедовы силы для пробки (справа) и области $B$ (слева), одинаковы.
Но у пробки центробежная сила слабей, поскольку её центр ближе к оси.
Это и заставляет её двигаться.
}
\label{pic:3.1}
\end{figure}

Теперь представим, что область $B$ начинает разбухать, превращаясь в пробку, при этом сохраняя ту же подводную форму.
В этом процессе частицы в области $B$ будут постепенно приближаться к оси вращения.
Следовательно, его центробежная сила будет ослабевать, тогда как сила Архимеда останется прежней.
Это несоответствие приводит к тому, что пробку начинает толкать в сторону оси.%
\footnote{Попробуйте убедиться, что частицы области $B$ обязаны (в среднем) обязаны приблизиться к оси.
Для этото придётся воспользоваться двумя наблюдениями
(1) вертикальная составляющая силы Архимеда области $B$ уровновешивается весом воды в $B$ и
(2) то, что пробка однородна.
Пробка со смещённым центром тяжести, может наоборот будет плыть к краю;
см. раздел~\ref{sec:sails}.
\pr}

\section{Параболические зеркала и две кухонные головоломки}

\paragraph*{Параболические зеркала.}
Зеркала телескопов имеют форму параболоидов (параболоид — это поверхность, образуемая вращающейся вокруг своей оси симметрии параболой, как показано на рисунке 3.1).
Параболоиды столь полезны потому, что собирают пучок лучей, идущих параллельно оси, в одну точку.%
\footnote{По этой же причине тарелки микрофонов для подслушки, спутниковые и другие антенны также имеют форму параболоидов.
Датчик размещается в фокусе параболоида, чтобы собирать все «лучи», отражающиеся от антенны.}
Среди всех поверхностей только параболоиды обладают этим фокусирующим свойством.
И вот по замечательному совпадению природа предоставляет простой способ получить параболоидную форму: поверхность вращающейся жидкости, как на рисунке 3.1, автоматически принимает форму параболоида.

Если дать расплавленному стеклу в вращающемся контейнере медленно остыть, получится отличный параболоид — без всякой механической обработки.
В случае с параболоидом природа выступает одновременно и как аналоговый компьютер, и как токарный станок.

\paragraph*{Съедобная головоломка.}
Налейте жидкий желатин в миску, поставьте её на проигрыватель и крутитите его пока желатин не застынет. Поверхность застывшего желатина будет представлять собой аккуратный параболоид.
Ваши друзья могут озадачиться (а возможно, и немного встревожиться), решив, что вы провели часы, кропотливо вырезая углубление до такой невероятной гладкости.
Позже можно будет их угостить, наполнив параболоид взбитыми сливками.

\paragraph*{Кулинарное приложение теоремы Тейлора --- Праудмена.}
А вот способ сделать интересные цветные узоры в желатине.
Поставьте стакан с жидким желатином на проигрыватель, дайте ему несколько секунд, чтобы он стал вращаться вместе со столом, и влейте туда столовую ложку или две желатина другого цвета.
Эти две жидкости смешаются неожиданным образом: добавленный желатин сформирует завесу в виде закрученного рулона.
Подождите, пока этот удивительный узор застынет.
Спросите свох друзей как такое сделать
(будем надеятся, что от этого у вас не убавится друзей).
Это может их озадачить, хотя скорее всего, они догадаются, что здесь не обошлось без вращения.

Это необычное перемешивание происходит из-за гироскопического эффекта в жидкостях.
Явление описывается теоремой Тейлора --- Праудмена, которая, грубо говоря, утверждает, что быстро вращающаяся жидкость приобретает направленную «жёсткость» и ведёт себя так, будто состоит из «зубочисток», параллельных оси вращения.
Чем быстрее вращение (по сравнению с внутренним движением жидкости), тем больше эта жёсткость.

Этот эффект играет важную роль в движении атмосферы и океанов.
Подробное обсуждение (интуитивно понятная часть которого не требует знания математического анализа) можно найти в классической книге Дж. Бэтчелора «Введение в динамику жидкостей».
%??? а картинки есть???

\section{Параболическая антенна}

\parbf{Задача.}
Вообразите сосуд с водой, стоящий на плавно вращающейся платформе (см. рисунок 3.2).
\begin{figure}[ht!]
\centering
\begin{lpic}[t(2mm),b(2mm),r(0mm),l(0mm)]{pics/3.2}
\lbl[t]{32.5,8;$x$}
\lbl[tl]{2,6;$S=$ поперечная площадь}
\lbl[l]{42.8,19;$y$}
\lbl[l]{26,40;$\omega$}
\lbl[tr]{22.5,10;$A$}
\lbl[tl]{42.5,10;$B$}
\lbl[bl]{42.5,28;$C$}
\lbl[b]{35,12;$F_c$}
\end{lpic}
\caption{Почему поверхность принимает параболическую форму?
}
\label{pic:3.2}
\end{figure}
Её поверхность принимает форму параболоида — как показано на рисунке.
Если разсечь эту поверхность вертикальной плоскостью, проходящей через ось вращения, то получится парабола \( y = kx^2 \).
Её крутизна определяется параметром \( k \), который можно выразить через угловую скорость \( \omega \) и ускорение свободного падения \( g \):
\begin{equation}
k = \frac{\omega^2}{2g}.
\label{eq:3.1}
\end{equation}
На Луне, где \(g_{\text{лун}} \approx g/6\), этот параболоид будет в шесть раз куче.
А на Юпитере раза в 2{,}5 более пологим.
Если же сменить скорость вращения с 33 до 78 оборотов в минуту — чуть больше, чем в два раза — то \(k\) вырастет почти в шесть раз.
Можно добиться того же же, что на Луне, но дешевле.

Почему же вода выбирает форму параболоида из бесконечного множества возможных форм?
Вот простое объяснение без использования анализа.
Единственное, что нужно знать, — это то, что центробежная сила, действующая на материальную точку массой \( m \), вращающуюся по окружности радиуса \( r \), равна:
\begin{equation}
m\omega^2 r,
\label{eq:3.2}
\end{equation}
всё это объясняется в приложении на стр. ???.

Задача — найти глубину на любом расстоянии \( x \) от оси вращения (см. рисунок 3.2). Вот простой ключ к решению.

Центробежная сила, действующая на горизонтальный столбик воды \(AB\), создаёт дополнительное давление в точке \(B\).
Это давление поддерживает вертикальный столб воды \(BC\).
То есть:
\[
p_{AB} = p_{BC}.
\label{eq:3.3}
\]

С одной стороны, давление \(p_{AB}\) в точке \(B\) равно центробежной силе \(F_c\), действующую на трубку \(AB\), делённую на её поперечную площадь \(S\):
$p_{AB} = F_c/S$.
Центробежная сила равна \( F_c = m\omega^2 r \); это объясняется на стр. ???.
Здесь масса трубки: $m = \text{плотность} \cdot \text{объём}\z= \rho x S$,
а расстояние от центра масс трубки до оси равно \( r = \frac{x}{2} \).
Значит
\[
p_{AB} = \frac{F_{AB}}S= \frac{\rho x S \cdot \omega^2 \cdot x/2}S
= \frac{1}{2} \rho \omega^2 x^2.
\]
С другой стороны, давление от вертикального столба $BC$:
\[
p_{BC} = \rho g y,
\]
где \(y\) — глубина.
Подставив всё это в уравнение \ref{eq:3.3}, получим
\[
\frac{1}{2} \rho \omega^2 x^2 = \rho g y,
\quad\text{или}\quad
y = \frac{\omega^2 x^2}{2g},
\]
что и требовалось.

Заметим, что в формуле плотность \(\rho\) сократилась, то есть на форму поверхности влияют только \(\omega \) и \(g\).
Значит, при прочих равных, у ртути будет та же поверхность что и у воды.
Так что при изготовлении зеркала телескопа, вращением расплавленного материала, о плотности можно не беспокоиться — одной заботой меньше.

\section{Моторная лодка на склоне}

\paragraph*{Вопрос.}
Представьте снова воду, вращающуюся в миске, как на рисунке 3.3.
Игрушечная лодка с дистанционным управлением плавает на поверхности.
Оператор хочет, чтобы лодка оставалась в фиксированной точке относительно земли в стороне от центра, как показано на рисунке.
В каком направлении он должен повернуть нос лодки?
И в какую сторону он должен ею управлять: прямо, вправо или влево?

\begin{figure}[ht!]
\centering
\begin{lpic}[t(2mm),b(2mm),r(0mm),l(0mm)]{pics/3.3}
\lbl[b]{36,19;$1$}
\lbl[b]{33,25;$2$}
\lbl[b]{27,28;$3$}
\end{lpic}
\caption{Куда должна направляться лодка во вращающейся воде, чтобы оставаться в неподвижном положении относительно земли?
}
\label{pic:3.3}
\end{figure}

\paragraph*{Ответ.}
Нос лодки должен быть указывать в направлении $2$.
Действительно, лодке необходимо иметь скорость, противоположную направлению потока, а также некоторую тягу от центра, чтобы не скатываться вниз по склону: центробежной силы, удерживающей лодку на склоне, уже нет, ведь лодка покоится относительно земли.

\section{Без вёсел и парусов}\label{sec:sails}

\paragraph*{Вопрос.} Вы плывёте на лодке во вращающемся бассейне.
Сможете ли вы управлять движением лодки без вёсел, гребного винта и парусов?
Сопротивлением воздуха можно пренебречь. %??? паруса не рефмуются с тем, что воздухом можно пренебречь.

\paragraph*{Ответ.}
Чтобы двигаться к центру, следует \emph{встать} во весь рост.
Однако «вверх» во вращающейся системе отсчёта не то же самое, что «вверх» относительно земли:
вставая, вам придётся придвинуться к оси.
Центробежная сила при этом уменьшается, и лодка может начать дрейфовать к оси.
Чтобы удалиться от оси, надо лечь на дно лодки, максимально увеличив расстояние до оси, тем самым усилив дрейф к краю.

\paragraph*{Вопрос.}
Моя плавучесть чуть выше нуля. Смогу ли я плавать во вращающемся бассейне?

\paragraph*{Ответ.}
Не обязательно.
Ноги имеют большую плотность, чем грудная клетка, из-за воздуха в лёгких, поэтому я обычно держусь вертикально в воде (педпочитаю держать голову сверху).
Теперь представьте, что я нахожусь ближе к краю вращающегося бассейна, где поверхность воды имеет крутой уклон. Если моё тело перпендикулярно поверхности, то оно лежит почти горизонтально.
В этом положении мои ноги окажутся дальше от оси, и центробежная сила, потянет их вниз;%
\footnote{Моя голова будет ближе к оси вращения, чем остальное тело --- я буквально почувствую лёгкость в голове и тяжесть в ногах.}
это может превзойти мою плавучесть и я утону.

Однако есть несколько способов спастись.
Во-первых, можно попробовать держаться на воде так, чтобы ноги были ближе к поверхности и направлены вниз по склону.
А ещё можно добраться до стенки басейна, опуститься до дна и ползти по дну к центру (стараясь двигаться ногами вперёд).
В центре центробежная сила исчезнет, и можно всплывать.

\section{Айсберг}

\paragraph*{Вопрос.} Чувствуют ли айсберги вращение Земли?
(Разумеется, на айсберги влияют течения и ветры, которые сами по себе зависят от вращения Земли, но я спрашиваю о более прямом воздействии.)

\paragraph*{Ответ.}
Вращение Земли создаёт силу, тянущую айсберги к экватору.
Это уже объяснялось в задаче о плавающей пробке (страница ???),
но я повторю его для айсберга.
Представьте, что айсберг ни откуда не приплыл, а намёрз из воды, которую он и вытеснил.
Вода расширяется, превращаясь в айсберг, часть которого поднимается над поверхностью.
Это увеличивает среднее расстояние этой воды до оси вращения Земли.
Поскольку центробежная сила возрастает с удалением от оси, айсберг испытывает б\'{о}льшую центробежную силу — в направлении от оси!
Эта сила и тянет айсберг к экватору, как показано на рисунке \ref{pic:3.4}.

\begin{figure}[ht!]
\centering
\begin{lpic}[t(2mm),b(2mm),r(0mm),l(0mm)]{pics/3.4}
\lbl[t]{12.5,21,-14;{\footnotesize экватор}}
\end{lpic}
\caption{Из-за вращения Земли айсберги тянуться к экватору.}
\label{pic:3.4}
\end{figure}

А насколько эта сила велика?
Грубая оценка, которой я не хочу вас утомлять (а может, и раздражать), показывает, что для айсберга протяжённостью около \(\approx 10\) км и толщиной \(0{,}2\) км скорость устойчивого дрейфа, вызываемого этой силой, будет порядка \(1\) м/с:
при этой скорости сопротивление уравновешивает центробежную силу.

Метр в секунду — это скорость пешехода, и она не такая уж малая даже по сравнению со скоростью некоторых океанических течений.
Однако есть одно «но»: чтобы достичь этой скорости, начиная с нуля, потребуется примерно год — настолько мало ускорение.
С другой стороны, некоторые айсберги не тают год и два, так что времени вполне достаточно.
Действительно, если принять среднюю скорость за \(0{,}5\) м/с, то за 1 год $\approx 3{,}15\cdot10^7\approx  \pi\cdot10^7$%
\footnote{Это забавное приближение $1$ год $\approx\pi\cdot10^7$ секунд я узнал от Тадаси Токиэды.}
секунд мы проделаем расстояние порядка $(\pi/2)\cdot10^7$ м, или около $15\,000$ км — примерно от полюса до экватора!
Это меня поразило.
Разумеется, я пренебрегал гораздо более сильным влиянием ветров и течений,
но центробежный эффект хоть и слаб, но действует постоянно;
а ветры и течения --- сильны, но переменчивы.
Неясно (по крайней мере мне), приводит ли эта слабая сила к тому, что айсберги действительно дрейфуют в сторону экватора.%
\footnote{Чтобы разобраться в этом вопросе, нужно больше информации о течениях и ветрах.
Теоретически можно представить как течения, при которых слабый дрейф не оказывает существенного влияния,
так и течения при которых влияние дрейфа огромно.
При наличии более полной информации о течениях и ветрах, сделав некоторые естественные допущения
задача может быть решена использованием теории динамических.
Компьютерный эксперимент может сильно помочь, особенно в случае течений, с которыми теоретические методы не справляются.
Течения океанов, вероятно, как раз относятся к такому типу.}
\chapter{Плававание и дайвинг}

\section{Ванна на колесиках}\label{Ванна на колесиках}

Вот вариант задачи о шарике с гелием, но на Земле.

\paragraph{Вопрос.}
Лодочка плавает у конца ванны.
Ванна установлена на колёсиках способных без трения ездить по полу.
В начале всё находится в состоянии покоя.
С помощью пульта управления вы заставляете лодочку переместиться с одного конца ванны в другой и ждёте пока ванна и её содержимое опять не перейдут в состояние покоя.
В какую сторону сместилась ванна?

Будем считать, что $m$ и $M$ --- массы лодочки и ванны соответственно, а $L$ --- длина ванны.

\begin{figure}[ht!]
\centering
\begin{lpic}[t(2mm),b(2mm),r(0mm),l(0mm)]{pics/4.1}
\end{lpic}
\caption{Насколько сместится ванна после того, как лодочка проплывёт слева направо?}
\label{pic:4.1}
\end{figure}

\paragraph{Ответ}
\[
\text{расстояние}=\frac{m}{m + M}L
\]
неверен.
На самом деле ванна окажется там же, где была вначале.

\paragraph{Объяснение.}
По закону Архимеда, масса лодочки равна массе вытесненной ею воды.
Поэтому перемещение лодочки с одного конца в другой эквивалентно перестановке двух равных масс.
Но такая перестановка не сдвинет центр масс системы вода$+$лодка относительно ванны.

Центр масс также не может сдвинуться относительно пола, поскольку к ванне не приложены внешние силы.
Следовательно, ванна не передвинулась.

\paragraph{Вопрос.}
Круглое блюдо, наполненное водой, уравновешено на узкой опоре (рисунок \ref{pic:4.2}a).
Резиновая уточка плавает у края блюда.
Вы медленно вытаскиваете уточку.
В какую сторону опрокинется блюдо, если вообще опрокинется?

\begin{figure}[ht!]
\centering
\begin{lpic}[t(2mm),b(2mm),r(0mm),l(0mm)]{pics/4.2}
\lbl[bl]{0,22;(a)}
\lbl[bl]{38,22;(b)}
\end{lpic}
\caption{Опрокинется ли блюдо, если осторожно вытащить уточку?}
\label{pic:4.2}
\end{figure}

\paragraph{Ответ.}
Блюдо не опрокинется; оно останется в равновесии.
Это видно из следующего мысленного эксперимента.
Во-первых, по закону Архимеда, уточку можно заменить вытесненной ей водой --- это влияет на равновесие.
Теперь вытаскивание уточки эквивалентно высасыванию этой области воды.
Поскольку всё это делается медленно, вода успевает перераспределиться, и можно считать, что высасывается слой воды равной толщины, а это не нарушает равновесия.

\paragraph{Вопрос.}
Изменится ли ответ, если блюдо не было бы симметричным, как например на рисунке~\ref{pic:4.2}b?

\paragraph{Ответ.}
Равновесие может нарушиться.
Например, если удалять воду выше пунктирной линии на рисунке \ref{pic:4.2}b, то центр масс воды будет смещаться влево, соответсвенно и блюдо опрокинется влево.

\section{Задача с углублением}\label{Углублённая задача}

Не начинайте эту задачу, не разобравшись с предыдущей.

\paragraph{Загадка.}
На рисунке \ref{pic:4.3} показана лодочка из предыдущей задачи, но теперь она подцеплена снизу натянутым тросиком, который прикреплённ к колесику, катящемуся по подводной рельсе, прикреплённой к ванне.
Как и в предыдущей задаче,
в начале всё покоится,
далее лодочка плывёт к противоположному концу ванны,
останавливается,
и мы ждём некоторое время пока ванна и её содержимое не придут в состояние покоя.
Вопрос тот же: сдвинется ли ванна? и если да, то в каком направлении?

\begin{figure}[ht!]
\centering
\begin{lpic}[t(2mm),b(2mm),r(0mm),l(0mm)]{pics/4.3}
\lbl[r]{1,2.3;колёсико}
\lbl[t]{37,5.5;рельса}
\end{lpic}
\caption{В каком направлении сдвинется ванна, если тросик подтягивет лодочку вниз?}
\label{pic:4.3}
\end{figure}

\paragraph{Кое-какие соображения.}
В предыдущей задаче, где троссик отсутствовал, мы установили, что ванна останется на месте.
Но ведь вертикальная сила, создаваемая тросиком, не может повлиять на горизонтальное движение ванны и лодочки.
Значит, ванна останется неподвижной, как и в предыдущей задаче.

Верно ли сказанное, а если нет, то где ошибка?

\paragraph{Ответ (теперь правильный).}
Ванна сдвинется в том же направлении, что и лодочка.
Объяснение аналогично задаче про шарик с гелием на странице~\pageref{Гелиевый шар} --- по сути, это та же задача.

Поскольку троссик тянет лодочку вниз, масса вытесненной ею воды превысит массу самой лодочки, то есть лодочка вытесняет больший объём, при той же массе --- как и шарик с гелием в задаче на странице~\pageref{Гелиевый шар}.
То есть, менее массивная лодочка поменялась местами с более массивной водой.
Значит, относительно ванны, центр масс смещается влево.
Но центр масс обязан оставаться неподвижным относительно земли, и, значит, сама ванна смещается вправо.

Иными словами, раз центр масс движется влево относительно ванны, сама ванна обязана двигаться вправо, ведь импульс всей системы должен оставаться нулевым.

\paragraph{Вопрос.}
А где же обман в доказательстве, что ванна не сдвинется?

\paragraph{Ответ.}
Обман в заявлении, что трос не влияет на горизонтальные силы, действующие на ванну.
Когда лодочка движется, она передвигает воду, а вода взаимодействует со стенками ванны.
Таким образом, вертикальное натяжение троса оказывает горизонтальное воздействие (хоть и непрямое).%
\footnote{Представим себе, что теперь у нас появилась возможность дистнационно подтягивать и осаблять торсик. Как мы только-что выяснили, если перегонать лодочку вправо, то и ванна перевинется вправо. Далее расслабим тросик и перегоним лодочку влево.
Согласно предыдущей задаче ванна не сдвинется.
Похоже, что можно повторять этот цикл заставляя ванну двигаться всё дальше и дальше.
Но, тогда и центр масс ванны (с её содеримым) будет двигаться вправо --- что-то не так. Где же ошибка? \pr}

\paragraph{Отрицательные массы.}
Поскольку троссик тянет лодочку вниз, её масса меньше массы вытесненной ею воды.
Поэтому можно думать, что вместо лодочки мы передвигаем в ванне отрицателную массу.
Именно эта отрицательность и приводит к неожиданному результату.

\section{Как сбросить вес за долю секунды}

\paragraph{Загадка.}
Стоя на напольных весах, смогу ли я сделать так, чтобы они показали меньший вес (ни на что не опираясь и не снимая одежду)?

\paragraph{Ответ.} Вопрос с подвохом: это можно сделать, но только на короткое время.
Для этого достаточно подогнуть колени.
Если сделать это \emph{очень} быстро, то мои ноги могут на какое-то мгновение оторваться от опоры, и весы покажут нулевой вес.
Если подгибать колени помедленней, то разница будет не столь заметной, но уменьшение веса всё равно произойдёт.
Довольно скоро придётся прекратить ускорение вниз и начать ускоряться вверх;
весы начнут показывать больший вес, пока движение не прекратится.

Всё это результат работы закона Ньютона:%
\footnote{Объясняется в приложении на странице\pageref{Законы Ньютона}.}
\[
ma=F,
\]
где $a$ --- ваше ускорение, а $F=S-W$ --- сила внешних воздействий;
здесь $S$ --- сила, с которой весы толкают меня вверх, а $W$ --- мой вес в покое.
Весы всегда показывают силу $S$.
Если я сгибаю колени, то ускоряюсь вниз: $a<0$, и, следовательно,
\[
ma=S - W < 0,
\qquad\text{то есть,}\qquad
S < W,
\]
и показания весов меньше моего веса.
Если я стою неподвижно, то $a=0$ и $S-W=0$; весы показывают печальную правду.
Если же я начну прыжок, то $a>0$, значит, $S-W>0$ и весы показывают больше моего веса.

\medskip

Следующая задача потребует немного математического анализа.

\paragraph{Задача.}
Докажите, что независимо от того, как я прыгаю на весах, среднее значение, которое они покажут будет стремиться к моему истинному весу, предполагая, что я готов ждать.

\paragraph{Решение.}
Пусть $T$ --- достаточно большое время ожидания.
Интегрируя уравнение $ma(t)=S(t) - W$, получаем
\[
\int\limits_0^T ma(t)\, dt
=
\int\limits_0^T(S(t)-W)\, dt.
\]
По основной теореме анализа (см. страницу~\pageref{Основная теорема анализа}),
\[
\int\limits_0^T a(t)\, dt
=
\int\limits_0^T v'(t)\, dt
=
v(T)-v(0).
\]
Значит
\[
m\big(v(T) - v(0)\big)
=
\int\limits_0^T S(t)\, dt - W T.
\]
Поделив на $T$, получим
\[
\frac mT \big(v(T) - v(0)\big) =
\!\!\!\!\underbrace{\frac1T \int\limits_0^T S(t)\, dt}_{\tiny \text{среднее значение}\ S}\!\!\!\!- W.
\]
Пусть $T\to\infty$ (будем считать, что я доживу до этого момента),
тогда левая часть стремится к нулю, так как $v$ ограничена
в силу моих человеческих возможностей.
Следовательно, и правая часть тоже стремится к нулю,
то есть, среднее значение
\[
\frac1T\int\limits_0^T S(t)\, dt
\]
стремится к $W$, что и требовалось доказать.

\section{Шарик под водой}\label{Шарик под водой}

\paragraph{Вопрос.}
Шарик, надутый воздухом, удерживается под водой нитью, привязанной ко дну банки, а сама банка стоит на весах.
Что произойдёт с показаниями весов сразу после того как нить разорвётся?

\begin{figure}[ht!]
\centering
\begin{lpic}[t(2mm),b(2mm),r(0mm),l(0mm)]{pics/4.4}
\end{lpic}
\caption{Что произойдёт с показаниями весов (увеличиваться, уменьшаться или не изменятся) сразу после разрыва нити?}
\label{pic:4.4}
\end{figure}

\paragraph{Правда или нет?}
До разрыва нить тянула дно банки вверх.
После разрыва эта сила исчезает, от этого банка становится как бы тяжелее.
Поэтому сразу после разрыва нити весы покажут больший вес.

\paragraph{Теперь правда.}
На самом деле всё наоборот: весы вначале покажут \emph{меньший вес}, то есть банка покажется легче.
Чтобы это понять, проследим за центром масс всей системы (банки с её содержимым).
Как только нить рвётся, шарик ускоряется вверх, и тот же объём воды ускоряется вниз.
Поскольку вода значительно плотнее воздуха, центр масс содержимого банки ускоряется вниз.
А это и означает, что сила, действующая на весы, уменьшается --- так же, как в предыдущей задаче, где я подгибал колени, уменьшая этим показания весов.

Более формально, по второму закону Ньютона (страница \pageref{Законы Ньютона}), ускорение $a$ центра масс определяется всеми силами, действующими на банку, их всего две --- реакция опоры (весов) и сам вес; то есть,
\[\text{реакция}-\text{вес}=ma.\]
Когда нить рвётся, начальное ускорение направлено вниз, то есть $a\z<0$, это означает, что
\[
\text{реакция}-\text{вес}<0
\qquad\text{или}\qquad
\text{реакция}<\text{вес},
\]
то есть сила реакции опоры (её и показывают весы) меньше веса.

\paragraph{А где была ошибка?}
В предыдущем рассуждении я сказал правду, но не всю.
Не была упомянута сила \emph{воды}, действующая на банку.
Действительно, в момент разрыва нити вода начинает опускаться, из-за чего давление на дно банки уменьшается, и банка кажется легче.
Утверждение, что разрыв нити делает банку тяжелее верно, но в то же самое время банка становится и легче за счёт уменьшения давления на дно.
Как показывает рассуждение с центром масс, облегчение побеждает утяжеление.

\paragraph{Мораль.}
Эта задача подтверждает, что внешность обманчива.
Шар хорошо виден, но лёгок.
Вода же гораздо массивнее и, следовательно, куда важнее, но её легко упустить из вида (может, потому что она прозрачная).
Я потратил время на видимое и несущественное, а упустил существенное и невидимое.
В физике как и в жизни, пустышки могут привлечь больше внимания, чем того заслуживают.

\section{Аквалангист в цистерне}

\paragraph{Вопрос.}
Аквалангист со слегка положительной плавучестью плавает на глубине большой прямоугольной цистерны; ему приходиться работать ластами чтобы не всплыть.
Рядом с этой цистерной стоит такая же, заполненная водой до того же уровня, но без аквалангиста.%
\footnote{Иными словами, объём воды вместе с аквалангистом в первой цистерне равен объёму воды во второй.}
Какая из цистерн тяжелее?

\paragraph{Парадокс.} Вот два противоречащих друг другу рассуждений:

(A): С одной стороны, первая цистерна явно легче, поскольку объём её содержимого совпадает с объёмом содержимого второй цистерны, а плотность аквалангиста меньше плотности воды.

(B): С другой стороны, так как глубина воды в цистернах одинакова, давление воды на их дно также одинаково.
Значит, цистерны и весят одинаково.

Где же ошибка?

\paragraph{Ответ.}
Верно рассуждение (A) --- против алгебры не попрёшь.
Остаётся найти ошибку в (B).
Плавучий аквалангист должен работать ластами, чтобы оставаться под водой;
в движущейся воде давление не обязано быть таким же, как в неподвижной воде на той же глубине.%???воде-воде
Более того, само понятие глубины плохо определено, ведь поверхность воды не вполне плоская.
Когда аквалангист работает ластами он направляет воду вверх, чтобы удержаться под водой,
и над ним образуется небольшой водяной холм.
Из рассуждения (A) мы заключаем, что среднее давление воды на дно цистерны с аквалангистом меньше, чем в другой цистерне.

Следующий вопрос я оставлю читателю:

\paragraph{Задача.}
Вертолёт завис над водой.
Поток воздуха от его лопастей создаёт на поверхности воды небольшое углубление.
Выполняется ли для него закон Архимеда?
Иначе говоря, равен ли вес вертолёта приблизительно весу вытесненной воды?
Будем считать, что движение воздуха и поверхность воды установились.

\section{Проблема с весом}\label{Проблема с весом}

Два сосуда на рисунке \ref{pic:4.5} наполнены водой до одного уровня.
Днища обоих имеют одинаковую форму и размер.

\paragraph{Вопрос.}
Верно ли, что сила давления воды на дно в левом сосуде больше, чем в правом?

\begin{figure}[ht!]
\centering
\begin{lpic}[t(2mm),b(2mm),r(0mm),l(0mm)]{pics/4.5}
\end{lpic}
\caption{Верно ли, что вода в левом сосуде оказывает большее давление на дно?}
\label{pic:4.5}
\end{figure}

\paragraph{Ответ.}
Нет, неверно: на оба дна действует одинаковая сила.
Причина в том, что давление (то есть сила, приходящаяся на единицу площади) зависит только от глубины и не зависит от формы сосуда: будь то кружка или озеро, давление на одной и той же глубине одинаково!
Поскольку дно обоих сосудов находятся на одной глубине, давления там равны.
А поскольку площадь дна та же, то и силы, действующие на них, равны.

\paragraph{Вопрос.}
А что не так вот с таким рассуждением: «Поскольку вода в правом сосуде весит меньше, она давит на дно с меньшей силой»?

\paragraph{Ответ.} На рисунке \ref{pic:4.6} показано, что дно сосуда с горлышком испытывает силу, б\'{о}льшую, чем вес самой воды.
Эта избыточная сила равна направленной вниз силе от свода сосуда.
Действительно, согласно первому закону Ньютона, силы, действующие на воду, находятся в равновесии: силы, направленные вниз, уравновешивают силу, направленную вверх:
\[\text{вес} + \text{сила вниз}=\text{сила вверх}
\qquad\text{и}\qquad
\text{вес}<\text{сила вверх}.
\]
Итак, вес воды меньше, чем сила, с которой дно действует на воду снизу.

Узкое горлышко создаёт то же давление, что и широкое!

\begin{figure}[ht!]
\centering
\begin{lpic}[t(2mm),b(2mm),r(0mm),l(0mm)]{pics/4.6}
\end{lpic}
\caption{Сила, с которой вода действует на дно, зависит только от площади дна и глубины, но не от веса воды.}
\label{pic:4.6}
\end{figure}

\chapter{Потоки и струи}

\section{Закон Бернулли и шприц}

\paragraph{Вопрос.}
Представим, что вы брызгаетесь водой из шприца, нажимая на поршень.
Вспомните о первом законе Ньютона (движение сохраняется равномерным, если на тело не действуют силы) и ответьте, надо ли прикладывать какое-то усилие, чтобы перемещать поршень с постоянной скоростью при условии, что поршень скользит без трения, а вода имеет нулевую вязкость?
Другими словами, что случится если я толкну поршень, сообщив ему некоторую скорость и отпущу;
будет ли он продолжать двигаться с той же скоростью по инерции?

\begin{figure}[ht!]
\centering
\begin{lpic}[t(2mm),b(2mm),r(0mm),l(0mm)]{pics/5.1}
\end{lpic}
\caption{Будет ли поршень двигаться равномерно по инерции при отсутствии вязкости и трения?}
\label{pic:5.1}
\end{figure}

\paragraph{Ответ.}
Чтобы двигать поршень с постоянной скоростью, придётся применить силу, даже в идеальном мире без трения и вязкости.
Первый закон Ньютона о равномерном движении по инерции здесь нельзя применять, так как \emph{некоторая часть воды ускоряется}.
А именно, вода ускоряется при подходе к выходу из цилиндра шприца; см. рисунок \ref{pic:5.2}.
\begin{figure}[ht!]
\centering
\begin{lpic}[t(2mm),b(2mm),r(0mm),l(0mm)]{pics/5.2}
\lbl[rb]{3,15;{\footnotesize медленное}}
\lbl[rt]{3,13;{\footnotesize движение\ }}
\lbl{23,14;{\footnotesize ускорение}}
\lbl[b]{57.5,17;{\footnotesize быстрое}}
\lbl[t]{57.5,11;{\footnotesize движение}}
\end{lpic}
\caption{Каким-то частицы воды приходится ускоряться, и для этого необходима разница давлений.}
\label{pic:5.2}
\end{figure}
Причина в том, что частицы жидкости подталкиваются сзади, то есть давление сзади частицы выше, чем впереди неё.
В этом и состоит \label{эффект Бернулли}\emph{закон Бернулли}, который обычно формулируют как «чем ниже давление, тем выше скорость».
Из-за этого может показаться, что увеличение скорости вызывает понижение давления.
На самом деле всё наоборот: «жидкость ускоряется в направлении понижения давления».
В частности, если давление вдоль потока понижается, то повышается скорость.

Мы одновременно сформулировали закон Бернулли и объяснили его.
Таким образом, закон Бернулли это частный случай второго закона Ньютона.

Закон Бернулли наводит на аналогию с падающим камнем: чем ниже уровень/потенциальная энергия камня, тем быстрее он движется.
Более того, про закон Бернулли можно думать и как о частном случае закона сохранения энергии.%
\footnote{Эта связь ещё более очевидна, если всё же записать закон Бернулли формулой: \[\tfrac12\rho v^2 + p = \text{const},\]
здесь $v$ --- скорость устоявшегося потока несжимаемой жидкости, $\rho$ --- плотность, а $p$ --- давление в той же точке; силой тяжести пренебрегли.\pr}

\paragraph{Задача.}
Какую силу нужно приложить, чтобы двигать поршень с постоянной скоростью $v$?
Площадь поршня равна $A$, а площадь выходного отверстия — $a$.

\paragraph{Решение.}
Когда мы прикладываем силу $F$, чтобы переместить поршень на расстояние $D$, мы совершаем работу $W = F D$.
Вся эта работа тратится на увеличение кинетической энергии воды (ведь мы предположили отсутствие трения и вязкости):
\begin{equation}
F \cdot D = \frac{m v_{\text{вых}}^{2}}{2} - \frac{m v^{2}}{2},
\label{eq:5.1}
\end{equation}
где $m$ — масса вытесненной воды.
Остаётся выразить $F$ через $v$, $a$ и $A$.
Сначала заметим, что $m = \rho A D$, где $\rho$ — плотность воды.

Кроме того, так как вода несжимаема, объём, вытесняемый поршнем в секунду ($vA$), равен выходящему объёму:
\[
vA = v_{\text{вых}} a
\qquad\text{или}\qquad
v_{\text{вых}} = \frac{A}{a} v,
\]
о чём вы наверно сразу догадались.
Подставляя всё это в~\eqref{eq:5.1}, получаем
\begin{equation}
F = \frac12\rho A v^{2} \left[ \left(\frac{A}{a}\right)^2  - 1 \right].
\label{eq:5.2}
\end{equation}

Приведём пару интересных следствий этой формулы.
Давайте посмотрим, что будет при разных отношениях площадей $A/a$ и фиксированной скорости поршня $v$.
\begin{enumerate}
\item
Предположим, что выходное отверстие узкое, то есть отношение $A/a$ велико.
Тогда, согласно~\eqref{eq:5.2}, требуемая сила $F$ также велика.
Проталкивать воду через узкое отверстие трудно, но не из-за вязкости, как можно было бы подумать, а из-за того, что надо постоянно ускорять частицы.
Совершаемая нами работа идёт не в тепло (кинетическую энергию хаотического движения), а в кинетическую энергию упорядоченного движения выбрасываемой воды.
\item
Теперь предположим, что трубка расширяется, а не сужается; то есть $A/a < 1$, как на рисунке~\ref{pic:5.3}.
Тогда, согласно~\eqref{eq:5.2}, сила $F$ отрицательна.
Это означает, что, для поддержания постоянной скорости, поршень придётся придерживать!
Опять же, это можно увидеть и без формулы, ведь вода выходит с меньшей скоростью, чем скорость внутри цилиндра.
Это значит, что придётся замедлять частицы воды, придерживая поршень.
\end{enumerate}

\begin{figure}[ht!]
\centering
\begin{lpic}[t(7mm),b(2mm),r(0mm),l(0mm)]{pics/5.3}
\lbl[b]{20,17.5;\parbox{45mm}%
{\centering\footnotesize  Оттягиваем поршень\\для поддержания\\постоянной скорости.}}
\lbl[t]{18,3;\parbox{12mm}%
{\centering\footnotesize низкое\\давление}}
\lbl[t]{27,11;\parbox{12mm}{\centering\footnotesize быстрое\\движение}}
\lbl[b]{46,15.5,7;{\footnotesize замедление}}
\lbl[t]{46,12.5,-7;{\footnotesize движения}}
\lbl[b]{63,13,90;\parbox{20mm}{\centering\footnotesize медленное\\движение}}
\lbl[t]{73,13,90;\parbox{20mm}{\centering\footnotesize атмосферное\\давление}}
\lbl[b]{48,26,40;\parbox{30mm}{\centering\footnotesize снижение давления\\тормозит частицы}}
\end{lpic}
\caption{Поддержание постоянной скорости в расширяющейся трубке требует постоянного придерживания (торможения) поршня.}
\label{pic:5.3}
\end{figure}

\section[Коктейльная трубочка]{Коктейльная трубочка и\\ необратимость времени}

\paragraph{Вопрос.}
Требуется ли больше усилий, чтобы втянуть, или выдуть воду из коктейльной трубочки (см. рисунок \ref{pic:5.4})?
Предполагается, что вода в трубочке движется с постоянной скоростью и что сила тяжести вязкость не играют заметной роли.

\begin{figure}[ht!]
\centering
\begin{lpic}[t(2mm),b(2mm),r(0mm),l(0mm)]{pics/5.4}
\end{lpic}
\caption{Одинаковые ли усилия нужны, чтобы поддерживать скорость постоянной?}
\label{pic:5.4}
\end{figure}


\paragraph{Ответ.}
Втягивать трудней; причина объясняется на рисунке \ref{pic:5.5}.
При всасывании вода поступает в трубочку со всех направлений, как на части (b) рисунка.
В среднем, частицы воды заметно увеличивают скорость, подходя к отверстию.
Это ускорение как раз и обеспечивается всасыванием.%
\footnote{Тот самый эффект Бернулли со страницы \pageref{эффект Бернулли}.}
Иными словами, мы должны затратить энергию на разгон жидкости, а это требует усилий.

\begin{figure}[ht!]
\centering
\begin{lpic}[t(2mm),b(2mm),r(0mm),l(0mm)]{pics/5.5}
\lbl[tl]{0,47;(a)}
\lbl[tl]{40,47;(b)}
\end{lpic}
\caption{Что трудней всасывать или выдувать воду (поддерживая скорость постоянной)?}
\label{pic:5.5}
\end{figure}

С другой стороны, как показано на рисунке \ref{pic:5.5}a, вытекающая вода образует струю.
Поскольку струя расширяется медленно, давление тоже меняется медленно вдоль потока.

\paragraph{Направление времени.}
Можно было бы подумать, что, изменив направление потока в трубочке, мы просто изменим направление движения воды повсюду;
но с детства всем известно, что так не получится.
Задуть свечу легко, а погасить её, втягивая воздух (с безопасного расстояния, не обжигая губ), невозможно.
Я был бы рад услышать подробности от тех, кому это удалось (но не от их адвокатов).

\begin{figure}[hb!]
\centering
\begin{lpic}[t(2mm),b(2mm),r(0mm),l(0mm)]{pics/5.6}
\end{lpic}
\caption{Поток, начинающийся как входящая струя, нестабилен и через мгновение начнёт выглядеть так, как на рисунке \ref{pic:5.5}.}
\label{pic:5.6}
\end{figure}

Вопрос предпочтительного направления времени много лет занимал учёных.
Его суть в следующем кажущемся противоречии.
В классической механике, законы Ньютона обратимы во времени.
И всё же, если рассматривать систему с большим числом классических частиц, например идеальный газ, то кажется, что обратимость времени исчезает.
Разрешение этого противоречия состоит в том, что ситуация, показанная на рисунке \ref{pic:5.6} (та же, что на рисунке \ref{pic:5.5}a, но с обращённым временем), хоть и возможна теоретически, но крайне неустойчива: даже если искусственно заставить жидкость двигаться как на рисунке \ref{pic:5.6}, то через мгновение всё сломается и жидкость начнёт двигаться как на рисунке \ref{pic:5.5}b.

\section{Как двигаться в космическом корабле}\label{Как двигаться в космическом корабле}

\paragraph{Вопрос.}
Представьте, что вы зависли посредине кабины космического корабля.
Вы уже отдохнули и хотели бы добраться до стены.
Можно чего-нибудь бросить (например, ботинок или ремень%
\footnote{Они ведь вообще не нужны в невесомости --- ходить негде и штаны не спадают.}) и начать движение в противоположную сторону, но предположим, что кидаться предметами запрещено.
Как же добраться до стены?

\begin{figure}[ht!]
\centering
\begin{lpic}[t(2mm),b(2mm),r(0mm),l(0mm)]{pics/5.7}
\lbl[b]{8,41;вдох}
\lbl[b]{27.5,41;выдох}
\lbl[b]{50,41;вдох}
\lbl[b]{73,40;выдох}
\end{lpic}
\caption{Движение в невесомости за счёт дыхания.}
\label{pic:5.7}
\end{figure}

\paragraph{Ответ.}
Просто дышите.
При вдохе вы втягиваете воздух со всех сторон, а при выдохе он выходит струёй, как показано на рисунке~\ref{pic:5.7}.
В совокупности один цикл вдох-выдох выбрасывает воздух в направлении этой струи, это заставит вас двигаться в противоположную сторону.
Вы начнёте передвигаться как крайне малоэффективный кальмар.%
\footnote{Кальмары движутся по тому же принципу, только выбрасывают воду сзади.}
Если дышать ртом, то движение будет медленнее (что неудивительно), ведь воздух выталкивается через рот с меньшей скоростью, чем через нос.

\section{Загадка о садовой поливалке}

\paragraph{Вопрос.}
Садовая поливалка состоит из $S$-образной трубки, вращающаяся вокруг точки $P$, как показано на рисунке \ref{pic:5.8}.
Вода подаётся через шланг, и сила струи заставляет поливалку вращаться.
В каком направлении будет вылетать вода относительно наблюдателя на земле?
Считайте, что вращение происходит без трения и с постоянной скоростью.


\begin{figure}[ht!]
\centering
\begin{lpic}[t(2mm),b(2mm),r(0mm),l(0mm)]{pics/5.8}
\lbl[bl]{0,35;(a)}
\lbl[bl]{50,35;(b)}
\lbl[br]{20,0;шланг}
\lbl[br]{67,0;шланг}
\lbl[b]{23,28;вращение}
\lbl[b]{22,23;$P$}
\lbl[b]{77,33.5;полуокруность}
\end{lpic}
\caption{В каком направлении струя выходит из трубок?}
\label{pic:5.8}
\end{figure}

\paragraph{Ответ.}
Вода будет выходить в радиальном направлении, то есть прямо от точки $P$; направление, показанное на рисунке \ref{pic:5.8}a, неверно!
Касательная составляющая скорости воды равна нулю, как на рисунке 5.9.

\paragraph{Объяснение.}
Вода, подаваемая через шланг, не вращается вокруг вертикальной оси.
Единственное, что могло бы заставить её вращаться, --- это трение в шарнире, но по условию оно отсутствует.
Поэтому вода выходит с тем же нулевым вращением, с каким вошла.
Значит ей придётся вылетать в строго радиальном направлении.%
\footnote{Чтобы сделать это рассуждение совсем точным, достаточно заменить расплывчатый термин \emph{вращение} на точный термин \emph{момент импульса} и сказать, что момент импульса не меняется из-за отсутствия момента силы.}

\paragraph{Ещё вопрос.}
Что если переделать поливалку, придав каждому плечу форму полуокружности, как показано на рисунке \ref{pic:5.8}b.
Хорошая ли это идея?

\paragraph{Ответ.}
Идея любопытна, но поливалка станет непригодной для полива.
Вода будет выходить с нулевой скоростью относительно земли, то есть литься аккуратно вниз!
Действительно, как уже установлено, вода обязана выходить в радиальном направлении.
Но радиальная составляющая скорости на выходе должна быть равна нулю, потому что плечо имеет форму полуокружности.%
\footnote{Действительно, радиальная составляющая этой скорости та же что в системе отсчёта крутящейся вместе с поливалкой.
А в этой системе струя воды должна быть направлена по касательной к трубке, то есть перпендикуляно к направлению к $P$; то есть у неё нулевая радиальная составляющая. \pr}
Получится довольно странная поливалка: вода будет входить с положительной скоростью, а выходить с нулевой.

\paragraph{Вопрос.}
Итак, в нашу странную поливалку вода втекает с положительной кинетической энергией, а вытекает с нулевой скоростью и, следовательно, с нулевой кинетической энергией.
Куда же девается кинетическая энергия?

\begin{figure}[ht!]
\centering
\begin{lpic}[t(2mm),b(2mm),r(0mm),l(0mm)]{pics/5.9}
\lbl[l]{33,26;\parbox{32mm}%
{\centering\footnotesize  поршень движется с\\ постоянной скоростью}}
\lbl[r]{18,8;\parbox{22mm}%
{\centering\footnotesize  я придерживаю\\ поршень}}
\end{lpic}
\caption{Если поршень движется вверх, то его надо придерживать, чтобы скорость оставалась постоянной.}
\label{pic:5.9}
\end{figure}

\paragraph{Ответ.}
Энергию высасывает поливалка --- в самом прямом смысле;
это не фигура речи, а объяснение физического процесса.
Поливалка сосёт в том смысле, что давление в шланге отрицательно%
\footnote{Точнее, оно ниже атмосферного давления.}
(рисунок \ref{pic:5.9}).
Это происходит потому, что вода во вращающейся трубе выбрасывается наружу центробежным эффектом, создавая разрежение.

\paragraph{Водяной кнут.}\label{Водяной кнут}
Что же будет с поливалкой, если поршень на рисунке \ref{pic:5.9} не удерживать?

Вращающиеся плечи создают центробежное разрежение, которое ускоряет движение воды.
Поливалка начнёт вращаться быстрее и быстрее, пока вся вода не выльется.%
\footnote{Попробуйте разобраться не следует ли из этого наблюдения существование вечного двигателя.\pr}
Это поведение схоже, как это ни странно, со щелчком кнута.
Если послать волну вдоль кнута, то двигаясь к кончику, она укорачивается%
\footnote{Так как толщина кнута уменьшается от начала к концу.\pr}%
, и та же энергия концентрируется в очень короткой волне у конца.
При правильном движении концентрация энергии может достичь того, что кончик кнута превысит скорость звука.
Похожая, хотя и менее впечатляющая, концентрация энергии происходит и в нашем мысленном эксперименте с поливалкой.

\section{Быстрый слив с нулевой скоростью}

\paragraph{Вопрос.}
На рисунке \ref{pic:5.10} показан бак с присоединённым резиновым шлангом.
Можно ли сливать воду из шланга так, чтобы она вытекала с нулевой скоростью?%
\footnote{Скорость меряется относительно земли.}

\begin{figure}[ht!]
\centering
\begin{lpic}[t(2mm),b(2mm),r(0mm),l(0mm)]{pics/5.10}
\end{lpic}
\caption{Можно ли сливать воду из бака так, чтобы вода при выходе из шланга имела (почти) нулевую скорость?}
\label{pic:5.10}
\end{figure}

\paragraph{Ответ.}
Надо просто двигать конец шланга со скоростью, противоположной скорости выходящей воды;
тогда вода будет выходить с нулевой скоростью.
Это то же, что бросить яблоко назад из движущегося автомобиля со скоростью, равной скорости автомобиля.
В момент броска яблоко будет иметь нулевую скорость относительно земли.

Странная поливалка, показанная на рисунке \ref{pic:5.8}b, как раз может сливать воду с нулевой скоростью, если подсоединить её к баку.
Более того, эта поливалка идеально подходит для опорожнения бака: она позволяет очень быстро сливать жидкость из одного бака в другой, без брызг и без насоса.
Действительно, поскольку поливалка создаёт разрежение (как объясняется на странице \pageref{Водяной кнут}), она сама работает как насос!%

\section{Загадка о замороженной струе}

Следующий вопрос пришёл ко мне в голову, пока я мыл посуду и наблюдал, как струя воды из крана ударяется о дно раковины.

\paragraph{Вопрос.}
Вода равномерно льётся из банки на плоскую платформу весов, и разлетается в стороны, как показано на рисунке \ref{pic:5.11}.
Показания весов измеряют силу удара струи.
Что больше: сила удара струи или вес падающей струи?
Или же они примерно равны?
\begin{figure}[ht!]
\centering
\begin{lpic}[t(2mm),b(2mm),r(0mm),l(0mm)]{pics/5.11}
\lbl[ul]{19,9;$W_1$}
\lbl[ul]{61,9;$W_2$}
\lbl[l]{54,30;\parbox{32mm}%
{\footnotesize  замороженная\\ водяная струя}}
\end{lpic}
\caption{Как соотносится сила удара падающей воды с её весом?}
\label{pic:5.11}
\end{figure}
Будем игнорировать сопротивление воздуха, поверхностное натяжение и другие относительно несущественные эффекты.
Вертикальной скоростью воды у горлышка банки и внутри неё следует пренебречь.

\paragraph{Ответ.}
В каждый момент времени о платформу бьётся лишь небольшая часть воды.
Это может навести на мысль, что сила удара меньше веса всей струи, но на самом деле эти силы примерно равны!

\paragraph{Неформальное объяснение.}
Импульс%
\footnote{Импульс объясняется в приложении, на странице~\pageref{Импульс}, здесь говорится об импульсе в вертикальном направлении.}
всей воды не меняется, пока струя льётся равномерно.
Действительно, у струи импульс не меняется, так как сама струя не меняется в процессе течения, а импульс остальной воды равен нулю.

Как объяснялось в конце раздела~\ref{sec:A.4}, постоянство импульса означает, что сумма сил, действующих на воду, равна нулю:
\[
W - R = 0,
\]
где $W$ — вес воды, а $R$ — сумма реакций со стороны банки, весов и нулевого уровня.
Соотношение $W = R$ даёт
\begin{equation}
W_{\text{банка}} + W_{\text{струя}} + W_{\text{ноль}}
= R_{\text{банка}} + R_{\text{весы}} + R_{\text{ноль}}.
\label{eq:5.3}
\end{equation}
Но $W_{\text{банка}} = R_{\text{банка}}$ и $W_{\text{ноль}} = R_{\text{ноль}}$, ведь можно считать, что вода в банке и на нулевом уровне находится в покое.
Выбросив эти слагаемые из~\eqref{eq:5.3}, получаем:
\[
W_{\text{струя}} = R_{\text{весы}},
\]
что и требовалось доказать.

\section{Загадка о завихрении}

\paragraph{Краткая справка.}
Для этой задачи нам потребуется следующее утверждение;
оно будет расшифровано чуть ниже.
\begin{quote}
\emph{Зав\'{и}хренность невязкой жидкости остаётся равной нулю, если она была равна нулю изначально}.
\end{quote}
Это частный случай теоремы Кельвина.%
\footnote{Формулировку и доказательство этой теоремы можно найти, например, в учебнике Бэтчелора.
Отмечу, что для двумерных жидкостей, есть другое более наглядное доказательство, которое почти сразу вытекает из следующих двух фактов: (1) круглый сгусток жидкости не испытывает крутящего момента (относительно своего центра) из-за отсутствия вязкости и (2) площадь сгустка не изменяется при его переносе потоком жидкости (при этом сгусток не обязан оставаться круглым; предполагается лишь, что в один из моментов времени он был круглым).}

\begin{figure}[ht!]
\centering
\begin{lpic}[t(7mm),b(2mm),r(0mm),l(0mm)]{pics/5.12}
\lbl[b]{58,34;\parbox{32mm}%
{\footnotesize\centering  вертикальная\\составляющая\\скорости}}
\lbl[b]{85,34;\parbox{32mm}%
{\footnotesize\centering  горизонтальная\\составляющая\\скорости}}
\lbl{70,7; $\mathrm{curl}=\omega_1+\omega_2$ %??? или $\mathrm{rot}=\omega_1+\omega_2$???
}
\lbl[t]{60,19;$\omega_1$}
\lbl[r]{73.5,25;$\omega_2$}
\end{lpic}
\caption{Завихренность жидкости есть сумма угловых скоростей двух бесконечно малых штрихов, сходящихся в данной точке, в тот момент когда они перпендикулярны друг другу.}
\label{pic:5.12}
\end{figure}

\paragraph{Что такое зав\'{и}хренность?}\label{def:завихренность}
Название подсказывает, что завихренность%
\footnote{Ударение на первый слог.\pr}
мерит как вращается жидкость.
Сейчас я объясню как её мерить для двумерных жидкостей.
Представим, что в какой-то момент я добавил к жидкости краситель, нарисовав на ней плюс из двух коротких перпендикулярных штрихов, как показано на рисунке~\ref{pic:5.12}.
Этот плюс станет поворачиваться по мере переноса потоком (он также будет растягиваться, но это нас не интересует);
давайте измерим угловые скорости $\omega_1$ и $\omega_2$ этих штрихов в начальный момент, когда они ещё перпендикулярны.
По определению, завихренность жидкости в точке $p$ равна сумме $\omega_1+\omega_2$.
Если бы жидкость, вращалась как твёрдое тело, то её завихренность равнялась бы удвоенной угловой скорости твёрдого вращения.

\paragraph{Загадка о развороте жидкости.}

На рисунке \ref{pic:5.13} показана идеальная жидкость, заполняющая кольцо с поршнем внутри.
Всё находится в покое.
Затем мы проталкиваем поршень по всему кольцу и останавливаем его.
Согласно теореме Кельвина, на протяжении всего процесса завихренность должна оставаться нулевой.
Но ведь вся вода при этом разворачивается, как же это возможно?

\begin{figure}[ht!]
\centering
\begin{lpic}[t(7mm),b(2mm),r(0mm),l(0mm)]{pics/5.13}
\lbl{33,19;{\scriptsize поршень}}
\lbl[l]{41,19;\parbox{22mm}%
{\footnotesize  движение\\поршня}}
\end{lpic}
\caption{Как может жидкость вращаться, сохраняя нулевую угловую скорость
(точнее, зав\'{и}хренность)?}
\label{pic:5.13}
\end{figure}

\paragraph{Решение.}
Когда поршень движется против часовой стрелки, вода не перемещается как твёрдое тело; если бы это было так, то завихренность действительно была бы ненулевой.
Чтобы сохранять нулевую завихренность, воде приходится двигаться быстрее у внутренней стенки, чем у внешней, как показано на рисунке \ref{pic:5.14}a.
Жидкость одновременно совершает два движения:
(1) вращается против часовой стрелки вместе с поршнем и
(2) циркулирует по часовой стрелке.
В системе отсчёта, вращающейся вместе с поршнем, течение выглядит как на рисунке \ref{pic:5.14}b.

\begin{figure}[ht!]
\centering
\begin{lpic}[t(2mm),b(5mm),r(0mm),l(0mm)]{pics/5.14}
\lbl[tl]{0,40;(a)}
\lbl[tl]{48,40;(b)}
\lbl[t]{54,18.5;$S$}
\lbl[b]{41,22,-85;\parbox{22mm}%
{\footnotesize\centering  движение\\поршня}}
\lbl{33,19;{\scriptsize поршень}}
\lbl{81,19;{\scriptsize поршень}}
\lbl[t]{20,-1;{\scriptsize относительно земли}}
\lbl[t]{67,-1;\parbox{32mm}%
{\footnotesize\centering  относительно поршня\\ точка $S$ фиксирована}}
\end{lpic}
\caption{(a) Относительно земли поток быстрее на внутренней стороне.
(b) Поток в системе отсчёта, вращающейся вместе с поршнем.}
\label{pic:5.14}
\end{figure}

\paragraph{Вопрос.} Есть ли в жидкости точка, которая возвращается на место после одного оборота поршня?

\paragraph{Ответ.}
Точка $S$, %???где на отрезке она находится???S=P???
расположенная диаметрально противоположно поршню, возвращается в исходное положение. Более того, эта точка остаётся неподвижной относительно поршня на протяжении всего оборота.%Почему??? видимо здесь подразумевается, что линии потока не зависят от времени???
\footnote{Существование такой точки следует из теоремы Боля --- Брауэра о неподвижной точке, формулировку и доказательство которой можно найти в любом учебнике по топологии для студентов, например, в книге Дж. Манкриза.}

\section{Вопрос о струйном принтере}\label{Вопрос о струйном принтере}

Струйные принтеры работают, выбрасывая тонкие струи чернил на бумагу.

\paragraph{Вопрос.}
Вода (или чернила) выпрыскиваются из тонкой трубки и из-за поверхностного натяжения распадается на капли (рисунок \ref{pic:5.15}).
Летят ли капли с той же скоростью $v$ с которой вода выходит из трубки?
Силой тяжести и сопротивлением воздуха следует пренебречь.

\begin{figure}[ht!]
\centering
\begin{lpic}[t(2mm),b(5mm),r(0mm),l(0mm)]{pics/5.15}
\lbl[l]{10,2.5;{\footnotesize $v_1$}}
\lbl[l]{48,2.5;{\footnotesize $v_2$}}
\lbl[t]{23,0;$J$}
\lbl[b]{35,9;$T$}
\end{lpic}
\caption{Будут ли капли продолжать лететь с той же скоростью, с какой вылетает струя?
Сопротивлением воздуха следует пренебречь.}
\label{pic:5.15}
\end{figure}

\paragraph{Ответ.}
Капли движутся медленнее, чем струя в трубке.
Поверхностное натяжение заставляет струю $J$ стягиваться подобно резиновой ленте.
Это натяжение тянет кончик $T$ влево, к выходу из трубки, замедляя его.
Позднее этот кончик отрывается и превращается в каплю, которая теперь движется медленнее, чем чернила в трубке.%???непонятно.

\section{Загадка о водопаде}

\paragraph{Вопрос.}
Рассмотрите горизонтальную трубу на рисунке \ref{pic:5.16}.
Из трубы вытекает вода, слой $A$, выйдя из трубы, начинает падать,
слой $B$, всё ещё находящийся в трубе, пока ещё не падает.
То есть наш поток сдвигается вниз, и значит завихривается по часовой стрелке.
Значит по выходе из трубы вода приобретает ненулевую завихренность.
Но, как мы знаем, это противоречит теореме Кельвина (сформулированной выше), согласно которой завихренность меняться не может.

Где ошибка рассуждения?

\begin{figure}[ht!]
\centering
\begin{lpic}[t(7mm),b(5mm),r(0mm),l(0mm)]{pics/5.16}
\lbl[tl]{0,32;(a)}
\lbl[tl]{30,32;(b)}
\lbl[tl]{61,32;(c)}
\lbl[lb]{22,25;{\footnotesize завихривается}}
\lbl[rt]{12,11;{\footnotesize не завихривается}}
\lbl[l]{22,15;{\footnotesize $A$}}
\lbl[l]{17,18;{\footnotesize $B$}}
\lbl[b]{74,22;\parbox{42mm}%
{\footnotesize\raggedleft  река или водопроводная\\ труба}
}
\end{lpic}
\caption{Слои воды начинают опускаться, по выходу из трубы, приобретая завихренность --- или всё же нет?}
\label{pic:5.16}
\end{figure}

\paragraph{Ответ.}
Ошибка в неправильно нарисованном слое $A$.
На самом деле слои не остаются вертикальными, а наклоняются, как показано на рисунке \ref{pic:5.16}b.
Этот наклон компенсирует и вращение в моём исходном рассуждении.

\paragraph{Поток в трубе.}
Подобное противовращение наблюдается в воде, проходящей через изгиб трубы или реки.
На рисунке \ref{pic:5.16}c показано, как линия красителя движется по изогнутой трубе.
Когда труба поворачивает направо, линия красителя поворачивается влево так, чтобы сохранить нулевую завихренность. На втором изгибе трубы происходит обратное: труба поворачивает налево, линия — направо.
\chapter{Велосипеды, гимнасты и ракеты}

\section{Как качаться на качелях?}\label{Как качаться на качелях?}

\paragraph{Вопрос.}
Иногда сказать легче, чем сделать.
Но бывает, что легче сделать, чем сказать.
Обычные качели --- хороший тому пример.
Объясните, как именно ребёнок переводит энергию своих мышц в раскачивание качелей.
Ответ совсем не простой.%
\footnote{Однажды внук спросил деда:
«Ты спишь с бородой над одеялом или под?», после этого дед не мог заснуть, клал бороду и так, и эдак; оба варианта казались неудобными.}

\paragraph{Ответ (анатомия резонанса).}
Представим себе, что вы качаетесь на качелях.
Наибольшую $g$-силу вы ощущаете в самой нижней точке,
а наименьшую --- в самых высоких (крайних) точках траектории.%
\footnote{Меньшая $g$-сила в верхней точке обусловлена сразу двумя причинами: (1) центробежная сила меньше и (2) проекция силы тяжести тоже меньше.}
Теперь вообразим, что у вас в руках груз.
Внизу траектории вы поднимаете его к плечам, держите до самой верхней точки, а там быстро опускаете на колени.
Далее всё повторяется: внизу поднимаем, наверху опускаем и так далее.

\begin{figure}[ht!]
\centering
\begin{lpic}[t(2mm),b(2mm),r(0mm),l(0mm)]{pics/6.1}
\lbl[r]{19.5,4.2,-25;{\scriptsize поднимаем}}
\lbl[r]{-0.2,10.2;{\scriptsize опускаем}}
\lbl[l]{48.2,10.2;{\scriptsize опускаем}}
\end{lpic}
\caption{Движение центра масс, раскачивающее качели.}
\label{pic:6.1}
\end{figure}

Эти действия будут раскачивать качели; давайте поймём почему.
Заметим, что вы поднимаете груз, когда он тяжелей, а опускаете, когда он лёгче.
То есть суммарно вы совершаете положительную работу, и она идёт на раскачивание.

Конечно, вовсе не обязательно держать в руках груз: можно использовать голову (в самом прямом смысле), торс или ноги.
Именно это и делают дети:
внизу они выпрямляют колени и спину (поднимая вес),
а наверху сгибают колени и откидываются назад (опуская вес).

Ребёнком я всё это делал, но объяснить не мог.
Теперь --- наоборот.

\section{Почему дорожает энергия?}\label{Почему дорожает энергия?}

\paragraph{Вопрос.}
Камень падает с постоянным ускорением (сопротивлением воздуха пренебрегаем).
На рисунке \ref{pic:6.2} показана скорость камня после каждого пройденного метра.
Почему прирост скорости уменьшается с каждым следующим метром?

\begin{figure}[ht!]
\centering
\begin{lpic}[t(2mm),b(2mm),r(0mm),l(0mm)]{pics/6.2}
\lbl[l]{25,8;{\scriptsize 1 м \quad $\Delta E$}}
\lbl[l]{25,18.2;{\scriptsize 1 м \quad $\Delta E$}}
\lbl[l]{25,28.4;{\scriptsize 1 м \quad $\Delta E$}}
\lbl[l]{25,38.6;{\scriptsize 1 м \quad $\Delta E$}}
\lbl{15,38.6;{\footnotesize $16$ км/ч}}
\lbl{15,28.4;{\footnotesize $6{,}6$ км/ч}}
\lbl{15,18.2;{\footnotesize $5{,}1$ км/ч}}
\lbl{15,8;{\footnotesize $4{,}3$ км/ч}}
\lbl{5,24,90;прирост скорости}
\end{lpic}
\caption{Чем ниже падает камень, тем меньше прирост скорости на каждый пройденный метр.}
\label{pic:6.2}
\end{figure}

\paragraph{Ответ.}
По мере того как камень ускоряется, он тратит всё меньше времени на прохождение каждого следующего метра и, соответственно, имеет всё меньше времени, чтобы увеличить скорость.

Вот более строгое объяснение.
Если камень падает с высоты $h$, то в начале его потенциальная энергия%
\footnote{Потенциальная энергия, по определению, есть работа, необходимая для поднятия массы на высоту $h$.
Эта работа равна: сила $\cdot$ высоту $=$ вес $\cdot\  h=mgh$, так как вес равен $mg$.}
равна $mgh$
(мы считаем, что он отпущен без начальной скорости, то есть его кинетическая энергия равна нулю).
Непосредственно перед ударом о землю вся энергия становится кинетической:
$\tfrac{mv^2}2$.
Приравняв эти два выражения и сократив $m$, получим
\[
v^2=2gh
\qquad\text{или}\qquad
v=\sqrt{2gh}.
\]
Таким образом, зависимость $v$ от $h$ задаётся параболой, с рогами в бок; наклон её касательной убывает по $h$.
В частности, по мере увеличения $h$, равные приращения $h$ дают меньшие приращения~$v$.

\paragraph{Вопрос.} А откуда мы знаем, что вес тела массы $m$ равен $mg$?

\paragraph{Ответ.}
По определению, $g$ --- это ускорение, вызываемое силой тяжести $W$ (то есть весом) действующей на свободно падающую массу.
Следовательно, по второму закону Ньютона ($F=ma$, стрaница~\pageref{Законы Ньютона}), имеем $W=mg$.

\section[Большие обороты на перекладине]{Большие обороты на перекладине\\и хомячок в колесе}\label{Большие обороты на перекладине}

Большие обороты --- это базовый гимнастический элемент на перекладине, гимнаст стоит на руках, совершает оборот, прокручиваясь под перекладиной, снова поднимается в стойку на руках, и так далее (рисунок \ref{pic:6.3}a).

\begin{figure}[ht!]
\centering
\begin{lpic}[t(2mm),b(2mm),r(0mm),l(0mm)]{pics/6.3}
\lbl[tl]{0,77;(a)}
\lbl[tl]{46,77;(b)}
\lbl[tl]{30,31;(c)}
\lbl[b]{47,51,90;{\scriptsize начало}}
\lbl[rb]{60,64;\parbox{22mm}{\footnotesize\raggedleft  ваша\\сила}}
\lbl[lb]{63,64;\parbox{22mm}{\footnotesize  сила\\на вас}}
\lbl[lt]{85,39;{\footnotesize центр масс}}
\end{lpic}
\caption{(a) Выполнение больших оборотов на перекладине.
(b) Как используется момент веса.
(c) Как подкачивать энергию в большие обороты.}
\label{pic:6.3}
\end{figure}

\paragraph{Вопрос.}
Гимнаст неподвижно висит на перекладине.
Будем считать, что между его руками и перекладиной нет трения: хват абсолютно скользящий --- запястиями невозможно создать момент силы.
Сможет ли он выполнить большие обороты?

Давайте игнорировать прогиб перекладины, сопротивление воздуха и другие малозначительные факторы.

Многие физики и математики отвечают приблизительно следующее:
\begin{quote}
\emph{«Без трения нет момента силы, а без момента силы невозможно создать момент импульса,
и значит, вращение невозможно.
В частности, без трения гимнаст не сможет выполнить большие обороты.»
\footnote{Моменты силы и импульса обсуждаются в приложении.}}
\end{quote}

Это рассуждение неверно.
К счастью, юные гимнасты, которые выполняют большие обороты, не знают физики достаточно хорошо, чтобы их остановила подобная логика.
Они не слыхали ни о моменте силы, ни о моменте импульса ---
зачастую всех прожитых ими лет не хватило бы для подготовки диссертации.
Иногда меньшие знания дают преимущество.

\paragraph{Где же ошибка?}
Неверно уже предположение об отсутствии момента силы: сила тяжести может создавать момент.
Этот момент равен нулю, пока гимнаст висит неподвижно --- это правда --- но, сгибая тело, он может сделать этот момент ненулевым.
Давайте разберёмся как.

Представим себе, что вы висите на перекладине (рисунок \ref{pic:6.3}b) и сгибаетесь в поясе, как будто пытаетесь дотянуться руками до пальцев ног.
Напрягая мышцы живота, вы тянете руки вперёд, тем самым толкая перекладину вперёд (стрелка влево на рисунке).
Перекладина в ответ толкает вас вправо --- по третьему закону Ньютона (действие равно противодействию).
Значит, ваш центр масс тоже смещается вправо --- по второму закону Ньютона.
Теперь центр масс уже не совсем под перекладиной, и сила тяжести начинает менять момент импульса, увлекая ваше тело под перекладину, как в маятнике.
Итак, сгибая тело, вы заставляете силу тяжести создавать ненулевой момент.

\paragraph{Вопрос.}
Предположим, что гимнаст уже выполняет большие обороты.
Что ему (в принципе) следует делать, чтобы их ускорить?

\paragraph{Ответ.}
Принцип точно такой же, как на качелях: нужно совершать положительную суммарную работу.
Для этого следует приближать центр масс к перекладине тогда, когда это труднее (внизу), и отдалять его от перекладины тогда, когда это легче (вверху).

Грубая пародия на это показана на рисунке \ref{pic:6.3}c.

\paragraph{Несбалансированное колесо.}
А вот другой способ понять движения гимнаста.
На рисунке \ref{pic:6.3}c показана траектория его центра масс.
Представим, что масса распределена по всей траектории --- будем думать, что она лежит на ободе колеса, с осью на перекладине.
Так как колесо смещено влево, момент силы тяжести будет направлен против часовой стрелки.
Такое смещение можно поддерживать, постоянно подстраивая спицы колеса;
это сделает момент силы тяжести постоянным.
Именно это позволит набирать скорость и компенсировать трение.

Заметим, что хомяк в колесе делает то же самое:
он смещает центр масс и тем самым заставляет момент силы тяжести, раскручивать колесо.
(Хотя сам хомяк, вряд ли, думает в этих терминах.)

\paragraph{Задача.}
Изменение длины спиц в регулируемом колесе требует работы.
Объясните поподробней, как нужно менять длины спиц, чтобы поддерживать смещение обода.

Обратите внимание на то, что натяжение у укорачивающихся спиц в среднем должно быть больше, чем у удлиняющихся.
Как и раньше, длина спицы определяет расстояние от центра масс гимнаста до перекладины.

\section{Машина на льду}

Следующую задачу я узнал от Энди Руины %???
из Корнеллского университета.

\paragraph{Задача.}
Представьте, что вы за рулём машины;
вы едете по прямой по замёрзшему озеру и надавили тормоз.
Разумеется, лучше всего, чтобы колёса продолжали крутиться,
но если всё-таки не повезло, какие бы колёса вы предпочли заблокировать: передние или задние?
Ваша цель --- остановиться, двигаясь прямо без заноса.

\paragraph{Решение.}
Как это ни странно, лучше, если заблокируются передние колёса.%
\footnote{В этом случае рулить бессмысленно, но ваша цель лишь двигаться прямо.}
В этом случае машина продолжит двигаться прямо.
Если же заблокируются задние, то машина развернётся и до полной остановки будет  двигаться задом на перёд (при условии, что руль фиксирован).
Резко вдавив задний тормоз на велосипеде, можно заметить схожий эффект:
если удалось заблокировать заднее колесо, то оно начнёт скользить в сторону.

\begin{figure}[ht!]
\centering
\begin{lpic}[t(2mm),b(2mm),r(0mm),l(0mm)]{pics/6.4}
\lbl[t]{7.5,0,9.8;{\footnotesize крутится}}
\lbl[t]{42,5,9.8;{\footnotesize скользит}}
\lbl[b]{5,15,9.8;{\footnotesize стабилизатор}}
\end{lpic}
\caption{Как оперение удерживает стрелу прямо, так и катящиеся задние колёса удерживают прямо машину (с заблокированными передними колёсами).}
\label{pic:6.4}
\end{figure}

\paragraph{Объяснение.}
Очень помогает сравнение с движением стрелы.
Стрела летит прямо, потому что оперение не даёт её хвосту уходить в сторону (см. рисунок \ref{pic:6.4}).
Тоже происходит и с машиной, когда передние колёса заблокированы, катящиеся задние колёса играют ту же роль стабилизирующего оперения.
Поэтому если задние колёса продолжают катиться, то машина приобретает устойчивость, но теряет её, если катятся только передние.

Однажды вечером, в снежную погоду, я провёл этот эксперимент на пустой заснеженной парковке.
Блокировка задних колёс (ручным тормозом), легко разворачивала машину на 180°.

\section{Как поворачивать на велосипеде}\rindex{велосипед}

\paragraph{Задача.}
Велосипедист едет прямо и внезапно решает свернуть налево.
Что при этом он делает с рулём?

\paragraph{Решение.}
Чтобы поворачивать налево, нужно чтобы велосипед наклонялся влево;
наклон нужен чтобы скомпенсировать центробежную силу.
Для создания этого наклона велосипедист на мгновение подворачивает руль вправо, при этом колёса сдвигаются из-под него вправо, а тело по инерции продолжает двигаться прямо (рисунок \ref{pic:6.5}).
Добившись так нужного наклона влево, следует поворачивать руль влево, вписываясь желаемый поворот.
Если вы проверите это на себе (как сделал я), то заметите, что руки подсознательно делают первоначальный обратный поворот.%
\footnote{Это может зависеть от стиля вождения велосипедом, попробуйте повернуть двигая только рулём, но не телом. \pr}
Наши рефлексы отлично разбираются в механике.%
\footnote{Те читатели, которые умеют управлять велосипедом не держась за руль,
могут попытаться разобраться как это у них получается.
Лучше всего сначала привести доказательство того, что это невозможно, а потом найти в нём ошибку.
Сравнение с большими оборотами (раздел~\ref{Большие обороты на перекладине}) может оказаться полезным. \pr}

\begin{figure}[ht!]
\centering
\begin{lpic}[t(2mm),b(2mm),r(0mm),l(0mm)]{pics/6.5+}
%\lbl[b]{32,8,-3;\parbox{32mm}{\footnotesize\centering  руль слегка вправо,\\ а наклон влево}}
%\lbl[tr]{72,15,13;{\footnotesize создав наклон,}}
%\lbl[tl]{72,15,56;{\footnotesize поворачиваем}}
\lbl[bl]{-4,-5;
\begin{tikzpicture}
\path [
    decorate,
    decoration={
        text along path,
        text/.expanded=\bracetext{Сначала прямо, руль чуть вправо,},
        text align=left,
    }
](0,-.8) .. controls (3,-.0) and (8,-3) .. (8.3,2.5);
\end{tikzpicture}
}
\lbl[bl]{-4,-10;
\begin{tikzpicture}
\path [
    decorate,
    decoration={
        text along path,
        text/.expanded=\bracetext{а наклон влево. Создав наклон, поворачиваем.},
        text align=right,
    }
](0,-.8) .. controls (3,-.0) and (8,-3) .. (8.3,2.5);
\end{tikzpicture}
}
\end{lpic}
\caption{Чтобы свернуть влево, велосипедист сначала чуть поворачивает руль вправо, создавая наклон влево.}
\label{pic:6.5}
\end{figure}

В дополнение к сказанному, у быстро движущегося велосипеда с массивными шинами
наклон усиливается за счёт гироскопического эффекта.
Повернув руль вправо, вы поворачиваете переднее колесо  вызываете заметный наклон влево.%
\footnote{Марк, а можно (и нужно ли) здесь сказать, «... поворачиваете переднее колесо, создавая момент импульса относительно оси в направлении движения, который должен компенсироваться наклоном велосипеда влево.»?\pr}

\section{Разгон одним наклоном}\label{Разгон одним наклоном}\rindex{велосипед}

\paragraph{Вопрос.}
Можно ли изменить скорость велосипеда, используя только руль?
Не разрешается крутить педали и двигать телом.

\paragraph{Ответ.}
Предположим, что велосипед движется по прямой.
Тогда при входе в поворот с наклоном скорость автоматически увеличится.%
\footnote{Как наклоняется велосипед при входе в поворот, описано в предыдущей задаче.}
Причина: от наклона уменьшается потенциальная энергия.
Следовательно, должна увеличиться кинетическая, а вместе с ней и скорость.%
\footnote{Строго говоря, часть кинетической энергии превращается в кинетическую энергию вращения --- ведь велосипедист теперь вращается.
Тем не менее можно показать, что остаётся достаточно энергии, чтобы вызвать увеличение скорости.}

А как же зависит прирост скорости от начальной скорости (при одном и том же угле наклона)?
Удивительно, но прирост оказывается больше при меньших скоростях.%
\footnote{По той же причине, что прыжок с в четыре раза большей высоты даёт прирост скорости лишь в два раза.}%
\footnote{Марк, а можно (и нужно ли) сравнить это с \ref{Почему дорожает энергия?}?\pr}
Один и тот же угол наклона при движении со скоростью $2$ км/ч даст больший прирост скорости, чем при $20$ км/ч.
Вот объяснение.
Когда я наклоняюсь и тем самым опускаю свой центр масс на
величину $h$, я увеличиваю кинетическую энергию, на столько же на сколько я уменьшил потенциальную:
\[
\frac{mV^2}2 - \frac{mv^2}2=mgh,
\]
где $V$ --- новая скорость, а $v$ --- начальная скорость.%
\footnote{Замечу, что я поворачиваю, и часть кинетической энергии ушла на вращение, но я этим пренебрёг.}
Сократив $m$, получим
\[
V^2 - v^2=2gh,\]
и после пары алгебраический манипуляций
\[V - v=\frac{2gh}{v+V}=\frac{2gh}{v+\sqrt{2gh+v^2}}.\]
В частности,
с ростом начальной скорости $v$
её прирост $V - v$ уменьшается.

\section{Как разогнаться на велосипеде, двигая только тело?}\label{Как разогнаться на велосипеде, двигая только тело?}\rindex{велосипед}

Как мы только что выяснили, велосипедист может разогнаться зайдя в поворот.
Но это даёт лишь небольшой, однократный прирост скорости --- его нельзя повторять, разгоняя велосипед.

\paragraph{Задача.}
Может ли велосипедист (теоретически) увеличивать свою скорость бесконечно, не крутя педали, а только двигая телом?

Чтобы исключить возможные лазейки, считаем, что ветра нет, нельзя пользоваться двигателями и так далее.

\paragraph{Подсказка.}
Колесо напоминает конёк на льду: и то и другое легко движется куда направлено и не хочет двигаться вбок.

\paragraph{Решение.}
Я начинаю движение по прямой, сидя прямо.
Моя цель --- оказаться в том же положении, но с большей скоростью.
Добиться этого можно в три шага:
\begin{enumerate}
\item Наклониться вперёд к рулю, опустив тем самым центр масс.
\item Войдя в крутой поворот, выпрямиться, тем самым поднимая центр масс.
\item Снова начать ехать прямо.
\end{enumerate}

Почему же эти действия увеличат скорость?
Заметим, что при движении по окружности, перегрузка ($g$-сила) становится больше из-за дополнительного центробежного эффекта.%
\footnote{Это подробно объясняется в следующей задаче.}
Когда я выпрямляюсь против этой большей силы тяжести, я совершаю больше работы, чем ту, которую «получаю обратно», опуская центр масс.%
\footnote{Когда я говорю, что «получаю обратно» энергию, я имею в виду использование гравитационной энергии для зарядки батареи, подобно тому, как гибридный автомобиль делает это при торможении.}
Разность этих энергий переходит в приращение моей кинетической энергии.
Тот же принцип использует ребёнок, раскачивая качели, или гимнаст, выполняющий большие обороты.

\paragraph{Другое объяснение.}
Прирост скорости можно объяснить через сохранение момента импульса.
При движении по окружности мой момент импульса относительно центра окружности остаётся постоянным (так как момент силы, действующей на меня относительно центра окружности, равен нулю --- закон сохранения момента импульса обсуждается на странице~\pageref{Закон сохранения момента импульса}).
Когда я выпрямляюсь, центр масс приближается к центру окружности, и чтобы сохранить момент импульса, должна возрасти скорость.%
\footnote{Строго говоря, велосипед с человеком не образуют замкнутую систему, ведь они взаимодействуют с землёй. Значит надо дополнительно убедиться, что импульс не вкачивается и не выкачивается из системы.\pr}


\section{Как набрать вес на мопеде}\rindex{велосипед}

\paragraph{Вопрос.}
Вы едете на мопеде ровно по окружности с постоянной скоростью, наклоняясь в сторону поворота под фиксированным углом.
Какую перегрузку ($g$-силу) вы испытываете?
Другими словами, какой вы ощущаете вес?

\paragraph{Ответ.}
На рисунке~\ref{pic:6.6} показаны две силы, действующие на мопед:
(1) реакция опоры $R$ со стороны земли и (2) его вес $W$.
Сила реакции $R$ и есть ощущаемый вес.
Равнодействующая этих двух сил --- это центростремительная сила, которая заставляет мопед двигаться по окружности.
Следовательно, эта равнодействующая направлена к центру окружности, а значит горизонтально, как это и показано на рисунке~\ref{pic:6.6}.
Треугольник $ABC$ прямоугольный, и угол при $A$ совпадает с углом наклона~$\theta$.

\begin{figure}[ht!]
\centering
\begin{lpic}[t(2mm),b(2mm),r(0mm),l(0mm)]{pics/6.6}
\lbl[br]{3,18;$B$}
\lbl[r]{29,17;$B$}
\lbl[b]{25,34;$R$}
\lbl[bl]{31,4;$\theta$}
\lbl[bl]{31,21;$\theta$}
\lbl[t]{29,-.5;$A$}
\lbl[bl]{51,18;$C$}
\lbl[br]{40,26;$R$}
\lbl[tl]{33,19;\parbox{22mm}{\footnotesize центростре-\\мительная\\сила }}
\lbl[t]{81,15;\parbox{22mm}{\footnotesize\centering центр\\ пути\\ мопеда}}
\end{lpic}
\caption{Прирост ощущаемого веса при движении по окружности.}
\label{pic:6.6}
\end{figure}

Из $\triangle ABC$ получаем
\begin{equation}
\frac{W}{R}=\cos \theta.
\label{eq:6.1}
\end{equation}
Поскольку $\cos \theta < 1$ получаем, что $R > W$ --- при равномерном повороте всегда будет перегрузка.
При $\theta=30\degree$ ощущаемый вес увеличивается на $15\%$.
Если удастся удерживать наклон в $45\degree$, то вес увеличится на $41\%$: 70-килограммовый человек будет чувствовать себя как 100-килограммовый.
При $60\degree$ (огромном наклоне) вес будет удваиваться.

Та же формула работает и для самолёта в равномерном повороте.
Например, чтобы выдерживать постоянную перегрузку $2g$,
пилот должен наклонить самолёт на $60\degree$
(так что крылья будут составлять $30\degree$ с вертикалью).
Это также показывает, что войдя в поворот на машине, вы становитесь тяжелее.
Можно определить, насколько именно, измерив угол~$\theta$ с помощью подвешенного на нитке груза
и подставив этот $\theta$ в~\eqref{eq:6.1}.

\section[Как почувствовать квадрат в mv²/2]{Как почувствовать квадрат в $\tfrac{mv^2}{2}$ через велосипедные педали?}\rindex{велосипед}

Кинетическая энергия $K$ определяется как работа, необходимая для разгона массы $m$ из состояния покоя до заданной скорости $v$.
Как мы знаем, $K=mv^2/2$;
это объясняется на странице~\pageref{Кинетическая энергия}
 в приложении.

\paragraph{Вопрос.}
Как почувствовать квадрат в $\tfrac{mv^2}{2}$  через велосипедные педали?
Будем считать, что вы едете по ровной дороге, без сопротивления воздуха и трения качения.%

\paragraph{Ответ.}\label{Ответ:Как почувствовать квадрат}
Так как $v$ возводится в квадрат, то чем быстрее вы едете,
тем больше топлива требуется, чтобы прибавить $1$ км/ч.
Действительно, чтобы разогнаться от $v$ до $v+1$, нужна энергия
\[
\frac{m(v+1)^2}{2} - \frac{mv^2}{2}
  =\frac{m(2v+1)}{2}
  =mv + \tfrac{m}{2} > mv,
\]
и эта энергетическая цена ростёт с увеличением $v$.

Чтобы интуитивно ощутить эту растущую цену, представьте, что вы нажимаете на педали с постоянной силой, тем самым ускоряясь с постоянным ускорением.
То есть на то, чтобы прибавить $1$ км/ч, уходит та же секунда, независим от того, быстро или медленно вы движетесь.
А тут возникает неприятность:
при быстром движении, педали приходится крутить тоже быстро, и за ту же секунду ваши ноги должны пройти большее расстояние.
Это означает, что за секунду быстрой езды вы совершаете больше работы, чем за секунду медленной.
Итак, чтобы поддерживать постоянное ускорение, ваш двигатель должен работать всё быстрее и быстрее, сохраняя ту же силу.
Иначе говоря, его мощность должна постоянно расти.
Так что спешка изнуряет даже без трения, а с трением дела ещё хуже.

\paragraph{Задача.}
На сколько больше топлива требуется,
чтобы разогнаться до $70$ км/ч, чем до $10$ км/ч
(пренебрегая всеми потерями на трение и предполагая идеальный двигатель со 100 \% КПД)?

\paragraph{Решение.}
Почти в $50$ раз!
Действительно,
\[
\frac{K_{70}}{K_{10}}
 =\frac{m \cdot 70^2/2}{m \cdot 10^2/2}
 =\frac{70^2}{10^2}
 =49.
\]

\section{Парадокс с ракетами}\label{Парадокс с ракетами}

\paragraph{Ускоряющаяся ракета.}
Когда ракета сжигает единицу топлива, её скорость увеличивается на определённую величину.
\emph{Эта величина не зависит от того, с какой скоростью ракета двигалась до начала сжигания топлива.}%
\footnote{Всё происходит в невесомости; скорости меряются в инерциальной системе.}
Действительно, ускорение ракеты не зависит от её текущей скорости;
в этом ракета отличается от велосипеда: чем быстрее движется велосипед, тем труднее его разгонять.%
\footnote{Дополнительные пояснения на странице \pageref{Ответ:Как почувствовать квадрат}.}
Это различие приводит к следующему любопытному выводу.

\begin{figure}[ht!]
\centering
\begin{lpic}[t(2mm),b(2mm),r(0mm),l(0mm)]{pics/6.7}
\lbl[r]{1,5;$v$}
\lbl[l]{32,30;$v$}
\lbl[t]{16,16;$O$}
\lbl[tl]{41,8;\parbox{22mm}{\footnotesize струя\\двигателя}}
\end{lpic}
\caption{Ракеты получают больше кинетической энергии, чем её было в топливе, но разве такое возможно?}
\label{pic:6.7}
\end{figure}

\paragraph{Парадокс.}
На рисунке~\ref{pic:6.7} изображены две ракеты, закреплённые на стержне, который может свободно вращаться вокруг точки $O$.
Сообщив ракетам начальное вращение со скоростью $v$, мы включаем их двигатели.
После сгорания топлива скорость каждой ракеты увеличивается на $1\ \text{м/с}$.
Это приращение одинаково независимо от начальной скорости $v$.
Теперь скорость равна $v+1$, суммарная кинетическая энергия имеет вид
\[
E_{\text{после}}=m (v+1)^2,
\]
а прирост кинетической энергии равен
\[
\Delta K=\underbrace{m(v+1)^2}_{\text{после}} - \underbrace{\phantom{(}m v^2\phantom{)}}_{\text{до}}=2mv + m.
\]
Согласно этой формуле, выбрав $v$ достаточно большим, приращение энергии $\Delta K$ можно сделать произвольно большим.
Получается, что ракеты могут приобрести больше кинетической энергии, чем содержится в их топливе!
Неужели это правда?

\paragraph{Ответ.}
Это может прозвучать неожиданно, но ответ на последний вопрос утвердительный --- ракеты могут получить за время работы двигателей больше энергии, чем содержалось в топливе.
Однако это не нарушает закон сохранения энергии, потому что есть ещё один участник процесса --- выброшенное топливо, которое теряет значительную часть своей кинетической энергии.
При больших скоростях ракеты эта потеря особенно велика.
Если учесть эту энергию, то парадокс исчезнет.

Более подробное обсуждение схожей задачи приведено на страницах~\pageref{Мячик из машины}---\pageref{end:Мячик из машины}.

\section{Ракета-кофеварка}

Некоторые кофеварки снабжены рычагом, нажимая на который, вы перекачиваете кофе в чашку (рисунок~\ref{pic:6.8}).
Обратите внимание, что струя кофе, бьющая вниз, создаёт подъёмную  реактивную силу, действующую на кофеварку.
Обычно эта сила слишком мала, чтобы поднять кофеварку, не говоря уже о преодолении давления руки, нажимающей на рычаг,
но давайте поймём может ли подобная кофеварка взлететь в принципе.

\begin{figure}[ht!]
\centering
\begin{lpic}[t(2mm),b(2mm),r(0mm),l(0mm)]{pics/6.8}
\lbl[l]{34,48;$f$}
\lbl[l]{34,2;$F$}
\lbl[l]{45,25;$\ell$}
\lbl[lb]{45,49;$L$}
\end{lpic}
\caption{Можно ли заставить что-то взлететь, надавив сверху?}
\label{pic:6.8}
\end{figure}

\paragraph{Вопрос.}
Можно ли, хотя бы в принципе, сконструировать кофеварку с такими пропорциями, чтобы она поднималась со стола, когда нажимают на рычаг?

\paragraph{Ответ} (прыгающая кофеварка).
Удивительно, но это можно сделать.
Если отношение плеч рычага $L/\ell$ на рисунке~\ref{pic:6.8} очень велико,
то сила $F$ на поршне будет огромной.
Следовательно, можно достичь произвольно реактивной тяги, прикладывая лишь слабую силу $f$
(заглавные буквы $F, L$ и их строчные варианты $f, \ell$ указывают кто из них больше, а кто меньше).
Так, что реактивная сила может превысить сумму веса и приложенной силы $f$.
К сожалению рычаг в такой кофеварке придётся двигать очень быстро, ведь для поддержания высокого давления требуется достаточно большая скорость (закон Бернулли\rindex{закон Бернулли} объясняется на странице \pageref{eq:5.2}).
Так что, к сожалению, у вас не получится разыграть кого-нибудь прыгающей кофеваркой.
А это печально, ведь если бы идея сработала, то кофе давило бы на чашку с силой превышающей вес кофеварки в сумме с силой давления на рычаг.

Теперь более подробно.
Чтобы кофеварка поднялась, реактивная сила струи $F_J$ должна превысить сумму веса%
\footnote{Вес кофеварки уменьшается из-за выливающегося кофе, но не будем обращать внимание на эти мелочи.
Для тех же, кто настаивает на большей строгости, можно сказать так: пусть $W$ --- это начальный (наибольший) вес кофеварки, до того как кофе начал выливаться.}
$W$ и силы $f$:
\begin{equation}
F_J > W + f.
\label{eq:6.2}
\end{equation}
Для определённости будем считать, что $f=W/2$.
Тогда условие отрыва примет вид
\begin{equation}
F_J > \tfrac32W.
\label{eq:6.3}
\end{equation}
Должно быть ясно, что если сила поршня $F$ достаточно велика,
то и реактивная сила струи $F_J$ окажется достаточно большой;
так что для $F_J$ будет выполняться условие~\eqref{eq:6.3}.
Таким образом, всё, что нужно, --- это создать очень большую силу поршня.
А этого можно добиться, выбрав отношение плеч рычага достаточно большим.
Действительно, по закону рычага
\[
F=\frac{L}{\ell} f,
\]
и, следовательно, силу $F$ можно сделать сколь угодно большой,
увеличивая отношение $L/\ell$.

\paragraph{Задача.}
Можно ли достичь отрыва при $f > W$; то есть нажимая с силой превышающей вес кофеварки?

\paragraph{Задача.}
Вода под давлением вытекает из одного сосуда в другой сосуд, стоящий под ним, с постоянной скоростью.
Оба сосуда находятся на платформе весов.
Что будут показывать весы по сравнению с суммарным весом сосудов и воды?

\section{Мячик из машины}\label{Мячик из машины}

\paragraph{Ситуация.}
Двигаясь в машине, я бросаю вперёд мячик, сообщая ему дополнительную кинетическую энергию.
Однако прирост энергии мячика (с точки зрения наблюдателя на земле) может превысить ту энергию, которую затратили мои мышцы при броске, не правда ли, странно?
Следующий абзац проясняет в чём дело.

\paragraph{Подробности.}
Изначально мячик двигался вместе с машиной со скоростью $V$.
Я бросил мячик вперёд со скоростью $v=1$, и его новая скорость теперь равна $V+1$.
Изменение кинетической энергии мячика равно
\begin{equation}
\Delta K
=
\underbrace{\frac{m(V+1)^2}{2}}_{\text{после}}
-
\underbrace{\phantom{(}\frac{mV^2}{2}\phantom{)}}_{\text{до}}
= \frac{m}{2}+ mV.
\label{eq:6.4}
\end{equation}

\paragraph{Парадокс.}
Согласно \eqref{eq:6.4}, чем быстрее движется машина, тем больше кинетической энергии приобретает мячик (при той же скорости броска $v=1$)!
Ещё удивительнее то, что энергия, приобретённая мячиком, может превышать энергию, затраченную моими мышцами, если машина движется достаточно быстро.
Как это объяснить?

\paragraph{Решение.}
Хотя в \eqref{eq:6.4} содержится ошибка (см. следующий абзац), странный вывод остаётся в силе: мячик действительно может получить больше энергии, чем произвела моя рука.
Однако этот прирост происходит за счёт потери кинетической энергии машины,
Нельзя игнорировать то, что происходит с машиной, как это было сделано в \eqref{eq:6.4}, даже если этот эффект кажется малым,
ведь когда я бросаю мячик вперёд, машина получает толчок назад.
Поэтому её кинетическая энергия уменьшается, и с учётом этого уменьшения получается верное значение общего прироста кинетической энергии.

\begin{figure}[ht!]
\centering
\begin{lpic}[t(2mm),b(2mm),r(0mm),l(0mm)]{pics/6.9}
\lbl[b]{14,14;$m$}
\lbl[b]{77,14;$m$}
\lbl{14,7;$M$}
\lbl{55,7;$M$}
\lbl[b]{34,10;$V$}
\lbl[b]{71,7;$V_1$}
\lbl[t]{85,11;$V_1+v$}
\lbl[t]{13,0;{ до броска}}
\lbl[t]{54,0;{ после броска}}
\end{lpic}
\caption{Как перераспределяется кинетическая энергия.}
\label{pic:6.9}
\end{figure}

\paragraph{Баланс энергии без жульничества.}
Вот точное решение парадокса (предполагается отсутствие трения, сопротивления воздуха
и других факторов, отвлекающих от основной идеи).
Сначала найдём скорость машины после броска.
Акт броска не изменяет импульса системы (см. рисунок~\ref{pic:6.9}):
\begin{equation}
(M+m)V=M V_1 + m(V_1+v),
\label{eq:6.5}
\end{equation}
где $M$ --- масса машины, $m$ --- масса мячика, $V_1$ --- новая скорость машины,
а $v$ --- скорость мячика относительно машины в момент броска.

Общее изменение кинетической энергии равно
\begin{equation}
\Delta K_{\text{общ}}
=
\underbrace{\frac{MV_1^{2}}{2}}_{\text{машина после}}
+
\underbrace{\frac{m(V_1+v)^{2}}{2}}_{\text{мячик после}}
-
\underbrace{\frac{(m+M)V^{2}}{2}}_{\text{машна и мячик до}},
\label{eq:6.6}
\end{equation}
Воспользовавшись \eqref{eq:6.5} и перескочив через алгебраические выкладки, получаем
\begin{equation}
\Delta K_{\text{общ}}=\frac{m v^{2}}2\cdot\frac M{m+M}.
\label{eq:6.7}
\end{equation}

\paragraph{Обсуждение.}
\begin{enumerate}
\item Согласно \eqref{eq:6.7}, $\Delta K_{\text{общ}}$ не зависит от начальной скорости машины $V$, как и ожидалось.

\item Если $M \gg m$, как в случае машины и мячика, то из \eqref{eq:6.7} получаем
$\Delta K_{\text{общ}} \approx  m v^2/2$,
как если бы машина была неподвижно закреплена на земле --- тоже согласуется с ожиданиями.

\item Разница энергий $\Delta K_{\text{общ}}$ разбивается на две части:
\[
\Delta K_{\text{общ}}=\Delta K_{\text{мячик}} + \Delta K_{\text{машина}}
= \frac{m v^{2}}2\cdot\frac M{m+M}.
\]
Если машина движется быстро, то $\Delta K_{\text{мячик}}$ будет сильно положительной,
а $\Delta K_{\text{машина}}$ --- сильно отрицательной.
То есть мячик ворует много энергии у машины.
В то время как общая разница $\Delta K_{\text{общ}}$ не зависит от скорости машины,
то как $\Delta K_{\text{общ}}$ разбивается на $\Delta K_{\text{мячик}}$ и $\Delta K_{\text{машина}}$ очень даже зависит.
\end{enumerate}
\label{end:Мячик из машины}
\chapter{Парадоксы силы Кориолиса}

\section{Что такое сила Кориолиса}

\paragraph{Вопрос.}
Представим себе, что вы играете в мяч на платформе крытой карусели, так что вам ничего не видно снаружи.
Вы кидаете мяч из центра точно в цель на краю (как показано на рисунке \ref{pic:7.1}), но он откланяется вправо и вы промахиваетесь.
В чём же дело?
Забудем пока о силе тяжести и будем считать, что вращение происходит против часовой стрелки.

\begin{figure}[ht!]
\centering
\begin{lpic}[t(7mm),b(2mm),r(30mm),l(0mm)]{pics/7.1}
\lbl[b]{50,28;$A$}
\lbl[t]{52,13;$A'$}
\lbl[l]{61,30;$B$}
\lbl[tl]{58,7;$B'$}
\lbl[r]{33,12;\parbox{22mm}{\footnotesize\raggedleft  след мяча\\на платформе}}
\lbl[b]{35,38;\parbox{42mm}{\footnotesize\centering  прямолинейное движение\\мяча относительно земли}}
\lbl[l]{61,39;\parbox{41mm}{\footnotesize
когда мяч долетит до точки
$A$, он окажется над точкой $A'$, а позже, точка $B'$ окажется под мячом в точке $B$.}}
\end{lpic}
\caption{Объяснение силы Кориолиса.}
\label{pic:7.1}
\end{figure}

\paragraph{Ответ.}
Проще всего объяснить дело так:
«На самом деле мяч летит прямо, но из-за вращения платформы, смещается сама цель.
Поэтому мяч и пролетает правее.
Просто в закрытом вращающемся помещении создаётся впечатление, будто это мяч отклонился вправо».

Чтобы получше в этом разобраться, вообразите, что мяч отмечает свой путь на платформе, выпуская вниз струйку чернил.%
\footnote{Напомним, что мы пренебрегаем силой тяжести. Соответсвенно мяч летит по горизонтальной прямой.}
Хотя мяч летит прямо, из-за вращения платформы, след, который он оставит, окажется изогнутым как показано на рисунке.
Для нас, находящихся на земле, в таком искривлении нет ничего загадочного.
Но наблюдатель на платформе%
\footnote{ Земля — это пример такой платформы, хотя на протяжении большей части человеческой истории люди не осознавали её вращения.}%
, который воспринимает её неподвижной, испытывает иллюзию действия невидимой силы.
Эта мнимая сила и называется \emph{силой Кориолиса}.

Мы с вами живём во вращающемся мире, где сила Кориолиса проявляется повсюду.
Она вызывает вращение циклонов и антициклонов, а также влияет на океанические течения.
Сила Кориолиса обсуждается во многих книгах, например у \label{Арнольд-Лифшиц} Арнольда, Гольдштейна и Ландау --- Лифшица.%
\footnote{Марк, по-моему здесь стоит сказать что она обсуждается в приложении.}

\paragraph{Вопрос.}\label{Гудзон}
Река Гудзон течёт на юг.
В каком направлении сила Кориолиса действует на текущую в ней воду?

\paragraph{Ответ.}
Сила отклоняет воду на запад.
Действительно, представте порцию воды, движущуюся на юг вдоль меридиана.
Из-за вращения всё на Земле движется на восток, причём чем дальше от северного полюса, тем быстрее.
Поэтому, когда порция воды в Гудзоне удаляется от полюса, её скорость в восточном направлении возрастает.
Значит вода будет прижиматься к западному берегу реки,
сопротивляясь этому возрастанию по инерции.
Соответственно, самой воде будет казаться, что какая-то сила толкает её на восток.
Объясняет ли это то, что западный берег Гудзона (напротив Манхэттена, со стороны Нью-Джерси) крутой, а восточный (манхэттенский) --- пологий?
Скорее всего, нет.

\section{Кориолис в самолёте}\label{Кориолис в самолёте}

\paragraph{Вопрос.}
Как велика сила Кориолиса, действующая на человека в реактивном самолёте (обычная скорость около $250$ м/с)?

\paragraph{Ответ.}
Чтобы упростить вычисления, будем считать, что самолёт летит над Северным полюсом.
В этом случае можно думать, что Земля это плоская платформа --- огромная карусель, вращающуюся вокруг своей оси.
Тогда пассажир испытывает силу Кориолиса%
\footnote{Эту формулу можно найти в упомянутых выше книгах. На странице ??? я «выведу» эту формулу без множителя 2 и предложу найти ошибку.}
\begin{equation}
F = 2 m \omega v,
\label{eq:7.1}
\end{equation}
где $m$ --- его масса,
$\omega$ --- угловая скорость вращения Земли,
а $v$ — скорость самолёта.
Округлим человека до $m = 70$ кг (извините за каламбур),
$\omega = \tfrac{2\pi}{24 \cdot 3600}\,\text{рад/с}$ и
$v = 250 \,\text{м/с}$.
Подставив всё это в \eqref{eq:7.1}, получим
$F/g \approx 240 \,\text{г}$ --- \emph{этой силы хватило бы чтобы поддерживать чашку с водой!}

Другой способ почувствовать величину этой силы — посмотреть на угол~$\theta$,
на который она отклонила бы подвешенный маятник.
Этот угол (в радианах) близок к отношению силы Кориолиса к весу $mg$:
\[
\theta \;\approx\; \tan\theta \;=\; \frac{2 \omega v}{g}.
\]

Получается примерно $1/600$??? радиана, или около
$0{,}1$°???.
Именно на такой угол самолёт теоретически должен накрениться, чтобы избежать бокового сноса.
Что это означает в терминах разницы высот между концами крыльев?
Примерно $1/600$??? от размаха крыла.
Размах крыла боинга 747 около $60$~м, а значит разница высот порядка $10$??? см.
Немного, но заметить можно.

\section{Кориолис в канализации}

\paragraph{Вопрос.}
Многие считают, что в северном полушарии вода стекает в слив по часовой стрелке из-за силы Кориолиса, но правда ли это?

\paragraph{Ответ.}
Нет, не правда.
Сила Кориолиса действует и в унитазе и в ванне, но она ничтожно мала.
Эту силу непросто заметить даже в самолёте (см. страницу предыдущий раздел), и ещё труднее в воде, которая движется в тысячи раз медленнее.
Вода закручивается при сливе по другим причинам.
В некоторых унитазах, например, вода подаётся под углом, так что она уже закручена.
В ванне вращение может возникнуть из-за того, что воду взболтнули и она приобрела небольшую завихренность%
\footnote{Завихренность определена на странице \pageref{def:завихренность}.}%
, которая становится заметной лишь при подходе к сливному отверстию.
Ещё одна причина, по которой вода может начать вращаться при сливе даже из состояния покоя, — это сочетание асимметрии ванны и вязкости воды.
\begin{figure}[ht!]
\centering
\begin{lpic}[t(2mm),b(2mm),r(0mm),l(0mm)]{pics/7.2}
\lbl[t]{18,17;{\footnotesize глубоко}}
\lbl[b]{18,1;{\footnotesize мелко}}
\lbl[lt]{5,9;{\footnotesize слив}}
\end{lpic}
\caption{Закручивание воды при сливе может быть вызвано асимметрией ванны в сочетании с вязкостью воды.}
\label{pic:7.2}
\end{figure}
На рисунке \ref{pic:7.2} приведён пример ванны, которая будет сливаться против часовой стрелки и в Бостоне, и в Буэнос-Айресе.

\section{Высокое давление и хорошая погода}

\paragraph{Вопрос.}
Массы воздуха повышенного давления, называются антициклонами.%
\footnote{Приствка \emph{анти-} указывает на вращение противположное вращению Земли.}
В северном полушарии они вращаются по часовой стрелке.
Каким же образом повышенное давление связывается с вращением по часовой стрелке?

\paragraph{Ответ.}
Это объясняется силой Кориолиса.
Представьте, что воздух сначала расходится от центра (рисунок \ref{pic:7.3}a).
Тода сила Кориолиса будет отклонять каждую частицу воздуха вправо от направления потока.%
\footnote{Как и порцию воды в Гудзоне на странице \pageref{Гудзон}.}
Это приведёт к тому, что частицы будут отклоняться от радиального пути, закручиваясь по часовой стрелке.
Можно представить себе установившееся вращение воздушных масс, при котором сила Кориолиса движущихся по кругу частиц как бы сдерживает высокое давление в центре антициклона --- они как овчарки бегают вокруг стада овец, собирая его вместе (рисунок \ref{pic:7.3}).

\begin{figure}[ht!]
\centering
\begin{lpic}[t(7mm),b(8mm),r(0mm),l(0mm)]{pics/7.3}
\lbl[tl]{0,34;(a)}
\lbl[tl]{42,34;(b)}
\lbl[tl]{22,10.5;{\footnotesize $f$}}
\lbl[b]{24.5,14.5;{\footnotesize $v$}}
\lbl{13,13;\parbox{22mm}{\footnotesize\centering высокое\\давление}}
\lbl{62,13;\parbox{22mm}{\footnotesize\centering высокое\\давление}}
\lbl[b]{21,28;\parbox{22mm}{\footnotesize\centering движение\\наружу}}
\lbl[l]{28,20;\parbox{22mm}{\footnotesize сила\\Кориолиса}}
\lbl[l]{75,22;\parbox{22mm}{\footnotesize сила\\Кориолиса}}
\lbl[b]{55,27,25;\parbox{22mm}{\footnotesize\centering движение\\воздуха}}
\lbl[t]{13,-1;\parbox{42mm}{\footnotesize\centering высокое давление\\
запускает вращение\\
силой Кориолиса}}
\lbl[t]{62,-1;\parbox{42mm}{\footnotesize\centering
сила Кориолиса\\
удерживает\\
высокое давление}}
\end{lpic}
\caption{Как сила Кориолиса связывает высокое давление с антициклонами.}
\label{pic:7.3}
\end{figure}

\paragraph{Как антициклоны связаны с хорошей погодой?}
Благодаря «подушке» повышенного давления в центре воздух движется вниз, нагревается от сжатия%
\footnote{Награвание от сжатия объясняется на странице ???.}%
, и облака «растворяются».
В циклонах происходит противоположное: воздух поднимается, охлаждаясь при расширении, и влага конденсируется, образуя облака.

\section{Что вызывает пассаты?}

Пассаты образуют тропический пояс ветров, который постоянно дует с востока на запад.
Откуда они берутся?

\begin{figure}[ht!]
\centering
\begin{lpic}[t(2mm),b(2mm),r(0mm),l(0mm)]{pics/7.4}
\lbl[t]{19.5,42;\parbox{22mm}{\footnotesize\centering
холодней}}
\lbl[b]{19.5,18;\parbox{22mm}{\footnotesize\centering
теплей}}
\lbl[bl]{60,50;\parbox{22mm}{\footnotesize\centering
вращение}}
\lbl{19.5,29;\parbox{22mm}{\footnotesize\centering
циркуляция\\
без вращения\\
Земли}}
\lbl{63,29;\parbox{22mm}{\footnotesize\centering
Отклонение\\
из-за силы\\
Кориолиса}}
\end{lpic}
\caption{Пассаты возникают в результате действия силы Кориолиса на потоки воздуха.}
\label{pic:7.4}
\end{figure}

Причина в сочетании атмосферной циркуляции и силы Кориолиса.
Ниже приведено сильно упрощённая схема, которая всё же даёт представление о том, что происходит.
\begin{enumerate}
\item Более холодный воздух течёт, с высоких широт в сторону экватора.
Это происходит в нижних слоях атмосферы и напоминает то как из открытой двери холодный воздух растекается по полу комнаты.
\item Теперь вступает в игру сила Кориолиса.
Она отклоняет этот поток к западу, как показано на рисунке \ref{pic:7.4}.
\item Ближе к экватору, воздух нагревается, поднимается вверх и движется обратно к полюсям, но уже в верхних слоях атмосферы.
\end{enumerate}

Атмосфера подобна двигателю на солнечной батарее.
Она питается энергией излучения Солнца, и возвращает её обратно в космос через излучение.
Небольшая часть солнечного тепла заставляет атмосферу двигаться, преодолевая трение.
Трение превращает эту энергию в тепло, которое и излучается наружу.
То есть крохотная часть солнечной энергии встряхивает атмосферу Земли по пути от Солнца в космос.
Земля и всё, что на ней, включая нас с вами, подобна организму, который поглощает энергию, и выделяет её в том же количестве, но другой форме, а точнее, в другой части спектра излучения.
\chapter{Центробежные парадоксы}

\section{Куда дешевле лететь, на запад или на восток?}

\paragraph{Задача.}
Известно, что на полёт из Бостона в Лондон расходуется меньше топлива, чем на обратный рейс.
Это происходит потому, что ветер дует примерно в восточном направлении.
Но представим себе, что ветер волшебным образом исчез.
Исчезло бы тогда и различие в расходе топлива?
Чтобы не отвлекаться на мелочи, давайте заменим Бостон с Лондоном двумя точками $A$ и $B$ на экваторе и спросим: в отсутствие ветров будет ли перелёт на восток из $A$ в $B$ требовать столько же топлива, сколько перелёт на запад из $B$ в $A$?

\paragraph{Решение.}
Из-за вращения Земли полёт на восток потребует меньше топлива.
Действительно, каждая точка экватора совершает орбитальное движение вокруг центра Земли.
Если самолёт движется на восток, то это \emph{увеличивает} его орбитальную скорость.
Значит увесличивается центробежная сила, и самолёт становится чуть легче.
Ну а если самолёт легче то и расход топлива меньше.

Ну а насколько легче?
При скорости полёта 250 м/с разница в весе составит около $
\tfrac23 \%$ (две трети процента).%
\footnote{Чуть ниже мы увидим, что отношение разности весов к истинному весу вычисляется по формуле
$4v\omega/g$,
где $v$ --- скорость самолёта,
$\omega$ --- угловая скорость вращения Земли,
а $g$ --- ускорение свободного падения.}
Загруженный Boeing 747 может весить 300 тонн.
Значит разница составит тонны две, то есть человек 30 без багажа!

Можно думать, что самолёт это спутник, только очень медленный:
б\'{о}льшую часть веса держат его крылья, но немного веса приходится на центробежную силу.

На самом деле, влияние ветра конечно же сильней центробежной силы.

\paragraph{Задача.}
А как подсчитать, на сколько самолёт станет легче по отношению к его весу за счёт центробежной силы?

\paragraph{Решение.} Разница между весом при полёте на восток и на запад равна разнице центробежных сил%
\footnote{См. страницу \pageref{sec:A.9}.}%
:
\[\Delta W
=
\frac{m v_{\text{восток}}^{2}}{R} - \frac{m v_{\text{запад}}^{2}}{R}
=
\frac mR\left((\omega R + v)^{2} - (\omega R - v)^{2}\right),
\]
где $\omega$ --- угловая скорость вращения Земли, $R$ --- радиус Земли, а $v$ --- скорость самолёта.
Раскрыв скобки и сократив, получим
\[
\Delta W=4 m \omega v.
\]
Значит, отношение к весу равно
\[
\frac{\Delta W}{W}=\frac{4 m \omega v}{mg}=\frac{4 \omega v}{g}.
\]

\section{Парадокс с Кориолисом}\label{Парадокс с Кориолисом}

Человек, идущий с (постоянной) скоростью $v$ по платформе,
вращающейся с (постоянной) угловой скоростью $\omega$, испытывает
ускорение Кориолиса:
\begin{equation}
a=2 \, \omega v.
\label{eq:8.1}
\end{equation}
Эта формула приводится в нескольких учебниках механики, упомянутых
на странице \pageref{Арнольд-Лифшиц}.
Чуть ниже я «выведу» похожую формулу, но без множителя $2$:
\begin{equation}
a=\omega v.
\label{eq:8.2}
\end{equation}

\paragraph{Вопрос} Попробуйте выяснить, куда потерялась половина силы Кориолиса в следующем рассуждении?

\begin{figure}[ht!]
\centering
\begin{lpic}[t(2mm),b(2mm),r(0mm),l(0mm)]{pics/8.1}
\lbl[tr]{10,19;$v\,\Delta t$}
\lbl[br]{3,23;$\omega v\,\Delta t$}
\end{lpic}
\caption{Что не так с этим «доказательством»?}
\label{pic:8.1}
\end{figure}

\paragraph{«Вывод» формулы \eqref{eq:8.2}.}
Предположим, что я иду по радиусу вращающейся платформы
(см. рисунок~\ref{pic:8.1}) со скоростью $v$.
За время $\Delta t$ я отойду на расстояние $r=v\,\Delta t$ от центра.
Из-за вращения платформы, моя скорость, перпендикулярная радиусу, будет равна
\[\omega r=\omega v\,\Delta t.\]
Таким образом, за время $\Delta t$, моя скорость, в направлении перпендикулярном радиусу, изменилась на величину
$\Delta v=\omega v\,\Delta t$;
следовательно, ускорение равно
\[\Delta v/\Delta t
= \omega v\,\Delta t/\Delta t
= \omega v.
\]
Итак, я получил равенство \eqref{eq:8.2} --- где же ошибка?

\begin{figure}[ht!]
\centering
\begin{lpic}[t(2mm),b(2mm),r(0mm),l(0mm)]{pics/8.2}
\lbl[l]{12,37;\parbox{32mm}{\centering
забытый член:\\
$v\sin(\omega\,\Delta t)\approx v\omega\,\Delta t$}}
\lbl[r]{-1,25;$v\omega\,\Delta t$}
\lbl[r]{-1,17;$v\omega\,\Delta t\cos(\omega\,\Delta t)\approx v\omega\,\Delta t$}
\end{lpic}
\caption{Объяснение куда делся множитель 2.}
\label{pic:8.2}
\end{figure}

\paragraph{Решение.}
Рисунок~\ref{pic:8.1} не вполне верен; его надо заменить на рисунок~\ref{pic:8.2}.
Я упустил из виду то, что радиус по которому я шёл повернулся на угол $\omega\Delta t$,
а вместе с ним повернулся и вектор моей скорости.
Это добавляет слогаемое $v \sin(\omega \,\Delta t) \approx \omega v \,\Delta t$ к изменению скорости в перпендикулярном направлении к \emph{изначальному} радиусу --- та самая потерянная половинка!

Итак, множитель $2$ в формуле \eqref{eq:8.1} складывается из двух частей:
(1) из различия скоростей между точками платформы, и
(2) из изменения направления движения вследствие поворота платформы.


\section{Что держит стоячий маятник?}

\paragraph{Вопрос.}
Маятник --- это грузик на стержне.
У него есть два положения равновесия: висячее устойчиво, а стоячее --- неустойчиво.
Если установить маятник стоя на опоре, то малейшее движение заставит его упасть.
Разумеется, можно удерживать маятник в равновесии, как метлу на ладони.
Это требует осмысленной реакции на движение маятника.
А что произойдёт, если мы просто будем трясти точку его подвеса (скажем, в вверх-вниз)?

\paragraph{Ответ.}
Если точку подвеса достаточно быстро трясти в вертикальном направлении, то стоячее положение окажется устойчивым.
Это поразительное явление известно больше сотни лет.%
\footnote{A. Stephenson, ``On a new type of dynamical stability,'' \emph{Manchester Memoirs} 52 (1908), p. 110.}
При этом равновесие удерживается совсем не так, как при использовании обратной связи:
трясущемуся подвесу наплевать на то, что делает маятник, и он никак не реагирует на его движение.
Удивительно, что тряска может оказывать столь умное действие:
в конце концов, почему бы быстрым движениям вверх-вниз просто не компенсировать друг друга?
И почему тряска помогает устойчивости, а не неустойчивости?

\begin{figure}[ht!]
\centering
\begin{lpic}[t(7mm),b(10mm),r(0mm),l(0mm)]{pics/8.3}
\lbl[tl]{0,95;(a)}
\lbl[tl]{0,52;(b)}
\lbl[t]{25,96;\parbox{32mm}{\footnotesize\centering
стоячий маятник\\
нестабилен}}
\lbl[rb]{5,88;{\footnotesize грузик}}
\lbl[t]{9,75,-70;{\footnotesize стержень}}
\lbl[tr]{36,79;$\theta$}
\lbl[tl]{37,70;$mg$}
\lbl[tl]{43,86;$mg\sin\theta$}
\lbl[t]{13,63;\parbox{32mm}{\footnotesize\centering
упор\\
(стационарный)}}
\lbl[b]{21,70,-90;{\footnotesize притяжение}}
\lbl[lb]{15,50;{\footnotesize стабилен!}}
\lbl[r]{7,24;{\footnotesize шарнир}}
\lbl[b]{18,24,-90;{\footnotesize быстро елозит}}
\lbl{12,4;\parbox{22mm}{\footnotesize\centering электро-\\лобзик}}
\lbl[b]{42,30,37;{\footnotesize направление тряски}}
\lbl[l]{68,24;\parbox{22mm}{\footnotesize
сила\\
тяжести}}
\lbl[t]{11,-1;\parbox{42mm}{\footnotesize\centering
стоячий маятник стабилен если опора трясётся достаточно часто}}
\end{lpic}
\caption{(a) Стоячий маятник нестабилен, но (b) он становится стабильным, если опору сильно трясти.}
\label{pic:8.3}
\end{figure}

\paragraph{Эксперимент.}
На рисунке \ref{pic:8.3} показан алюминиевый стержень, который нежёстко приделан к лезвию электролобзика.%
\footnote{Это демонстрируется в моём ролике на YouTube: http://www.youtube.com/user/MarkLevi51\#p/a/u/2/cHTibqThCTU.}
Когда я включаю лобзик, лезвие начинает быстро ходить вверх-вниз --- примерно 30 раз в секунду.
При этом возникает ощущение, что невидимая пружина пытается выровнить стержень параллельно движению лезвия.
Эта воображаемая пружина достаточно сильна, она способна держать маятник в почти горизонтальном направлении, когда я поворачиваю лобзик, как показано на рисунке \ref{pic:8.3}b, справа.

\paragraph{Вопрос.}
Как же тряске удаётся стабилизировать маятник?
(Только без формул!)

\paragraph{Ответ.}
Вместо того чтобы рассматривать стержень, как на рисунке
\ref{pic:8.3} или в моём видеоролике, будем думать о грузике закреплённом на конце невесомого стержня.
Ускорение точки подвеса настолько велико, что гравитацией (малой по сравнению с ним) можно пока пренебречь.
Со стороны стержня, грузик испытывает сильное притягивание и отталкивание поочерёдно.
Поскольку это толкательнотянущие усилия направлены \emph{точно вдоль стержня}, грузик старается двигаться в направлении стержня.
Таким образом, грузик стал бы двигаться по кривой траектории, как на рисунке \ref{pic:8.4}a.
Такая траектория называется \emph{кривой погони} или \emph{трактрисой}.%
\footnote{Трактриса определяется как кривая, для которой данная кривая отсекает от касательной отрезки равной длины.
Например, если катить переднее колесо велосипеда по прямой, то его заднее колесо будет двигаться по трактрисе.}
Давайте временно считать, что движение грузика ограничено этой трактрисой.
Это довольно безобидное ограничение, ведь оно не мешает сильным толчкам и рывкам со стороны стержня.
При таком ограничении грузик будет совершать быстрые колебания взад-вперёд по короткой дуге $AB$.
Так как дуга изогнута, грузик будет испытывать центробежную силу (см. рисунок \ref{pic:8.4}b), и грузик захочет двигаться в направлении этой центробежной силы!
Если теперь снять ограничение, то грузик подчинится своему желанию.
Если тряска достаточно сильна, эта сила превзойдёт дестабилизирующее действие силы тяжести%
\footnote{Более подробное обсуждение даётся в \cite[стр. 158]{levi1999}.
Удивительно, но для того, чтобы превратить интуитивную идею в критерий устойчивости, требуется всего пара строк: стоячее положение устойчиво, если  $
\langle v^2 \rangle \ge g \ell$,
где $v$ --- скорость точки подвеса, $\langle \cdot \rangle$ обозначает среднее за период работы электролобзика, а $\ell$ --- длина маятника.
Наше физическое объяснение заменяет гораздо более длинное формальное вычисление, с дополнительным преимуществом, что оно объясняет что происходит на самом деле.}%
, и маятник встанет вертикально, что и завершает объяснение.

\begin{figure}[ht!]
\centering
\begin{lpic}[t(7mm),b(2mm),r(0mm),l(0mm)]{pics/8.4}
\lbl[tl]{0,37;(a)}
\lbl[tl]{35,37;(b)}
\lbl[t]{8,12,75;{\footnotesize трактриса}}
\lbl[tl]{11,23;$A$}
\lbl[tl]{18,30;$B$}
\lbl[br]{40,14;$v$}
\lbl[bl]{41.8,9;$\theta$}
\lbl[bl]{61,31;$u=v\sin\theta$}
\lbl[tl]{60,17;$g\sin\theta$}
\lbl[b]{44,31;$ku^2$}
\end{lpic}
\caption{Неочевидная центробежная сила отвечает за устойчивость стоячего маятника.}
\label{pic:8.4}
\end{figure}

\paragraph{Ловушка Паула.}
Описанное явление известно не меньше века;
самое раннее упоминание, которое мне удалось найти, содержится в указанной выше статье Стивенсона 1908 года.
Тот же самый эффект стабилизации с помощью тряски, но в иной форме, используется в так называемой ловушке Паула --- устройстве для удерживания заряженных частиц в вакууме с помощью вибрирующих электрических полей.%
\footnote{Если бы я оказался в системе отсчёта стержня в эксперименте с лобзиком, то мне бы казалось, что вибрирует сила тяжести.}
За это изобретение Вольфганг Пауль был удостоен Нобелевской премии.
Объяснение Паула основано на дифференциальных уравнениях.
Но я не советую расказывать его случайному прохожему (даже не пытайтесь, особенно если вы не своём районе).

\section{Антигравитационная патока}

\paragraph{Вопрос.}
Банка с крышкой наполовину наполнена патокой или другой густой тяжёлой жидкостью вроде мёда.
Если перевернуть банку вверх дном, патока, разумеется, начнёт перетекать вниз.
А можно ли двигать банку так, чтобы патока не вытекала даже тогда, когда банка перевёрнута (рисунок \ref{pic:8.5})?
Иными словами, можно ли удержать патоку в перевёрнутой банке?

\begin{figure}[ht!]
\centering
\begin{lpic}[t(2mm),b(2mm),r(0mm),l(0mm)]{pics/8.5}
\lbl{10,11;\textcolor{white}{\footnotesize патока}}
\end{lpic}
\caption{Тряска удерживает патоку от перетекания вниз.}
\label{pic:8.5}
\end{figure}

\paragraph{Ответ.}
Если банку быстро трясти в направлении её оси, то патока не будет вытекать даже после переворачивания.%
\footnote{Этот эксперимент обсуждается в статье M. M. Michaelis,
T. Woodward, American Journal of Physics 59(9) (1991), pp. 816--821.
Теоретическое обоснование приводится в G. H. Wolf, Physical Review Letters 24 (1970), pp. 444--446.}
Я воспроизвёл этот опыт, воспользовавшись подручными средствами собранными в мастерской и на кухне, сделав так, чтобы при включённом электролобзике банка тряслась вдоль оси.
Если всю эту конструкцию перевернуть вверх дном, то патока не вытечет, а чудесным образом остаётся наверху, как будто сила тяжести изменила своё направление.
Удивительным образом тряска удерживает поверхность патоки ровной и не даёт ей течь.
Ещё больше впечатляет следующее: если повернуть банку вбок (при работающем лобзике), то поверхность патоки будет оставаться вертикальной, будто это стена воды в расступившемся Красном море.

\section{Почему праща не может работать}\label{Почему праща не может работать}

\paragraph{Парадокс.}
Представим себе, что я раскручиваю камень на верёвке,
верёвка натягивается и действует на камень силой $T$.
Эта сила направлена прямо к опоре --- точке, скажем $P$, где мои пальцы держат верёвку.
Следовательно, момент силы%
\footnote{Обсуждается на страницее \pageref{sec:A.5}.}
$T$ относительно точки $P$ равен нулю.
Но нулевой момент силы означает, что момент импульса камня не меняется.
То есть, $Lv=\mathrm{const}$, где
$L$ --- длина верёвки, а $v$ --- скорость камня, перпендикулярная к верёвке.
Значит и $v$ не меняется.
Однако со времён Голиафа известно, что это не так --- где же ошибка?

\begin{figure}[ht!]
\centering
\begin{lpic}[t(2mm),b(2mm),r(0mm),l(0mm)]{pics/8.6}
\lbl[br]{12,42;$B$}
\lbl[l]{19,11;\parbox{32mm}{\footnotesize траектория\\ руки}}
\lbl[l]{28,36;\parbox{28mm}{\footnotesize камень ускоряется к убегающему равновесию в точке $B$}}
\lbl[r]{-1,0;\parbox{32mm}{\footnotesize\raggedleft  ускорение\\ руки}}
\lbl[b]{8,27,70;\parbox{32mm}{\footnotesize\centering ощущаемая\\ гравитация}}
\end{lpic}
\caption{Праща --- это маятник бегущий к постоянно ускользающему от него положению равновесия.}
\label{pic:8.6}
\end{figure}

\paragraph{Ответ.} То, что
\[
\text{момент силы}=0 \quad\Rightarrow\quad \text{момент импульса}=\text{const}
\]
справедливо в инерциальных системах отсчёта.
А система отсчёта, связанная с моими пальцами,
вовсе не инерциальна ---  мне приходится ускорять свои пальцы при вращении.

\paragraph{Как же работает праща?}
Ответ можно увидеть на рисунке~\ref{pic:8.6}.
По сути, мы создаём маятник, который всё время скользит вниз по наклонной, \emph{в погоне за ускользающим от него положением равновесия~$B$}.
Моя рука~--- точка подвеса маятника~$P$~--- движется (скажем) по окружности,
всё быстрее и быстрее.
Наблюдатель, связанный с точкой~$P$, будет ощущать перегрузку, показанную на рисунке.
В частности, он увидит, что камень ускоряется как бы вниз к положению равновесия~$B$,
так же как обычный маятник ускоряется вниз к своей нижней точке.
При этом точка $B$ постоянно ускользает от камня против часовой стрелки, и камню приходится гнаться за~$B$, всё время ускоряясь.

Следующая задача раскрывает другой неожиданный аспект, связанный с физикой пращи.

\section{Задача Давида и Голиафа}\label{Задача Давида и Голиафа}

Напомним из предыдущей задачи, что праща --- это камень на верёвке, который раскручивают, а затем отпускают.%
\footnote{Здесь и далее мы пренебрегаем гравитацией.}
Следующий парадокс возник при попытке разобраться в физике пращи.
В ответ может быть трудно поверить.

\paragraph{Задача о праще.}
Я веду конец верёвки по окружности так, что камень на другом её конце движется
по большей концентрической окружности, и при этом угол опережения верёвки по отношению к скорости камня остаётся постоянным, скажем $\theta=45$°.
Камень будет вращаться всё быстрее благодаря касательной составляющей
$T_{\text{кас}}$ силы натяжения верёвки.
(Мне придётся вращать пальцы также быстрее, чтобы обеспечить постоянный угол 45°).
Предположим, что камень движется по окружности радиуса $1 \,\text{м}$.
Попробуйте прикинуть, сколько понадобится времени,
чтобы разогнать его с начальной скорости $1 \,\text{м/с}$
до скорости звука $(330 \,\text{м/с})$?
До скорости света $(300{,}000{,}000 \,\text{м/с})$?%
\footnote{Притворимся, что ньютоновская механика применима при любых скоростях, так что объекты могут двигаться быстрее скорости света.
Также не будем учитывать гравитацию и сопротивление воздуха, а также ограничения на прочность верёвки и человеческие возможности.}

\begin{figure}[ht!]
\centering
\begin{lpic}[t(2mm),b(2mm),r(0mm),l(0mm)]{pics/8.7}
\lbl[r]{10,8;{\footnotesize 45°}}
\lbl[r]{56.7,6.6;{\footnotesize 45°}}
\lbl[rb]{14,10;{\footnotesize 1\,м}}
\lbl[b]{23,13,45;{\footnotesize тяга}}
\lbl[tl]{15,1;{\footnotesize ракета}}
\lbl[t]{62,1;{\footnotesize камень}}
\lbl[l]{30,12;\parbox{32mm}{\footnotesize путь\\ ракеты}}
\lbl[br]{67,10;$T$}
\end{lpic}
\caption{Ускорение прямо пропорционально скорости в квадрате.}
\label{pic:8.7}
\end{figure}

\paragraph{Задача о ракете.}
Игрушечная ракета летит по окружности радиусом $1 \,\text{м}$,
её двигатели постоянно направлены с опережением под фиксированным углом
$\alpha=45$° к центру (см. рисунок~\ref{pic:8.7}).%
\footnote{Это требует постоянно увеличивающейся тяги.}
Сколько времени понадобится ракете,
чтобы увеличить свою скорость от $1 \,\text{м/с}$ до скорости звука?
До скорости света?

\paragraph{Решение.}
Камень (как и ракета) \emph{превысит скорость света --- не говоря уже о скорости звука --- менее чем за секунду!}
Скорость стремится к бесконечности, когда время приближается к отметке в одну секунду.
Это попросту означает, что в принципе невозможно продолжительно раскручивать камень по окружности, поддерживая угол опережения 45° (или любой другой положительный угол).
Вот тому объяснение.

\paragraph{Объяснение.}
Я покажу, что касательное ускорение камня $a_{\text{кас}}$ прямо пропорционально квадрату скорости $v$ ---
а именно,
\[
a_{\text{кас}}=v^{2}.
\]
В частности скорость изменения $v$ прямо пропорциональна $v^2$.
И, как покажут выкладки следующего абзаца, такая величина подойдёт к бесконечности за конечное время.

\paragraph{Подробности.}
Так как угол между силой натяжения $T$ и касательной равен 45°,
из рисунка~\ref{pic:8.7} следует, что касательная и радиальная компоненты силы $T$ равны.
То же самое справедливо и для касательного и центростремительного ускорений:
$a_{\text{кас}}=a_{\text{цен}}$.
Но центростремительное ускорение (см.~страницу~\pageref{sec:A.9}) задаётся формулой
\[
a_{\text{цен}}=\frac{v^2}{r}=v^2 \quad (\text{ведь}\  r=1\,\text{м}).
\]
Значит,
\begin{equation}
a_{\text{кас}}=v^2. \label{eq:8.3}
\end{equation}
Теперь, начинается матанализ.
Сейчас мы покажем, что в силу этого соотношения
скорость $v$ достигнет бесконечности за конечное время.
Уравнение~\eqref{eq:8.3} можно переписать как
\begin{equation}
\frac{1}{v^2}\,\frac{dv}{dt}=1. \label{eq:8.4}
\end{equation}
Взяв первообразную, получим
\[
-\frac{1}{v}=t + c,
\]
где $c=-1$, ведь $v=1$ при $t=0$.
Значит
\[
\frac{1}{v}=1 - t.
\]
При $t=0.9999 \,\text{с}$ получаем $v=10{,}000 \,\text{м/с}$ ---
достаточно, чтобы запустить камень на орбиту Земли и почти достаточно,
чтобы преодолеть её гравитацию.
При $t=0.999999$ скорость превысит скорость света.
За какое-то время до 1 секунды кинетическая энергия камня
превысит суммарную энергию, запасённую в Солнце и во всех остальных
звёздах Вселенной.
Вот такими могут оказаться вполне реалистичные предположения.

\paragraph{Вопрос.}
При $t > 1$ мы получаем
\[
v=\frac{1}{1 - t} < 0,
\]
то есть камень будет двигаться назад.
Как объяснить такую нелепость?

\paragraph{Ответ.}
При $t > 1$ формула $v=1/(1 - t)$ попросту не применима.\footnote{Напомним, что $v=1/(1 - t)$ решает уравнение $\tfrac{dv}{dt}=v^2$ при $t<1$ и $t>1$, но в момент $t=1$ правая часть не определена и формула не даёт решения в окрестности~$1$.}

\paragraph{Чудесный банковский счёт.}
Мы увидели, что если скорость изменения $\tfrac{dv}{dt}$
некоторой величины $v$ прямо пропорциональна её квадрату $v^{2}$,
то $v$ подходит к бесконечности за конечное время.
Вообразим на минуту, что банк решает начислять проценты по такому принципу,
позволяя балансу $v$ изменяться по этому закону%
\footnote{То есть $\tfrac{dv}{dt}=kv^2$ вместо зависимости $\tfrac{dv}{dt}=kv$, исползуемой при обычном непрерывном начислении процентов.}%
, то есть начисляемые проценты прямо пропорциональны квадрату текущего баланса.
Для клиента это было бы прекрасно (а для банка --- ужасно).
В частности, баланс достиг бы бесконечности за конечное время.
Однако если клиент проворонит определённый момент (например, $t=1$ в нашем предыдущем примере), то баланс станет отрицательным.%
\footnote{Если считать, что наше решение
$v=\frac{1}{1 - t}$ продолжается за точку разрыва. Вопрос применимости этой формулы с момента $t=1$ спорный и должен решаться в суде.}
Внезапно огромное состояние превратится в огромный долг (в математике, прямо как в жизни).
Если что-то убегает на $+\infty$ за конечное время, то часто возвращается из $-\infty$.

При этом странном начислении процентов клиенты станут выигрывать от объединения своих счетов.
Например, если два равных счёта объединяются в один,
то проценты увеличиваются в четыре раза, ведь
$(2v)^2=4v^2$,
то есть доход \emph{каждого} человека удвоится.
Это приблизит момент, когда клиенты станут бесконечно богатыми.
Обычное,
применяемое в банках, экспоненциальное начисление процентов
$\frac{dv}{dt}=kv$, является единственно разумным ---
в частности клиенты не выигрывают и не проигрывают
от объединения своих счетов.%
\footnote{Иными словами, дифференциальное уравнение, описывающее баланс, является линейным.
Для такого уравнения сумма двух решений также является решением.
Это означает, что объединённый счёт будет иметь тот же баланс, что и сумма двух счетов, если их вести раздельно.}
И ещё одно замечание: при экспоненциальном начислении ваша прибыль будет той же, независимо от того в чём мерить баланс: в рублях, копейках или долларах.
Но это вовсе не так в случае начисления по закону $\tfrac{dv}{dt}=v^2$:
как только вы убедите банк мерить ваше богатство в копейках, скорость обогащения возрастёт в сотню раз!

\section{Вода в трубе}

\paragraph{Вопрос.}
На рисунке \ref{pic:8.8}, вода течёт по изогнутой трубе.
Подойдя к изгибу, она пытается продолжать двигаться прямо
и давит на трубу как показано на рисунке,
в том же направлении, в котором двигалась до поворота.
Верно ли указано направление силы?

\begin{figure}[ht!]
\centering
\begin{lpic}[t(5mm),b(2mm),r(10mm),l(0mm)]{pics/8.8}
\lbl[tl]{-10,41;(a)}
\lbl[tl]{35,41;(b)}
\lbl[tl]{-10,17;(c)}
\lbl[r]{-1,35;{\footnotesize ток}}
\lbl[b]{28,36;{\footnotesize сила}}
\lbl[br]{47,36;$v$}
\lbl[l]{52,29;$v'$}
\lbl[b]{71,37;$\Delta v=v'-v$:}
\lbl[l]{77,29;$v'$}
\lbl[tr]{73,23;$-v$}
\lbl[b]{71,31,45;$\Delta v$}
\lbl[t]{8,4, 45;\parbox{22mm}{\footnotesize\centering  сила на\\воду}}
\lbl[t]{44,4, 45;\parbox{22mm}{\footnotesize\centering  сила на\\трубу}}
\end{lpic}
\caption{В каком направлении вода действует на трубу при повороте?}
\label{pic:8.8}
\end{figure}

\paragraph{Ответ.}
Нет, рисунок неверен.
На самом деле сила направлена вверх и вправо, под углом 45° к обоим прямым участкам трубы.
Нельзя забывать, что вода поворачивает вниз, и, значит, она обязана толкать трубу вверх.
Точнее говоря, посмотрим на изменение импульса частицы воды при прохождении поворота.
Скорость частицы изменилась с $v$ на $v'$ (см. рисунок~\ref{pic:8.8}).
Приращение скорости $\Delta v=v' - v$ идёт по биссектрисе прямого угла, как показано на рисунке.
Согласно второму закону Ньютона, средняя сила, действующая на частицу, направлена вдоль приращения скорости.
А согласно третьему закону Ньютона, вода прикладывает к трубе равную по величине и противоположно направленную силу.

\paragraph{Ещё раз чуть по-другому.}
А вот ещё способ увидеть, что ответ на рисунке~\ref{pic:8.8}a неверен.
Хоть это не вполне строго, но можно считать, что сила, действующая на трубу, складывается из центробежных сил всех частиц на повороте.
Но центробежная сила, действующая на частицу, равна $mv^{2}/r$.
Тут важно, что скорость $v$ возводится в квадрат;
в частности, замена $v$ на $-v$ ничего не меняет.
Но если рассуждать как на рисунке~\ref{pic:8.8}a, то при обратном токе воды сила была бы иной.
Значит, рисунок неверен.

\section{Натяжение колец}\label{Что сильней натягивается?}

Следующая задача имеет неожиданный ответ, простое решение и ещё более удивительное следствие, описанное на странице \pageref{Скользящие тросики в невесомости}.

\paragraph{Задача.}
Два кольца разного радиуса вращаются с той же (линейной) скоростью.
Они сделаны из тросиков разной длины, но идентичных во всём остальном.
Центробежная сила растягивает оба кольца.
Какое кольцо больше натягивается?
Будем считать, что тросик идеально гибок и не растяжим, а внешние силы, включая силу тяжести, отсутствуют.

\begin{figure}[ht!]
\centering
\begin{lpic}[t(2mm),b(2mm),r(0mm),l(0mm)]{pics/8.9}
\lbl[lb]{1,1;{\footnotesize картонка}}
\lbl[tr]{21,6;$A$}
\lbl[br]{21,38.5;$B$}
\lbl[b]{15,7.7;$F$}
\lbl[t]{15,35;$F$}
\lbl[b]{8,38;$v$}
\lbl[t]{37,6;$v$}
\end{lpic}
\caption{Натяжение тросика.}
\label{pic:8.9}
\end{figure}

\paragraph{Ответ.}
Тросики натягиваются одинаково.
Чтобы это увидеть (почти без вычислений), рассмотрим половину кольца (см. рисунок~\ref{pic:8.9}) — закроем левую часть картонкой, просто чтобы её не было видно.
Мы увидим, что тросик вырастает из точки $A$ и исчезает в точке $B$ с одной и той же скоростью $v$.
Между входом и выходом каждая частица изменяет свою скорость на $2v$.
Это изменение вызывается силой натяжения $F$ в точках $A$ и $B$.
Чтобы найти $F$, подождём время $\Delta t$ (скоро оно сократится).
За время $\Delta t$ некоторая масса $\Delta m$ выросла из точки $A$ и та же масса $\Delta m$ исчезла в точке $B$.
То есть масса $\Delta m$ изменила скорость на $2v$ за время $\Delta t$.
По второму закону Ньютона ($F=ma=m \Delta v / \Delta t$) получаем:
\[(2F)\Delta t=\Delta m \cdot (2v),
\quad\text{или}\quad
F=\frac{\Delta m}{\Delta t} v.
\]
Теперь заметим, что $\Delta m=\rho \cdot (v \Delta t)$, где $\rho$ --- линейная плотность тросика (масса на единицу длины).
Подставив это в последнее выражение, получим
\[
F=\rho v^2,
\]
так что натяжение действительно не зависит от радиуса окружности --- только от $v$ и $\rho$.

А вот более короткое доказательство того, что натяжение $F$ не зависит от радиуса.
Рассматрим полуокружность и заметим, что сила $2F$ удерживает её центр масс на круговой орбите:
\[2F=\frac{m u^{2}}{r},\]
где $m$ --- масса полуокружности,
$r$ --- расстояние от центра масс полукольца до центра кольца,
а $u$ --- скорость центра масс.
Теперь отметим, что
(1) отношение $m/r$ не зависит от радиуса кольца $R$, ведь и $m$, и $r$ пропорциональны $R$;
(2) скорость $u$ не зависит от $R$ (а только от $v$).
Следовательно, и $F$ не зависит от $R$.

\section{Скользящие тросики в невесомости}\label{Скользящие тросики в невесомости}

Тросики могут вести себя неожиданно.
Вот например, если тросик замкнуть в кольцо%
\footnote{Будем считать, что наш тросик идеален: он не растяжим, не сопротивляется изгибу и очень тонкий.
Можно думать о цепочке из маленьких шариков, как на офисных авторучках.}
и раскрутить
(рисунок~\ref{pic:8.9}), то в условиях невесомости, будет сохранятся его круглая форма.


\begin{figure}[ht!]
\centering
\begin{lpic}[t(2mm),b(2mm),r(0mm),l(0mm)]{pics/8.10}
\end{lpic}
\caption{Какие из тросиков (если таковые найдутся) не будут меняться двигаясь в невесомости, если придать их частицам равные начальные скорости вдоль стрелок?}
\label{pic:8.10}
\end{figure}

\paragraph{Вопрос.}
Есть ли другие кривые, кроме окружности, вдоль которых тросик может скользить в невесомости, оставаясь неподвижным как целое?
Например, будут ли обладать этим свойством какие-нибудь из кривых на рисунке~\ref{pic:8.10}, если придать тросику равную начальную скорость вдоль кривой?

Иначе говоря, представьте себе абсолютно гибкий шланг,
наполненный водой, в котором вода циркулирует без трения.
Какие из начальных положений шланга на рисунке~\ref{pic:8.10} не будут меняться во времени?

\paragraph{Ответ.}
В это и трудно поверить, но при условиях, описанных в предыдущем абзаце \emph{любая} гладкая кривая будет сохранятся.%
\footnote{E. J. Routh, \emph{Dynamics of a System of Rigid Bodies}, Part 2, 4th ed. (London: MacMillan and Co., 1884), pp. 299--300.}

\paragraph{Объснение.}
Всё станет ясно если вспомнить (раздел \ref{Что сильней натягивается?}), что натяжение \emph{кругового} тросика не зависит от его радиуса.
Заметим, что любую кривую можно приблизить кривой составелнной из дуг окружностей разного радиуса.
Поскольку натяжение в каждой из этих дуг будет одинаковым, тросик с радостью сохранит свою форму.

\paragraph{Более строгое объяснение.}
Сейчас потребуется немного векторного анализа.
Нам надо будет применить второй закон Ньютона к движущемуся тросику.
Для этого запараметризуем тросик его длиной $s$ от отмеченной точки.
Обозначим через $\mathbf{r}(s,t)$ радиус-вектор частицы с параметром $s$ в момент времени $t$ (рисунок~\ref{pic:8.11}).

\begin{figure}[ht!]
\centering
\begin{lpic}[t(2mm),b(2mm),r(0mm),l(0mm)]{pics/8.11}
\lbl[r]{12,10;$0$}
\lbl[b]{32,19,20;$\mathbf{r}(s,t)$}
\lbl[t]{39,12,7;$\mathbf{r}(s+ds,t)$}
\lbl[b]{38,31,-22;$T \mathbf{r}'(s,t)$}
\lbl[b]{68,8,-50;$T \mathbf{r}'(s+ds,t)$}
\end{lpic}
\caption{Второй закон Ньютона для двигающегося тросика.}
\label{pic:8.11}
\end{figure}

В следующем абзаце я покажу, что второй закон Ньютона можно записать как
\begin{equation}
\rho\, \ddot{\mathbf{r}}=(T \mathbf{r}')',
\label{eq:8.5}
\end{equation}
где точка $\dot{}=\partial/\partial t$ обозначает производную по времени,
штрих $'\z=\partial/\partial s$ --- производную по $s$,
a $T=T(s,t)$ --- натяжение тросика.
Поскольку тросик не растяжим,
\begin{equation}
\|\mathbf{r}'\|=1,
\label{eq:8.6}
\end{equation}
где $\|\cdot\|$ обозначает длину вектора.
Уравнения \eqref{eq:8.5}–\eqref{eq:8.6} образуют полную систему для неизвестных функций
$\mathbf{r}$ и $T$.
Теперь уже несложно показать, что любое скользящее движение тросика
сохраняет форму.
Пусть $\mathbf{R}=\mathbf{R}(s)$ --- параметризация кривой длиной её дуги.
Если частица начинает движение в точке $\mathbf{R}(s)$
и скользит вдоль кривой со скоростью $v$, то в момент $t$
она будет находиться в точке $\mathbf{R}(s+vt)$;
то есть $\mathbf{r}(s,t)=\mathbf{R}(s+vt)$ и $T=\rho v^2$.
Подставив это в \eqref{eq:8.5}–\eqref{eq:8.6}, получим равенства (убедитесь в этом сами).
Это и доказывает наше утверждение.

Остаётся вывести \eqref{eq:8.5}.
На дугу $(s, s+ds)$ действуют только две силы натяжения на концах.
Их равнодействующая равна
$(T \mathbf{r}')_{s+ds} - (T \mathbf{r}')_s$,
здесь я не указал зависимость от $t$.
Центр масс дуги находится в точке $
\tfrac{1}{ds} \int_s^{s+ds} \mathbf{r}(\sigma, t) d\sigma$.
Ускорение центра масс --- это вторая производная по времени:
$\mathbf{a}=\tfrac{1}{ds} \int_s^{s+ds} \ddot{\mathbf{r}}(\sigma, t) d\sigma$.
Масса дуги вычисляется как $m \z= \rho ds$, где $\rho$ --- линейная плотность, то есть масса тросика на единицу длины.
Таким образом закон Ньютона $m\mathbf{a}=\mathbf{F}$ принимает вид
\begin{equation}
(\rho ds)
\cdot
\frac{1}{ds}
\int\limits_s^{s+ds}
\rho\,\ddot{\mathbf{r}}(\sigma, t)\, d\sigma
= (T \mathbf{r}_s)_{s+ds} - (T \mathbf{r}_s)_s .
\end{equation}
Разделив обе части на $ds$ и устремив $ds \to 0$, получаем \eqref{eq:8.5}.

\medskip

Вот ещё несколько любопытных тем/задач для читателей знакомых с векторным анализом:

\begin{enumerate}
\item Покажите, что момент импульса скользящего движения плоского тросика равен
\begin{equation}
L=\rho v A ,
\end{equation}
где $v$ --- скорость, а $A$ --- площадь, охватываемая тросиком.
\item Покажите, что $z$-координата момента импульса скользящего движения пространственного тросика равна
\begin{equation}
L_z=\rho v A_{xy},
\end{equation}
где $A_{xy}$ --- (ориентированная) площадь проекции тросика на плоскость $xy$.
Тоже верно и для любой прямой и плоскости, перпендикулярной ей.
\item Определим \emph{циркуляцию} тросика как интеграл
$\mathcal{C}=\int v_{\text{кас}} ds$,
то есть интеграл касательной скорости $v_{\text{кас}}$ по длине кривой $s$.
Покажите, что при \emph{любом} (не только скользящем) свободном движении тросика в невесомости
его циркуляция не меняется.
\end{enumerate}

\chapter{Парадоксы гироскопа}

\section{Как волчок уклоняется от земного притяжения?}

Вращающийся волчок удерживается в вертикальном положении, вовсе не силой, противодействующей притяжению земли.
Всё совсем по-другому, есть странная \emph{уклоняющая} сила --- сила, которая всегда остаётся \emph{перпендикулярной} направлению движения оси волчка.
Эта уклоняющая сила противостоит неустойчивости: волчок начинает падать, но затем отклоняется в сторону, и в результате движется так, как показано на рисунке \ref{pic:9.3}.
В следующем абзаце я постараюсь показать, как эта странная «гироскопическая сила» выводится из второго закона Ньютона.

\paragraph{Антигравитационное велоколесо.}\label{Антигравитационное велоколесо}
Будем использовать велосипедное колесо как волчок.
Подвесим его на двух верёвках, как показано на рисунке \ref{pic:9.1}, и хорошо раскрутим.
Теперь перережем одну из верёвок.
Удивительно, но не поддерживаемый конец оси не упадёт вниз.%
\footnote{Предполагаем, что колесо хорошо раскручено.}
Вместо этого он начнёт медленно поворачивать (прецессировать).
Причём чем быстрее вращается колесо, тем медленнее будет идти прецессия.

\begin{figure}[ht!]
\centering
\begin{lpic}[t(2mm),b(2mm),r(0mm),l(0mm)]{pics/9.1}
\lbl[b]{15,31;$F$}
\lbl[b]{57,20;$F$}
\lbl[lb]{71,8;$1$}
\lbl[l]{67,20;$2$}
\lbl[lt]{71,30;$3$}
\lbl[b]{60,35;$1$}
\lbl[b]{65.3,36;$2$}
\lbl[b]{71,35;$3$}
\lbl[b]{10,8,95;{\footnotesize вращение}}
\lbl[r]{4,33;\parbox{22mm}{\footnotesize\raggedleft  поворачивание\\ оси}}
\lbl[l]{38,21;{\footnotesize разрыв}}
\lbl[t]{66,3;{\footnotesize вид сверху}}
\end{lpic}
\caption{Как инерция частиц гироскопа не даёт ему упасть после разрыва верёвки.}
\label{pic:9.1}
\end{figure}

\paragraph{Вопрос.}
Что же мешает колесу упасть?
Как ему удаётся противостоять силе тжести?


\paragraph{Ответ.}\label{Антигравитационное велоколесо:Ответ}
Можно сказать, что за это отвечает \emph{центробежная сила}, но не та, что первой приходит на ум, а другая --- перпендикулярная ей!

\begin{figure}[ht!]
\centering
\begin{lpic}[t(2mm),b(2mm),r(0mm),l(0mm)]{pics/9.2}
\lbl[t]{3,3;\parbox{22mm}{\footnotesize\centering  момент\\ силы}}
\lbl[]{17,38.6,-20;\parbox{22mm}{\footnotesize\centering  центробежная\\ сила}}
\lbl[b]{67,20;{\footnotesize вид сверху}}
\lbl[tl]{21,16;$L$}
\end{lpic}
\caption{Разбор гироскопического эффекта.}
\label{pic:9.2}
\end{figure}

Давайте посмотрим, что происходит с колесом, когда его ось поворачивает.
На рисунке \ref{pic:9.2} показана траектория одной частицы обода, когда она проходит вблизи верхней точки колеса.
Эта траектория искривлена из-за поворота оси.
Повенуясь инерции, частица пытается двигаться прямо, сопротивлясь отклонению с некоторой центробежной силой $F$ показаной на рисунке.
Подобно этому, сила $-F$ действует на частицу вблизи нижней точки.
Совместный эффект таков, как если бы на колесо действавал момент невидимых сил вокруг линии $L$; именно этот момент и не даёт колесу упасть.

\paragraph{Странная сила.}\label{Антигравитационное велоколесо:Странная сила}
Вращающееся колесо демонстрирует пример очень странной силы, напоминающей силу магнитного на движущийся заряд.
В отличие от трения, эта странная сила направлена \emph{перпендикулярно} движению оси гироскопа.
Чтобы выразиться точнее, превратим наше колесо в волчок, закрепив один конец оси на земле как показано на рисунке \ref{pic:9.3} (чтобы этот конец не мог скользить, но мог поворачиваться во все стороны).
Что случится, если попытаться сдвинуть свободный конец $A$ оси?
Чтобы не путаться в мелочах, пренебрежём силой тяжести.

\begin{figure}[ht!]
\centering
\begin{lpic}[t(2mm),b(2mm),r(0mm),l(0mm)]{pics/9.3}
\lbl[tl]{-4,37;(a)}
\lbl[tl]{47,37;(b)}
\lbl[tr]{9,26;$A$}
\lbl[b]{1,30;$F$}
\lbl[lb]{59,15;$F$}
\lbl[l]{16,34;$v$}
\lbl[l]{65.5,12;$v$}
\lbl[b]{32,1;{\footnotesize опора}}
\lbl[b]{40,11;\parbox{22mm}{\footnotesize\centering  шаровой\\шарнир}}
\end{lpic}
\caption{(a) Для поддержания неизменного направления оси требуется постоянное усилие в перпендикулярном направлении.
(b) Постоянное уклонение волчка не даёт ему упасть.}
\label{pic:9.3}
\end{figure}

\paragraph{Задача.}\label{Антигравитационное велоколесо:Задача}
В каком направлении нужно толкать конец $A$ оси вращающегося волчка, чтобы перемещать $A$ с постоянной скоростью?

\paragraph{Решение.}
Следует приложить силу перпендикулярно желаемому направлению движения (рисунок \ref{pic:9.3})!
Объяснение уже было дано при обсуждении вращающегося и поворачивающего велосипедного колеса.
Игра с настоящим вращающимся колесом даёт странное ощущение: если толкнуть ось, то она уйдёт под прямым углом к толчку.
Осознав это поведение, становится легко переориентировать колесо в любом направлении без особых усилий.

Точно также действует магнитное поле на движущийся заряд, эта сила направлена перпендикулярно скорости заряда.

\paragraph{Отступление.} Часто реакция людей ортогональна приложенному стимулу,
и это очень напоминает гироскопический/магнитный эффект.
Однако с магнитами сравнивают людей совсем другого типа.

\paragraph{Устойчивость за счёт уклонения.}
Волчок удерживается не потому, что сопротивляется гравитации, а более хитрым способом.
Любое движение оси порождает гироскопическую силу%
\footnote{Надо разъяснить, что эта сила фиктивная.
Когда я говорю «сила», я имею в виду, что волчок ведёт себя так, как если бы на него действовала внешняя сила.}%
, перпендикулярную этому движению, как показано на рисунке \ref{pic:9.3}.
На рисунке видно: волчок может сначала начинать падать, но затем отклоняется от падения вниз.
Постоянное действие этой уклоняющей силы приводит к траектории, изображённой на рисунке \ref{pic:9.3}.
Такой механизм можно назвать «уклончивой устойчивостью».

\paragraph{Энергетические соображения.}
То, что ось реагирует силой, \emph{перпендикулярной} навязанному движению (рисунок \ref{pic:9.3}), можно объяснить через закон сохранения энергии.
Действительно, если я перемещаю конец оси с постоянной скоростью $v$, то я не изменяю энергию вращения гироскопа.
Ведь если подшипники идеальны, я никак не могу влиять на скорость вращения.
Следовательно, я не совершаю работы, а значит, силе действия моей руки придётся быть перпендикулярной к скорости её движения.

\section{Гироскоп в велосипеде}

Современный велосипед прошёл долгой путь дарвиновскотехнологической эволюции и похоже.
Он идеален, насколько насколько таковым может быть творение цивилизации.
Научиться ездить на велосипеде гораздо легче, чем объяснить физику этого действия.%
\footnote{Похоже, что тело умнее головы.}
Следующие две задачи посвящены этому весьма сложному вопросу.

\begin{figure}[ht!]
\centering
\begin{lpic}[t(8mm),b(2mm),r(0mm),l(0mm)]{pics/9.4}
\lbl[tl]{-10,37;(a)}
\lbl[tl]{28,37;(b)}
\lbl[b]{9,31;\parbox{22mm}{\footnotesize\centering  вид сверху на\\переднее колесо}}
\lbl[b]{47,31;\parbox{32mm}{\footnotesize\centering вид спереди без\\поступательного движения}}
\lbl[l]{19,15;\parbox{22mm}{\footnotesize  направление\\движения}}
\lbl[bl]{14,4;$1$}
\lbl[l]{10,15.5;$2$}
\lbl[tl]{14,27;$3$}
\lbl[r]{46,15.5;$1$}
\lbl[lr]{42,27;$2$}
\lbl[t]{0,14;$F$}
\lbl[t]{55,14;$F$}
\end{lpic}
\caption{Гироскопический эффект переднего колеса.
(a) Поворот колеса вправо наклоняет велосипед влево.
(b) Наклон велосипеда влево также поворачивает колесо влево.}
\label{pic:9.4}
\end{figure}

\paragraph{Вопрос.}
Я еду на велосипеде прямо и слегка повeрнул руль вправо.
Как на меня повлияет гироскопический эффект от переднего колеса?

\paragraph{Ответ.}
Рама наклонится влево; это объясняется на рисунке \ref{pic:9.4}a.

\paragraph{Вопрос.}
Теперь я еду на велосипеде без рук по прямой, я наклоняю раму влево --- скажем, согнувшись в сторону в поясе.
В какую сторону гироскопический эффект будет пытаться повернуть переднее колесо?

\paragraph{Ответ.}
Тоже влево, как объясняется на рисунке \ref{pic:9.4}b.

\section{Как катится монета?}

Устойчивость катящейся монеты похожа на чудо.
Создаётся впечатление, что монета разумна или, по крайней мере, обладает рефлексом опытного моноциклиста (уж точно она справляется лучше неопытного).
Случайные неровности поверхности ей не помеха --- она легко под них подстраивается;
она умудриться устоять даже если её слегка толкнуть.
Без каких-либо подвижных частей монета воплощает предельную простоту; но как же объяснить её рефлексы?
Как безмозглый кусок метала умудряется управлять своим движением, удерживаясь на тонкой грани между падением влево или вправо?

Следующая задача чуть приблизит нас к ответу.
Но даже разобравшись, как всё устроено, меня не перестаёт удивлять ловкость и естественность движения монеты.
Несмотря на объяснение (которое скоро будет дано), кажется счастливой случайностью, что гироскопический эффект \emph{помогает} монете устоять, а не, наоборот, упасть.
Более всего восхищает контраст между устойчивостью монеты и отсутсвием у неё мозгов.%
\footnote{В этом смысле некоторые люди тоже меня восхищают.}

\begin{figure}[ht!]
\centering
\begin{lpic}[t(2mm),b(2mm),r(0mm),l(0mm)]{pics/9.5}
\lbl[rw]{38,27;{\footnotesize 1: наклон вправо}}
\lbl[r]{25,21;\parbox{13mm}{\footnotesize 2: момент силы}}
\lbl[tl]{39,21;\parbox{16mm}{\footnotesize 3: поворот направо не даёт упасть}}
\end{lpic}
\caption{Гироскопический эффект подправляет движение катящейся монеты, предотвращая её падение.
Из-за наклона (1) сила тяжести создаёт гироскопический момент (2), который заставляет монету поворачивать вправо (3).}
\label{pic:9.5}
\end{figure}

\paragraph{Задача.}
Если монета катится вперёд с уклоном вправо, как показано на рисунке \ref{pic:9.5}, то её траектория поворачивает тоже вправо.
Из-за этого счастливого совпадения монета не падает.
Похоже, что монета умна --- наклонившись вправо, она поворачивает вправо, как это делает наклонившийся велосипедист, тем самым избегая падения.
Почему же монета поворачивает в сторону своего наклона?

\paragraph{Решение.}
Когда монета катится вперёд с наклоном вправо,
она начинает падать (рисунок \ref{pic:9.5}).
Это падение --- то есть рост её наклона ---
вызывает гироскопический момент (объясняется в следующем предложении;
см. также страницу ???), который поворачивает монету вправо,
заставляя её траекторию поворачивать вправо; эта коррекция и предотвращает падение.
Для понимания механизма происходящего, вообразим, что бы произошло, если бы монета двигалась прямо, продолжая всё больше наклоняться.
В системе отсчёта, движущейся вместе с монетой, мы бы наблюдали изгиб траекторий частиц на её ободе, как показано
на рисунке \ref{pic:9.2}.
Возникающая центробежная сила будет стремиться повернуть монету вправо.

Приведённое рассуждение говорит лишь, что монета может не упасть, оно вовсе не доказывает, что монета не упадёт.
Откуда нам знать, например, что этот самокорректирующий эффект достаточно силён, чтобы удерживать монету?
А может и наоборот, этот эфект окажется чрезмерно сильным, вызывая нестабильный колебательный процесс?
Конечно же, чтобы сохранять равновесие, монета должна катиться достаточно быстро.
Подробное обсуждение устойчивости катящейся монеты
можно найти на страницуах ???---??? \emph{«Динамики неголономных систем»} Неймаркa и Фуфаева.

\section{Как удержаться на скользком куполе?}

\paragraph{Задача.}
Можно ли поместить твёрдое кольцо на идеально гладкий полусферический купол (рис. 9.6) так, чтобы оно не соскользнуло, даже если его слегка подтолкнуть?
Удержать кольцо точно на верху не получиться, ведь малейший толчок заставит его соскользнуть.
Использование внешних опор, включая магнитные или тому подобные устройства, не разрешается.

\begin{figure}[ht!]
\centering
\begin{lpic}[t(2mm),b(2mm),r(0mm),l(0mm)]{pics/9.6}
\end{lpic}
\caption{Как не сокользнуть с идеально скользкого купола?}
\label{pic:9.6}
\end{figure}

\paragraph{Решение.}
Надо поместить кольцо на купол,
хорошо раскрутить быстро вокруг его центральной оси и отпустить.
Если кольцо вращается достаточно быстро и оно расположено не слишком далеко от вершины, то оно не соскользнёт, а будет двигаться так, как показано на рисунке \ref{pic:9.7}d.%
\footnote{Мы считаем, что трения нет вовсе. Иначе кольцо стало бы  замедлятся и соскользывать.}

\paragraph{Почему это сработает.}
Наше кольцо на сфере --- это по сути волчок: колесо с длинной осью, вращающееся на шарнире (рис. \ref{pic:9.3}a, страница \pageref{pic:9.3}).
Чтобы пояснить эту аналогию: вместо того чтобы строить идеально скользкую сферу (что очень не просто), можно прикрепить кольцо к концу стержня с помощью невесомых спиц, получится велосипедное колесо на длинной оси.
Конец оси опирается на стол или прикреплён к столу с помощью шарнира без трения.
Таким образом, кольцо фактически будет ограничено невидимой сферой, без трения.
Итак, кольцо на сфере --- это волчок.%
\footnote{Чтобы аналогия с волчком стала полной, придётся предположить, что кольцо не может отрываться от поверхности купола (хотя на самом деле такое может произойти).
Для этого можно считать, что кольцо удерживается на сфере какой-то силой, например, магнитной.}
А волчок не падает, если его раскрутить достаточно быстро. Читатель может обратиться к объяснению на странице \pageref{Антигравитационное велоколесо:Ответ} или прочитать следующий абзац.

\begin{figure}[ht!]
\centering
\begin{lpic}[t(7mm),b(2mm),r(0mm),l(0mm)]{pics/9.7}
\lbl[bl]{-5,75;(a)}
\lbl[bl]{47,75;(b)}
\lbl[bl]{-5,34;(c)}
\lbl[bl]{47,34;(d)}
\lbl[lt]{79,50;$1$}
\lbl[lt]{73.5,47.5;$2$}
\lbl[tl]{69,43;$3$}
\lbl[br]{59,58;$1'$}
\lbl[br]{65,60;$2'$}
\lbl[br]{69,65;$3'$}
\lbl[bl]{70,54;$F$}
\lbl[tr]{67,54;$F'$}
\lbl[bl]{5,32;$F$}
\lbl[tr]{14.5,18;$F'$}
\lbl[tl]{23,9;\parbox{12mm}{\footnotesize центр сферы}}
\lbl[b]{24,13,39.5;{\footnotesize ось момента $F$ и $F'$}}
\lbl[t]{69,10;{\footnotesize траектория центра кольца}}
\end{lpic}
\caption{Почему кольцу удаётся не соскользнуть вниз.}
\label{pic:9.7}
\end{figure}

\paragraph{Прямое объяснение.}
Вскоре после того как кольцо отпускают, оно начинает скользить вниз.
Поэтому, если смотреть из системы отсчёта, связанной с центром кольца, оно как бы поворачивается, как показано на рисунке \ref{pic:9.7}b.
Этот поворот вызывает искривление траекторий частиц кольца --- например, траектории 1–2–3.
В силу инерции частица сопротивляется этому искривлению центробежной силой $F$, перпендикулярной к плоскости кольца.
Диаметрально противоположная частица кольца создаёт такую же по величине, но противоположно направленную силу $F$.
Эти две силы действуют на кольцо с моментом сил (рис. 9.7c).
(Я рассмотрел только две частицы, но остальные в совокупности создают такой же эффект.)
Этот гироскопический момент заставляет кольцо отклоняться от направления своего движения, и кольцо будет двигаться так, как показано на рисунке \ref{pic:9.7}d.

Именно так это и работает: не сопротивление падению, а уклонение от него!

Чтобы лучше развить в себе физическую интуицию
предствьте себя пузом на большой скользкой сфере, быстро крутящимся колесом (как при выполнении акробатического колеса).

\section{Как найти север с помощью гироскопа}

\paragraph{Вопрос.}
Как найти направление на географический север, используя гироскоп?
Считаем, что у вас идеальный гироскоп, без трения способный вращаться вечно.

\paragraph{Ответ.}
Если установить гироскоп горизонтально на плотик, который плавает в ёмкости с водой,
то ось гироскопа медленно повернётся точно по меридиану!
При этом направление на север будет то, в которое гироскоп вкручивается по правилу правого винта:
если думать, что ось это винт, то он будет вращаться так, чтобы двигаться на север.
Иными словами, гироскоп пытается как можно лучше выровнять своё вращение с вращением Земли, принимая во внимание, что его ось долна остоваться горизонтальной.

\paragraph{Почему же гироскоп ищет север?}
Для простоты разместим нашу установку на экваторе (рисунок \ref{pic:9.8}).
Вращение Земли заставляет вращаться нашу установку, и можно считать, что она вращается вокруг пунктирной линии север---юг.
Предположим, что ось гироскопа изначально направлена в каком-то другом направлении --- скажем, восток–запад, как на рисунке.
Вращения Земли толкает ось гироскопа вдоль стрелок ($A$);
гироскоп же отвечает поворотом в направлении ($B$).
Гироскоп ведёт себя ровно так, как объяснялось на странице \pageref{Антигравитационное велоколесо:Задача}: если толкнуть его ось, то он отреагирует движением в перпендикулярном направлении.
В итоге ось плавающего гироскопа ориентируется вдоль меридиана, как показано на рисунке.

\begin{figure}[ht!]
\centering
\begin{lpic}[t(7mm),b(2mm),r(0mm),l(0mm)]{pics/9.8}
\lbl[r]{0,86;$A$}
\lbl[t]{18,69;$A$}
\lbl[t]{13,71;$B$}
\lbl[b]{6,83;$B$}
\lbl[tl]{-5,97;{\footnotesize НАЧАЛО:}}
\lbl[tl]{-5,34;{\footnotesize КОНЕЦ:}}
\lbl[b]{18,88,50;{\scriptsize меридиан}}
\lbl[b]{21,28,60;{\scriptsize меридиан}}
\lbl[b]{43,30,64;{\scriptsize меридиан}}
\lbl{47.5,40.5;{\footnotesize экватор}}
\lbl[tl]{19,76;\parbox{16mm}{\scriptsize с земным\\ вращением}}
\lbl[t]{10,64;\parbox{20mm}{\footnotesize\centering ответ на\\ вращение}}
\lbl[l]{43,22;\parbox{16mm}{\footnotesize земнoe\\ вращение}}
\lbl[]{46,58;\parbox{18mm}{\footnotesize\centering плотик\\ на воде}}
\lbl[lt]{53,76.5;
\begin{tikzpicture}[
  decoration={
    text effects along path,
    text/.expanded=\bracetext{выстраивается по вращению земли },
    text effects/.cd,
    text along path,
    character count=\i,
    character total=\n,
    characters={scale=.7}
    }
]
\draw [decorate] (0,.6) .. controls (1.6,.9) and (3.8,.1) .. (2.7,-1.2);
\end{tikzpicture}
}
\end{lpic}
\caption{Как работает гирокомпас.}
\label{pic:9.8}
\end{figure}

\paragraph{Больше свободы.}
Вместо того чтобы помещать гироскоп на плотик, можно оставить его ось несвязанной, установив гироскоп на карданном подвесе или погрузив его в жидкость так, чтобы он имел нейтральную плавучесть.
Тогда гироскоп будет постепенно выравниваться по оси Земли. А угол его оси с горизонтом укажет широту!

В итоге, независимо от способа подвеса, гироскоп пытается по возможности согласовать своё вращение с вращением Земли.

\paragraph{Задача.} Почему подвешенный гирокомпас выравнивает свою ось по оси Земли?
(Подсказка: трение с жидкостью заставляет гироскоп переориентироваться.)

Полезность гирокомпаса в его независимости от магнитных аномалий.
Это даёт большое преимущество на стальном корабле.
Кроме того он указывает на географический, а не магнитный полюс.

\paragraph[Немного истории.]%
{Немного истории.%
\footnote{Больше подробностей можно найти о гирокомпасе в статье Википедии.}
}
Гирокомпас был запатентован в 1908 году Э. А. Сперри, автором множества других изобретений.
На счету Сперри более 400 патентов и гирокомпас, пожалуй, самый известный.
Гирокомпас Сперри сыграл важную роль в Первой мировой войне.
После смерти Сперри в его честь был назван корабль ВМС США (плавбаза подводных лодок).
Этот корабль был спущен на воду через десять дней после нападения на Перл-Харбор, а после долгой службы, завершившейся в 1982 году, его перевели в музей.
Гирокомпас используется на кораблях и по сей день.

Ещё в 1916 году, в разгар Первой мировой войны,
Сперри и Питер Хьюитт изобрели БПЛА --- самолёт-дрон.
Сперри пророчески назвал его «бомбой будущего».
В конеце войны Сперри и Хьюитт в длинной череде проб и ошибок пытались сделать из этой идеи надёжное оружие.
Сын изобретателя, Лоуренс Сперри, учавствовал в опасных испытаниях, и несколько раз чуть не погиб.
\chapter{Горячее и холодное}

\section{Может ли холодное нагреть горячее?}

Вопрос в заголовке конечно же имеет отрицательный ответ: при контакте двух тел тепло переходит от горячего к холодному.%
\footnote{Это является следствием второго закона термодинамики, экспериментально установленного принципа.}
Поэтому даже спрашивать следующее кажется глупым:

\paragraph{Задача.}
Можно ли, используя стакан воды нагретый до 100~°C, нагреть стакан молока с начальной температурой 0~°C до температуры выше чем 50~°C, то есть их общей температуры при смешивании?
Предполагаем, что стаканы одного размера.
Будем считать, что вода и молоко полностью идентичны всеми своими тепловыми свойствами.%
\footnote{В частности, у них одинаковые удельные теплоёмкости. То есть одинаковое количество тепла одинаково повышает температуры у равных масс воды и молока.}
Тепло не поступает извне, \emph{но разрешается использовать дополнительные сосуды.}

\paragraph{Решение.}
Это можно проделать, не нарушая второй закон термодинамики.
Для этого нам понадобится ещё один пустой стакан и крохотный ковшик.
Зачерпнём ковшик холодного молока, опустим его в горячую воду и подождём, пока температура не уравновесится.
Перельём содержимое ковшика в пустой стакан.
Повторим процесс, пока всё молоко не окажется в третьем стакане.
По пути в третий стакан каждая порция молока получает немного тепла от воды.
Я утверждаю, что после всех переливаний молоко окажется теплее воды.
Чтобы это понять, представьте, что вы вытаскиваете \emph{последний ковшик} молока из воды.
В этот момент молоко в ковшике той же температуры, что вода.
А остальное молоко ещё теплей, ведь ранние порции нагревались сильнее.
Значит, после добавления этого последнего ковшика к остальному молоку, оно окажется теплее воды.

\begin{figure}[ht!]
\centering
\begin{lpic}[t(2mm),b(2mm),r(0mm),l(0mm)]{pics/10.1}
\lbl{8,7;\parbox{20mm}{\footnotesize\centering молоко\\ при 0~°C}}
\lbl{41.2,10;\textcolor{white}{\parbox{20mm}{\footnotesize\centering вода\\ изначально \\была при\\ 100~°C}}}
\lbl{41.2,31;\parbox{28mm}{\footnotesize\centering температура\\уравновесилась}}
\lbl{74.4,3.7;\parbox{20mm}{\footnotesize\centering тёплое\\молоко}}
\end{lpic}
\caption{В стакане $N$ ковшиков, и
каждый ковшик молока при $0\degree$ охлаждает воду в $1+1/N$ раз.}
\label{pic:10.1}
\end{figure}

\paragraph{Приготовление лосося и постоянная Эйлера $\bm{e}$.}
Мы увидели, что этот метод нагревает молоко до температуры выше 50~°C, но насколько именно?
Если ковшик достаточно мал, то молоко нагреется приблизительно до 63~°C.
То есть можно обжечься, однако заметим, что это рекомендованная температура до которой надо прогревать внутренность филе лосося при приготовлении.

Математику должно показаться занятным, что предельно достижимая температура воды при этом методе равна
\[\frac{100\degree}{e},\]
где $e = 2{,}718\ldots$ — постоянная Эйлера, то есть предел последовательности $(1 \z+ 1/N)^N$ при $N \to \infty$.

Чтобы это обосновать,
давайте считать, что стакан вмещает $N$ ковшиков, где $N$ — целое число.
Один ковшик холодного молока при $0\degree$ и стакан с водой температуры $T$, придут в тепловое равновесие при температуре
\[
\frac{T}{1 + 1/N}.
\]
Действительно, при добавлении одного холодного ковшика к $N$ тёплым весь запас тепла $N$ ковшиков равномерно распределяется между $N+1$ ковшиками, и, следовательно, в расчёте на один ковшик, тепло  уменьшается в $\tfrac{N+1}{N} = {1+1/N}$ раз.
Значит и температура уменьшается в то же число раз.
Таким образом, при переходе от шага $k$ к шагу $k+1$, для температуры воды выполняется рекуррентное соотношение:
\[
T_{k+1} = \frac{T_k}{1 + 1/N},
\qquad T_0 = 100\degree.
\]
Во время всей процедуры исходная температура $T_0 = 100\degree$ делилась на одну и ту же величину $N$ раз, и в конце охладится до
\[
T_N = \frac{100\degree}{(1 + 1/N)^N} \;\;\approx\;\; \frac{100\degree}{e}.
\]

\paragraph{Температура тела.}
Взяв $N$ достаточно большим, получим, что $(1+1/N)^N \approx e = 2{,}718\ldots$,
и, значит,
\[
T_N \approx \frac{100\degree}{e} \approx 36{,}8\degree.
\]
Эта температура подозрительно близка к температуре человеческого тела.
Если вдруг нужно вычислить $e$ в экстренной ситуации и под рукой есть градусник, то можно измерить свою температуру в градусах Цельсия и подставить её в соотношение
\[
\frac{100\degree}{T_{\text{человек}}} \approx e.
\]
(Если у вас жар, то оценка получится заниженной, а если гипотермия, то завышенной.)
Получается, что натуральный логарифм (тот самый, что с основанием $e$) связан с человеческой натурой.

Раз уж пошла речь о совпадениях, напомню, что температура человеческого тела связана с оптимальной температурой для приготовленного лосося ($63\degree$):
\[
T_{\text{человек}} + T_{\text{лосось}} \;\approx\; 100\degree.
\]

\paragraph{Ещё теплей.}
Оказывается, что можно добиться почти идеального обмена температурами двух жидкостей — по крайней мере, в теории.
Для этого нужно мелко дробить обе жидкости, а не только молоко.
Практически это можно реализовать, пропуская воду и молоко в противоположных направлениях через две трубки, находящиеся в тесном тепловом контакте, как на рисунке \ref{pic:10.2}. Если прокачивать молоко слева, а воду справа, то мы получим почти идеальный обмен теплом.
Это простое устройство называется (противоточным) теплообменником.

\begin{figure}[ht!]
\centering
\begin{lpic}[t(2mm),b(2mm),r(0mm),l(0mm)]{pics/10.2}
\lbl[br]{6,17;\parbox{20mm}{\footnotesize\raggedleft холодное\\молоко}}
\lbl[tr]{6,1.5;\parbox{20mm}{\footnotesize\raggedleft холодная\\вода}}
\lbl[lb]{71,17.5;\parbox{20mm}{\footnotesize горячее\\молоко}}
\lbl[lt]{71,2;\parbox{20mm}{\footnotesize горячая\\вода}}
\lbl{39,9.5;{\footnotesize передача тепла}}
\end{lpic}
\caption{Противоточный теплообменник обеспечивает почти полный обмен температурами.}
\label{pic:10.2}
\end{figure}

Идея теплообменника используется в природе.
Например наши руки снабжены теплообменниками, ведь кровь глубоких вен идёт противотоком с кровью артерий.
В холодных условиях холодная кровь от кистей возвращается по этим венам, получая тепло от идущей наружу артериальной крови. Согретая поступающая кровь помогает поддерживать температуру тела.
При этом охлаждённая артериальная кровь, направляющаяся к конечностям, уже не отдаёт так много тепла наружу.
В жарких условиях этот механизм отключается: кровь идёт по поверхностным венам, помогая рассеивать избыточное тепло.

Теплообменниками снабжены собаки, овцы, верблюды и другие животные.
Они помогают поддерживать температуру мозга ниже, чем у остального тела: более холодная венозная кровь из рта и носа охлаждает артериальную кровь, питающую мозг (наиболее уязвимый к перегреву орган).
Кролики, у которых нет такого механизма, рискуют погибнуть от перегрева, если в жаркую погоду их долго преследует собака.
У серых китов по всей поверхности языка имеется множество противоточных теплообменников (язык не может быть утеплён слоем жира).
И этот механизм не ограничивается только млекопитающими: некоторые рыбы, например тунец, используют противоточные теплообменники, чтобы поддерживать температуру мышц на целых 14 °C выше температуры воды.

\section{Насос и молекулярный пинг-понг}

\paragraph{Вопрос.}
Можно заметить, что насос нагревается когда вы накачиваете шину велосипеда.
Происходит ли это нагревание только из-за трения или ещё по какой-то другой причине?

\paragraph{Ответ.}
Основная причина не в трении.
Воздух нагревается при сжатии, и уже он нагревает его стенки.

\paragraph{Вопрос.} А почему сжатие нагревает воздух?

\paragraph{Ответ.}
Если вы решите сыграть в пинг-понг или теннис, то ответ окажется в ваших руках.
Ракетка, ударяющая по летящему шарику, подобна движущемуся поршню насоса, подтолкающего молекулу воздуха.
Благодаря движению ракетки шарик после удара приобретает б\'{о}льшую скорость (см. рисунок \ref{pic:10.3}).
(Прибавка скорости равна удвоенной скорости налетающей ракетки.
Предполагается, что столкновение совершенно упругое и масса шарика мала по сравнению с массой ракетки.)
Точно так же и молекулы, отталкиваясь от поршня, ускоряются и нагревают воздух.

\begin{figure}[ht!]
\centering
\begin{lpic}[t(2mm),b(2mm),r(0mm),l(0mm)]{pics/10.3}
\lbl{23,23,30;{\footnotesize холодный}}
\lbl{34,7,52;{\footnotesize горячий}}
\lbl[tl]{58,26;$v$}
\lbl[t]{66,23;$V$}
\lbl[t]{75,20;$V+2v$}
\lbl{65,7;\parbox{48mm}{\footnotesize\centering молекула ускоряется на\\ удвоенную скорость поршня}}
\end{lpic}
\caption{Молекула ускоряется на удвоенную скорость поршня.}
\label{pic:10.3}
\end{figure}

На рисунке \ref{pic:10.4} показано, как меняется температура воздуха внутри насоса.
Согласно графику, средняя температура в насосе выше, чем снаружи.
При этом стенка насоса нагревается до некоторого среднего уровня.

\begin{figure}[ht!]
\centering
\begin{lpic}[t(2mm),b(2mm),r(0mm),l(0mm)]{pics/10.4}
\lbl[t]{45,-1;{\footnotesize время}}
\lbl[b]{0,12.5,90;{\footnotesize температура}}
\lbl[t]{45,7;{\footnotesize температура снаружи}}
\lbl[l]{10,3.5;{\footnotesize сжатие}}
\lbl[b]{14,17;\parbox{22mm}{\footnotesize\centering выход\\ воздуха}}
\end{lpic}
\caption{Зависимость температуры в насосе от времени.}
\label{pic:10.4}
\end{figure}

\section{Теплонасос из велонасоса}

Теплонасос — это холодильник, которым пользуются для нагрева.
Холодильник перекачивает тепло изнутри наружу.
Если холодильник называют теплонасосом, то всего лишь хотят сказать что он используется для нагрева, а не охлаждения,
однако эти две функции неотделимы друг от друга.

\paragraph{Вопрос.}
Можно ли использовать велонасос как теплонасос, то есть перекачивать им тепло внутрь тёплого помещения из холодной зимней стужи снаружи?
Меня интересует лишь возможность этого в принципе, решение не обязано быть практичным.

\paragraph{Ответ.}
Насос с закрытым выходным отверстием — это просто поршень в цилиндре с воздухом (см. рисунок \ref{pic:10.3}).
Поместим насос снаружи, не сжатый и холодный.
Затем надавим на поршень достаточно сильно, чтобы сжатый воздух нагрелся выше комнатной температуры (сжатие вызывает нагрев).
Далее занесём насос внутрь чтоб он отдал часть тепла в помещению.
Как только его температура сравнится с температурой комнаты, вытащим его наружу и расслабим поршень.
Расширяющийся воздух станет холоднее уличного воздуха, ведь он отдал часть своего тепла помещению.
Став холоднее, насос будет втягивать тепло из холодного зимнего воздуха!
Это тепло компенсирует то, которое мы передали воздуху внутри.
После того как насос снова достигнет наружной температуры, цикл завершён и его можно повторять пока не надоест.

Настоящие теплонасосы устроены хитрей, но принцип работы тот же.
Вместо воздуха в них используют хладагент%
\footnote{например один из фреонов \pr}%
; сжатие и расширение заменены конденсацией и испарением хладагента.
Хладагент перекачивается по трубам, и нет нужды бегать с ним туда-сюда.

\paragraph{Эффективность.}
Удивительно, но тепловой насос расходует меньше энергии, чем сжигание топлива для получения того же количества тепла.
Это происходит потому, что часть работы, затраченная на сжатие поршня снаружи, возвращается, когда я снова выношу насос наружу и отпускаю поршень; то есть, поршень отдаёт обратно часть энергии, которую я потратил на его сжатие.

\section{Две комнаты}

\paragraph{Вопрос.}
Одна из двух одинаковых комнат в доме теплее другой.
Верно ли что у молекул воздуха в тёплой комнате
суммарная кинетическая энергия больше, чем в холодной?

\paragraph{Ответ.}
Энергия одинакова!%
\footnote{Я предполагаю, что воздух является идеальным газом и не учитываю вторичные эффекты, такие как расширение стен при нагревании и невозможность поддерживать температуру совершенно одинаковой во всей комнате.}
У воздуха в тёплой комнате средняя энергия молекулы выше чем в холодной, однако в тёплой комнате меньше молекул, так как нагретый воздух расширяется, и какая-то его часть просачивается в щель под дверью.
Эти два противоположных эффекта (молекулы быстрее, но их меньше) уравнивают друг друга.
Действительно, число молекул в комнате обратно пропорционально температуре $T$ (считая от абсолютного нуля), тогда как кинетическая энергия каждой молекулы прямо пропорциональна $T$.
Сейчас я разберу всё это подробней.

\paragraph{Подробное объяснение.}
Согласно уравнению состояния
идеального газа (достаточно хорошее приближение при рассматриваемых температурах),
\begin{equation}
    pV = NkT,
    \label{eq:10.1}
\end{equation}
где $p$ --- давление в комнате, $V$ --- объём комнаты, $T$ --- абсолютная температура воздуха,
$N$ --- число молекул в комнате, а $k$ --- константа, не зависящая от перечисленных величин
(она называется постоянной Больцмана).

С другой стороны, известно, что средняя кинетическая энергия $E$ молекулы прямо пропорциональна температуре газа: $E = (3k/2)T$.
Поэтому полная кинетическая энергия всех $N$ молекул в комнате равна
\[E_{\text{полная}}
= N E
= N \frac{3k}{2}T
= \tfrac{3}{2}NkT
\stackrel{\text{\eqref{eq:10.1}}}{=} \tfrac{3}{2}pV.
\]
Поскольку при нагревании комнаты и давление $p$, и объём $V$ остаются постоянными,
постоянной остаётся и $E_{\text{полная}}$, что и требовалось.

\section{Как морозить велосипедной шиной}


\paragraph{Вопрос.}
Как можно создать температуру ниже нуля в жаркий летний день, используя велосипедную шину?

\paragraph{Ответ.}
Надо просто открыть ниппель в накачанной шине.
Допустим, что шина надута до 3 атмосфер.
Это означает, что давление в внутри шины на 3 атмосферы больше чем снаружи, то есть давление в шине 4 атмосферы.
При выходе из шины, воздух значительно расширяется, ведь давление падает с 4 до 1 атмосферы.
В свою очередь, расширение вызывает охлаждение.
При таком падении давления абсолютная температура воздуха уменьшается процентов на сорок.
Начав с температуры $27 \,\text{°C} \approx 300\, \text{°K}$, мы получим $171 \,\text{°K} \approx -102\, \text{°C}$!
Здесь не учитывается нагрев за счёт вязкости при прохождении через узкое отверстие, но с уверенностью можно сказать, что температура будет ниже точки замерзания.

\paragraph{Вычисления.}
Рассмотрим небольшую область воздуха, которая за очень короткое время перемещается из шины наружу.
Поскольку всё происходит быстро, можно пренебречь теплообменом между этой областью и окружающим воздухом.
Для такого расширения без теплообмена (называемого \emph{адиабатическим расширением}) температура воздуха пропорциональна давлению в степени $2/5$:
\[\frac{T_2}{T_1} = \left(\frac{p_2}{p_1}\right)^{\tfrac{2}{5}},\]
где индексы температуры и давления обозначают начальное и конечное состояния.
Мы знаем, что $p_2/p_1=1/4$.
Возведя это в степень $2/5$, получаем примерно $0{,}57$.
Начав с $T_1 = 300 \,\degree\text{K}$ получаем
\[T_2 = 300 \cdot 0{,}57 \approx 171 \,\text{°K},\]
что соответствует примерно $-102\, \text{°C}$!
Таким образом, открывая ниппель, мы получаем самое холодное место на Земле (не считая криогенных лабораторий).
Хоть оно и очень маленькое и существует очень короткое время, но всё равно этим можно гордиться.

\chapter{Пара вечных двигателей}

Вечный двигатель — это утопическая мечта, а мечты тянут к себе чудаков.
К счастью, в отличие от многих социальных утопистов прошлого (да и настоящего), эти не опасны; они редко убивают ради идеи.
Общее для всех утопий — это попытка нарушить какой-нибудь закон, будь то закон сохранения энергии, закон экономики, закон человеческой психологии или закон общества.

Изобретатель вечного двигателя должен обладать безграничным умом, ведь ему надо справиться с бесконечно трудной задачей — изобрести невозможное.
Изобретатели вечных двигателей бывают очень умны, но мудрых среди них немного.

Два вечных двигателя этой главы, — это головоломки, в каждой из которых требуется найти изъян.%
\footnote{Вскоре после Галилея разоблачать ложные теории физики  стало довольно безопасно.
В экономике это случилось гораздо позже.
В Советском Союзе, приблизительно в 1947 году, мой знакомый получил 12 лет за устные рассуждение в классе экономики о том, что стремление к прибыли может быть необходимым условием хорошо работающей экономики.}

\section{Вечный двигатель на капелярной тяге}

Вода поднимается в тонкой трубочке благодаря капиллярному эффекту, как показано на рисунке \ref{pic:11.1}.
Можно ли как-то использовать работу, поднимающейся воды?
она могла бы тянуть что-то за собой, но, к сожалению, этот двигатель останавливается, как только вода достигает определённой высоты.
Но вот способ обойти эту проблему.
\begin{figure}[ht!]
\centering
\begin{lpic}[t(2mm),b(2mm),r(0mm),l(0mm)]{pics/11.1}
\lbl[r]{6,25;\parbox{20mm}{\footnotesize\raggedleft сила\\капиляра}}
\lbl{43,17;\parbox{20mm}{\footnotesize\centering вид\\сверху}}
\lbl[b]{72.8,18.1,10;{\tiny $+--+$}}
\end{lpic}
\caption{Вечный двигатель на капелярной тяге}
\label{pic:11.1}
\end{figure}
Разместим поршень внутри тонкой замкнутой трубки и добавим каплю воды вплотную к поршню, как показано на рисунке \ref{pic:11.1}, чтоб не было пузырьков между водой и поршнем.
Поршень может скользить практически без трения.
Положим трубку горизонтально на стол, чтобы не приходилось бороться с силой тяжести.
Теперь капиллярный эффект будет тянуть поршень по трубке, и нет ничто, что остановило бы его движение;
то есть, ему придётся двигаться бесконечно.
Чтобы всё работало, надо только добиться, чтобы сила трения поршня была меньше, тянущей его капиллярной силы.
Так мы получим бесконечный источник энергии, не требующий топлива, ведь при малом, но ненулевом трении выделяется тепло.

\paragraph{Вопрос.}
Если это не первый в мире работающий вечный двигатель, то в рассуждении должна быть ошибка, но где?
То есть где именно ошибка в моём «доказательстве», что поршень будет двигаться вечно?
(Она вовсе не в реализации поршня с малым трением.)

\paragraph{Ответ.}
Проблема не в отсутсвии хорошей смазки, она серьёзнее.
Сначала разберёмся, как вода поднимается в капиляре.
Это происходит по двум причинам:
(1) из-за поверхностного натяжения: поверхность воды ведёт себя как натянутая резиновая плёнка из-за особенностей расположения молекул воды;
(2) наличие электростатического притяжения воды к стенкам трубки.
Электростатическое притяжение притягивает воду к внутренней стенке трубки, из-за чего поверхность воды принимает вогнутую форму мениска.
Тут же включается поверхностное натяжение: стремясь выпрямить вогнутую поверхность, оно тянет за собой водяной столб.
Эти два процесса — растекание вдоль стенки и подтягивание — происходят одновременно, и вода поднимается.
Когда сила тяжести уравновесит капилярный эффект, подъём прекратится.

Теперь вернёмся к нашему двигателю.
Тот же самый эффект, который поднимал воду в трубочке, здесь тоже присутствует: вогнутый мениск по-прежнему тянет воду, пытаясь увлечь за собой поршень.
Но что происходит рядом с поршнем?
Сейчас мы выясним, что там возникает противоположный эффект, сводящий на нет всю идею.
Из-за электростатического притяжения давление воды у стенок оказывается выше, и, в частности, оно выше возле поршня.
Это дополнительное давление толкает поршень в противоположном направлении.
Именно этот противодействующий эффект и был упущен.

\section{Вечный двигатель из эллиптического отражателя}

Эту задачу, я узнал от Петера Унгара в середине
1970-х годов.
Она основана на прекрасном свойстве эллиптических зеркал:
любой луч света, исходящий из одного фокуса эллипса,
после отражения%
\footnote{Предполагается, что поверхость отражает идеально.}
пройдёт через другой его фокус.
Следующий вечный двигатель использует это свойство.

\paragraph{Идея теплопередачи.}
Внутренняя поверхность эллипсоидальной оболочки (рисунок \ref{pic:11.2}) представляет собой идеальное зеркало.
\begin{figure}[ht!]
\centering
\begin{lpic}[t(2mm),b(2mm),r(0mm),l(0mm)]{pics/11.2}
\lbl[t]{29,42.5;\parbox{20mm}{\footnotesize\centering идеальное\\зеркало}}
\end{lpic}
\caption{Излучение шаров в фокусах идеально отражается эллипсоидным зеркалом.}
\label{pic:11.2}
\end{figure}
В двух фокусах помещены два шара разного радиуса, находящиеся при одной и той же температуре $T$.
Оба наших шара излучают, (как и любое тело с температурой выше абсолютного нуля).
Из двух шаров при одинаковой температуре больше энергии излучает больший (если предположить, что они сделаны из одного и того же материала и имеют одинаковый цвет).
Так как каждый луч, исходящий из одного фокуса, после отражения проходит через другой, все лучи от большего шара попадают на меньший, и наоборот.
Отсюда следует, что при равной температуре большой шар отдаёт больше тепла, чем получает, и поэтому маленький будет нагреваться, а большой — охлаждаться.
Разность температур можно использовать чтобы приводить в действие двигатель%
\footnote{Например, она может вызывать движение воздуха, которое можно использовать для вращения колеса.}%
, так что наше устройство обеспечивает вечный источник энергии.

\paragraph{Вопрос.} Где ошибка?

\paragraph{Ответ.} Не всякий луч, исходящий от большего шара, попадёт на маленький.
Некоторые вернутся обратно к большому;
это относится, например, к лучам, выходящим из большого шара влево как на рисунке \ref{pic:11.2}.
Но ещё важнее то, что лучи исходят с поверхности тела во всех направлениях, а не только радиально, и такие нерадиальные лучи не проходят через фокусы.%
\footnote{Интенсивность излучения в данном направлении прямо пропорциональна элементу площади поверхности тела и косинусу угла между направлением и нормалью. Более того, если бы нашёлся матеиал для которого интенсивность излучения описывалась другим законом, то вечный двигатель стал бы реальностью. \pr}
Это и разрушает исходное рассуждение.





{

\sloppy

\nocite{*}%печатает всё
\printbibliography[heading=bibintoc]
\fussy

}


\newpage

{

\tableofcontents

}

\end{document}

БИНОМ. Лаборатория знаний
Манн, Иванов и Фербер (МИФ)
Альпина нон-фикшн
Интеллект

углубленную/олимпиадную литературу (БИНОМ. Лаборатория знаний, Интеллект, МЦНМО, Физматлит).



