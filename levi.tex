\documentclass[twoside]{book}
\usepackage{geometry}
\usepackage{book-ru}



\geometry%{top=0.9in, bottom=0.9in,left=0.9in, right=0.9in, paperwidth=6in, paperheight=9in}
{top=0.9in, bottom=0.9in,inner=0.9in, outer=0.7in, paperwidth=6in, paperheight=9in}


\def\thetitle{Почему кошки падают на лапы}
\def\theauthor{Марк Леви}

\hypersetup{
pdftitle={\thetitle},
pdfauthor={\theauthor},
pdfsubject={Математика}}

\makeatletter
\newcommand{\rindex}[2][\imki@jobname]{%
  \index[#1]{\detokenize{#2}}%
}
\makeatother


\begin{document}
%\pagestyle{empty}

\title{\thetitle\\
и ещё 76 парадоксов и головоломок}
\author{\theauthor}
\date{}
\maketitle

\thispagestyle{empty}

Предварительное издание, предназначенное исключительно для отлова ляпов. 

Исправления слать по адресу 
\url{petrunin@math.psu.edu}.

\vfill

\pagebreak

\thispagestyle{empty}
\chapter[Парадоксы, головоломки, задачи]{Физические парадоксы, головоломки и задачи}

\section{Введение}

Хороший физический парадокс --- это (1) неожиданность, (2) головоломка и (3) урок и всё это в прикольной обёртке.
Парадокс часто основан на убедительном рассуждении, которое приводит либо к ошибочному, но правдоподобному выводу, либо к верному но неожиданному выводу, который кажется ошибочным;
так, что трудно не поддаться искушению разобраться что к чему.
В годы холодной войны ходила байка, что ЦРУ мешало работе советских военных НИИ, подбрасывая листовки с головоломками и логическими задачами.
Времена изменились, теперь те же задачи используются при приёме на работу, советский пропагандист сказал бы, что они остались орудиями капитализма.

Парадоксы не только увлекательны, но и полезны.
Они развивают интуицию, логическое мышление и критический подход, так, что
у человека развивается внутренний детектор лжи.
Хороший парадокс также учит скромности и осторожности, показывая, как легко ошибиться даже в элементарных вопросах физики.
Успокаивает то, что даже очень умные люди допускают ошибки в, казалось бы, очевидных вопросах.
А ведь объекты, с которыми имеют дело другие области --- астрономия, биология, медицина, экономика, климатология, политика и СМИ посложней%
\footnote{Я не пытаюсь сравнивать сложности наук, просто хочу сказать, что типичный объект в физике (например, кристалл) проще типичного объекта в других областях (например, клетки в биологи).}%
, чем в физике, а значит, там ещё легче ошибиться.
Кроме того, некоторые «ошибки» могут приносить пользу, по крайней мере, некоторое время.

В этой книге я хотел поделиться прикольными размышлениями о том, как устроен мир.
Надеюсь они помогут вам понять суть некоторых физических явлений, и при этом не измучают%
\footnote{Я говорю о «мучениях» с иронией --- математика, разумеется, незаменима и прекрасна, по крайней мере для меня, уже потому, что это моя профессия.}
математической.

Эта книга по физике — науке, которая ходит на двух ногах, одна нога --- это математика, а другая --- физическая интуиция.
К сожалению, школьная физика часто выходит храмоногой.

\paragraph{Сравнение с музыкой.}
Если бы музыке учили так же, как зачастую учат физике, то мы знали бы отдельные ноты, но не мелодии, которые из них получаются.
Увы, слишком много учеников видят в физике набор формул, которые надо лишь применить в подходящем случае, и,
как следствие, много способных учеников теряют к физике интерес.

\paragraph{Сначала интуиция.}
Эта книга призвана натренировать вашу физическую интуицию.
Слишком часто курсы физики пренебрегают интуицией, делая упор на подбор формулы, подходящей в конкретном случае.
В этой книге всё наоборот:
я хотел добиться максимума интуитивного понимания при минимуме формул.
Обсуждение волчка --- хороший тому пример;
без каких-либо формул я объясню, почему вращающийся волчок остаётся в вертикальном положении.
Нужны годы освоения математики и физики в том объёме, чтобы научиться записывать дифференциальные уравнения движения волчка и понять, как из этих уравнений следует его устойчивость.
Пройдя весь этот путь, лишь немногие придут к интуитивному пониманию, почему волчок не падает.
Обидно, когда всё это время мощнейший инструмент --- физическая интуиция --- оказывается не у дел.

\section{Предварительные сведения}

Б\'{о}льшая часть книги (хоть и не вся) должна быть доступна читателям без специальной подготовки в физике;
все необходимые понятия объяснены в приложении.
Обычно математика остаётся в рамках алгебры, но изредка используется математический анализ.
Но даже в этих местах читатель, готовый принять на веру кое-какие математические выкладки, должен сносно себя чувствовать.%
\footnote{Речь идёт, например, о задаче с пращой на странице \pageref{???}, где камень достигает бесконечной скорости за секунду.}

Тяга к новому — основной инстинкт большинства живых существ, или, по крайней мере, млекопитающих.
Побуждая нас к исследованию, этот инстинкт помогает выживать — за исключением некоторых случаев, вроде лауреатов «Премии Дарвина» или героев шоу «Чудаки» (также известного как «Придурки», англ. Jackass).
Тот же самый инстинкт, который привёл Эйнштейна к его великим открытиям, толкает любопытного ребёнка разобрать механические часы и заглянуть внутрь.
Он же побуждает щенков и котят исследовать окружающий мир,
а у некоторых людей этот инстинкт настолько силён, что способен противостоять системе школьного образования.

\section{Источники}

Собирать подобные задачи посоветовал мне отец после того, как увидел одну мою головоломку;
она пришла мне в голову после школьного урока о капиллярном эффекте (см. страницу \pageref{???}).
Из этой коллекции и выросла данная книга.
Несмотря на то, что многие задачи книги моего собственного сочинения%
\footnote{Например, \ref{Гелиевый шар}, \ref{Парадокс с кометой}, \ref{Хочешь медленнее}, 3.1, 3.2, 3.5, 3.6, 4.1, 4.2, 4.4---4.6, 5.3---5.8, 6.6, 6.7, 6.10---6.12, 8.2, 8.5, 8.6, 9.4, 11.1, 12.3, 13.2, 14.6, 14.8.???
}, \emph{я уверен, что кто-то уже задумывался над ними или чем-то похожем задолго до моего рождения.}
Если мне известен автор или источник, то я его указываю.

\paragraph{Книжная полка.}
К счастью, %???
многое из основ физики можно понять, получив от этого удовольствие и (почти) без формул ---
несколько замечательных научно-популярных книг тому подтверждение;
среди них — \emph{The Flying Circus of Physics} Уолкера,
\emph{Thinking Physics} Эпштейна,
\emph{Mad about Physics} Яргодзки и Поттера,
а также классическая «Занимательная физика» Перельмана.
К сожалению, шикарная книга Маковецкого \emph{Смотри в корень}, которая разошлась тиражом более миллиона экземпляров в Советском Союзе, похоже, так и не была переведена на английский.
Книга Миннарта \emph{The Nature of Light and Color in the Open Air} (посвящёна оптическим явлениям в природе) никогда не устареет и доставит радость любому любознательному человеку, которому посчастливится её открыть.

\chapter{В открытом космосе}

\section{Шарик с гелием}\label{Гелиевый шар}

\paragraph{Задача}
Два космонавта, Андрей и Боря, пристёгнуты к противоположным концам космической капсулы, как показано на рисунке \ref{pic:2.1}.
В начале всё находится в покое и Андрей держит в руках большой шарик надутый гелием.
Он толкает шарик, и тот начинает двигаться в сторону Бори.
В каком направлении начнёт двигаться вся капсула с точки зрения наблюдателя, парящего в открытом космосе снаружи?
Поскольку Андрей и Боря пристёгнуты к стенкам, их можно считать частью капсулы.


\begin{figure}[ht!]
\centering
\begin{lpic}[t(2mm),b(2mm),r(0mm),l(0mm)]{pics/2.1(1)}
\lbl[b]{29,24;куда?}
\lbl{29,20;воздух}
\lbl{29,14;{\footnotesize гелий}}
\lbl[lt]{0,-.5;Андрей}
\lbl[rt]{58,-.5;Боря}
\end{lpic}
\caption{Куда двинется капсула когда Андрей толкнёт шар?}
\label{pic:2.1}
\end{figure}

\paragraph{Правдоподобное рассуждение.}\label{Првдоподобное рассуждение}
Поскольку Андрей толкает шарик вправо, шарик отталкивает Андрея влево, ведь по третьему закону Ньютона «действие равно противодействию».
А раз шарик толкает Андрея влево, то он и вся капсула начнут двигаться влево.

Похоже ли это на правду?

\paragraph{Ответ.}
На самом деле --- нет: капсула тоже будет двигаться вправо!

\paragraph{Объяснение через центр масс.}
Центр масс всей системы (капсулы и её содержимого) остаётся неподвижным, поскольку на систему не действуют внешние силы (все понятия этого предложения объяснены в приложении, страница \pageref{???}).

Рассмотрим движение внутри капсулы \emph{с точки зрения Андрея}, как показано на рисунке~\ref{pic:2.2}.
Шарик имеет гораздо меньшую массу, чем вытесняемый им воздух.
Значит, с точки зрения Андрея, центр масс смещается \emph{влево}.
И поскольку центр масс всей системы без внешних сил остаётся неподвижным,
Андрей и вся капсула движутся вправо с точки зрения внешнего наблюдателя.

Ошибка состояла в том, что мы слишком много думали о шарике и недостаточно о более массивном воздухе, который перемещается влево, занимая его место.

\begin{figure}[ht!]
\centering
\begin{lpic}[t(2mm),b(2mm),r(0mm),l(0mm)]{pics/2.2(1)}
\lbl[tl]{22,20;воздух}
\lbl[tl]{22,3.5;воздух}
\lbl{21.5,12.5;{\footnotesize гелий}}
\lbl[bl]{58,12;капсула}
\end{lpic}
\caption{Движение с относительно капсулы (и Андрея).}
\label{pic:2.2}
\end{figure}

\paragraph{То же самое через импульс.}
В приложении (страница \pageref{???}) объяснятся, что неподвижность центра масс эквивалентна тому, что суммарный импульс остаётся равным нулю.

С точки зрения Андрея, вытесненный воздух движется влево.
Это означает, что сам Андрей (вместе с капсулой) обязан двигаться вправо, чтобы скомпенсировать движение воздуха и держать суммарный импульс нулевым.

Всё станет совсем очевидным, если довести соотношение масс до крайности, как на рисунке~\ref{pic:2.3}, где вместо пары гелий-воздух берётся гелий-вода.
Поскольку почти вся масса приходится на воду, она практически не двигается.
А значит, когда шарик с гелием перемещается вправо, оболочка капсулы (чьей массой мы пренебрегаем) тоже движется вправо, уступая место гелию.

\begin{figure}[ht!]
\centering
\begin{lpic}[t(2mm),b(2mm),r(0mm),l(0mm)]{pics/2.3}
\lbl[r]{-1,36.5;до}
\lbl{7,37;гелий}
\lbl{36,36.5;вода}
\lbl[r]{12,10;после}
\lbl{64,11;гелий}
\lbl{36,10.5;вода}
\end{lpic}
\caption{Вода остаётся на месте, а почти невесомая оболочка капсулы сдвигается вправо.}
\label{pic:2.3}
\end{figure}

\paragraph{Назойливое сомнение.}
Приведённый выше ответ верен.
Но разве я не доказал противоположное на странице~\pageref{Првдоподобное рассуждение}, в подразделе с правдоподобным рассуждением?
В чём же ошибка того рассуждения?

\paragraph{Ответ.}
Ошибка в том, что не были учтены \emph{все} силы, действующие на Андрея.
Мы забыли о силе со стороны оболочки капсулы!
Когда Андрей толкает шарик, эта сила передаётся через воздух к оболочке, и та в свою очередь толкает Андрея в спину.
Удивительно, что этот толчок оказывается сильнее толчка шарика.
По сути, ударив шарик, он ещё сильнее ударил себя в спину!
Довольно удивительно (особенно для Андрея).
А как же это происходит?
Интуитивное объяснение этому даст следующий параграф.

\paragraph{Как пнуть себя в спину своим же коленом?}
Этот параграф объяснит, как может капсула толкать Андрея сильнее, чем он толкает шарик.
Чтобы легче это прочувствовать, упростим ненадолго задачу:
будем считать, что шарик надут не гелием, а воздухом.
Значит, когда Андрей толкает шарик, он лишь перераспределяет воздух внутри капсулы.%
\footnote{Предполагается, что оболочка шарика не имеет массы.}
Перемещения воздуха внутри капсулы не меняет его центр масс, а значит ни капусла, ни Андрей не будут двигаться.
По первому закону Ньютона, равнодействующая сила на Андрея будет равна нулю: \emph{его ладони и спина ощутят равные противоположнонаправленные силы}.

А что изменится, если воздух в шарике заменить гелием?
У шарика уменьшится инерция, то есть его будет легче ускорять.
Поэтому при той же силе в спину ладони Андрея ощутят меньшую силу.

\paragraph{Детское воспоминание.}
Ту же ошибку, я допустил в детстве, реализуя свою мечту летать.
В один прекрасный день меня осенило --- я забрался на стул, схватился за сиденье и начал изо всех сил тянуть его вверх.
Я думал, что сиденье потащит меня вверх, а сам я превращусь в человека-ракету (предполагалось, что ножки стула разойдутся в стороны, как антенны спутника).
Но взлёт не удался, ведь я упустил из виду, что мои руки (косвенно) соединены с моим задом
(это стечение обстоятельств сейчас воспринимается удачным и логичным),
таким образом мой зад толкал сиденье вниз, сводя на нет подъёмную силу рук.
Я давил на стул двумя частями своего тела, но одну не учёл.

Сходство с разобранным парадоксом должно быть очевидно.
Как и в первом рассуждении, я не учёл все силы.
Андрей связан с шариком не только руками, но и спиной — через оболочку и воздух;
эту вторую связь мы как раз и упустили в первом рассуждении.

\section[Управление спутником]{Управление спутником без реактивных двигателей}
\label{Управление спутником}

\paragraph{Задача.}
Может ли спутник изменить свою орбиту вокруг Земли, не пользуясь реактивными двигателями, солнечным ветром, и тому подобными средствами тяги?
Разрешается использовать солнечные панели для сбора энергии.

\paragraph{Подсказка.}
Восспользуйтесь тем, что сила тяжести зависит от расстояния до Земли; спутник не обязан быть материальной точкой.

\paragraph{Ответ.}
Самый простой спутник, с которым это сработает, состоит из двух масс, соединённых тросом.
Мотор, питающийся солнечной энергией, будет менять длину троса.
Давайте считать, что спутник движется по орбите, вращаясь, как показано на рисунке~\ref{pic:2.4}.

\begin{figure}[ht!]
\centering
\begin{lpic}[t(2mm),b(2mm),r(40mm),l(40mm)]{pics/2.4}
\lbl{4.7,17;{\footnotesize Земля}}
\lbl[bl]{23,28;орбита}
\end{lpic}
\caption{Гантелеобразный спутник регулируемой длины.}
\label{pic:2.4}
\end{figure}

Орбиту нашего спутника можно поднять или опустить, если надлежащим образом регулировать длину троса!
Будем считать, что спутник изначально вращается и движется по орбите, как показано на рисунке~\ref{pic:2.4}.
Допустим мы хотим понизить орбиту.
Для этого будем следовать инструкциям на рисунке~\ref{pic:2.5}:
подтягивать трос, когда он направлен к Земле, и ослаблять его, когда он почти перпендикулярен направлению к Земле. %??? я не вижу там инструкций!!!
Повторяя это раз за разом на каждом обороте, мы заставим спутник снижаться.
Если же надо поднять орбиту, то следует делать всё наоборот.

\paragraph{Суть.}
Из-за вращательного (кувыркательного) движения трос испытывает центробежное натяжение.
Но не только центробежное, натяжение слегка меняется из-за приливного эффекта, это объясняет рисунок~\ref{pic:2.6}.

\begin{figure}[ht!]
\centering
\begin{lpic}[t(2mm),b(2mm),r(0mm),l(0mm)]{pics/2.5}
\lbl{4.7,17.5;{\footnotesize Земля}}
\lbl[l]{47,17;\parbox{20mm}{движение\\ по орбите}}
\end{lpic}
\caption{Для понижения орбиты, укорачивайте трос, когда он направлен к Земле.}
\label{pic:2.5}
\end{figure}

\begin{figure}[ht!]
\centering
\begin{lpic}[t(2mm),b(2mm),r(0mm),l(0mm)]{pics/2.6}
\lbl{4.7,4.5;{\footnotesize Земля}}
\lbl[t]{32,3;$A$}
\lbl[t]{47,3;$B$}
\end{lpic}
\caption{Тросс натягивается потому, что притяжение в точке $A$ сильнее, чем в точке $B$.}
\label{pic:2.6}
\end{figure}

Как показано на рисунке~\ref{pic:2.6}, натяжение троса большее, когда он направлен к Земле,
ведь массы $A$ и $B$ в этот момент находятся на разных расстояниях от Земли,
и разница сил притяжения в $A$ и $B$, растягивает трос.%
\footnote{Тот же эффект отвечает за приливное вытягивание Земли вдоль прямой Луна---Земля.}
Эту переменную силу натяжения можно использовать для ускорения вращения.
Натягивая трос, когда натяжение больше, и ослабляя его, когда оно меньше, мы совершаем работу.
Эта работа идет на увеличение скорости вращения.%
\footnote{По этому же принципу раскачиваются качели.
Mы поднимаем часть тела, когда $g$-сила больше, и опускаем её, когда $g$-сила меньше.
Так мы совершаем работу, которая переходит в движение качелей
(подробнее это обсуждается на страницах \pageref{Как качаться на качелях?}---\pageref{Почему дорожает энергия?}).
Такой приём «покупай дорого, продавай дёшево» прекрасно работает для подкачки энергии в систему,
но привёл бы к разорению при игре на бирже.}
Поскольку суммарный момент импульса%
\footnote{Справка о моменте импульса дана на странице \pageref{???}.} сохраняется увеличивая вращательный момент, мы уменьшаем орбитальный момент.
А уменьшение орбитального углового момента означает, что спутник \emph{понижает} свою орбиту --- это будет объяснено чуть ниже.

\paragraph{Объяснение размазыванием.}
Есть ещё один способ понять, как увеличивается момент импульса спутника.
Представьте, что массы двух шариков, составляющих гантель, размазаны вдоль их траекторий,
как на рисунке~\ref{pic:2.5}.
То есть мы заменили две массы на что-то вроде обруча из провода.
Суть в том, что этот обруч наклонён по отношению к направлению на Землю, и поэтому на него действует момент  силы, вызванный тем, что ближе к Земле притяжение сильней.
Этот момент пытается повернуть обруч против часовой стрелки,
то есть \emph{увеличивает} его момент импульса — в точности как мы утверждали в исходном объяснении.
(Та же идея с размазыванием используется при объяснении гимнастического элемента на странице~\pageref{Большие обороты на перекладине}.)

\paragraph{Уточнения.}
Вот пара уточнений к предыдущим объяснениям, включая то, что чем выше орбита, тем больше момент вращения.

Поскольку сила тяжести направлена строго к центру Земли, она не может менять полный момент импульса спутника.%
\footnote{Это не совсем правда, поскольку Земля не совсем круглая, но не будем обращать на это внимание.}
Значит полный момент импульса $M$ спутника относительно центра Земли остаётся постоянным, независимо от того, что мы делаем с тросом; то есть,
\[
M = M_{\text{вращ}} + M_{\text{орб}} = \mathrm{const}.
\]
Таким образом, увеличивая вращательный момент $M_{\text{вращ}}$, мы уменьшаем орбитальный момент $M_{\text{орб}}$.
А этот момент связан с радиусом орбиты $r$ формулой
\begin{equation}
\sqrt{M_{\text{орб}}} = k \sqrt{r}, \label{eq:2.1}
\end{equation}
где $k = m \sqrt{G M}$; здесь $m$ --- масса спутника, $G$ --- гравитационная постоянная, а $M$ --- масса Земли.%
\footnote{Действительно, при движении по круговой орбите радиуса $r$, непосредственно из определения момента импульса получаем, что
\begin{equation}
M_{\text{орб}} = m v r. \label{eq:2.2}
\end{equation}
Здесь $v$ — это скорость и она связана с $r$ вторым законом Ньютона.
Согласно этому закону, центростремительное ускорение $\frac{v^2}{r}$ определяется силой притяжения:
\[
\frac{m  v^2}{r} = \frac{G m M}{r^2},
\]
и значит
$v = m  \sqrt{G M}/\sqrt{r}$.
Подставив это выражение в \eqref{eq:2.2}, получим \eqref{eq:2.1}.}
По этой формуле, уменьшение $M_{\text{орб}}$ влечёт уменьшение радиуса орбиты $r$.
Это подтверждает ранее сделанное утверждение: ускоряя вращение (кувыркание), мы уменьшаем орбитальный момент, и значит спутник снижается.

\paragraph{Другие манёвры.}
Можно добиться и большего.
Например, можно менять эксцентриситет орбиты,
но как это делать предоставляется выяснить читателю.

\section{Парадокс с кометой}\label{Парадокс с кометой}

Обсуждение движения пушечных ядер часто начинают с того, что горизонтальная скорость ядра не меняется, поскольку в горизонтальном направлении не действуют никакие силы (сопротивлением воздуха пренебрегаем).
Вот попытка применить то же рассуждение для движения в открытом космосе.

\begin{figure}[ht!]
\centering
\begin{lpic}[t(2mm),b(2mm),r(0mm),l(0mm)]{pics/2.7}
\lbl[lb]{6,60.5;Солнце}
\lbl[lt]{6,27;Солнце}
\lbl[lt]{11,-.5;Солнце}
\lbl[lb]{10,3.5;$S$}
\lbl[l]{50,12;\parbox{40mm}{в этих направлениях\\силы не действуют}}
\lbl[lt]{29,34;притяжение}
\lbl[rb]{23,40;$v$}
\lbl[b]{36,44;$v_\perp$}
\lbl[rt]{42.5,9;$C$}
\lbl[t]{2,53;\parbox{25mm}{быстро\\вращается}}
\lbl[tl]{49,52;\parbox{40mm}{медленно\\вращается}}
\end{lpic}
\caption{(a) Никакая сила не действует перпендикулярно к линии $SC$.
Остаётся ли $v_\perp$ постоянной?
(b) Линия Солнце---комета поворачивается быстрее, когда комета ближе к Солнцу.
%??? не надо ли поменять местами катинки (или подписи)
}
\label{pic:2.7}
\end{figure}

Солнце может только притягивать к себе комету, и не может раскручивать её вокруг себя,
ведь сила притяжения всегда направлена точно в сторону Солнца (см. рисунок~\ref{pic:2.7}).
Иными словами, сила солнечного притяжения не имеет компоненты
в направлении, перпендикулярном прямой, соединяющей Солнце и комету: \( \bm{F}_\perp = 0 \).
Поскольку сила в перпендикулярном направлении равна нулю,
по первому закону Ньютона (если сила нулевая, то скорость не меняется), получаем,
что, компонента скорости \(v_\perp\) в этом направлении не меняется.

Но тут появляются сомнения.
Момент импульса%
\footnote{Определения и пояснения даны на странице ???.}
кометы обязан сохраняться:
\[
M = mv_\perp \cdot r = \text{const},
\]
где \(r\) — расстояние от кометы до центра Солнца.
Поскольку \(r\) меняется по мере движения кометы по эллиптической орбите,
компоненте скорости \(v_\perp\) придётся меняться, чтобы оствить произведение \(v_\perp \cdot r\) неизменным.
Что же из этого верно (если верно хоть что-то)?

\parbf{Решение.}
Ошибка в первом рассуждении: я неверно использовал первый закон Ньютона.
Этот закон справедлив только в инерциальной системе отсчёта,%
\footnote{То есть в системе отсчёта, которая движется без ускорения и без вращения.
Подробности по закону Ньютона даны в приложении на странице ???.}
то есть я неявно и неверно предположил, что прямая $SC$ есть ось координат инерциальной системы.
Но ось $SC$ вращается, а значит эта система не инерциальна.

\section[Хочешь медленнее --- разгоняйся!]{Хочешь медленнее --- разгоняйся!\footnote{Эта же задача обсуждается в книге «Смотри в корень!» Маковетского \cite[Задачa 22 «Хочешь быстрее --- тормози»]{makovetskij}. \pr}}
\label{Хочешь медленнее}




\paragraph{Вопрос.}
Космический корабль движется по круговой орбите вокруг планеты.
Желая повысить орбиту, космонавт включает двигатели, разгоняя корабль вперёд.
После выхода на новую круговую орбиту, двигатели выключаются.
Пока работали двигатели кораблю сообщалась ускорение примерно в направлении движения; движется ли теперь он быстрее, чем раньше?

\paragraph{Ответ.}
На самом деле корабль сбавил скорость.

\paragraph{Обяснение.}
Ответ покажется не столь странным, если подумать о езде на велосипеде;
ведь въезжая на горку можно замедлиться даже сильно давя на педали.
То же самое происходит и с космическим кораблём: подъём орбиты — это та же горка.
Энергия двигателей тратится не на ускорение, а на преодоление тяготения.
Корабль получает потенциальную энергию, но теряет кинетическую;
при этом прирост потенциальной энергии превышает потерю кинетической.

\begin{figure}[ht!]
\centering
\begin{lpic}[t(2mm),b(2mm),r(0mm),l(0mm)]{pics/2.8}
\lbl[t]{16,8;толчок}
\lbl[b]{16,33;толчок}
\lbl[ul]{35,21;траектория}
\lbl[r]{9.5,16;$1$}
\lbl[l]{27,16;$2$}
\lbl[rb]{4,27;$3$}
\end{lpic}
\caption{Разгон замедляет.
Два коротких импульса в направлении движения прикладываются в отмеченных точках траектории.
В результате орбита поднялась, но корабль сбавил скорость.
}
\label{pic:2.8}
\end{figure}

А как именно зависит орбитальная скорость \(v\) от радиуса орбиты~\(r\)?
Оказывается, что:
\[
v = \frac{k}{\sqrt{r}},
\]
где \(k = \sqrt{GM}\),
\(G\) — универсальная гравитационная постоянная,
а \(M\) — масса планеты.
Когда спутник находится на круговой орбите, второй закон Ньютона \(ma = F\) говорит, что центростремительное ускорение%
\footnote{То есть ускорение по направлению к центру; подробности и пояснения на странице ???.}
$a = \frac{v^2}{r}$ обеспечивается силой притяжения $F = {GmM}/{r^2}$, и, значит,
\[
m \frac{v^2}{r} = \frac{GmM}{r^2}.
\]
где \(m\) — масса спутника.
Отсюда получаем
\[
v = \sqrt{\frac{GM}{r}}.
\]
Согласно этой формуле при подъёме (то есть при увеличении \(r\)) спутник замедляется.
%дважды сказано про G???


{

\sloppy

\printbibliography[heading=bibintoc]


\fussy

}


\newpage

{

\tableofcontents

}

\end{document}
