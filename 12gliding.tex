\chapter{Парус и крыло}

\paragraph{Вопрос.}
Можно ли в безветренный день двигаться под парусом по реке?

\paragraph{Ответ.}
Да, парус работает без ветра благодаря течению; см. рисунок \ref{pic:12.1}.
В неподвижном воздухе парус двигается, как нож в масле, рассекая воздух, он двигается вдоль своей плоскости.
А вот киль толкает течение, и поэтому лодка может плыть, например, под прямым углом к берегу, как это показано на рисунке.
Можно думать, что киль играет роль паруса, а текущая вода — роль дующего ветра,
сам же парус, рассекая воздух, ведёт себя как киль!
Получается обычное движение под парусом, только вверх ногами.

\begin{figure}[ht!]
\centering
\begin{lpic}[t(7mm),b(2mm),r(0mm),l(0mm)]{pics/12.1}
\lbl[r]{-1,20;{\footnotesize течение}}
\lbl[t]{41,20;{\footnotesize киль}}
\lbl[r]{33,33;\parbox{20mm}{\footnotesize\raggedleft направление\\паруса}}
\lbl[l]{35,7;\parbox{35mm}{\footnotesize направление движения\\(относительно берега)}}
\lbl[b]{39,42;\parbox{25mm}{\footnotesize\centering удерживающий\\трос}}
\end{lpic}
\caption{Когда воздух неподвижен парус и воздух меняются ролями с килем и водой.}
\label{pic:12.1}
\end{figure}

\paragraph{Симметрия паруса и киля.}
Мы только что рассмотрели лодку с точки зрения наблюдателя на берегу.
Но представьте себя на лодке,
для вас вода неподвижна, а ветер, наоборот, дует вверх по течению.
То есть вы окажетесь в привычной ситуации — лодка идёт под парусом при ветре.
Ваш парус будет ловить ветер как обычно, и киль, как обычно, рассекать воду.
Получается симметрия: парус и киль меняются ролями в зависимости от системы отсчёта!
В этом отношении у паруса и киля равные права.%
\footnote{На самом деле есть одно различие: киль выровнен по корпусу, а парус — нет.
Удобнeй считать киль с корпусом единым целым, и думать, что лодка состоит из двух частей: (1) киль+корпус и (2) парус.}

\paragraph{Задача.}
В каких направлениях может двигаться лодка по реке?
Считаем, что воздух неподвижен, а река течёт.

\paragraph{Решение.}
В обычной ситуации (нет течения и дует ветер) лодка может идти во всех направлениях, за исключением направлений почти против ветра.
Но наша лодка на реке находится в точно такой же ситуации, только киль играет роль паруса и наоборот.
Поэтому парус (а значит и лодка) может двигаться во всех направлениях, за исключением направлений почти против течения.

\section{Вишнёвые косточки и паруса}

Поедание вишни и стральба друг в друга её косточкам предоставляют неиссякаемый источник детских радостей.
Как мы знаем, чтобы стрельнуть косточкой, надо сжать её между пальцами и выпустить в нужном направлении.

\begin{figure}[ht!]
\centering
\begin{lpic}[t(2mm),b(2mm),r(0mm),l(0mm)]{pics/12.2}
\lbl[bl]{-2,34;(a)}
\lbl[bl]{35,34;(b)}
\lbl[bl]{64,34;(c)}
\lbl[t]{3,2;{\footnotesize ветер}}
\lbl[t]{22,2;{\footnotesize течение}}
\lbl[b]{34,19;{\footnotesize ветер}}
\lbl[b]{55,19;{\footnotesize \phantom{р}течение}}
\lbl[t]{24,22;{\footnotesize киль}}
\lbl[b]{5,15,60;{\footnotesize парус}}
\lbl{60,0;{\footnotesize похоже}}
\lbl[l]{14,34;\parbox{20mm}{\footnotesize направление\\ движения}}
\end{lpic}
\caption{
(a, b)
Ветер и течение стараются раздвинуть клин из паруса и киля, заставляя его скользить в направлении стрелки.
(c) Если же клин сжимать, то он будет скользить в противоположном направлении.}
\label{pic:12.2}
\end{figure}

\paragraph{Вопрос.}
Чем сила, разгоняющая вишнёвую косточку, похожа на силу, движущую парусную лодку?

\paragraph{Ответ.}
На рисунке \ref{pic:12.2}a ветер и течение протвоположно направлены и их скорости равны,
а парусная лодка движется перпендикулярно их направлениям.
Обратите внимание на то, что ветер и течение подобны двум пальцам, которые давят на парус и киль изнутри, как показано на рисунке \ref{pic:12.2}b.
Эти пальцы расходятся, заставляя лодку плыть по стрелке.
По сути то же самое происходит, когда мы стреляем косточкой, только наши пальцы сходятся, а расходятся.

\paragraph{Влияние системы отсчёта.}
Предположив, что воздух и вода движутся с равными и противоположными скоростями, мы выбрали систему отсчёта, движущуюся со средней скоростью ветра и течения.
Если же мы предположим, что воздух неподвижен, то тем самым привяжем систему отсчёта к воздуху.
Всё это показано на рисунке \ref{pic:12.3}.
Прямо как в жизни --- сменив точку зрения, мы получаем новое понимание.

\begin{figure}[ht!]
\centering
\begin{lpic}[t(2mm),b(7mm),r(0mm),l(0mm)]{pics/12.3}
\lbl[t]{8,2;{\footnotesize без течения}}
\lbl[t]{39,2;\parbox{40mm}{\footnotesize\centering безветрие:\\
киль работает как парус,\\
а парус как киль}}
\lbl[b]{17,23;{\footnotesize киль}}
\lbl[b]{4,16,60;{\footnotesize парус}}
\lbl[t]{67,12;{\footnotesize вода}}
\lbl[t]{65,7;{\footnotesize воздух}}
\lbl[t]{38,7;{\footnotesize вода}}
\lbl[t]{8,7;{\footnotesize воздух}}
\end{lpic}
\caption{Движение под парусом в трёх системах отсчёта.}
\label{pic:12.3}
\end{figure}

\section{Как плыть точно против ветра}

\paragraph{Вопрос.}
Парусная лодка способна двигаться во всех направлениях, кроме тех, что направлены почти против ветра.
Чтобы попасть из точки $A$ в точку $B$ прямо против ветра, лодке приходится идти зигзагами (лавировать).
Но ведь былo бы здорово двигаться в точности против ветра.
Можно ли сконструировать такую лодку?

\paragraph{Ответ.}
Лодка подходящей конструкции показана на рисунке \ref{pic:12.4}.
Более того, такая лодка будет автоматически, подстраивать направление при изменении ветра.%
\footnote{Более впечатляющую конструкцию можно найти в заметке ``Wind-propeller sails proposed for liners'' Modern Mechanix and Inventions, January 1935, p. 49.
%\href{http://blog.modernmechanix.com/2007/11/27/wind-propeller-sails-proposed-for-liners/}{blog.modernmechanix.com/2007/11/27/wind-propeller-sails-proposed-for-liners}.
}
Устройство очень простое:
длинный стержень одним концом уходит в воду, а другим висит в воздухе,
на концах стержня воздушный пропеллер и гребной винт.
При правильном выборе размеров пропеллера и винта, ветер будет вращать пропеллер, а тот, в свою очередь, будет заставлять гребной винт ввинчивать лодку против ветра.
Поскольку пропеллер находится на корме, лодка сама будет разворачиваться против ветра.


\begin{figure}[ht!]
\centering
\begin{lpic}[t(2mm),b(2mm),r(0mm),l(0mm)]{pics/12.4}
\lbl[b]{24,18,-20;{\footnotesize стержень}}
\lbl[l]{54,25;{\footnotesize ветер}}
\end{lpic}
\caption{Такая лодка сама будет разворачиватся и двигаться против ветра.}
\label{pic:12.4}
\end{figure}

\section{Против ветра на велосипеде}

Ехать на велосипеде против сильного встречного ветра может быть тяжело.
А может ли в этом помочь ветрогенератор, установленный на велосипеде?
Можно поытатся использовать энергию для езды.
У этой идеи есть очевидные плюсы и минусы,
но давайте отбросим практические соображения,
и зададимся вопросом могло бы это помочь в принципе.

Рассмотрим конструкцию на рисунке \ref{pic:12.5}.
Ветрогенератор питает электродвигатель, который помогает крутить педали.
Будем считать, что \emph{ветрогенератор и двигатель идеальны}.
В частности, если я протащу ветрогенератор сквось неподвижный воздух, то вся мощность, потраченная на сопротивление воздуха, возместиться электроэнергией.
Иными словами, в безветренную погооду общее влияние моего приспособления окажется нулевым.%
\footnote{На практике, эта затея генерировала бы в основном смех.}
Ветер обладает энергией, есть ли способ её использовать?

\begin{figure}[ht!]
\centering
\begin{lpic}[t(2mm),b(2mm),r(0mm),l(0mm)]{pics/12.5}
\lbl[bl]{-2,37;(a)}
\lbl[bl]{35,37;(b)}
\lbl[br]{2,18;$F_{\text{вел}}$}
\lbl[bl]{40,18;$F_{\text{вел}}$}
\lbl[bl]{30,20.5;$F_{\text{вел}}$}
\lbl[b]{75,20.5;$F_{\text{вел}}+F_{\text{проп}}$}
\lbl[bl]{40,33;$F_{\text{проп}}$}
\lbl[tr]{2,16;{\footnotesize (воздух)}}
\lbl[tl]{30,18.5;{\footnotesize (я)}}
\lbl[t]{27,14;$v$}
\lbl[t]{71,14;$v$}
\lbl[tl]{78,16;$v_{\text{ветер}}$}
\end{lpic}
\caption{Велосипед с ветрогенератором.}
\label{pic:12.5}
\end{figure}

\paragraph{Вопрос.} Будет ли преимущество от моего ветрогенератора при езде против ветра?

\paragraph{Ответ.}
Давайте вычислим разницу между мощностью%
\footnote{Мощность определяется как работа, совершаемая за еденицу времени.
Работа же (согласно определению) равна силе, которую я прикладываю, умноженной на пройденное расстояние в направлении этой силы.
Следовательно, мощность равна произведению силы на скорость в направлении действия этой силы.}
%
, затрачеваемую с ветрогенератором и без него.
Будем считать, что я еду на велосипеде против ветра с фиксированной скоростью $v$.

\begin{enumerate}
\item \emph{Без ветрогенератора}, для езды с постоянной скоростью $v$, надо прикладывать силу, равную (и противонаправленой) сопротивлению воздуха велосипеда и велосипедиста $F_{\text{вел}}$.
Значит, требуемая мощность будет
\begin{equation}
F_{\text{вел}} v.
\label{eq:12.1}
\end{equation}
\item \emph{С ветрогенератором} придётся тащить велосипед, велосипедиста, и ещё пропеллер.
Поэтому понадобится б\'{о}льшая сила:
\[
F_{\text{вел}} + F_{\text{проп}},
\]
где $F_{\text{проп}}$ — это сила, действующая на пропеллер со стороны ветра.
Общая затрачиваемая мощность теперь равна
\[
(F_{\text{вел}} + F_{\text{проп}})v.
\tag{12.2}
\]
Это больше, чем раньше, но часть энергии возмещается ветрогенератором.
Сколько же именно?
Ветрогенератор движется сквозь воздух со скоростью $v + v_{\text{ветер}}$.
Согласно нашему прежнему предположению об отсутствии потерь его мощность равна
\[
F_{\text{проп}}(v + v_{\text{ветер}}).
\]
Следовательно, на мою долю, остаётся
разность между общей мощностью, и мощностью ветрогенератора:
\[
(F_{\text{вел}} + F_{\text{проп}})v - F_{\text{проп}}(v + v_{\text{ветер}})
= F_{\text{вел}} v - F_{\text{проп}} v_{\text{ветер}}.
\]
Сравнив этот результат с \eqref{eq:12.1}, видим, что выйгрыш мощности равен
\[
F_{\text{проп}} v_{\text{ветер}}.
\]
\end{enumerate}

Итак, \emph{при езде на велосипеде против ветра, ветрогенератор%
\footnote{Учитывая выше упомянутые соглашения.}%
, даёт тот же выйгрыш в мощности, как если бы он работал стационарно.}

\paragraph{Вопрос.} Предположим, что у пропеллера ветрогенератора и у рекламного щита равные сопротивления воздуха.
Как отличаются кинетическая энергия потоков воздуха позади этих двух препятствий?

\paragraph{Ответ.}
Ветрогенератор превращает часть кинетической энергии воздуха в электроэнергию.
Рекламный щит почти не меняет кинетичесую энергию воздуха.
Поэтому воздух за ветрогенератором будет спокойней (он будет обладать меньшей кинетической энергией), чем за щитом.

\section{Парение без восходящих потоков}

\paragraph{Вопрос.}
Может ли планер равномерно набирать высоту без восходящих потоков?
Отсутствие восходящих потоков означает, что воздух движется исключитлььно в горизонтальном направлении.

\paragraph{Ответ.}
Да, по крайней мере в принципе — например, если скорость ветра меняется с высотой.

\paragraph{А почему?}
Представьте себе, что скорость ветра возрастает с высотой (см. рисунок \ref{pic:12.6}).
Запустим планер против ветра.
В неподвижном воздухе набирающий высоту планер быстро теряет скорость и начал бы падать.
Но поскольку ветер усиливается с высотой, происходит удивительное: по мере подъёма планер попадает в область более сильного встречного ветра.
Поднимаясь, он как бы встраивается во всё более быстро движущийся поток, удерживая свою скорость относительно воздуха без двигателей.
Разумеется, при этом планер также сносит «назад».
Но, набрав скорость, он может спуститься и вернуться в исходную точку, продолжая цикл бесконечно.
Некоторые морские птицы умеют использовать этот механизм,
они способны почти без усилий планировать при отсутствии восходящих потоков.

\begin{figure}[ht!]
\centering
\begin{lpic}[t(2mm),b(2mm),r(0mm),l(0mm)]{pics/12.6}
\lbl[bl]{-2,37;(a)}
\lbl[bl]{39,37;(b)}
\lbl[t]{14,32;{\footnotesize 80 км/ч}}
\lbl[t]{61,32;{\footnotesize 80 км/ч}}
\end{lpic}
\caption{Набор высоты без восходящих потоков.}
\label{pic:12.6}
\end{figure}

\paragraph{О ловкости птиц.}
Океанские волны заставляют воздух двигаться вверх-вниз, создавая кратковременные восходящие и нисходящие потоки.
При приближении гребеня волны, вместе с поверхностью воды поднимается воздух, а когда гребень удаляется, воздух опускается.
Это не те устойчивые термические восходящие потоки, которые возникают над сушей.
Но если птица летит близко к воде, оставаясь чуть впереди гребня волны, она будет всё время находится в зоне восходящего потока и может долго парить, без усилий.
Это умеют делать пеликаны, альбатросы и некоторые другие морские птицы.
Они поступают так же, как сёрферы-люди, но им не нужно касаться воды!
Крылья птиц работают как доски для воздушного сёрфинга.
Это даёт ещё один способ понять, что делают сёрферы: они удерживаются на той части волны, которая движется вверх.

\begin{figure}[ht!]
\centering
\begin{lpic}[t(2mm),b(2mm),r(0mm),l(0mm)]{pics/12.7}
\lbl[tl]{43,6;\color{white}{\footnotesize вода поднимается}}
\lbl[tr]{23,6;\color{white}{\footnotesize вода опускается}}
\lbl[bl]{43,22;{\footnotesize воздух поднимается}}
\lbl[br]{23,22;{\footnotesize воздух опускается}}
\lbl[b]{30,33;{\footnotesize распространение волны}}
\end{lpic}
\caption{Чтобы оседлать волну надо оставаться в зоне потока, движущегося вверх.}
\label{pic:12.7}
\end{figure}

\paragraph{Может ли дельтаплан оседлать волны?}
Хоть это и кажется неверояным,
но в принципе дельтаплан мог бы использовать гигантские волны для воздушного серфинга
Он мог бы планировать, двигаясь вдоль гребня, поднимаясь перед тем, как волна сломается, переходя на следующую волну и так далее; см. рисунок \ref{pic:12.8}.
Я бы хотел попробовать такое проделать с радиоуправляемым планером.

\begin{figure}[ht!]
\centering
\begin{lpic}[t(7mm),b(2mm),r(0mm),l(0mm)]{pics/12.8}
\lbl[t]{10,11,-18;{\footnotesize волна}}
\lbl[t]{34,3,-18;{\footnotesize волна}}
\end{lpic}
\caption{Воздушный серфинг.}
\label{pic:12.8}
\end{figure}

\paragraph{Вопрос.}
Сёрфер движется по волне с постоянной скоростью.
Что ему следует сделать, чтобы ускориться?
Будем считать, что волна движется с постоянной скоростью, не меняя формы.

\paragraph{Ответ.}
Чтобы ускориться, надо уменьшить угол между направлением своего движения и линией гребня.
При этом придётся выйти на более крутую часть волны.

\section{Опасность горизонтального ветра со сдвигом скорости}

Парение без восходящих потоков привлекает, но у него есть и опасная сторона.
Предствьте, что скорсть ветра уменьшается при уменьшении высоты, планер летит против ветра с малой скорстью относительно воздуха (рисунок \ref{pic:12.6}b).
Желая разогнаться, пилот опускает нос;
без ветра это увеличило бы скорость.
Но в нашем случае у планера есть шанс \emph{свалиться}; то есть он потеряет скорость относительно воздуха настолько, что потеряет управление.

\paragraph{Вопрос.}
А что будет чувствовать парашютист при горизональном ветре, который меняет скорость с высотой?

\paragraph{Ответ.}
Если ветра нет, то парашютист не будет ощущать горизонтального ветра.
Однако при изменении скорости ветра от высоты парашютист будет чувствовать, будто на него дует горизонтальный ветер, как будто на него дует невидимый пропеллер в горизонтальном направлении.
